\subsection{Estimators}

% TODO: entire section

% boilerplate code
% graph and session management
%\textcolor{blue}{}
\subsubsection{Input data}

% TODO: input function overview

\paragraph{Specifying Hyper Parameters}

\subparagraph{Epochs}
\textcolor{blue}{by default, training will continue until the training data is exhausted, or the number of specified epochs is reached}

Options:

%TODO: code example
\begin{enumerate}
	\item input\_fn
	\item steps
	\item max\_steps --- will potentially do nothing if the checkpoint has already reached this value
\end{enumerate}


\paragraph{In Memory Data}

\textcolor{blue}{Usually this is in the form of either numpy arrays or pandas dataframes. These can both be used directly.}

% TODO: examples for Numpy array

% TODO: examples for Pandas DF

\paragraph{Out of Memory Data}

\textcolor{blue}{In the ``real world'' the dataset will likely not fit into memory. To (sanely) address this, estimators play nicely with the tf.Data API (please see \textcolor{red}{local ref} for more information)}

% TODO: show quick demo example

% tf.estimator base class allows you to build your own model 

% premade models (TODO: Show quick list)

% main advantage -- estimators are interchangable

% "reasonable" defaults for each estimator

\subsubsection{Checkpoints}

\textcolor{blue}{directory specified when creating model. By default, predictions will be made from the latest checkpoints in this directory. training also resumes from the latest checkpoint in the directory. --- to start from scratch, the directory will need to be deleted or specified to a new location.}

\subsubsection{Distributed}

% you need: 1. estimator, 2. run config, 3. train spec, eval spec

% final call tf.estimator.train_and_evaluate(estimator, train_spec, eval_spec)

\paragraph{tf.estimator.RunConfig}

% TODO: example

\textcolor{blue}{the directory for checkpoints and Tensorboard logs and freq of checkpoints (save\_checkpoint\_steps) and frequency of logs (save\_summary\_steps)  }

\paragraph{tf.estimator.TrainSpec}

\textcolor{blue}{pass in input function (likely through data API (see \textcolor{red}{localref})), }


\paragraph{tf.estimator.EvalSpec}

\textcolor{blue}{pass in input function for evaluation dataset (likely through data API (see \textcolor{red}{local ref})),}

\textcolor{blue}{creates model and loads latest checkpoint, then runs eval. Therefore, you cannot get a frequency greater than the checkpoints created. They can be obtained less frequently by using the `throttle\_specs` parameter}

\paragraph{Notes}

\textcolor{blue}{Shuffling considerations.use the dataset = tf.data.Dataset().list\_files().shuffle() command --- each worker a different seed? Even if the data is shuffled on disk. Can also use the dataset().shuffle()}

\subsubsection{TensorBoard}

% TODO: example -- to create: tf.estimator.RunConfig(model_dir='some_dif')

\textcolor{blue}{to visualize, then open tensorboard by issuing `tensorboard --logdir output\_dir` and the dashboard will appear on localhost:6006}

\textcolor{blue}{pre-made estimators already export relevant metrics, embeddings, histograms, etc.. for more information on how to use tensorboard please see \textcolor{red}{local ref}}

\textcolor{blue}{if building a custom estimator or would like to add additional information to tensorboard, summaries can be added with any of the following: tf.summary.scalar, tf.summary.image, tf.summary.audio, tf.summary.text, tf.summary.histogram}

\paragraph{Adding Custom}

% tf.summary.scalar('meanVar_01', tf.reduce_mean(var_01))

\subparagraph{tf.summary.scalar}

\subparagraph{tf.summary.image}

\subparagraph{tf.summary.audio}

\subparagraph{tf.summary.text}

\subparagraph{tf.summary.histogram}


\subsubsection{Deployment}

% TODO: examples

%\textcolor{blue}{two things 1) export\_latest = tf.estimator.LatestExporter(serving\_input\_receiver\_fn=serving\_input\_fn) and then eval_spec = tf.estimator.EvalSpec(input\_fn=eval\_input\_fn, exporters=export\_latest)}

% will map from JSON from REST API and the model

% important to use tf commands in the input transformation/parsing function

\paragraph{Exporters}

\textcolor{blue}{there are many types of exporters and exporter schemes. The simplest may be the tf.estimator.LatestExporter}


