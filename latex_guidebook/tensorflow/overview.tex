\chapter{Overview}

% TODO: this chapter needs serious restructuring/organization consideration

\section{Introduction}

\textcolor{green}{TODO: Tensorflow is an open-source library for numerical computation (not only as a deep/machine learning framework), released by google XXXXXXXXX.}

\textcolor{green}{TODO: There are many different deep learning frameworks}

\textcolor{green}{TODO: why use a deep learning framework}

\section{High-level Overview of Components}

\textcolor{green}{TODO: paras with local refs to components}

\textcolor{green}{TODO: Diagram of tensorflow at a high level}


\section{TensorFlow Essentials}

\subsection{API Hierarchy}

\subsection{Placeholders, feed\_dict, Variables and Constants}

\textcolor{blue}{using get\_variable to create the variable}

\subsection{Coding Style}

\textcolor{blue}{Imperative vs lazy evaluation.}

\subsection{Graphs}

\textcolor{blue}{TODO: Graphs.}

\textcolor{blue}{directed}

\textcolor{blue}{acyclic}

\textcolor{blue}{Why directed acyclic graph (DAG) representation? Language and Hardware Portability -- Developers get to develop programs in a high level language (like python) and have the TensorFlow execution engine execute (written in C++) the model on different platforms (targeted for exact hardware)}

\subsection{Sessions}

\textcolor{blue}{TODO: Sessions.}

\subsection{Tensors}

\textcolor{blue}{TODO: what is a tensor. Tensor --- an n dimensional array of data. they ``flow" through the (directed, acyclic) graph --- hence, TensorFlow \textcolor{red}{local ref to tensor/matrix}}



\section{Workflow}

\textcolor{green}{TODO: Diagram of a typical workflow}

% lazy evaluation
\textcolor{blue}{first step is to build the graph, the second step is to execute the graph (in a session, which will evaluate to numpy arrays)}

% eager mention
\textcolor{blue}{There exists another mode of operation called ``eager'', in which the operations are executed imperatively (tf.eager is discussed in further detail in \textcolor{red}{local ref})}

\textcolor{blue}{TODO: list of overloaded operations, common arithmetic operators/shorthand}

% TODO: this belongs somewhere else
\textcolor{blue}{Note about how training is more computationally expensive than inference}

\section{Debugging TensorFlow}

\textcolor{blue}{Before writing any TensorFlow, I'd like to share some tips and tools to debug TensorFlow programs.}


\textcolor{blue}{error messages are your friend. Use the error message to isolate and debug the operation that is causing problems.}

\textcolor{blue}{two pieces of information: i) stack trace and ii) error message (type+message and operation) }

\subsection{Common Errors}
% tensor shape
% scalar-vector mismatch
\subsubsection{Shape}

% TODO: show how to use these common operations and their output
\textcolor{blue}{some common operations to change the shape of a tensor may included i) tf.reshape() ii) tf.expand\_dims(), iii) tf.slice(), and iv) tf.squeeze() --- expand and squeeze are inverses} 

\subsubsection{Data Type}
% data type mismatch

\subsection{Debugging Tools}

\subsubsection{tf.Print}
% log specific tensor values
%TODO: Code Examples

\subsubsection{tfdbg}

%TODO: Code Examples

% python super_tf_model.py --debug

\subsubsection{TensorBoard}

%TODO: diagram examples

\textcolor{blue}{Using TensorBoard for monitoring is discussed in a later section(\textcolor{red}{local ref}). The scope in this section will focus on using TensorBoard to debug a program}

\subsubsection{Logging and Verbosity}

% different modes overview debug --> fatal
% info = development, warn = production
% tf.logging.set_verbosity(tf.logging.INFO)





