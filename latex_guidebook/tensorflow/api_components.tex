\section{Datasets}

%%%%%%%%%%%%%%%%%%%%%%%% Bias
\section{Bias}

\subsection{Types of Bias}

\subsubsection{Interaction Bias}

\subsubsection{Latent Bias}

\subsubsection{Selection Bias}

\subsubsection{Recency Bias}

\subsection{Evaluation}

\textcolor{blue}{Performance evaluation on subsets of your data}

\textcolor{blue}{Evaluating false positives and false negatives for these subsets in the context of the problem/application.}

\textcolor{blue}{Equality of Opportunity -- gives individuals an equal chance}


%%%%%%%%%%%%%%%%%%%%%%%% Acquiring Data
\section{Data Acquisition}


\subsection{Resources}

\subsection{Generating Fake Data}

\textcolor{green}{TODO: generating fake data with SKL}

\textcolor{blue}{make\_blobs}

% {{{datagen_blobs_2dcode}}}
\begin{lstlisting}[style=pyInStyle]
X, y = datagen.make_blobs(centers=4, n_samples=100, n_features=2,cluster_std=1.0,
                          center_box=(-10, 10),
                          random_state=42, shuffle=True)
\end{lstlisting}

% {{{datagen_blobs_2dimg}}}
\begin{figure}
\centering
\includegraphics[width=0.65\textwidth]{./sync_imgs/datagen/blobs/2dimg.png}
\label{fig:datagen_blobs_2dimg}
\end{figure}

\textcolor{blue}{Data can also be generated in three (multiple) dimensions}

% {{{datagen_blobs_3dcode}}}
\begin{lstlisting}[style=pyInStyle]
X, y = datagen.make_blobs(centers=4, n_samples=100, n_features=3, random_state=42)
\end{lstlisting}

% {{{datagen_blobs_3dimg}}}
\begin{figure}
\centering
\includegraphics[width=0.65\textwidth]{./sync_imgs/datagen/blobs/3dimg.png}
\label{fig:datagen_blobs_3dimg}
\end{figure}

%\textcolor{blue}{More dataset types can be generated, the documentation can be found at (http://scikit-learn.org/stable/modules/classes.html#module-sklearn.datasets)}


\textcolor{blue}{see \textcolor{red}{local ref?} for more examples on how to generate data}


%%%%%%%%%%%%%%%%%%%%%%%% Data types
\section{Data Types}

\TD{create hierarchical diagram here}

\TD{another ``levels of measurement'' diagram}

\subsection{Numerical (quantitative)}

\subsubsection{Discrete}

\subsubsection{Continuous}

\subsection{Categorical (qualitative)}

\r{main subclasses are nominal and ordinal. Another subclass may be something called ``cyclic''. Cyclic variables are varibles that occur in cycles, for example, the months of a year Jan (2019), Feb (2019), ... Dec (2019), Jan (2020), Feb (2020), etc.}

\subsubsection{Nominal}

\r{unordered}

\TD{example}

\subsubsection{Ordinal}

\r{ordered}

\TD{example}

%%%%%%%%%%%%%%%%%%%%%%%% Data Pre-processing
\section{Data Pre-processing}

\textcolor{blue}{Data is rarely obtained in a form that is necessary for optimal performance of a learning algorithm. Data can be missing, can contain a mix of categorical and quantitative, can contain values on vastly different scales, etc.}

\textcolor{blue}{It is important to note that any parameters related to data pre-processing, such as feature scaling and dimensionality reduction, are obtained solely from observing the training set. The parameters for these methods obtained on the training set are then later applied to the test set. This is important since if these preprocessing parameters were obtained on the entire dataset and included the test set, the the model performance may be overoptimistic since then when applying the methods to the unseen data.}

\subsection{Handling Missing Data}

\subsubsection{Filtering Out}

\textcolor{blue}{Simply removing any entries that are missing data. This is convenient and easy but may not be practical -- any time data is being removed, potentially useful information is lost and too much data may be removed.}

\textcolor{green}{TODO: Code in jupyter on how to do this with pandas and dropna -- key params - how, thresh, subset}

\subsubsection{Filling In}

\textcolor{blue}{Estimating the missing data}

\subsection{Handling Categorical Data}

\subsubsection{Encoding}

\subsection{Feature Scaling, Normalization}

\subsubsection{Min-Max scaling (Normalization)}

\textcolor{blue}{values are shifted and rescaled so they end up on a [0,1] range}

\subsubsection{Standardization}

\textcolor{blue}{(Eq.~\ref{eq:preprocess_standardization}) first, subtract the sample mean, then divide by standard deviation variance}

\textcolor{blue}{pros: unlike min-max, not bound to specific range}

\textcolor{blue}{standardized values always have a zero mean and a standard deviation of 1.}

\textcolor{blue}{gives our data the property of a standard normal distribution}

\begin{equation}
{X' = \frac{X - \mu}{\sigma}}
\label{eq:preprocess_standardization}
\end{equation}

\textcolor{green}{TODO: create code sample - numpy, and sklearn methods}


\subsection{Others}

\subsubsection{Removing Duplicates}

\subsubsection{Outliers}

\subsubsection{Discretization and Binning}


%%%%%%%%%%%%%%%%%%%%%%%% Data Type Considerations + Feature Extraction
% TODO: this section is rough.. and contains overlap on feat engineering in previous section
\section{Feature Extraction from Various Datatypes}

\textcolor{green}{TODO: Feature Extraction}

\subsection{Feature Engineering}

\textcolor{blue}{acquisition and/or systematic improvement of features}

\textcolor{blue}{TODO: features are learned,not engineered in deep learning models}

% TODO: placement and naming
\subsubsection{Kernel}

\textcolor{blue}{the following can't be separated linearly as is.}

% {{{kernelized_2class4clust_2dimg}}}
\begin{figure}[h]
\centering
\includegraphics[width=0.65\textwidth]{./sync_imgs/kernelized/2class4clust/2dimg.png}
\label{fig:kernelized_2class4clust_2dimg}
\end{figure}

\textcolor{blue}{but what if we produce a new feature (feature 1 ** 2)}

% {{{kernelized_2class4clust_3dimg}}}
\begin{figure}
\centering
\includegraphics[width=0.65\textwidth]{./sync_imgs/kernelized/2class4clust/3dimg.png}
\label{fig:kernelized_2class4clust_3dimg}
\end{figure}


\subsubsection{Feature Crosses}

% TODO: figure of linearly separable dataset and or a xor dataset
% makes some non-linear problems (xor) linear

\textcolor{blue}{combine two or more categorical features. Feature crossing is only possible when working with categorical features. When working with continuous features, the values can be discretized prior to the feature cross}


% how much to translate feature 1 and 2 are parameters that need to be learned
% "discreteize" the input space
% feature crosses "memorize" -- "Memorization works when LOTS of data for a 
% single cell in the input space & the distribution of data is statistically significant."
% not used as often in traditional ML, but powerful on large datasets
% Rome/New york yellow/white Taxi example

\textcolor{blue}{number of inputs.}
% example of 24hrs a day, 7 days a week = 168 inputs in a feature cross of the two.
% TF uses a sparse representation for inputs to address this (one hot encoding and feature crosses)
% input will only activate one input at a time, thus the input is very sparse

\textcolor{blue}{It is possible that the feature cross may cause the model to overfit the data.}

% TODO: show example of this happening -- X1,X2 (2 blobs ) = good, X1X1,X2X2,X1X2 = overfit

\textcolor{blue}{it is possible to look at the relative weights for the inputs and determine how much each feature is contributing to the decision. This can help determine if maybe the features cross isn't necessary -- L1 regularization (see \textcolor{red}{local ref}) may work to zero out this feature as well.}

% TODO: implementation details and choosing the number of hashbuckets.
% if too small, there could be collisions, "rule of thumb" 1/2sqrt(N) and 2N
% trade off is memorization vs sparsity

% adding an embedded layer (real values, learned)
% learns how to ``embedd'' the feature cross in a lower dimesnsional space
% the features learned in embedded features may be useful to other problems from a seperate/maybe related domain
% using learned embeddings in one city for another city on the same types of inputs

\subsection{Images}

\textcolor{green}{TODO: Images}


\subsubsection{Video}

\textcolor{green}{TODO: Video}


\subsection{Natural Language}

\textcolor{green}{TODO: Natural Language}

\subsubsection{Terminology}

\textcolor{blue}{A {corpus}\index{corpus} is a collection of documents. {vocabulary}\index{vocabulary} is a corpus's unique words}

\subsubsection{Pre-processing}

\textcolor{green}{TODO: Pre-processing}

\textcolor{blue}{converting all letters to lowercase}

\textcolor{blue}{stemming and lemmatization --- Condensing word forms (derived and inflected) into a single feature. These methods are used to reduce the dimensionality of the features space.}

\paragraph{Stop Word Filtering}

\textcolor{blue}{todo: removing words that are common throughout the language as well as potentially to most of the documents in a corpus. Typically stop words do not convey meaning through their meaning, but rather through their grammatical meaning.}

\paragraph{Tokenization}

\textcolor{blue}{Tokenization is the process of splitting and grouping characters together into meaningful sequences. \textcolor{red}{If a document is tokenized, the result is a set of tokens (words).} Tokens are not limited to words however, and may also be shorter sequences like punctuation characters and affixes.}

\textcolor{green}{TODO: Tokenization example}

\paragraph{Lemmatization}

\textcolor{green}{TODO: Lemmatization. converting words into their base form --- determining the lemma (morphological root) of an inflected word.}

\paragraph{Stemming}

\textcolor{green}{TODO: Stemming. There exist many stemming algorithms. Stemming removes all character patterns that appear to be affixes to a word. Note: the resulting word may or may not be a valid word e.g. \textcolor{red}{XXXXXXXX}.}

\subparagraph{Porter Stemming}

\subsubsection{Encoding}

\paragraph{Encoding Methods}

\subparagraph{Bag-of-Words}

\textcolor{blue}{{bag-of-words}\index{bag-of-words} similar to one-hot-encoding, it encodes words that appear in text as one feature for each word of interest. Does not encode any other information like syntax, grammar, or order of the words.}

\textcolor{blue}{Bag-of-Words encodes the corpus's vocabulary as a feature vector to represent each document. The intuition for using bag-of-words is that documents that contain similar words are likely to be similar to one another.}


\paragraph{tf-idf}

\textcolor{green}{TODO: tf-idf\index{tf-idf} (Eq.\ref{eq:tf_idf_def}) Inverse Document Frequency is a measure of how common/rare a term is in a corpus --- explain importance}

\begin{equation}
{log\frac{N}{1|XXXXXXXXTODOXXXXXXXXXX|}}
\label{eq:tf_idf_def}
\end{equation}

\subsubsection{Embedding}

% TODO: this section may need to be promoted

% an embedding can be created for any categorical column

% ``embeddings cab be thought of as latent features''

% good starting point for number of dimmensions may be cube root of the possible values

\textcolor{blue}{Embeddings are }

\subparagraph{glove}

\textcolor{green}{TODO: glove}

\subparagraph{word2vec}

\textcolor{green}{TODO: word2vec}

\subsubsection{Other Notes}

% 'hashing trick' --- see p59 of Mastering ML with SKL

\subsection{Audio}


\textcolor{green}{TODO: Audio}


%%%%%%%%%%%%%%%%%%%%%%%% Feature selection
\section{feature selection}

\TD{TODO: selecting features}


%%%%%%%%%%%%%%%%%%%%%%%% Data sampling and partitioning
\section{Partitioning Data}

\subsection{Sampling}

\textcolor{blue}{Training, validation, test}

\subsection{Transform}

\subsubsection{Overview}

\textcolor{blue}{TensorFlow Transform is a library used for preprocessing data with TensorFlow.}

\textcolor{blue}{MOTIVE: calculating values (such as $\mu$ and $\sigma$) for an entire dataset can be challenging for large datasets.}

\textcolor{blue}{Though preprocessing can already be accomplished with standard python, numpy, other libraries, or even in TensorFlow, tf.Transform extends these capabilities to support full passes over the dataset.}

\subsubsection{Installation}

\textcolor{green}{TODO: include snippet for installation}

\subsubsection{Implementation}

\textcolor{green}{TODO: include use examples}
% https://github.com/tensorflow/transform/blob/master/getting_started.md


% resources
% 1. https://github.com/tensorflow/transform
% 2. https://github.com/tensorflow/transform/blob/master/getting_started.md


\section{Building Architectures}

\subsection{Layers}

\subsubsection{Dense / Fully Connected}

\textcolor{blue}{TODO: \textcolor{red}{local ref to definition/explanation/example in earlier sec}}

\textcolor{blue}{$output = activation_fn(dot(weights, input) + bias)$}

\subsubsection{Convolution}

\textcolor{blue}{TODO: \textcolor{red}{local ref to definition/explanation/example in earlier sec}}

\subsubsection{Pooling}

\textcolor{blue}{TODO: \textcolor{red}{local ref to definition/explanation/example in earlier sec}}

\paragraph{Max}

\paragraph{Average}

\subsection{Estimators}

% TODO: entire section

% boilerplate code
% graph and session management
%\textcolor{blue}{}
\subsubsection{Input data}

% TODO: input function overview

\paragraph{Specifying Hyper Parameters}

\subparagraph{Epochs}
\textcolor{blue}{by default, training will continue until the training data is exhausted, or the number of specified epochs is reached}

Options:

%TODO: code example
\begin{enumerate}
	\item input\_fn
	\item steps
	\item max\_steps --- will potentially do nothing if the checkpoint has already reached this value
\end{enumerate}


\paragraph{In Memory Data}

\textcolor{blue}{Usually this is in the form of either numpy arrays or pandas dataframes. These can both be used directly.}

% TODO: examples for Numpy array

% TODO: examples for Pandas DF

\paragraph{Out of Memory Data}

\textcolor{blue}{In the ``real world'' the dataset will likely not fit into memory. To (sanely) address this, estimators play nicely with the tf.Data API (please see \textcolor{red}{local ref} for more information)}

% TODO: show quick demo example

% tf.estimator base class allows you to build your own model 

% premade models (TODO: Show quick list)

% main advantage -- estimators are interchangable

% "reasonable" defaults for each estimator

\subsubsection{Checkpoints}

\textcolor{blue}{directory specified when creating model. By default, predictions will be made from the latest checkpoints in this directory. training also resumes from the latest checkpoint in the directory. --- to start from scratch, the directory will need to be deleted or specified to a new location.}

\subsubsection{Distributed}

% you need: 1. estimator, 2. run config, 3. train spec, eval spec

% final call tf.estimator.train_and_evaluate(estimator, train_spec, eval_spec)

\paragraph{tf.estimator.RunConfig}

% TODO: example

\textcolor{blue}{the directory for checkpoints and Tensorboard logs and freq of checkpoints (save\_checkpoint\_steps) and frequency of logs (save\_summary\_steps)  }

\paragraph{tf.estimator.TrainSpec}

\textcolor{blue}{pass in input function (likely through data API (see \textcolor{red}{localref})), }


\paragraph{tf.estimator.EvalSpec}

\textcolor{blue}{pass in input function for evaluation dataset (likely through data API (see \textcolor{red}{local ref})),}

\textcolor{blue}{creates model and loads latest checkpoint, then runs eval. Therefore, you cannot get a frequency greater than the checkpoints created. They can be obtained less frequently by using the `throttle\_specs` parameter}

\paragraph{Notes}

\textcolor{blue}{Shuffling considerations.use the dataset = tf.data.Dataset().list\_files().shuffle() command --- each worker a different seed? Even if the data is shuffled on disk. Can also use the dataset().shuffle()}

\subsubsection{TensorBoard}

% TODO: example -- to create: tf.estimator.RunConfig(model_dir='some_dif')

\textcolor{blue}{to visualize, then open tensorboard by issuing `tensorboard --logdir output\_dir` and the dashboard will appear on localhost:6006}

\textcolor{blue}{pre-made estimators already export relevant metrics, embeddings, histograms, etc.. for more information on how to use tensorboard please see \textcolor{red}{local ref}}

\textcolor{blue}{if building a custom estimator or would like to add additional information to tensorboard, summaries can be added with any of the following: tf.summary.scalar, tf.summary.image, tf.summary.audio, tf.summary.text, tf.summary.histogram}

\paragraph{Adding Custom}

% tf.summary.scalar('meanVar_01', tf.reduce_mean(var_01))

\subparagraph{tf.summary.scalar}

\subparagraph{tf.summary.image}

\subparagraph{tf.summary.audio}

\subparagraph{tf.summary.text}

\subparagraph{tf.summary.histogram}


\subsubsection{Deployment}

% TODO: examples

%\textcolor{blue}{two things 1) export\_latest = tf.estimator.LatestExporter(serving\_input\_receiver\_fn=serving\_input\_fn) and then eval_spec = tf.estimator.EvalSpec(input\_fn=eval\_input\_fn, exporters=export\_latest)}

% will map from JSON from REST API and the model

% important to use tf commands in the input transformation/parsing function

\paragraph{Exporters}

\textcolor{blue}{there are many types of exporters and exporter schemes. The simplest may be the tf.estimator.LatestExporter}




\input{tensorflow/api_comp/eager}


\section{Design and Component Considerations}

\subsection{Initialization Strategies}

\textcolor{blue}{Discuss different initialization strategies and their importance}

\textcolor{blue}{truncated normal -- truncated Gaussian distribution}

\subsection{Hyperparameters}

\subsubsection{Training Related}

\paragraph{Learning Rate}

\subparagraph{Too High vs Too Low}

\textcolor{blue}{TODO: figure showing a convex cost function and the result of a learning rate being too high (overshoot, diverge) and too small (local minima)}

\paragraph{Batch Size}

\paragraph{Number of Training Iterations}

\paragraph{Momentum}

\paragraph{Weight Update}

\textcolor{red}{SGD, CG, L-BFGS, more complex more hyper-parameters}

\paragraph{Stopping Criteria}

\subsubsection{Model Related}

\paragraph{Architecture}

\paragraph{Weight Initialization}

\paragraph{Weight-decay}

L1

L2

\paragraph{Drop-out}



\subsection{Hyper-parameter optimization}

\textcolor{blue}{OVERVIEW}

\subsubsection{Coordinate Descent}

All hyper-parameters remain fixed, except for the hyper-parameter of interest. The hyper-parameter of interest is then adjusted such that the validation error is minimized.

\subsubsection{Grid Search}

\textcolor{green}{TODO: grid search explanation}

\subsubsection{Random Search}

\textcolor{green}{TODO: random search explanation}

\subsubsection{Grid vs Random Search}

\textcolor{green}{TODO: grid vs random search figure}

\subsubsection{Automated / Model-based Methods}

\section{Estimating Model Parameters}

\subsection{Optimizers}

\textcolor{blue}{Estimate the values of the model's parameters that minimize the value of the cost function}

\subsubsection{Gradient Descent}

\textcolor{blue}{Gradient Descent --- overview --- optimization algorithm that can be used to estimate the local minimum of a function}

\textcolor{blue}{Iteratively updates the model parameters by calculating the partial derivatives of the cost function at each step during training}

\textcolor{blue}{Gradient descent is only guaranteed to find the local minimum of the cost function.}


\paragraph{Batch Gradient Descent}

\textcolor{blue}{batch gradient descent --- taking a step (update the weights) opposite (down) the gradient calculated from the entire training set}

\textcolor{blue}{Batch gradient descent is deterministic --- will produce the same paramter values if the same dataset is used multiple times.}


\paragraph{Stochastic Gradient Descent}

\textcolor{blue}{Stochastic Gradient Descent (sometimes called iterative or on-line gradient descent) --- rather than update the weights based on the sum of the accumulated errors, the weights are updated for each training sample}

\textcolor{blue}{Stochastic gradient descent is deterministic --- may produce the different parameter values if the same dataset is used multiple times. May not minimize the cost function as well as gradient descent but the approximation is often ``close enough''.}


\paragraph{Mini-batch Gradient Descent}

\textcolor{blue}{mini-batch gradient descent --- compromise between batch and stochastic gradient descent where the gradient is calculated over a batch of training data}

\textcolor{blue}{Since the gradient is calculated on a single example, the error surface will appear noisier than if it was calculated over a batch or the entire training set.}

\textcolor{blue}{When using stochastic gradient descent, it is important to shuffle the data after each epoch.}

%% techniques
\chapter{Augmentation Techniques}
\label{app_aug_techniques}

\r{Including Imagery}

\r{flip, rotate. color/channel manipulation}

\r{mixup\cite{zhang2017mixup}}

\r{cutout\cite{devries2017improved}}

\r{cutmix\cite{yun2019cutmix}}

\TD{language -- back translation\cite{sennrich2015improving}}

% TODO: blog about this: https://ai.googleblog.com/2019/07/advancing-semi-supervised-learning-with.html
\TD{unsupervised augmentation\cite{xie2019unsupervised}}

%% learning augmentation

% TODO: not sure this belongs here
\r{Sample Pairing\cite{inoue2018data}}

\r{Smart Augmentation\cite{lemley2017smart}}

\r{GAN\cite{shrivastava2017learning}}

\r{population based augmentation (PBA)\cite{ho2019population}}

\r{Bayesian data augmentation\cite{tran2017bayesian}}

\TD{distortions, patches, jigsaw, color}

\TD{AugMix: A Simple Data Processing Method to Improve Robustness and Uncertainty \cite{Hendrycks2020AugMixAS}}

% image augmentation
\TD{Attacks Which Do Not Kill Training Make Adversarial Learning Stronger \cite{DBLP:journals/corr/abs-2002-11242}}

\TD{DuBIN --- AugMax: Adversarial Composition of Random Augmentations for Robust Training \cite{Wang2021AugMaxAC}}

% similar to dropout?
\TD{Random Erasing Data Augmentation \cite{DBLP:journals/corr/abs-1708-04896}}

%% autoaugment

\r{AutoAugment\cite{cubuk2018autoaugment}}

\r{Comment that AutoAugment can be applied directly to a dataset as well as transfer the learned policies to new datasets.}

\TD{figure of loop. controller, strategy, child network, update controller}

\r{Fast AutoAugment\cite{lim2019fast} improves upon the original search strategy in the original AutoAugment paper.}

\r{Unsupervised Augmentation}

\TD{autoaugment for object detection \cite{zoph2019learning}}

\section{Data Imbalance or Unbalanced Data}
\label{app_data_imbalance}

% TODO: special case of augmentation??

\TD{A Survey of Predictive Modelling under Imbalanced Distributions \cite{DBLP:journals/corr/BrancoTR15}}

\TD{\cite{krawczyk2016learning}}

\r{An ``imbalance in the data'' may have many meanings. That is, the labels could be imbalanced (e.g. in a binary classifier, there may be n times the number of instances with the label p when compared to the label q), the features may be imbalanced, (e.g. facial recognition is being performed on collected images that are composed of overwhemingly white, male, brown hair, clean shaven, hazel eye individuals), or it may mean a combination of the above.}

\r{this poses a problem, as typically the optimization process treats all samples individually and equally, which may (often does) pose problems when creating predictions on imbalanced data}

\r{Two main high level approaches to addressing this issue. You can either modify the (or utilize a combination of the listed):}
\begin{itemize}[noitemsep,topsep=0pt]
	\item data
	\item model
	\item post-processing
\end{itemize}

\r{data modification approaches aim to create a quisi-balanced representation, often through some re-sampling scheme. Simple example for would be to down-sampling the overrepresented class and upsampling the underrepresented class -- where class here might mean target variable or feature attribute.}

\r{when discussing class imbalance in reference to a continuous variable, the term skewed is often used to describe the data, whereas unbalanced or imbalanced is used for discrete variables}

\TD{relevance function --- }

\subsection{Methods}

\TD{paper to read: \TD{Self-paced Ensemble for Highly Imbalanced Massive Data Classification \cite{Liu2020SelfpacedEF}}}

\subsubsection{Data}

\r{as mentioned, under/oversampling scheme}

\TD{Method -- SMOGN}

\TD{Method --- SMOTE (\textbf{S}ynthetic \textbf{M}inority \textbf{O}versampling \textbf{T}echnique)}

\TD{method --- Tomek Links, select pairs of examples that are of opposite class, near one another. \TD{Figure}}

\TD{``shows that outcome imbalance is not a problem in itself, and that imbalance correction may even worsen model performance'' -- The harm of class imbalance corrections for risk prediction models: illustration and simulation using logistic regression~\cite{Goorbergh2022TheHO}}

\subsubsection{Model}

\TD{Utility-based Regression --- penalty based on \TD{relevance function}}

\paragraph{Losses}

% TODO: does this belong here?
% generalized loss function
\TD{Cyclical Focal Loss~\cite{Smith2022CyclicalFL}}
\TD{Asymmetric Loss For Multi-Label Classification~\cite{DBLP:journals/corr/abs-2009-14119}}

\subsubsection{Post processing}

% TODO: is this where calibration belongs??
\paragraph{Calibration}

%TODO: need to get a grasp on this...

% original
\TD{On Calibration of Modern Neural Networks \cite{DBLP:journals/corr/GuoPSW17}}

% recent
\TD{Revisiting the Calibration of Modern Neural Networks~\cite{DBLP:journals/corr/abs-2106-07998}}

% other, relevant
\TD{On the Dark Side of Calibration for Modern Neural Networks~\cite{Singh2021OnTD}}

% other, popular
\TD{Simple and Scalable Predictive Uncertainty Estimation using Deep Ensembles~\cite{Lakshminarayanan2017SimpleAS}}



\subsection{Evaluation}

\r{Difficult to discuss class imbalance without discussing the importance of having appropriate metrics in place to evaluate the methods. \TD{point to example of disease test where 1 out of N are positive --- acc is very high, yet...}}

\TD{point to metrics section and specific metrics that may be useful for various scenarios}


\section{Activation Functions}

\textcolor{blue}{Activation functions are XXXXXXXX}

\subsection{Why Non-linear}

\textcolor{blue}{Non-linear is necessary XXXXXXXXXX}

\subsection{Advancements}

\textcolor{green}{TODO: From step function to ?selu}

\subsection{Popular Activation Functions}

\textcolor{blue}{Activation functions can be grouped into two main categories -- smooth and not smooth. Smooth activation functions (such as sigmoid) are differentiable at every point along the function where as the other activation functions are not differentiable at every location (relu).}

\subsubsection{Smooth Non-linear}

\textcolor{blue}{The sigmoid\index{sigmoid} activation function}

% {{{act_smooth_sigmoid}}}
\begin{figure}
\centering
\includegraphics[width=0.65\textwidth]{./sync_imgs/act/smooth/sigmoid.png}
\label{fig:act_smooth_sigmoid}
\end{figure}

% {{{act_smooth_tangent}}}
\begin{figure}
\centering
\includegraphics[width=0.65\textwidth]{./sync_imgs/act/smooth/tangent.png}
\label{fig:act_smooth_tangent}
\end{figure}

% {{{act_smooth_elu}}}
\begin{figure}
\centering
\includegraphics[width=0.65\textwidth]{./sync_imgs/act/smooth/elu.png}
\label{fig:act_smooth_elu}
\end{figure}

% {{{act_smooth_selu}}}
\begin{figure}
\centering
\includegraphics[width=0.65\textwidth]{./sync_imgs/act/smooth/selu.png}
\label{fig:act_smooth_selu}
\end{figure}

% {{{act_smooth_softplus}}}
\begin{figure}
\centering
\includegraphics[width=0.65\textwidth]{./sync_imgs/act/smooth/softplus.png}
\label{fig:act_smooth_softplus}
\end{figure}

% {{{act_smooth_softsign}}}
\begin{figure}
\centering
\includegraphics[width=0.65\textwidth]{./sync_imgs/act/smooth/softsign.png}
\label{fig:act_smooth_softsign}
\end{figure}


\subsubsection{Not Smooth Non-linear}

% {{{act_notsmooth_relu}}}
\begin{figure}
\centering
\includegraphics[width=0.65\textwidth]{./sync_imgs/act/notsmooth/relu.png}
\label{fig:act_notsmooth_relu}
\end{figure}

% {{{act_notsmooth_leakyrelu}}}
\begin{figure}
\centering
\includegraphics[width=0.65\textwidth]{./sync_imgs/act/notsmooth/leakyrelu.png}
\label{fig:act_notsmooth_leakyrelu}
\end{figure}

% {{{act_notsmooth_relu6}}}
\begin{figure}
\centering
\includegraphics[width=0.65\textwidth]{./sync_imgs/act/notsmooth/relu6.png}
\label{fig:act_notsmooth_relu6}
\end{figure}

% {{{act_notsmooth_prelu}}}
\begin{figure}
\centering
\includegraphics[width=0.65\textwidth]{./sync_imgs/act/notsmooth/prelu.png}
\label{fig:act_notsmooth_prelu}
\end{figure}




\section{Fine-Tuning Architectures}

\input{tensorflow/api_comp/profiler}

\input{tensorflow/api_comp/debugger}

\section{Distributed Training}

\chapter{Distributed Methods}

\textcolor{blue}{Achieving distributed training in two main ways, and \textcolor{red}{federated}. Implementation details are discussed in \textcolor{red}{local ref}}

\textcolor{blue}{note about synthetic gradients \textcolor{red}{local ref}}

%  model is replicated and placed on multiple workers

\subsubsection{Data Parallelism}

\textcolor{blue}{data parallelism. \textcolor{green}{TODO: figure}}

\subsubsection{Model Parallelism}

\textcolor{blue}{Model parallelism. \textcolor{green}{TODO: figure}}

\subsubsection{Federated learning}

\textcolor{blue}{Federated learning -- consensus change}



\section{Model Evaluation}

\textcolor{green}{TODO: para about using metrics.}

\textcolor{green}{Para about using tensorboard during training and tfma after training}

\textcolor{blue}{Metrics computed during training (e.g. training and validation metrics) can be visualized in tensorboard. Tensorboard displays, and continuously updates during training, these metrics graphically against global training steps (or time) and is used to determine how well your model is being trained. TFMA computes and visualizes metrics from the final (presumably after training) model. These metrics are computed only once. \textcolor{red}{TFMA exports and computes the metrics once on a saved model which contains the eval graph and additional metadata}}

\textcolor{blue}{Where tensorboard will compare models to each other over time, TFMA will compare models at only one point in time. This is better displayed in \textcolor{red}{see figure xx}.}

\textcolor{green}{TODO: figure showing the difference between graphs from tensorboard and TFMA}

\emph{Cost} is frequently used interchangeably with loss. Technically, loss refers to the error on a single example and cost is the average of the loss across the entire training set.

% page 94 of AGtext
One-versus-all \emph{OvA} (also \emph{one-versus-rest})

One-versus-one (OvO) -- train a binary classifier for every pair


\section{Metrics}

%% Confusion matrix
\begin{table}
	\centering
	\begin{tabular}{l|l|c|c|}
		\multicolumn{2}{c}{}&\multicolumn{2}{c}{Ground Truth}\\ 
		\cline{3-4}
		\multicolumn{2}{c|}{}&Positive&Negative\\ 
		\cline{2-4}
		\multirow{2}{*}{\rotatebox{90}{Pred}}& Positive & $TP$ & $FP$ \\ 
		\cline{2-4}
		& Negative & $FN$ & $TN$ \\ 
		\cline{2-4}
	\end{tabular}
	\caption{Example confusion matrix}
	\label{tab:sample_conf_matrix}
\end{table}


\begin{itemize}
	
\item \textit{Accuracy}, (Eq.~\ref{eq:accuracy}): the ratio of correct predictions to the total number of predictions.

\begin{equation}
{\frac{TP+TN}{TP+TN+FP+FN}}
\label{eq:accuracy}
\end{equation}

\item \textit{Sensitivity}, (Eq.~\ref{eq:sensitivity}): the ratio of true positives that are correctly identified.

\begin{equation}
{\frac{TP}{TP+FN}}
\label{eq:sensitivity}
\end{equation}

\item \textit{Precision}, (Eq.~\ref{eq:precision}): the ratio of positives that are, in fact, positive. If the classifier predicts positive, how often is is correct?

\begin{equation}
{\frac{TP}{TP+FP}}
\label{eq:precision}
\end{equation}

\item \textit{AUC (Area Under the Curve)}, is a single value representing the area under an ROC curve. Though generally referred to as the AUC, the term is correctly abbreviated AUROC, specifying that the curve is an ROC curve.
\end{itemize}

\input{tensorflow/api_comp/tensorboard}

\subsection{TFMA: TensorFlow Model Analysis}

\subsubsection{Key Features}

\paragraph{Sliced Metrics}

\textcolor{blue}{Typically metrics are aggregated from the entire test dataset. TFMA allows for examining specific slices from the dataset which may indicate that certain instances or groups of instances aren't predicted as well as others.}

\textcolor{green}{TODO: figure showing how this might look}

\paragraph{Full-pass Metrics}

\textcolor{blue}{Metrics that are typically computed and visualized in tensorflow (such as on tensorboard) are actually approximations computed on mini-batches and known as ``streaming metrics''.}

\textcolor{blue}{TFMA performs a ``full pass'' on the evaluation dataset, leveraging \textcolor{yellow}{Apache Beam}. There are two main advantages to this approach: i) \textcolor{red}{more accurate calculation} and ii) scaling up to massive evaluation datasets.}

\subsubsection{Implementation}

\paragraph{Overview}

\textcolor{blue}{An evaluation graph from the trainer needs to be exported and saved as a `SavedModel'. TFMA will then use this saved model to compute metrics and provide visualization tools to analyze the metrics.}

\paragraph{Code}

\textcolor{blue}{The overall process is outlined below and assumes a trained model and test set already exist. At a high level, the steps are i) Exporting the Evaluation Graph ii) Computing Metrics and iii) Visualize Metrics (in a notebook). }

\subparagraph{Exporting the Evaluation Graph}

\textcolor{green}{TODO: code sample}

\subparagraph{Computing Metrics}

\textcolor{green}{TODO: code sample}

\subparagraph{Visualize Metrics (in a notebook)}

\textcolor{green}{TODO: code sample}


% Resources:
% 1. https://medium.com/tensorflow/introducing-tensorflow-model-analysis-scaleable-sliced-and-full-pass-metrics-5cde7baf0b7b


\section{Model Persistence}

\subsection{Saver}

%%%% may belong in serving, may belong in Saver -- then ref other?
\textcolor{blue}{SavedModel is the universal serialization format for TensorFlow models. SavedModel has support for multiple metagraphs -- this is important in serving where the model is slightly different than in training (removing dropout layers), and allows storage/access of models with tags. Supports SignatureDefs -- allows the specification of nodes as input/output (also supports multiple signatureDefs for multi-headed inference (see \textcolor{red}{local ref}))}

\input{tensorflow/api_comp/hub}


\section{Deployment}

\section{Deployment with TF Serving}

\subsection{Overview}
% https://github.com/tensorflow/serving/blob/master/tensorflow_serving/g3doc/architecture_overview.md
% https://www.youtube.com/watch?v=q_IkJcPyNl0

\subsubsection{Standard Abstractions}

\textcolor{blue}{Core components with APIs}

\paragraph{ServableHandle}

\paragraph{Manager}

\textcolor{blue}{Uses the Loader to load and unload a model}

\paragraph{Loader}

\textcolor{blue}{Loader for a TensorFlow <saved\_model>}

\textcolor{red}{the loader knows how to load a model and knows how to estimate the resources such as RAM or GPU}

\textcolor{red}{The loader signals to the manager that has an `aspired version' (ready for loading). The manager will then decide when to load this new version based on the version policy plugin (see below).}

\paragraph{Source}


\subsubsection{Plugins into Abstractions}


\paragraph{File System (Source)}

\textcolor{blue}{Monitors the file system.}


\paragraph{Version Policy (Manager)}

\textcolor{blue}{Preserve availability or preserve resources. Preserving availability might be more important in a live, user facing scenario and will keep keep one model loaded, load another beside, then point client to new model -- There's no down time, but more resources are consumed (two models loaded at once).  Preserving resources might be important if using a large model on a resource constrained environment or in an internal application where some downtime is considered acceptable. Under a policy preserving resources, the current model will be unloaded and a new model will be loaded -- cost is a slight hiccup in service, but the benefit is a saving on resources (like memory) since there's only the one model loaded at a time.}

\textcolor{red}{When loading a new model, the original/old model can't be immediately unloaded/deleted in case there are pending/queued jobs. TensorFlow keeps track of these jobs via ref-counting, and only then removes that model once all the jobs have completed}


\subsubsection{FIT}

\paragraph{ServerCore}

\textcolor{blue}{declare set of models to be loaded, pass them to ServerCore, and server core returns a manager of these models with the best practices out of the box.}


\paragraph{Binaries and APIs}

\subparagraph{Predict}

% coming soon?
%\subparagraph{Regress}

%\subparagraph{Classify}

%\subparagraph{MultiInference}

\paragraph{Servables}

\textcolor{blue}{The central abstraction to TensorFlow Serving -- the underlying object that will be used for inference.}

\textcolor{blue}{Servables may be big and complex (composite inference model) to small and simple (lookup table). }

\textcolor{red}{A composite model can be represented as either multiple independent servables or a single composite servable.}

\paragraph{Loaders}

\textcolor{blue}{Statdarize the API for loading and unloading a servable i.e. manage the servable's lifecycle.}

%\paragraph{Sources}
%\textcolor{blue}{}




\subsection{Example}


\section{Other}

\input{tensorflow/api_comp/probability}

\subsection{TensorFlow Extended}

\subsection{Keras}

\subsection{Image}

\subsection{Image Augmentation}

\subsection{Edward 2.0}

\subsection{TensorFlow Lite}
