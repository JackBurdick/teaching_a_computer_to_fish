\section{Overview}

\TD{learning to learn -- learning new tasks}



\section{Taxonomy}

\TD{Deep meta-leaning is often subdivided into three main subsections~\cite{vinyals2017model}}


\begin{itemize}[noitemsep,topsep=0pt]
	\item Metric-based
	\item Model-based
	\item Optimization-based
\end{itemize}


% TODO: meta learning?
\TD{Learning to learn by gradient descent by gradient descent \cite{DBLP:journals/corr/AndrychowiczDGH16} use a neural network to perform optimization -- could reuse between related tasks?}


% quote fron Finn lecture 2
\r{``in some ways multi-task is a prereq for meta learning -- if you can't learn/solve the training tasks, you won't be able to learn new tasks''-- two ways to view meta-learning algorithms i) mechanistic view ii) probabilistic view.}

\r{use our experience from before in some way, that is learn parameters on some tasks such that we can use the learned parameters to solve new tasks}

\TD{adaptation}

\TD{support set and query set}

\subsection{k-shot learning}

\TD{where $k$ is the number of examples you're learning from}

\TD{Evaluating meta-learning algorithms}

\section{evaluation}

\r{omniglot dataset --- 1623 characters, 50 alphabets, each character = 20 instances, many classes few examples}


\section{Adaptation approaches}

\subsection{Black-Box Adaptation}

\r{LSTMs or NMTs -- e.g. \cite{santoro2016meta}. NMT, external memory mechanism for storing information from training, and accessing that information when infering on new data points (in a differentiable way).}

% from lect 3, ~33min
\r{NMT is not permutation invariant, Feedforward and average is.\cite{garnelo2018conditional}}

\r{fast and slow weights -- where ``slow'' weights are related to the meta parameters and the ``fast'' weights are related to the task specific parameters.}

\r{meta-networks \cite{munkhdalai2017meta}}

\section{Optimization-based meta-learning}

\r{acquire task specific parameters through optimization -- differentiate through to the meta parameters.}

\r{meta-parameters act as a prior -- e.g. initialization then fine-tune. This is often associated with ``transfer learning'', where you train a network on some task then use those parameters for a new task. ``reusing features, and changing how those features are used for a new task -- not necessarily creating new features''.}


% ``computational graph with embedded gradient operator'' -- Fin Youtube 3
% learn initialization but replace gradient update with learned network Ravi + Larochelle '17
% MAML - benefit of inductive bias w/out losing expressivness of gradient d.
\r{Model-Agnostic Meta Learning \TD{CITE}}


% adaptation to individual tasks given an individual dataset

% optimization-based approach algo, lect 3 of Finn youtube
\begin{itemize}[noitemsep,topsep=0pt]
	\item sample task
	\item sample disjoint datasets
	\item update fine-tuning parameters?
	\item update meta parameters
\end{itemize}

% SNAIL, MetaNetworks

% in a probabilistic interp -- the meta parameters serve as a prior e.g. initialization for fine-tuning
% other forms of priors
% - Bayesian Linear regression on learned features (Rajeswaran et al implicit MAML '19)
% - Closed-form or convex optimization on learned features
% -- ridge regression, logistic regression (Bertinetto et al R2-D2 '19)
% -- SVM -- Lee et al. MetaOptNet '19


% Finn L4
\r{challenges}
\r{choosing an arhictecture that is effective for the inner gradient step. One idea is Progressive nerual architecture search + MAML (Kim et al. Auto-Meta). a finding; non-standard (deep and narrow) were effective for MAML}
% simple tricks can help optimization
\r{Bi-level optimization is tricky -- inner and outer optimization. Couple ideas: - automatically learn inner learning rate, tune outer (Li et al. Meta-SGD,Behl et al. AlphaMAML} \r{- optimize only a subset of params in the inner loop (Zhou et al. DEML, Zintgraf et al. CAVIA)} \r{- Decouple inner learning rate, Batch norm statistics per-step (Antoniou et al. MAML++)}\r{- Introduce context variables for increased expressive power (Finn et al. bias transformation, Zintgraf el al. CAVIA)}
r{Backpropagating through many inner gradient steps is compute and memory intensive}
% ~16min in L4
\r{approaches: - \TD{unsure -- (Finn et al. first-order MAML '17, Nichol et al. Reptile '18)}}


\section{Non-Parametric Few-Shot Learning}
\r{siamese networks, matching networks, prototypical networks}


\section{Meta-Learning Algorithms}
% comparisons of approaches
