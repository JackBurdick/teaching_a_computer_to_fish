
\section{losses}

% TODO: not sure where this section belongs
% TODO: It would maybe make sense to have an appendix section that covers the common losses

% potentially ``irrelvant''/~uninterpretible in value to the researcher

% did I already write this somewhere
\r{often we use a loss as a proxy for our performance metric. Most often because the performance metric is not differentiable \textit{e.g.} accuracy results in a binary output. We may also have a different loss function from our performance measure because we wish to penalize/constrain the model during training in a way that is seperate from how we report results \textit{e.g.} using MSE and MAE}

\subsection{fit somewhere}

\subsubsection{Contrastive Losses}

\TD{link to secion on self-supervised learning}


\subsection{Discrete}

\r{classification}


\subsubsection{Cross-Entropy}

%TODO: https://colah.github.io/posts/2015-09-Visual-Information/

\TD{A mathematical theory of communication \cite{shannon1948mathematical}}


\r{cross entropy --- difference between two probability distributions. similar, but not the same as KL-divergence}

\r{entropy, information theory --- the average length of bits necessary to encode a distribution of events}

\r{if cross entropy is perfect, it will be equal to the entropy of the distribution itself (the intrinsic iunpredictability)}

\r{binary cross entropy (in the context of loss functions, may be called/interchanged with logistic loss / log loss, even though they are not the exact same.)}

% https://machinelearningmastery.com/cross-entropy-for-machine-learning/
\r{KL-divergence is the measure of ``\textbf{extra bits}'' need to encode an event from $q$, instead of $p$, where as cross-entropy is the \textbf{total number} of bits}

\r{cross entropy of itself will be the entropy}

\r{NOTE: cross entropy is not symmetric,\textit{i.e.}calculating the cross entropy of one distribution $p$ from another $q$, is different from the cross entropy of $q$ from $p$}


 %  + \textrm{}(\textrm{}) \\



\begin{equation}
	\begin{split}
		\textrm{cross-entropy(preds, targets) } & =  \textrm{entropy} (\textrm{preds}) + \textrm{kl\_divergence}(\textrm{preds, targets})\\
		& = -(\textrm{sum}(\textrm{pred}) \times \log (\textrm{pred}) )+ KL(\textrm{preds, targets})\\
		& =  -\sum_{i=1}^{n}p(x_i)\log p(x_i)+ KL(\textrm{preds, targets}) \\
		& =  -\sum_{i=1}^{n}p(x_i)\log p(x_i)+ sum(\textrm{preds} * \log ( \frac{\textrm{preds}}{\textrm{targets}} ) \\
		& =  -\sum_{i=1}^{n}p(x_i)\log p(x_i)+ \sum_{i=1}^{n}p(x_i)\log \frac{p(x_i)}{q(x_i)} 
	\end{split}
\end{equation}

\r{sometimes (more often) the entropy equation is the negative sum, but it can also be written as the following positive case (using the identity $ \log ( \frac{1}{a} )  = - \log (a) $)}

\begin{equation}
	\begin{split}
		\textrm{entropy} (\textrm{preds})  & =  -(\textrm{sum}(\textrm{pred}) \times \log (\textrm{pred}) )\\
		& = - \sum_{i=1}^{n}p(x_i)\log p(x_i) \\
		& =   \sum_{i=1}^{n}p(x_i)\log ( \frac{1}{p(x_i) } )
	\end{split}
\end{equation}


\r{if using $log_{10}$, the units are ``bits'', if using $log_2$, the units are ``nats''}

\r{binary cross-entropy referes to cross-entropy of two classes, whereas categrorical cross-entropy referes to the cross-entropy of multiple ($n$) classes (where $n > 2$)}

% label smoothing
\subsection{label smoothing}
\TD{Label smoothing}
\TD{When Does Label Smoothing Help? \cite{DBLP:journals/corr/abs-1906-02629}}
\TD{Regularizing Neural Networks by Penalizing Confident Output Distributions \cite{DBLP:journals/corr/PereyraTCKH17}}



\subsection{Continuous}

\r{training objective -- continuous}

\subsubsection{Losses}

\TD{ELBO (Evidence Lower BOund)}

% TODO: index
\TD{Squared logarithmic error (SLE) and Mean SLE (MSLE)}
\TD{Root Mean Squared logarithmic error (RMSLE) and Mean RMSLE (RMSLE)}
\TD{Mean Absolute Percentage Error (MAPE)}


\subsection{Distribution}


