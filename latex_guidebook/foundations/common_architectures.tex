
\TD{Overview}

% TODO: I'm still not sure if I should divide this up by ``types'' of architectures or types of problems and the architectures used to solve them.. maybe both? unsure...


Images and Videos
\begin{itemize}[noitemsep,topsep=0pt]
	\item Classification
	\item Segmentation
	\begin{itemize}[noitemsep,topsep=0pt]
		\item Semantic segmentation (where we don't differentiate between instances)
		\item Instance segmentation
	\end{itemize}
	\item Object detection
\end{itemize}


\subsection{Spatial Data}

\r{types of problems -- spatial to spatial (1:1), spatial to sequence, spatial to value (single or multiple)}

\subsubsection{Convolutional Approaches}



\subsubsection{Image Classification}
% Graham Taylor talk
\r{network-in-network, inception/google lenet (end with $1 \times 1 \times N_{classes}$ into a global average pooling layer), vggnet}

\r{AlexNet\cite{krizhevsky2012imagenet}, won ImageNet Challenge 2012. ZFNet\cite{DBLP:journals/corr/ZeilerF13} won (not entirely true \TD{add footnote}) the following year (reducing convolution filter and stride size). ZFNet, though not called as such in the paper was the name given to the architecture (\textbf{Z}elier and \textbf{F}ergus Net)}

\TD{2014 winner GoogleLeNet\cite{DBLP:journals/corr/SzegedyLJSRAEVR14}}

\TD{Xception: Deep Learning with Depthwise Separable Convolutions \cite{DBLP:journals/corr/Chollet16a}}

% TODO: blog on ZFNet https://medium.com/coinmonks/paper-review-of-zfnet-the-winner-of-ilsvlc-2013-image-classification-d1a5a0c45103

% TODO: I'm not sure where this belongs
\TD{$1 \times 1 \times$ convolution explaination --- useful for adjusting the number of features in the feature dimension, either up or down.}

\TD{Inception module --- rather than manually choosing proper\``best'' filter size, ``let the model choose''.}

\r{Resnet (\TD{Deep Residual Learning for Image Recognition \cite{DBLP:journals/corr/HeZRS15}}, similar to highway networks) --- skip connections ``residual modual/block'' (dynamically adjusting depth) (others in this time: fractal nets, stochastic XXXX nets, where skip connections are shortcut paths)}

\r{DenseNet --- where these skip connections are taken to an extreme. also, concatenation, not summation}

% TODO: not sure where this belongs yet
\paragraph{skip connections}


\r{three popular explainations (as described in \cite{lakshmanan2021practical}).}
\begin{itemize}[noitemsep,topsep=0pt]
	\item Addition opperation makes the task ``easier''
	\begin{itemize}[noitemsep,topsep=0pt]
		\item Detailed by the authors of the original paper. That it is easier to predict the ``residue''/delta between the input and output rather than the solely the desired output.
	\end{itemize}
	\item Residual connections make the network effectively shallower 
	\begin{itemize}[noitemsep,topsep=0pt]
		\item Described in \TD{Residual Networks are Exponential Ensembles of Relatively Shallow Networks \cite{DBLP:journals/corr/VeitWB16}. The connections effectively create an ensemble of shallower networks. The network then learns to select the best path/subnetwork for each instance.}
	\end{itemize}
	\item Loss topological ``smoothing'' (TODO: change term)
	\begin{itemize}[noitemsep,topsep=0pt]
		\item \TD{show figure. \TD{Visualizing the Loss Landscape of Neural Nets \cite{DBLP:journals/corr/abs-1712-09913}}}
	\end{itemize}
\end{itemize}



\TD{SqueezeNet \cite{DBLP:journals/corr/IandolaMAHDK16} ``fire modules'', contraction and expansion}

\r{Squeeze and Excitation network --- adaptive recalibration of feature maps. \cite{DBLP:journals/corr/abs-1709-01507} }

% fit into an architecture ``timeline''
\TD{ResNeXt\cite{xie2017aggregated}}


\subsection{Object Detection}

\r{sliding window and use each window as input to a classifier. But the problem is that we need to apply the classifier to a huge number of locations and scales and aspect ratios. A potential solution to this problem is to use region proposals.}

\r{region proposals were introduced by R-CNN \TD{cite}, where conventional methods were used to propose regions for the CNN to classify. It's important to note that most regions are not square may need to be warped to be insert to the CNN}

\r{Bbox regression is also used --- how much to offset the region proposal.}


\r{fast R-CNN proposed regions by using the feature maps from a layer from a CNN that was previously trained (VGG or Resnet, for example). after identifying the regions, then crop-resize, then CNN over each region, then get output (class + bbox regression.)}

\r{faster r-cnn. Insert a region proposal network (RPN) to predict proposals from features --- in this architecture, there are actually four losses to jointly train 1) object/net RPN, 2) box/net RPN, 3) Class prediction, and 4) final bounding box score.}


% TODO: YOLO object detection
\TD{You Only Look Once: Unifie \cite{DBLP:journals/corr/RedmonDGF15}}
% TODO more object detection
\TD{Mask R-CNN \cite{DBLP:journals/corr/HeGDG17}}
\TD{Faster R-CNN: \cite{DBLP:journals/corr/RenHG015}}
\TD{SSD: \cite{DBLP:journals/corr/LiuAESR15}}

% 
\TD{Focal Loss (RetinaNet) \cite{DBLP:journals/corr/abs-1708-02002}}


\subsection{Segmentation}

\r{fully convolution --- dowsample to upsample: 1) efficency when using convs on smaller dimensional inputs 2) latent representation}

\r{CNN and RPN --- nice results \TD{cite}}

\r{mask R-CNN --- also used for pose prediction}

% code: https://github.com/facebookresearch/detectron2/tree/master/projects/PointRend
% blog: https://ai.facebook.com/blog/using-a-classical-rendering-technique-to-push-state-of-the-art-for-image-segmentation/
\TD{PointRend: Image Segmentation as Rendering \cite{Kirillov2019PointRendIS}}


\section{Text: Natural Language Processing}

\r{bag of words -- RNN -- RNN augmented (LSTM, GRU, etc) -- attention -- transformers }


\section{Structured data: Tablular}


\section{Other Common Architectures}

