
\TD{Overview}

% TODO: I'm still not sure if I should divide this up by ``types'' of architectures or types of problems and the architectures used to solve them.. maybe both? unsure...


Images and Videos
\begin{itemize}[noitemsep,topsep=0pt]
	\item Classification
	\item Segmentation
	\begin{itemize}[noitemsep,topsep=0pt]
		\item Semantic segmentation (where we don't differentiate between instances)
		\item Instance segmentation
	\end{itemize}
	\item Object detection
\end{itemize}


\subsection{Spatial Data}

\r{types of problems -- spatial to spatial (1:1), spatial to sequence, spatial to value (single or multiple)}

\subsubsection{Convolutional Approaches}



\subsubsection{Image Classification}
% Graham Taylor talk
\r{alexnet, network-in-network, inception/google lenet (end with $1 \times 1 \times N_{classes}$ into a global average pooling layer), vggnet}

\r{Resnet (similar to highway networks) --- skip connections ``residual modual/block'' (dynamically adjusting depth) (others in this time: fractal nets, stochastic XXXX nets, where skip connections are shortcut paths)}

\r{DenseNet --- where these skip connections are taken to an extreme. also, concatenation, not summation}

\r{Squeeze and Excitation network --- adaptive recalibration of feature maps. \TD{TODO}}

% fit into an architecture ``timeline''
\TD{ResNeXt\cite{xie2017aggregated}}


\subsection{Object Detection}

\r{sliding window and use each window as input to a classifier. But the problem is that we need to apply the classifier to a huge number of locations and scales and aspect ratios. A potential solution to this problem is to use region proposals.}

\r{region proposals were introduced by R-CNN \TD{cite}, where conventional methods were used to propose regions for the CNN to classify. It's important to note that most regions are not square may need to be warped to be insert to the CNN}

\r{Bbox regression is also used --- how much to offset the region proposal.}


\r{fast R-CNN proposed regions by using the feature maps from a layer from a CNN that was previously trained (VGG or Resnet, for example). after identifying the regions, then crop-resize, then CNN over each region, then get output (class + bbox regression.)}

\r{faster r-cnn. Insert a region proposal network (RPN) to predict proposals from features --- in this architecture, there are actually four losses to jointly train 1) object/net RPN, 2) box/net RPN, 3) Class prediction, and 4) final bounding box score.}


\subsection{Segmentation}

\r{fully convolution --- dowsample to upsample: 1) efficency when using convs on smaller dimensional inputs 2) latent representation}

\r{CNN and RPN --- nice results \TD{cite}}

\r{mask R-CNN --- also used for pose prediction}

% code: https://github.com/facebookresearch/detectron2/tree/master/projects/PointRend
% blog: https://ai.facebook.com/blog/using-a-classical-rendering-technique-to-push-state-of-the-art-for-image-segmentation/
\TD{PointRend: Image Segmentation as Rendering \cite{Kirillov2019PointRendIS}}


\section{Text: Natural Language Processing}

\r{bag of words -- RNN -- RNN augmented (LSTM, GRU, etc) -- attention -- transformers }


\section{Structured data: Tablular}


\section{Other Common Architectures}

