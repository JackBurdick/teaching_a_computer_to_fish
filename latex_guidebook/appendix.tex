\chapter{TODO:}


\section{Time Zones}

\r{Dealing with time zones are a colossal pain and should be avoided at all costs.}

\section{Types of missingness}

\TD{Data is rarely ever truly missing at random. Often the result of some unknown systematic error.}

% NOTE: I find this confusing, needs to be beefed up.
\r{Base types of missing data \cite{allison2001missing}}
\begin{itemize}[noitemsep,topsep=0pt]
	\item Missing at Random (MAR)
	\item Missing Completely at Random
	\item Ignorable (structurally missing)
	\item Nonignorable (missing not at random)
\end{itemize}

\paragraph{Missing at Random (MAR)}

\r{Can predict via other given data}

\paragraph{Missing Completely at Random}

\r{Completely unrelated to any given data}

\paragraph{ Ignorable (structurally missing)}

\r{Missing for a logical reason}

\paragraph{Nonignorable (missing not at random)}

\TD{UNSURE}



\section{Imputing Missing Values}
\label{appendix:imputing_missing_values}

\r{fill missing values}

\subsection{Imputation}

\r{Based on observations from the entire dataset}

% static
\TD{mean}
\TD{median}


\subsection{Interpolation}

\r{Based on observations from neighboring datapoints.  Interpolation can be viewed as a subset of imputation.}

% dynamic
\TD{linear}
\TD{Advanced}
\TD{LForward or reverse fill. Note that technically a reverse is a ``lookahead''}
\TD{Moving/Rolling values (\textit{e.g.} average or median)}

\TD{NOTE: averages could be either geometric or exponentially weighted moving averages (giving more weight to more recent data)}

% other
\TD{predicting}

\section{Anomaly Detection}

\r{difficult to evaluate -- often onnly evaluated on ``known'' anomalies.}

\subsection{methods}

\subsubsection{PCA}

\TD{use of dimensionality methods \ALR to reduce the feature space and look for anomalous instances.}

\r{sometimes PCA is ``chained'' i.e. PCA may be performed on an already reduced dataset}

\r{reconstruction error largely depends on the number of components kept --- balance in the number of components to keep: too few and the original data may not be constructed well enough and if there are too many they all may be proproduced}

\r{sparse could also be }

\TD{example}


\section{labeling data}

\TD{dimensionality reduction followed by clustering}

\section{Group Segmentation}

\TD{clustering}

\section{Time Series Concepts}

\TD{Stationarity --- the level of stability. Can we expect to predict the future given the past?}

\TD{Self-correlation --- }

\TD{Spurious correlations --- relationships that appear to be causal, but are not}




