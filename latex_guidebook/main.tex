\documentclass[12pt]{book} 

% TODO: create an inline \code command for monospace font

% The document preamble 

\usepackage[superscript,biblabel]{cite}

\usepackage{makeidx}

\usepackage{amssymb} % math notation (R)

\usepackage{url}

% had to add --shell-escape to pdflex command
\usepackage{markdown}
%\usepackage[hashEnumerators,smartEllipses]{markdown}

\usepackage{times} % Use PS times fonts 

\usepackage{listings} % code examples
\usepackage{xcolor} % textcolor
\usepackage{multirow}
\usepackage{rotating}
\usepackage{titlesec} % for using paragraphs as a "subsubsubsection"

% for encoded images
%\usepackage{filecontents}
%\newcommand{\generateimage}[2]{%
%\immediate\write18{convert.cmd #1 > #2}}
\usepackage{graphicx}	% Use pdf, png, jpg, or eps§ with pdflatex; use eps in DVI mode

\makeindex 

% https://tex.stackexchange.com/questions/45711/defining-lstset-parameters-for-multiple-languages
% Copied from the listings documentation
\lstdefinestyle{numbers} {numbers=left, stepnumber=1, numberstyle=\tiny, numbersep=10pt}
\lstdefinestyle{pyinput}{backgroundcolor=\color{gray!8},frame=shadowbox,showspaces=false, basicstyle=\scriptsize }
\lstdefinestyle{pyoutput}{backgroundcolor=\color{green!8},frame=shadowbox,showspaces=false, basicstyle=\scriptsize }
\lstdefinestyle{terminal}{backgroundcolor=\color{gray!8},frame=shadowbox,showspaces=false, basicstyle=\scriptsize }


% breaklines=true is used to wrap lines
\lstdefinestyle{pyInStyle} {language=Python,style=numbers,style=pyinput,frame=lines}
\lstdefinestyle{pyOutStyle} {language=Bash,style=numbers,style=pyoutput,frame=none}
\lstdefinestyle{terminalBash} {language=Bash,style=numbers,style=terminal,frame=lines, breaklines=true}

% captions for equations
\usepackage{caption}
\DeclareCaptionType{equ}[][]




% Details of the titlepage 
\title{Teaching a Computer to Fish} 
\author{Jack Burdick} 
\date{Last Update: \today} % Use the system date

% for using paragraphs as a "subsubsubsection"
\setcounter{secnumdepth}{4}
\setcounter{tocdepth}{4}
\titleformat{\paragraph}
{\normalfont\normalsize\bfseries}{\theparagraph}{1em}{}
\titlespacing*{\paragraph}
{0pt}{3.25ex plus 1ex minus .2ex}{1.5ex plus .2ex}

\begin{document} 
	
\frontmatter

\maketitle 

\tableofcontents


\mainmatter

\part{Background}

\chapter{Introduction}

But if you teach your computer to fish..


\section{Motivation}
There are many great resources that exist.

I wanted to create the guide I wish I found.

\textcolor{blue}{ML -- study of how programs learn from data. predictive analytics or statistical learning.}



\section{What \& Why ML}
\textcolor{blue}{Arthur Samuel -- ``ML is the study that gives computers the ability to learn without being explicitly programed.''}

\textcolor{blue}{Tom Mitchel -- ``A program can be said to learn from experience `E' with respect to some class of tasks `T' and performance measure `P', if its performance at tasks in `T', as measured by `P', improves with experience `E'.''}

\textcolor{blue}{ML is everywhere.... everyday --- personalized music, show,book,product recommendations, automatic image tagging to specific tasks, detecting fraudulent credit card activity -- analyzing medical images (benign, or malignant). }

\textcolor{green}{TODO: venn diagram, AI (knowledge bases -- logical inference rules (cyc))-> Machine Learning  (SVM, Logistic Regression, naive bayes) -> Representation Learning (...), Deep Learning (MLP, Deep CNN, RNN)}

% p.4 of DL
\textcolor{red}{Representation learning: learn the representation from from input to output, not just the mapping.}

\subsection{Deep Learning}


\textcolor{blue}{Model depth --- depends on the definition of what is considered a computational step --- what level of detail is being considered. This may be either the length of the longest path through the computational graph, where each multiplication, addition, etc. are considered, or this may be described by how the concepts are related to one another, where a group of opperations may be grouped together for a single count (such as in a dense or convolutional layer). In general, we will not be overly concerned with depth or how it is described. It is only important to be aware that some individuals may have different definitions for ``deep'' than others.}

\subsubsection{Why Now?}

\textcolor{blue}{Why is deep learning suddenly so popular?}

\textcolor{blue}{Deep learning is nothing ``new''. It has been rebranded, i.e. gone by many different names \textcolor{green}{cybernetics (1940s-1950s), connectionism (1980s-1990s), deep learning ($\approx 2006$) more prev+new examples} and its popularity has increased and decreased over time.}

\textcolor{green}{``Deep Learning = new electricity'' - Ng}

\textcolor{blue}{A few reasons that may be contributed to the recent surge in popularity may be the following:}

\begin{itemize}
	
	\item \textit{Data (a lot more of it)}: \textcolor{blue}{information about the increase in the collection of data}.
	
	\item \textit{Hardware (Computational Power/Price)}: \textcolor{blue}{GPUs, TPUs, examples of price -- both clock freq, number cores, memory, bandwidth}.
	
	\item \textit{Performance Benchmarks (Kaggle challenges, etc.)}: \textcolor{blue}{Examples like imagenet}.
	
	\item \textit{Software (algorithms)}: \textcolor{blue}{Advances in activation functions, weight-initialization schemes, optimization schemes --- batchnormalization, residual connections, separable convolutions --- allowed for deeper models to be trained.}
	
	\item \textit{Software (Libraries)}: \textcolor{blue}{More accessible --previously needed to have a deep understanding of C++ and/or CUDA --- python --- theano, tensorflow, abstractions on top, lasagne, keras, \textcolor{red}{yamlflow}. \textcolor{green}{TODO: CITE these} }
	
	\item \textit{Investment}: \textcolor{blue}{Rapid rise of machine learning investment and deployment}
	
\end{itemize}

\begin{figure}[htp]
	\centering
	\includegraphics[width=0.5\textwidth]{example-image-a}\hfil
	\caption{Cyclic figure of software, hardware, investment, benchmarks, \textcolor{green}{TODO}}
	\label{fig:cyclic_rise_of_dl_overview}
\end{figure}
\textcolor{green}{TODO: figure of how software, hardware, investment, benchmarks are related to the rise of ML .}



\section{Notes}

\subsection{Target Audience}
\textcolor{blue}{target audience: XXXXXXX}

\subsection{How to Read this Book}
\textcolor{blue}{sequential}

\textcolor{blue}{section by section (jk)}

\textcolor{green}{TODO: figure of the structure of the book. I really like figure 1.6 in DL}




\chapter{Resources and Communities}

\textcolor{blue}{There are many great resources and communities that I'd like to highlight}


\section{Online Communities}

\begin{itemize}
	
	\item Reddit
	
	\item Stack Overflow
	
	\item Slack
\end{itemize}


\section{Blogs}

\begin{itemize}
	
	\item \textcolor{blue}{XXXXXXXXXXXXXXX}
	
\end{itemize}



\section{Online Courses}

\begin{itemize}
	
	\item Udacity
	
	\item \textcolor{blue}{XXXXXXXXXXXXXXX}
	
\end{itemize}


\section{Text Books}

\begin{itemize}
	\item \textcolor{blue}{XXXXXXXXXXXXXXX}
	
\end{itemize}

\chapter{Prerequisites}

\textcolor{blue}{The following sections are a non-exhaustive, brief, refresher on some of the important underlying concepts and methodologies used by later concepts. Resources to further explore and learn these concepts are shared.}


\section{Math Notation}

\textcolor{blue}{Depending on the resource, the level of formal math education required to understand a passage may vary greatly. In order to demystify some of the resources that do not expand on the proofs and notations, below are some of the symbols and XXXXX used in math notation and their interpretation in simple text/natural language form.}

\textcolor{green}{TODO: Show symbols and examples in both equation and simple text/natural language form.}


\section{Boolean Logic}

\textcolor{green}{TODO: background/overview on boolean logic and importance}

\textcolor{green}{TODO: Examples}


\section{Linear Algebra}

\textcolor{green}{TODO: background/overview on linear algebra and importance}

\textcolor{green}{TODO: Examples}


\section{Graph Theory}

\textcolor{green}{TODO: Examples}


\part{ML}

% the organization and sec structure/naming needs work here.
\chapter{Basics}

\section{Overview}

%%%%%%%%%%%%%%%%%%%%%%%% obligatory "No Free Lunch"
I'm not sure it's possible to discuss machine learning without at least mentioning the ``No Free Lunch'' theorem, which states ``No single classifier works best across all possible scenarios''

%%%%%%%%%%%%%%%%%%%%%%%% Types of data
\textcolor{blue}{categorical or numerical. Numerical can be discrete or continuous}

%%%%%%%%%%%%%%%%%%%%%%%% Measurement Levels
\textcolor{blue}{qualitative or quantitative.} 

\textcolor{blue}{Qualitative can be nominal (aren't numbers and can't be put in any order -- e.g. the seasons: spring, summer, fall, winter) or ordinal (groups and categories that follow a strict order -- e.g. difficult levels: hard, medium, or easy)}

\textcolor{blue}{Quantitative are represented by numbers but can be interval (0 is meaningless -- e.g. temperature in C or F, where true zero is not 0) or ratio (has a true 0 -- e.g. temperature in K, weight or length)}


%%%%%%%%%%%%%%%%%%%%%%%% Acquiring Data
\section{Data Acquisition}


\subsection{Resources}

\subsection{Generating Fake Data}

\textcolor{green}{TODO: generating fake data with SKL}

\textcolor{blue}{make\_blobs}

% {{{datagen_blobs_2dcode}}}
\begin{lstlisting}[style=pyInStyle]
X, y = datagen.make_blobs(centers=4, n_samples=100, n_features=2,cluster_std=1.0,
                          center_box=(-10, 10),
                          random_state=42, shuffle=True)
\end{lstlisting}

% {{{datagen_blobs_2dimg}}}
\begin{figure}
\centering
\includegraphics[width=0.65\textwidth]{./sync_imgs/datagen/blobs/2dimg.png}
\label{fig:datagen_blobs_2dimg}
\end{figure}

\textcolor{blue}{Data can also be generated in three (multiple) dimensions}

% {{{datagen_blobs_3dcode}}}
\begin{lstlisting}[style=pyInStyle]
X, y = datagen.make_blobs(centers=4, n_samples=100, n_features=3, random_state=42)
\end{lstlisting}

% {{{datagen_blobs_3dimg}}}
\begin{figure}
\centering
\includegraphics[width=0.65\textwidth]{./sync_imgs/datagen/blobs/3dimg.png}
\label{fig:datagen_blobs_3dimg}
\end{figure}

%\textcolor{blue}{More dataset types can be generated, the documentation can be found at (http://scikit-learn.org/stable/modules/classes.html#module-sklearn.datasets)}


\textcolor{blue}{see \textcolor{red}{local ref?} for more examples on how to generate data}



%%%%%%%%%%%%%%%%%%%%%%%% Data Pre-processing
\section{Data Pre-processing}

\textcolor{blue}{Data is rarely obtained in a form that is necessary for optimal performance of a learning algorithm. Data can be missing, can contain a mix of categorical and quantitative, can contain values on vastly different scales, etc.}

\textcolor{blue}{It is important to note that any parameters related to data pre-processing, such as feature scaling and dimensionality reduction, are obtained solely from observing the training set. The parameters for these methods obtained on the training set are then later applied to the test set. This is important since if these preprocessing parameters were obtained on the entire dataset and included the test set, the the model performance may be overoptimistic since then when applying the methods to the unseen data.}

\subsection{Handling Missing Data}

\subsubsection{Filtering Out}

\textcolor{blue}{Simply removing any entries that are missing data. This is convenient and easy but may not be practical -- any time data is being removed, potentially useful information is lost and too much data may be removed.}

\textcolor{green}{TODO: Code in jupyter on how to do this with pandas and dropna -- key params - how, thresh, subset}

\subsubsection{Filling In}

\textcolor{blue}{Estimating the missing data}

\subsection{Handling Categorical Data}

\subsubsection{Encoding}

\subsection{Feature Scaling, Normalization}

\subsubsection{Min-Max scaling (Normalization)}

\textcolor{blue}{values are shifted and rescaled so they end up on a [0,1] range}

\subsubsection{Standardization}

\textcolor{blue}{(Eq.~\ref{eq:preprocess_standardization}) first, subtract the sample mean, then divide by standard deviation variance}

\textcolor{blue}{pros: unlike min-max, not bound to specific range}

\textcolor{blue}{standardized values always have a zero mean and a standard deviation of 1.}

\textcolor{blue}{gives our data the property of a standard normal distribution}

\begin{equation}
{X' = \frac{X - \mu}{\sigma}}
\label{eq:preprocess_standardization}
\end{equation}

\textcolor{green}{TODO: create code sample - numpy, and sklearn methods}


\subsection{Others}

\subsubsection{Removing Duplicates}

\subsubsection{Outliers}

\subsubsection{Discretization and Binning}



%%%%%%%%%%%%%%%%%%%%%%%% Data sampling and partitioning
\section{Partitioning Data}

\subsection{Sampling}

\textcolor{blue}{Training, validation, test}

\section{Some Terms}

\emph{input variable(s)} -- predictors, independent variables, features, regressors, controlled variables, exposure variables or simply variables.
 
\emph{output variable(s)} -- response or dependent variable. May also be known as regressands, criterion variables, measured variables, responding variables, explained variables, outcome variables, experimental variables, labels.

\textcolor{blue}{Both input and output variables may take on continuous or discrete values.}

\emph{relationship} $Y = f(x) + \epsilon$ \textcolor{blue}{estimate $f$. prediction and inference}.

\textcolor{blue}{\emph{reducible error\index{reducible error}} -- the estimated function $\hat{f}$ will likely not be perfect, and the reducible error is the error that could be corrected.  The \emph{irreducible error} is an error that can not be corrected. The irreducible error may be larger than zero due to \emph{unmeasured variables} \emph{e.g.} varibles that were not measured and \emph{unmeasurable variation} \emph{e.g.} an individual's feelings/emotions or variation in the production of a product. The irreducible error provides an upper bound on the performance of the predicted $\hat{f}$}

\section{Type of Learning}

\textcolor{blue}{Three main types of machine learning: supervised, unsupervised, and reinforcement.}

% From ML for Predictive Data Analytics
\textcolor{blue}{Another way to group types of learning -- Information-based, similarity-based, probability-based, and error-based}

\subsection{Supervised}

\textcolor{blue}{observe input variables with corresponding output values. A program that predicts an output for a in input by learning from pairs of labeled inputs and outputs. Classification \textcolor{red}{ref} and regression \textcolor{red}{ref} are subcategories of supervised learning}

\subsection{Unsupervised}

\textcolor{blue}{observe input variables without corresponding output values and attempts to discover patterns in the data.}

% p14 of mastering ml agorithms
\textcolor{blue}{There is no error signal to measure, rather, performance metrics report some attribute of structure discovered in the data, such as the distances within and between clusters.}
 
% clustering
\subsubsection{Clustering}

\textcolor{blue}{Finding sub groups where observations are more similar to eachother based on some similarity measure. Clustering is sometimes referred to as ``unsupervised classification'' and is often used to explore a dataset.}

\textcolor{blue}{An example of clustering may be to group a collection of documents into categories, or songs into genres.}

% principal components
\subsubsection{Dimensionality Reduction}

\textcolor{blue}{where the goal is to reduce the dimensionality of the data while retaining as much as the relevant information as possible}

\textcolor{blue}{A high number of features may be computationally costly. Ability to generalize may be reduced if some of the features capture noise or are irrelevant to the underlying relationship. The goal could be to find the features that account for the greatest changes in the response variable}


\subsection{Semi-supervised Learning}

\textcolor{blue}{`semi-supervised learning', another type of learning, makes use of both supervised and unsupervised data.}

\subsection{Reinforcement}

\textcolor{blue}{Reinforcement learning does not learn from labeled pairs of inputs and outputs, rather it learns from `feedback' from decisions that are not explicitly corrected.}

\textcolor{blue}{Goal -- develop an \emph{agent} that improves it's performance based on interactions with an \emph{environment} based on a \emph{reward}}

\subsection{Supervised vs Unsupervised}

\subsection{Classification vs Regression}

\subsubsection{Regression} -- Also called regression analysis \textcolor{red}{local ref?} predicting a continuous or quantitative output value. For example attempting to find a relationship between a given predictor/explainatory variables (age, job title, zip code) and a continuous response (an individuals outcome).

\subsubsection{Classification} -- predicting categorical (discrete) or qualitative output value (such as a non-numerical value). \textcolor{blue}{Binary classification (benign vs malignant) and multi-class classification (identifying many different skin diseases).}

% page 94 of AGtext
One-versus-all \emph{OvA} (also \emph{one-versus-rest}) -- 

One-versus-one (OvO) -- train a binary classifier for every pair


\textcolor{blue}{Binary classification can be extended to multi-class classification via the OvR method.}

%\subsubsection{Bayes Classifier}

\section{Training}

\section{Quality of Fit}

%% regression example

\subsection{Regression Example}

\textcolor{blue}{Mean Squared Error.$\hat{f}(x_i)$ is the prediction that $\hat{f}$ produces for the $i$th sample. The output will be small for predicted values that are similar to the ground truth}

\begin{equation}
{MSE = \frac{1}{n}\sum_{i=1}^{n}(y_i - \hat{f}(x_i))^2}
\label{eq:MSE_def}
\end{equation}

%% classification example

\subsection{Classification Example}

\textcolor{blue}{The proportion of mistakes that are made.}

\begin{equation}
{error\_rate = \frac{1}{n}\sum_{i=1}^{n}(y_i \ne \hat{y_i})}
\label{eq:class_error_rate_def}
\end{equation}

\textcolor{blue}{$\hat{y_i}$ is the predicted classification label for the $i$th observation using our predictor/model $\hat{f}$ and $y_i$ is the ground truth label}

\section{(Over|Under)fitting}

\subsection{Overfitting}

\textcolor{blue}{Overfitting\index{Overfitting} refers to a case in which a model fits the training data very well but does not fit validation/test set. If a model is overfitting, it is said to have a high variance and is analogous to memorizing the training set.}

\textcolor{blue}{Overfitting can arise from modeling data with too many parameters/too complex of a model.}

\textcolor{green}{TODO: figure showing an example of overfitting}

\subsection{Underfitting}

\textcolor{blue}{Underfitting\index{Underfitting} refers to a case in which a model does not fit the training data well. If a model is underfitting, it is said to have a high bias}

\textcolor{blue}{Underfitting can arise from modeling data with too few parameters/too simple of a model.}

\textcolor{green}{TODO: figure showing an example of underfitting}


\section{Bias Variance Trade-off}

\textcolor{blue}{Two fundamental causes of prediction error in a model -- the bias and the variance.}

\subsection{Variance}
\textcolor{blue}{variance\index{Variance} refers to the amount the model would change (consistency or variability) if it was re-trained/estimated multiple using a different subsets of the training data set. A model that has high variance is sensitive to randomness in the training data}

\textcolor{blue}{A model with high variance may be described as highly flexible and will likely overfit the data.}


\subsection{Bias}
\textcolor{blue}{Bias\index{Bias} refers to the amount of error that is introduced by approximating a problem with a model that is simpler than the (complex) problem}

\textcolor{blue}{A model with high bias will produce similar errors for instances regardless of the training data that is used to train the model -- the model is more strongly ``biased'' to its own assumptions of the relationship (as defined by the model), than the relationship the data may be indicating. A model with high bias may also be described as inflexible and will likely underfit the data.}


% not word-for-word, but example adapted from p35 of ISL
\textcolor{red}{For example, linear regression assumes a linear relationship between the features and labels. However, it is unlikely that a true linear relationship exists and so using linear regression to model this type of particular problem will likely introduce some bias.}

\subsection{Trade-Off}

% TODO: see page 34 of ISL for eq and explaination here

\textcolor{blue}{In general, as a more ``flexible'' model is used, the variance will increase and the bias will decrease.}


% see page 36 of ISL
\textcolor{blue}{It is easy to obtain a model with low bias but high variance (\emph{e.g.} drawing a squiggly line through every training observation) and it is easy to obtain a model with low variance but high bias (\emph{e.g.} drawing a straight line approximating every training observation) but it is difficult to obtain a model that has both low variance and low bias.}

\textcolor{blue}{It should be noted that in a real world example, it maynot be possible to explicitly calculate the test error, bias, or variance.}

\textcolor{green}{TODO: para about using regularization here/finding the right balance \textcolor{red}{local ref to regularization?}}

\subsection{Parametric vs non-parametric}

\subsubsection{parametric}

\textcolor{blue}{parametric models are models that learn a fixed number of parameters, independent from the number of training instances, that able to classify new data points without requiring the original dataset anymore. First, a function form is selected (linear, polynomial, etc.), then the coefficients for the function are learned form the training data.}
	
\textcolor{blue}{Examples of parametric models may be simple artificial neural networks, naive bayes, logistic regression, etc.}

\subsubsection{nonparametric}

%% unsure about this! 
\textcolor{red}{Nonparametric models are not models without parameters, rather they are models were the number of parameters are not fixed, they may grow with the number of training instances}

\textcolor{blue}{An Example of a nonparametric model may be k-Nearest neighbors -- where the model does not assume anything about the form of the mapping function and makes predictions based on the k most similar training instances.}

\textcolor{blue}{A disadvantage to this type of approach is that the computational complexity for classifying new samples grow linearly with the number of samples in the training set.}

\textcolor{blue}{May be useful when little is known about the underlying relationship in the data and there is an abundance of data.}

\subsection{Eager vs Lazy Learners}

\textcolor{green}{TODO: Eager vs Lazy overview}
\textcolor{blue}{Training an eager learner is often more computationally expensive, but typically prediction with the resulting model is inexpensive.}

\subsubsection{Eager Learners}

\textcolor{blue}{Eager learners estimate the parameters of a model that generalize to a training set}

\subsubsection{Lazy Learners}

\textcolor{blue}{Also known as Instance-based Learners}

\textcolor{blue}{Lazy learners store the training dataset with little to no processing.}


\subsection{Generative vs Discriminative Models}

\textcolor{green}{TODO: Generative vs Discriminative models overview}
%\textcolor{blue}{}

\subsubsection{Generative Models}

\textcolor{green}{TODO: Generative Models}

\subsubsection{Discriminative Models}

\textcolor{green}{TODO: Discriminative Models}

\subsection{Strong vs Weak Learners}

\textcolor{green}{TODO: Strong vs Weak learners (classifier, predictor, etc.) overview}
%\textcolor{blue}{}

\subsubsection{Strong Learners}

\textcolor{green}{TODO: Strong Learners are models that are arbitrarily better than weak learners.}

\subsubsection{Weak Learners}

\textcolor{green}{TODO: Weak Learners are models (typically simple models) that perform only slightly better than random chance. }


\section{Online Learning}

% See p.246 of Understanding Machine learning
\textcolor{blue}{difference to \textcolor{red}{PAC learning?}}

\section{Ensemble Methods}

\textcolor{green}{TODO: overview - discussed in more detail in \textcolor{red}{local ref?}}



%%%%%%%%%%%%%%%%%%%%%%%% Evaluation

\r{Importance of dataset partitioning \textcolor{red}{local ref?}}

% \textcolor{blue}{The best performance measure will vary depending on the task. For instance, in a medical setting, it may be life threating to classify an event as ``healthy'' when the patient is not healthy.}

\r{A performance measure is used to capture, empirically, how well a prediction made by the model aligns with the expected, ground truth, value.}

\r{Evaluation metrics allow for intuitive explaination of the results to those who may be non/less-technical}

\subsection{Creating a Test Set}

% rough para
\r{The most important rule regarding evaluating models, is to ensure that the data used to evaluate the model has never been used before to influence the during training or selection -- this means it was not used during training to update the parameters and it was not used to influence which models are `best' (like a validation set may be used for)}

\r{The performance of a model on a test set may be indicative of how well the model can generalize to unseen data. (This assumes your data sample is representative of the data population)}

\r{Hold-out test set -- created by randomly sampling the dataset. Again, it is important to emphasize that the instances in the test set are never used in the training process and are instead reserved for use only during the evaluation phase.}

\r{peeking\index{peeking}, is an issue that arises when part or all of the test set is included in the training set. This means the model has already seen the data on which the model will be evaluated and so it is possible, probable, that the model will produce high evaluation scores, which will likely translate to an overoptimistic estimation of the models performance when used in production.}

\r{Evaluating the performance of a model can be challenging and will vary depending on the task. For instance, accuracy may not always be the best measure of performance -- consider a medical setting in which sensitivity may be more important since a false negative may be life threatening where as a false positive may only require additional observation.}

\r{When comparing various models, it may be challenging to rank them on a single performance measure. \textcolor{green}{TODO: more.}}

\subsection{Qualitative Evaluation}

\r{generalization is a measure of how well the system preforms on previously unseen data. generalization error.}



\subsubsection{(Over$|$Under)fitting and Capacity}

\r{{Model capacity}\index{model capacity} helps control how likely a model is to overfit or underfit. Where a model with low capacity may have difficulty fitting a a training set and a model with high capacity may ``overfit'' the data by essentially memorizing the training data.}

\r{Model capcity is closely related to model complexity and the models {hypothesis space}index{hypothesis space} (The set of functions available to the learning algorithm --- \textcolor{green}{TODO: expand - for example a linear vs polynomial model})}

\TD{TODO: figure showing training and validation error and 1) optimal capacity, 2) under and overfitting region 3)generalization gap, 4) capacity}

\paragraph{Overfitting}

\r{Overfitting\index{Overfitting} refers to a case in which a model fits the training data very well but does not fit validation/test set. If a model is overfitting, it is said to have a high variance and is analogous to memorizing the training set.}

\r{Overfitting can arise from modeling data with too many parameters/too complex of a model.}

\r{learning ``particularities in the training set''}

\TD{TODO: figure showing an example of overfitting}

% addressing overfitting: 1) reduce number of features (manual selection or w/model selection algor) 2) regularization

\begin{figure}[htp]
	\centering
	\includegraphics[width=0.3\textwidth]{example-image-a}\hfil
	\includegraphics[width=0.3\textwidth]{example-image-b}\hfil
	\includegraphics[width=0.3\textwidth]{example-image-c}\hfil
	\caption{Figure example showing the same 2d dataset and an underfitting, overfitting, and ``good'' fitting. \textcolor{green}{TODO} circles=training, x=test -- include scores for each.. slight curve' under=linear, over=extreme poly, good=``smooth''}
	\label{fig:basics_eval_fitting_examples}
\end{figure}

\paragraph{Underfitting}

\r{Underfitting\index{Underfitting} refers to a case in which a model does not fit the training data well. If a model is underfitting, it is said to have a high bias}

\r{Underfitting can arise from modeling data with too few parameters/too simple of a model.}

\TD{TODO: figure showing an example of underfitting}

\subparagraph{Solution}

\TD{method for better optimization and increasing model capacity: greedy layer-wise --- unsupervised pre-training}

\r{better optimization --- use better optimization methods \ALR}


\subsubsection{Bias Variance Trade-off}

\r{Two fundamental causes of prediction error in a model -- the bias and the variance.}

\paragraph{Variance}
\r{variance\index{Variance} refers to the amount the model would change (consistency or variability) if it was re-trained/estimated multiple using a different subsets of the training data set. A model that has high variance is sensitive to randomness in the training data}

\r{A model with high variance may be described as highly flexible and will likely overfit the data.}


\paragraph{Bias}
\r{Bias\index{Bias} refers to the amount of error that is introduced by approximating a problem with a model that is simpler than the (complex) problem}

\r{A model with high bias will produce similar errors for instances regardless of the training data that is used to train the model -- the model is more strongly ``biased'' to its own assumptions of the relationship (as defined by the model), than the relationship the data may be indicating. A model with high bias may also be described as inflexible and will likely underfit the data.}


% not word-for-word, but example adapted from p35 of ISL
\textcolor{red}{For example, linear regression assumes a linear relationship between the features and labels. However, it is unlikely that a true linear relationship exists and so using linear regression to model this type of particular problem will likely introduce some bias.}

\paragraph{Trade-Off}

% TODO: see page 34 of ISL for eq and explaination here

\r{In general, as a more ``flexible'' model is used, the variance will increase and the bias will decrease.}

\r{One reason to choose a more restrictive model is that they are often more interpretable.}

\begin{figure}[htp]
	\centering
	\includegraphics[width=0.4\textwidth]{example-image-a}\hfil
	\includegraphics[width=0.4\textwidth]{example-image-b}\hfil
	\caption{\TD{side by side figure: a: complex vs simple training on trainnig data (nth poly vs linear), b: same models on test data}}
	\label{fig:basics_eval_tradeoff_examples}
\end{figure}


% see page 36 of ISL
\r{It is easy to obtain a model with low bias but high variance (\emph{e.g.} drawing a squiggly line through every training observation) and it is easy to obtain a model with low variance but high bias (\emph{e.g.} drawing a straight line approximating every training observation) but it is difficult to obtain a model that has both low variance and low bias.}

\textcolor{blue}{It should be noted that in a real world example, it may not be possible to explicitly calculate the test error, bias, or variance.}



\begin{figure}[htp]
	\centering
	\includegraphics[width=0.3\textwidth]{example-image-a}\hfil
	\includegraphics[width=0.3\textwidth]{example-image-b}\hfil
	\includegraphics[width=0.3\textwidth]{example-image-c}\hfil
	\caption{\TD{same 2D dataset with 3 layers and the hidden layer in a has few nodes, b: normal amount of nodes, and c: many nodes} \r{illistrative that the number of connections and complexity increases the chances for overfitting also increases}}
	\label{fig:basics_eval_nodesinhidden}
\end{figure}


\begin{figure}[htp]
	\centering
	\includegraphics[width=0.2\textwidth]{example-image-a}\hfil
	\includegraphics[width=0.2\textwidth]{example-image-b}\hfil
	\includegraphics[width=0.2\textwidth]{example-image-c}\hfil
	\includegraphics[width=0.2\textwidth]{example-image-a}\hfil
	\caption{\TD{same 2D dataset with one, two, three and four hidden layers}}
	\label{fig:basics_eval_numlayers}
\end{figure}

\r{observing a direct trade-off between overfitting and model complexity.}

\r{When we talk about deep learning, we're talking about deep and powerful models that are attempting to solve complex problems that are prone to overfitting and thus usually employ additional countermeasures, such as regularization, to help prevent overfitting.}


\TD{TODO: para about using regularization here/finding the right balance \textcolor{red}{local ref to regularization?}}



%%%%%%%%%%%%%%%%%%%%%%%% Metrics
\emph{Cost} is frequently used interchangeably with loss. Technically, loss refers to the error on a single example and cost is the average of the loss across the entire training set.

% page 94 of AGtext
One-versus-all \emph{OvA} (also \emph{one-versus-rest})

One-versus-one (OvO) -- train a binary classifier for every pair


\section{Metrics}

%% Confusion matrix
\begin{table}
	\centering
	\begin{tabular}{l|l|c|c|}
		\multicolumn{2}{c}{}&\multicolumn{2}{c}{Ground Truth}\\ 
		\cline{3-4}
		\multicolumn{2}{c|}{}&Positive&Negative\\ 
		\cline{2-4}
		\multirow{2}{*}{\rotatebox{90}{Pred}}& Positive & $TP$ & $FP$ \\ 
		\cline{2-4}
		& Negative & $FN$ & $TN$ \\ 
		\cline{2-4}
	\end{tabular}
	\caption{Example confusion matrix}
	\label{tab:sample_conf_matrix}
\end{table}


\begin{itemize}
	
\item \textit{Accuracy}, (Eq.~\ref{eq:accuracy}): the ratio of correct predictions to the total number of predictions.

\begin{equation}
{\frac{TP+TN}{TP+TN+FP+FN}}
\label{eq:accuracy}
\end{equation}

\item \textit{Sensitivity}, (Eq.~\ref{eq:sensitivity}): the ratio of true positives that are correctly identified.

\begin{equation}
{\frac{TP}{TP+FN}}
\label{eq:sensitivity}
\end{equation}

\item \textit{Precision}, (Eq.~\ref{eq:precision}): the ratio of positives that are, in fact, positive. If the classifier predicts positive, how often is is correct?

\begin{equation}
{\frac{TP}{TP+FP}}
\label{eq:precision}
\end{equation}

\item \textit{AUC (Area Under the Curve)}, is a single value representing the area under an ROC curve. Though generally referred to as the AUC, the term is correctly abbreviated AUROC, specifying that the curve is an ROC curve.
\end{itemize}



\chapter{Foundational Methods}

\section{Regression}

\subsection{Simple Linear Regression}


\begin{equation}
{Y \approx \beta_0 + \beta_1 X}
\label{eq:slr_ex}
\end{equation}

\textcolor{blue}{$\approx$ can be read as ``\emph{is approximately modeled as}''. $Y$ is a quantitative response (output/prediction) and $X$ predictor variable(input/feature). $\beta_0$ and $\beta_1$ are two unknown constants representing the intercept and slope, respectively. These unknown values that determine the behavior of the model are known as the model \emph{parameters} or \emph{coefficients}}

% see p62 of ISL for more

\subsection{Multiple Linear Regression}

\textcolor{blue}{Using $n$ predictors:}

\begin{equation}
{Y \approx \beta_0 + \beta_1 X_1 + \beta_2 X_2 + \cdots + \beta_n X_n}
\label{eq:mlr_ex}
\end{equation}


\subsection{Polynomial Regression}



\subsection{K-Nearest Neighbors}

\textcolor{blue}{The optimal value for k will depend on the bias-variance trade-off}

\section{Classification}

\subsection{Logistic Regression}

\textcolor{blue}{Despite the `regression' bit in the name, logistic regression is a classification model}

\textcolor{blue}{odds ratio\index{odds ratio} (Eq.~\ref{eq:odds_ratio}), where $p$ is representative of the probability of a positive (event we aim to predict) event.}

\begin{equation}
{\frac{p}{1-p}}
\label{eq:odds_ratio}
\end{equation}

\textcolor{blue}{A logit\index{logit} function (Eq.~\ref{eq:logit_def}) is the logarithm of the odds ratio (log-odds)}

\begin{equation}
{logit(p)=\log{\frac{p}{1-p}}}
\label{eq:logit_def}
\end{equation}

\textcolor{blue}{logistic function (sigmoid function) (Eq.~\ref{eq:sigmoid_def}) -- the inverse of a logit function and corresponds to the probability that a certain sample belongs to a particular class}

\begin{equation}
{S(x)={\frac{1}{1+e^{-x}}}={\frac{e^x}{e^x+1}}}
\label{eq:sigmoid_def}
\end{equation}


%%%%%%%%%%%%%%%%%%%%%%%%%%%%%% Support Vector Machines
\subsection{Support Vector Machines (SVM)}

\textcolor{blue}{Support Vector Machine (SVM)\index{Support Vector Machine (SVM)}. In order to minimize misclassification errors, the optimization objective is to maximize the margin (distance between the decision boundary (separating hyperplane) and the nearest training samples. These margins are called support vectors). Maximizing the margins, in theory, tend to have lower generalization error, where smaller margins may be more prone to overfitting.}

\textcolor{blue}{(Slack parameter?)}

\textcolor{blue}{Variable can be used to control the width of the margin and help tune the bias-variance trade-off.}

\textcolor{green}{TODO: figure showing difference in width of margins}

\subsubsection{Kernel SVM}

\textcolor{blue}{kernelized to solve nonlinear classification problems}

\paragraph{The `Kernel Trick'}

\textcolor{green}{TODO: paras about the kernel trick}

\textcolor{blue}{Transform the training data onto a higher dimensional feature space}

%%%%%%%%%%%%%%%%%%%%%%%%%%%%%% Decision Trees
\subsection{Decision Trees}

\textcolor{blue}{\textcolor{green}{(TODO: revise this para!)} Decision trees make classification decisions based on a series of questions that separate the data into subsets. These questions are chained and result in a tree of questions where the leaves are considered pure i.e. they contain samples that belong to the same class.}

\textcolor{blue}{Importance of pruning -- Decision trees can be very deep and can easily lead to overfitting. To help prevent this situation, a limit is set for the maximal depth of a tree. }

\subsubsection{Criterion -- Maximizing Information Gain}

\textcolor{blue}{Term - Information gain -- difference between the impurity of the parent node and the sum of the child node impurities -- the lower the impurity of the child nodes compared to the parent node, the higher the information gain}

\textcolor{blue}{Three commonly used splitting criteria used in binary decision trees: (i) Gini Impurity, (ii) entropy, and (iii) classification error}

\paragraph{Gini Impurity}

\paragraph{Entropy}

\paragraph{Classification Error}


\subsection{Random Forests}

\textcolor{blue}{Ensemble method. Combine various decision trees, where some may be weak learners\index{weak learner} (\textcolor{green}{def}) and some may be strong learners\index{strong learner} (\textcolor{green}{def}). The final classification will be determined by majority vote from the number of trees.}


\subsection{K-nearest Neighbors Classifier}

\textcolor{blue}{KNN is a lazy learner\index{lazy learner} (a special case of instance-based nonparametric model (see \textcolor{red}{local ref?})): the model memorizes the training dataset rather than learn a discriminative function}

\textcolor{blue}{KNN classification involves i) choosing the number of $k$ (nearest neighbors) and a distance metric, ii) finding the $k$ nearest neighbors, and iii) assigning a class label by majority vote.}

\textcolor{blue}{The value of $k$ is important when finding the balance between over and under fitting.}


 % algorithm foundations

\chapter{Ensemble Methods}

%%%%%%%%%%%%%%%%%%%%%%%%%%%%%% Ensemble Methods
% TODO: I'm still not sure how/where to structure this

\section{Dense}

\TD{TODO}

\section{Convolutions}

\TD{TODO}

\section{Pooling}

\TD{TODO}

\section{Recurrent Cells}

\TD{TODO}

\section{Capsule Networks}

\TD{TODO}

\section{Attention}

\r{overview can be found here\cite{weng2018attention}}

\TD{The original attention mechanism is introduced\cite{Bahdanau2015NeuralMT}.}

% TF attention implementation (https://www.tensorflow.org/tutorials/text/nmt_with_attention)

\TD{Effective Approaches to Attention-based Neural Machine Translation \cite{DBLP:journals/corr/LuongPM15}}

\TD{Massive Exploration of Neural Machine Translation Architectures \cite{DBLP:journals/corr/BritzGLL17}}

% TODO: index for transformer
\TD{Attention Is All You Need -- Transformer network --- multi-head self-attention mechanism, key-value pairs \cite{DBLP:journals/corr/VaswaniSPUJGKP17}}

% self-attention \TD{Self-attention, less commonly intra-attention}
\TD{Long Short-Term Memory-Networks for Machine Reading \cite{DBLP:journals/corr/ChengDL16}}


%\TD{Nice table comparing mechanisms https://lilianweng.github.io/lil-log/2018/06/24/attention-attention.html}

\TD{in above post\cite{weng2018attention}: soft vs hard attention and global vs local attention}


% TODO: read https://lilianweng.github.io/lil-log/2020/04/07/the-transformer-family.html


\section{NTM (Neural Turing Machines)}
% nerual network + external memory storage
% controller + memory
\TD{Neural Turing Machines \cite{DBLP:journals/corr/GravesWD14}}


\chapter{Term dump}

\emph{Collinearity} -- When two or more predictor variables are closely related to one another they are said to be collinear.

\emph{Curse of Dimensionality} --

\emph{dummy variable} --


\emph{Population vs Sample} -- the population (usually denoted $N$) is the collection of all the items of interest in a study where as the sample is a subset of a population (usually denoted $n$). The numbers obtained when working with a population are called the `parameters' and the numbers obtained when working with a sample are a called `statistics'. \textcolor{blue}{a random sample is obtained when each member of the sample is chosen from the population by chance and accurately reflects the population}


\part{Brief Reference}



\textcolor{blue}{Brief overview of how all packages and environments work together.}

%%%%%%%%%%%% Environment
\chapter{Environment}

\textcolor{green}{TODO: Diagram of an Overview of Environment}

\textcolor{blue}{The following sections are quick overviews of various components. Relevant, common, and useful information (in the context of AI/ML/DL) is provided but not expanded upon. Recommended references for further reading are included in each section. I would not advise ``reading'' this chapter as it is more of a reference and meant to be skipped around.}

\input{./nested/terminal}

\subsection{Hardware}

\subsubsection{CPU vs GPU}

\subsubsection{Cloud Providers}

\subsubsection{AWS Quickstart}

\section{Python}

\textcolor{green}{TODO: overview}

\textcolor{blue}{This section will not teach you everything you need to know to be python programmer. Rather, this section will assume you have programming experience and will focus on a few of the components of python that may be frequently encountered and may benefit from further explanation and examples.}

\subsection{Datatypes}

\subsubsection{Tuple}

\textcolor{blue}{fixed-length immutable sequence of Python objects.}

\subsubsection{List}

\textcolor{blue}{variable-length mutable sequence of Python objects.}

\paragraph{List Comprehensions}

\textcolor{blue}{para about list comprehensions}

% {{{py_nested_listcomp}}}
\begin{lstlisting}[style=pyInStyle]
matrix = [[1,2,3], [4,5,6], [7,8,9]]
# value is multiple of 3 and array sum >= 10
filtered = [[x for x in row if x % 3 == 0]
            for row in matrix if sum(row) >= 10]
print(filtered)
\end{lstlisting}
\begin{lstlisting}[style=pyOutStyle]
[[6], [9]]
\end{lstlisting}
\begin{markdown}
Using nested list comprehensions is possible but gets a little messy -- the rule of thumb is to not use more than 2 expressions in list comprehensions
\end{markdown}

\paragraph{Append Vs Extend}

\textcolor{blue}{Append vs extend.}

% {{{py_app_v_ext}}}
\begin{lstlisting}[style=pyInStyle]
x = [1, 2, 3]
x.append([4, 5])
print("Append: {}".format(x))

x = [1, 2, 3]
x.extend([4, 5])
print("Extend: {}".format(x))
\end{lstlisting}
\begin{lstlisting}[style=pyOutStyle]
Append: [1, 2, 3, [4, 5]]
Extend: [1, 2, 3, 4, 5]
\end{lstlisting}


\subsubsection{Dict}

\textcolor{blue}{para about dicts}

\subsubsection{Set}

\textcolor{blue}{para about sets}

\subsection{Functions}

\textcolor{blue}{para about functions}

\subsubsection{Keyword Only Arguments}

\textcolor{blue}{para about keyword only parameters}

% {{{py_func_kwonly}}}
\begin{lstlisting}[style=pyInStyle]
def func_with_kwargs(num, a, b,
                    *,
                    div_a=False,
                    div_b=False):
    if div_a:
        num /= a
    if div_b:
        num /= b
    return num

# print(func_with_kwargs(12, 2, 2, True, True)) # won't work
print(func_with_kwargs(12, 2, 2, div_a=True, div_b=True))
\end{lstlisting}
\begin{lstlisting}[style=pyOutStyle]
3.0
\end{lstlisting}
\begin{markdown}
all args after the `*` must be specified
\end{markdown}


\paragraph{Optional Parameters}

\textcolor{blue}{para about optional parameters}

% {{{py_opt_params}}}
\begin{lstlisting}[style=pyInStyle]
def log(message, *values):
    if not values:
        print(message)
    else:
        val_str = ", ".join(str(x) for x in values)
        print("{}: {}".format(message, val_str))

log("current number")
log("current numbers are", 1, 2)
\end{lstlisting}
\begin{lstlisting}[style=pyOutStyle]
current number
current numbers are: 1, 2
\end{lstlisting}

\subsubsection{Built-in Sequence Functions}

\paragraph{enumerate}

\textcolor{blue}{Enumerate is used for ...}

% {{{py_enumerate_01}}}
\begin{lstlisting}[style=pyInStyle]
for i, letter in enumerate(["a", "b", "c", "d"]):
    print("letter[{}]={}".format(i, letter))
\end{lstlisting}
\begin{lstlisting}[style=pyOutStyle]
letter[0]=a
letter[1]=b
letter[2]=c
letter[3]=d
\end{lstlisting}


% {{{py_enumerate_02}}}
\begin{lstlisting}[style=pyInStyle]
for i, letter in enumerate(["a", "b", "c", "d"], 1):
    print("letter#{}={}".format(i, letter))
\end{lstlisting}
\begin{lstlisting}[style=pyOutStyle]
letter #1=a
letter #2=b
letter #3=c
letter #4=d
\end{lstlisting}
\begin{markdown}
Using nested list comprehensions is possible but tgets a little messy -- the rule of thumb is to not use more than 2 expressions in list comprehensions
\end{markdown}


\paragraph{sorted}

\textcolor{blue}{para about sorted}

\paragraph{zip}

\textcolor{blue}{para about zip}

% {{{py_zip}}}
\begin{lstlisting}[style=pyInStyle]
pets = ["Siren", "Diesel"]
age = [3, 2]
for pet_info in zip(pets, age):
    print(pet_info)
\end{lstlisting}
\begin{lstlisting}[style=pyOutStyle]
('Siren', 3)
('Diesel', 2)
\end{lstlisting}

% {{{py_zip_longest}}}
\begin{lstlisting}[style=pyInStyle]
pets = ["Siren", "Diesel", "Bella"]
age = [3, 2]
print("---- Using zip")
for pet_info in zip(pets, age):
    print(pet_info)

print("------ using zip_longest")
for pet_info in zip_longest(pets, age):
    print(pet_info)
\end{lstlisting}
\begin{lstlisting}[style=pyOutStyle]
---- Using zip
('Siren', 3)
('Diesel', 2)
------ using zip_longest
('Siren', 3)
('Diesel', 2)
('Bella', None)
\end{lstlisting}

\paragraph{reversed}

\textcolor{blue}{para about reversed}

\subsection{Generators}

\textcolor{blue}{para about generators}

\subsection{Errors and Exception Handling}

\textcolor{blue}{para about errors and exception handling}

% {{{py_tryblock}}}
\begin{lstlisting}[style=pyInStyle]
try:
    # do something
except MyException as e:
    # handle exception
else:
    # runs when there are no exceptions
finally:
    # always runs after try:
\end{lstlisting}

\subsection{IO}

\textcolor{blue}{para about IO}

\subsection{Other}

\textcolor{blue}{para about other}




\input{./nested/conda}

\subsection{git}

\subsubsection{Overview}

\subsubsection{Commands}

\subsubsection{Github}

\section{IDE}

\subsection{Atom}


\subsubsection{Installation}

Linux (Ubuntu) instructions from https://flight-manual.atom.io/getting-started/sections/installing-atom/. Two options: (i) install package repository on system, or (ii) download and install from `.deb'

\paragraph{(i) Package Repository}

First, add the official package repository to the system
\begin{lstlisting}[style=terminalBash]
curl -L https://packagecloud.io/AtomEditor/atom/gpgkey | sudo apt-key add -
sudo sh -c 'echo "deb [arch=amd64] https://packagecloud.io/AtomEditor/atom/any/ any main" > /etc/apt/sources.list.d/atom.list'
sudo apt-get update
\end{lstlisting}

Next, install Atom
\begin{lstlisting}[style=terminalBash]
# Install Atom
sudo apt-get install atom
# Install Atom Beta
sudo apt-get install atom-beta
\end{lstlisting}

\paragraph{(ii) Using .deb}

First, download `.deb' package from atom. Then execute the following instructions.
\begin{lstlisting}[style=terminalBash]
# Install Atom
sudo dpkg -i atom-amd64.deb
# Install Atom's dependencies if they are missing
sudo apt-get -f install
\end{lstlisting}

\section{Jupyter}

\subsection{Environment}

\subsubsection{Styling}

% library for customizing theme

% p.500 of PDA
\subsubsection{Reloading Module Dependencies}

% p 498 of PDA
\subsubsection{Profiling}

\input{./nested/docker}

\section{Kube Flow}

\textcolor{blue}{Kubeflow...}

%%%%%%%%%%%%%%%%%%%%%%%% Common
\chapter{Common Libraries}

\textcolor{green}{TODO: Overview of Environment and libraries that are described}


\section{Numpy}

\textcolor{blue}{Designed to work with homogeneous numerical array data}

\subsection{ndarrays}

\subsubsection{Initialization}

\subsubsection{Indexing}

\subsubsection{Datatypes}

\subsection{Arithmetic}

\subsubsection{Basic}

\subsubsection{Statistical Methods}

\subsection{IO}

% save

% savez

\subsection{Other}

\subsubsection{transpose}

\subsubsection{Set Logic}

% unique





%%%%%%%%%%%%%%%%%%%%%%%% Image Packages
\chapter{Images}

\input{./nested/opencv}

%%%%%%%%%%%%%%%%%%%%%%%% NLP Packages
\chapter{Natural Language Processing}

\input{./nested/nltk}


%%%%%%%%%%%%%%%%%%%%%%%% Ingesting
\chapter{Data Acquisition}

\input{./nested/scrapy}

\input{./nested/beautifulsoup}

\chapter{Ingesting Data}

\section{Apache Beam}

\textcolor{green}{Apache Beam .....}

\chapter{Ingesting Data: Databases}

\section{SQL}

% TODO: this section will need major restructuring/reorganizing
% not looking to teach SQL from the ground up, just giving basics to *RETRIEVING* data

\textcolor{blue}{SQL (Structured Query Language) ...}

%\textcolor{blue}{variants}

\textcolor{blue}{This section will provide a brief overview of SQL and will focus on using SQL to \textbf{retrieve} data. A list of resources is provided for more information in case someone is interested in managing, maintaining, and updating databases.}

\subsection{Structure}


\subsubsection{Tables}

\textcolor{blue}{Table is made up of rows and columns}

\paragraph{Table Name}

\paragraph{Table Columns}

\paragraph{Primary Key}

\paragraph{Foreign Key}

\subsection{Relationships}

\subsection{Views}


\subsection{Writing Queries}

\textcolor{blue}{A (very) brief overview to various components of a SQL query statement.}

\textcolor{blue}{For more information please see the list of resources listed in \textcolor{red}{local ref}}

\subsubsection{Overview}



\subsubsection{Basic Query Statements}

\paragraph{SELECT}

\paragraph{FROM}

\paragraph{WHERE}

\paragraph{HAVING}
% https://javarevisited.blogspot.com/2013/08/difference-between-where-vs-having-clause-SQL-databse-group-by-comparision.html

\paragraph{AND}

\paragraph{ORDER BY}

\paragraph{GROUP BY}

\paragraph{LIMIT}

\paragraph{COUNT}

\paragraph{BETWEEN}

\paragraph{JOIN}

\paragraph{Other}

%\textcolor{blue}{*}

\input{./nested/mongo}

%%%%%%%%%%%%%%%%%%%%%%%% Exploring
\chapter{Analyzing Data}

\section{Pandas}

\textcolor{blue}{Open source data analysis and manipulation tool designed to make working with ``relational'' or ``labeled'' data.}

\textcolor{blue}{When working with pandas, there are two primary data structures (described below), the \code{series} (1-dimensional / single column) and the \code{DataFrame} (2-dimensional / rows+columnsv).}

\subsection{Installation}

Installation instructions can be found on the main website (https://pandas.pydata.org/) otherwise, if using anaconda (\textcolor{red}{local ref}) the following command may be used.

\code{conda install -c anaconda pandas}

\subsection{Series}

% TODO: create series

% TODO: create from other type

% TODO: read from file

\subsubsection{Attributes}

\textcolor{blue}{Attributes do not modify the data, they only view the data.}

% TODO: common

\textcolor{blue}{\code{.head()}}
\textcolor{blue}{\code{.tail()}}
\textcolor{blue}{\code{.count()}}
\textcolor{blue}{\code{.isnull()}}
\textcolor{blue}{\code{.sum()}}
\textcolor{blue}{\code{.mean()}}

% TODO: TRAP
\textcolor{red}{\code{.len()} vs \code{.count()}}

\textcolor{blue}{\code{.std()}}
\textcolor{blue}{\code{.min()}}
\textcolor{blue}{\code{.max()}}
\textcolor{blue}{\code{.median()}}
\textcolor{blue}{\code{.mode()}}
\textcolor{blue}{\code{.value\_counts()}}
\textcolor{blue}{\code{.describe()}}

\textcolor{blue}{\code{.idxmax()}}
\textcolor{blue}{\code{.idxmin()}}


\subsection{Dataframe}

\subsection{XXXXXX}

\textcolor{blue}{examples for both series and df}

\subsubsection{apply}

%TODO: overview + example series

%TODO: overview + example DF

\subsubsection{map}

%TODO: overview + example series

%TODO: overview + example DF

\subsubsection{sort}

%TODO: overview + example series

%TODO: overview + example DF

\subsubsection{inplace}

%TODO: overview + example series

%TODO: overview + example DF

\subsection{Hierarchical Indexing}

\subsection{Describing and Visualizing}

\subsection{Merging, Joining, Pivoting}

\subsection{Groups}

\subsection{Data Loading}

% functions for loading data


%%%%%%%%%%%%%%%%%%%%%%%% Visualizing
\chapter{Visualizing Data}

\section{Matplotlib}

\subsection{Basics}

%%%%%%%%%%%%%%%%%%%%%%%%%%%%%%%%%%
\subsection{Representation of types of data}

\subsubsection{Categorical Variables}

\paragraph{Frequency Distribution Tables}

\textcolor{blue}{two columns, one for the category and the other for the number of occurrences (frequency)}

\paragraph{Bar Charts}

\textcolor{blue}{Shows a table in a graphical form where each bar (each different category) height/length is representative of the value}

\paragraph{Pie Charts}

\textcolor{blue}{Shows a table in a graphical form where a circle (pie) shows the relative frequency of each categorical value}

\paragraph{Pareto Diagrams}

\textcolor{blue}{a special type of bar chart where the categories are shown in descending order of frequency and an additional curve shows the cumulative frequency (sum of relative frequencies)}


\subsubsection{Numerical Variables}



%%%%%%%%%%%%%%%%%%%%%%%%%%%%%%%%%%
\subsubsection{Figures, Subfigures}

\subsection{Chart Type Examples}

\subsubsection{Line}

\subsubsection{Scatter}

\subsubsection{Bar}

\subsubsection{Histograms}

\subsubsection{Pie}

\subsection{Customization}

\subsubsection{Colors}

\subsubsection{Markers}

\subsubsection{Ticks}

\subsubsection{Labels}

\subsubsection{Legends}

\subsubsection{Annotations}

\subsection{Saving to File}

\section{D3}

\textcolor{blue}{Overview of D3}

%%%%%%%%%%%%%%%%%%%%%%%% Analyzing
\chapter{Predicting Data}

\section{Scikit-Learn}

\subsection{Overview}

\textcolor{blue}{fit vs evaluate}

\textcolor{blue}{Optimizations -- justing the `njobs' parameter which allows the model to be parallelized onto multiple cores of the computer}


\subsection{Transformation Pipelines}

\subsection{Training}

\subsubsection{Cross-Validation}

\subsection{Fine-Tuning}

\subsubsection{Hyper-Parameter Optimization}

\paragraph{Grid Search}

\paragraph{Randomized Search}

\textcolor{blue}{Disucessed in ref to other setion}

%%%%%%%%%%%%%%%%%%%%%%%% Data Provanece & Reproducibility
\chapter{Data Provenance and Reproducibility}

\input{./nested/pachyderm}


%%%%%%%%%%%%%%%%%%%%%%%% Others
\chapter{Others}

\section{Regular Expressions}

\textcolor{blue}{brief overview and examples}

\input{./nested/tangent}

\section{Markdown}

\textcolor{blue}{Why is markdown included in a book about AI/ML/DL? Documentation is everything.}

\section{Other Libraries}

\r{clustering algorithms --- fastcluster, hdbscan, tslearn}


\part{TensorFlow}

% chapter
% TODO: I'm still not sure how/where to structure this

\section{Dense}

\TD{TODO}

\section{Convolutions}

\TD{TODO}

\section{Pooling}

\TD{TODO}

\section{Recurrent Cells}

\TD{TODO}

\section{Capsule Networks}

\TD{TODO}

\section{Attention}

\r{overview can be found here\cite{weng2018attention}}

\TD{The original attention mechanism is introduced\cite{Bahdanau2015NeuralMT}.}

% TF attention implementation (https://www.tensorflow.org/tutorials/text/nmt_with_attention)

\TD{Effective Approaches to Attention-based Neural Machine Translation \cite{DBLP:journals/corr/LuongPM15}}

\TD{Massive Exploration of Neural Machine Translation Architectures \cite{DBLP:journals/corr/BritzGLL17}}

% TODO: index for transformer
\TD{Attention Is All You Need -- Transformer network --- multi-head self-attention mechanism, key-value pairs \cite{DBLP:journals/corr/VaswaniSPUJGKP17}}

% self-attention \TD{Self-attention, less commonly intra-attention}
\TD{Long Short-Term Memory-Networks for Machine Reading \cite{DBLP:journals/corr/ChengDL16}}


%\TD{Nice table comparing mechanisms https://lilianweng.github.io/lil-log/2018/06/24/attention-attention.html}

\TD{in above post\cite{weng2018attention}: soft vs hard attention and global vs local attention}


% TODO: read https://lilianweng.github.io/lil-log/2020/04/07/the-transformer-family.html


\section{NTM (Neural Turing Machines)}
% nerual network + external memory storage
% controller + memory
\TD{Neural Turing Machines \cite{DBLP:journals/corr/GravesWD14}}


\chapter{Tensorflow API and Components}
% TODO: this entire chapter will likey be dropped -- the TF docs are exceptional now

\section{Datasets}

%%%%%%%%%%%%%%%%%%%%%%%% Bias
\input{./nested/basics/bias}


%%%%%%%%%%%%%%%%%%%%%%%% Acquiring Data
\input{./nested/basics/acquiring_data}

%%%%%%%%%%%%%%%%%%%%%%%% Data types
\input{./nested/basics/data_types}

%%%%%%%%%%%%%%%%%%%%%%%% Data Pre-processing
\input{./nested/basics/data_preprocessing}

%%%%%%%%%%%%%%%%%%%%%%%% Data Type Considerations + Feature Extraction
% TODO: this section is rough.. and contains overlap on feat engineering in previous section
\input{./nested/basics/data_type_considerations}

%%%%%%%%%%%%%%%%%%%%%%%% Feature selection
\input{./nested/basics/feature_selection}


%%%%%%%%%%%%%%%%%%%%%%%% Data sampling and partitioning
\input{./nested/basics/sampling_partitioning}

\subsection{Transform}

\subsubsection{Overview}

\textcolor{blue}{TensorFlow Transform is a library used for preprocessing data with TensorFlow.}

\textcolor{blue}{MOTIVE: calculating values (such as $\mu$ and $\sigma$) for an entire dataset can be challenging for large datasets.}

\textcolor{blue}{Though preprocessing can already be accomplished with standard python, numpy, other libraries, or even in TensorFlow, tf.Transform extends these capabilities to support full passes over the dataset.}

\subsubsection{Installation}

\textcolor{green}{TODO: include snippet for installation}

\subsubsection{Implementation}

\textcolor{green}{TODO: include use examples}
% https://github.com/tensorflow/transform/blob/master/getting_started.md


% resources
% 1. https://github.com/tensorflow/transform
% 2. https://github.com/tensorflow/transform/blob/master/getting_started.md


\section{Building Architectures}

\subsection{Layers}

\subsubsection{Dense / Fully Connected}

\textcolor{blue}{TODO: \textcolor{red}{local ref to definition/explanation/example in earlier sec}}

\textcolor{blue}{$output = activation_fn(dot(weights, input) + bias)$}

\subsubsection{Convolution}

\textcolor{blue}{TODO: \textcolor{red}{local ref to definition/explanation/example in earlier sec}}

\subsubsection{Pooling}

\textcolor{blue}{TODO: \textcolor{red}{local ref to definition/explanation/example in earlier sec}}

\paragraph{Max}

\paragraph{Average}

\subsection{Estimators}

% TODO: entire section

% boilerplate code
% graph and session management
%\textcolor{blue}{}
\subsubsection{Input data}

% TODO: input function overview

\paragraph{Specifying Hyper Parameters}

\subparagraph{Epochs}
\textcolor{blue}{by default, training will continue until the training data is exhausted, or the number of specified epochs is reached}

Options:

%TODO: code example
\begin{enumerate}
	\item input\_fn
	\item steps
	\item max\_steps --- will potentially do nothing if the checkpoint has already reached this value
\end{enumerate}


\paragraph{In Memory Data}

\textcolor{blue}{Usually this is in the form of either numpy arrays or pandas dataframes. These can both be used directly.}

% TODO: examples for Numpy array

% TODO: examples for Pandas DF

\paragraph{Out of Memory Data}

\textcolor{blue}{In the ``real world'' the dataset will likely not fit into memory. To (sanely) address this, estimators play nicely with the tf.Data API (please see \textcolor{red}{local ref} for more information)}

% TODO: show quick demo example

% tf.estimator base class allows you to build your own model 

% premade models (TODO: Show quick list)

% main advantage -- estimators are interchangable

% "reasonable" defaults for each estimator

\subsubsection{Checkpoints}

\textcolor{blue}{directory specified when creating model. By default, predictions will be made from the latest checkpoints in this directory. training also resumes from the latest checkpoint in the directory. --- to start from scratch, the directory will need to be deleted or specified to a new location.}

\subsubsection{Distributed}

% you need: 1. estimator, 2. run config, 3. train spec, eval spec

% final call tf.estimator.train_and_evaluate(estimator, train_spec, eval_spec)

\paragraph{tf.estimator.RunConfig}

% TODO: example

\textcolor{blue}{the directory for checkpoints and Tensorboard logs and freq of checkpoints (save\_checkpoint\_steps) and frequency of logs (save\_summary\_steps)  }

\paragraph{tf.estimator.TrainSpec}

\textcolor{blue}{pass in input function (likely through data API (see \textcolor{red}{localref})), }


\paragraph{tf.estimator.EvalSpec}

\textcolor{blue}{pass in input function for evaluation dataset (likely through data API (see \textcolor{red}{local ref})),}

\textcolor{blue}{creates model and loads latest checkpoint, then runs eval. Therefore, you cannot get a frequency greater than the checkpoints created. They can be obtained less frequently by using the `throttle\_specs` parameter}

\paragraph{Notes}

\textcolor{blue}{Shuffling considerations.use the dataset = tf.data.Dataset().list\_files().shuffle() command --- each worker a different seed? Even if the data is shuffled on disk. Can also use the dataset().shuffle()}

\subsubsection{TensorBoard}

% TODO: example -- to create: tf.estimator.RunConfig(model_dir='some_dif')

\textcolor{blue}{to visualize, then open tensorboard by issuing `tensorboard --logdir output\_dir` and the dashboard will appear on localhost:6006}

\textcolor{blue}{pre-made estimators already export relevant metrics, embeddings, histograms, etc.. for more information on how to use tensorboard please see \textcolor{red}{local ref}}

\textcolor{blue}{if building a custom estimator or would like to add additional information to tensorboard, summaries can be added with any of the following: tf.summary.scalar, tf.summary.image, tf.summary.audio, tf.summary.text, tf.summary.histogram}

\paragraph{Adding Custom}

% tf.summary.scalar('meanVar_01', tf.reduce_mean(var_01))

\subparagraph{tf.summary.scalar}

\subparagraph{tf.summary.image}

\subparagraph{tf.summary.audio}

\subparagraph{tf.summary.text}

\subparagraph{tf.summary.histogram}


\subsubsection{Deployment}

% TODO: examples

%\textcolor{blue}{two things 1) export\_latest = tf.estimator.LatestExporter(serving\_input\_receiver\_fn=serving\_input\_fn) and then eval_spec = tf.estimator.EvalSpec(input\_fn=eval\_input\_fn, exporters=export\_latest)}

% will map from JSON from REST API and the model

% important to use tf commands in the input transformation/parsing function

\paragraph{Exporters}

\textcolor{blue}{there are many types of exporters and exporter schemes. The simplest may be the tf.estimator.LatestExporter}




\input{tensorflow/api_comp/eager}


\section{Design and Component Considerations}

\subsection{Initialization Strategies}

\textcolor{blue}{Discuss different initialization strategies and their importance}

\textcolor{blue}{truncated normal -- truncated Gaussian distribution}

\subsection{Hyperparameters}

\subsubsection{Training Related}

\paragraph{Learning Rate}

\subparagraph{Too High vs Too Low}

\textcolor{blue}{TODO: figure showing a convex cost function and the result of a learning rate being too high (overshoot, diverge) and too small (local minima)}

\paragraph{Batch Size}

\paragraph{Number of Training Iterations}

\paragraph{Momentum}

\paragraph{Weight Update}

\textcolor{red}{SGD, CG, L-BFGS, more complex more hyper-parameters}

\paragraph{Stopping Criteria}

\subsubsection{Model Related}

\paragraph{Architecture}

\paragraph{Weight Initialization}

\paragraph{Weight-decay}

L1

L2

\paragraph{Drop-out}



\subsection{Hyper-parameter optimization}

\textcolor{blue}{OVERVIEW}

\subsubsection{Coordinate Descent}

All hyper-parameters remain fixed, except for the hyper-parameter of interest. The hyper-parameter of interest is then adjusted such that the validation error is minimized.

\subsubsection{Grid Search}

\textcolor{green}{TODO: grid search explanation}

\subsubsection{Random Search}

\textcolor{green}{TODO: random search explanation}

\subsubsection{Grid vs Random Search}

\textcolor{green}{TODO: grid vs random search figure}

\subsubsection{Automated / Model-based Methods}

\section{Estimating Model Parameters}

\subsection{Optimizers}

\textcolor{blue}{Estimate the values of the model's parameters that minimize the value of the cost function}

\subsubsection{Gradient Descent}

\textcolor{blue}{Gradient Descent --- overview --- optimization algorithm that can be used to estimate the local minimum of a function}

\textcolor{blue}{Iteratively updates the model parameters by calculating the partial derivatives of the cost function at each step during training}

\textcolor{blue}{Gradient descent is only guaranteed to find the local minimum of the cost function.}


\paragraph{Batch Gradient Descent}

\textcolor{blue}{batch gradient descent --- taking a step (update the weights) opposite (down) the gradient calculated from the entire training set}

\textcolor{blue}{Batch gradient descent is deterministic --- will produce the same paramter values if the same dataset is used multiple times.}


\paragraph{Stochastic Gradient Descent}

\textcolor{blue}{Stochastic Gradient Descent (sometimes called iterative or on-line gradient descent) --- rather than update the weights based on the sum of the accumulated errors, the weights are updated for each training sample}

\textcolor{blue}{Stochastic gradient descent is deterministic --- may produce the different parameter values if the same dataset is used multiple times. May not minimize the cost function as well as gradient descent but the approximation is often ``close enough''.}


\paragraph{Mini-batch Gradient Descent}

\textcolor{blue}{mini-batch gradient descent --- compromise between batch and stochastic gradient descent where the gradient is calculated over a batch of training data}

\textcolor{blue}{Since the gradient is calculated on a single example, the error surface will appear noisier than if it was calculated over a batch or the entire training set.}

\textcolor{blue}{When using stochastic gradient descent, it is important to shuffle the data after each epoch.}

%% techniques
\chapter{Augmentation Techniques}
\label{app_aug_techniques}

\r{Including Imagery}

\r{flip, rotate. color/channel manipulation}

\r{mixup\cite{zhang2017mixup}}

\r{cutout\cite{devries2017improved}}

\r{cutmix\cite{yun2019cutmix}}

\TD{language -- back translation\cite{sennrich2015improving}}

% TODO: blog about this: https://ai.googleblog.com/2019/07/advancing-semi-supervised-learning-with.html
\TD{unsupervised augmentation\cite{xie2019unsupervised}}

%% learning augmentation

% TODO: not sure this belongs here
\r{Sample Pairing\cite{inoue2018data}}

\r{Smart Augmentation\cite{lemley2017smart}}

\r{GAN\cite{shrivastava2017learning}}

\r{population based augmentation (PBA)\cite{ho2019population}}

\r{Bayesian data augmentation\cite{tran2017bayesian}}

\TD{distortions, patches, jigsaw, color}

\TD{AugMix: A Simple Data Processing Method to Improve Robustness and Uncertainty \cite{Hendrycks2020AugMixAS}}

% image augmentation
\TD{Attacks Which Do Not Kill Training Make Adversarial Learning Stronger \cite{DBLP:journals/corr/abs-2002-11242}}

\TD{DuBIN --- AugMax: Adversarial Composition of Random Augmentations for Robust Training \cite{Wang2021AugMaxAC}}

% similar to dropout?
\TD{Random Erasing Data Augmentation \cite{DBLP:journals/corr/abs-1708-04896}}

%% autoaugment

\r{AutoAugment\cite{cubuk2018autoaugment}}

\r{Comment that AutoAugment can be applied directly to a dataset as well as transfer the learned policies to new datasets.}

\TD{figure of loop. controller, strategy, child network, update controller}

\r{Fast AutoAugment\cite{lim2019fast} improves upon the original search strategy in the original AutoAugment paper.}

\r{Unsupervised Augmentation}

\TD{autoaugment for object detection \cite{zoph2019learning}}

\section{Data Imbalance or Unbalanced Data}
\label{app_data_imbalance}

% TODO: special case of augmentation??

\TD{A Survey of Predictive Modelling under Imbalanced Distributions \cite{DBLP:journals/corr/BrancoTR15}}

\TD{\cite{krawczyk2016learning}}

\r{An ``imbalance in the data'' may have many meanings. That is, the labels could be imbalanced (e.g. in a binary classifier, there may be n times the number of instances with the label p when compared to the label q), the features may be imbalanced, (e.g. facial recognition is being performed on collected images that are composed of overwhemingly white, male, brown hair, clean shaven, hazel eye individuals), or it may mean a combination of the above.}

\r{this poses a problem, as typically the optimization process treats all samples individually and equally, which may (often does) pose problems when creating predictions on imbalanced data}

\r{Two main high level approaches to addressing this issue. You can either modify the (or utilize a combination of the listed):}
\begin{itemize}[noitemsep,topsep=0pt]
	\item data
	\item model
	\item post-processing
\end{itemize}

\r{data modification approaches aim to create a quisi-balanced representation, often through some re-sampling scheme. Simple example for would be to down-sampling the overrepresented class and upsampling the underrepresented class -- where class here might mean target variable or feature attribute.}

\r{when discussing class imbalance in reference to a continuous variable, the term skewed is often used to describe the data, whereas unbalanced or imbalanced is used for discrete variables}

\TD{relevance function --- }

\subsection{Methods}

\TD{paper to read: \TD{Self-paced Ensemble for Highly Imbalanced Massive Data Classification \cite{Liu2020SelfpacedEF}}}

\subsubsection{Data}

\r{as mentioned, under/oversampling scheme}

\TD{Method -- SMOGN}

\TD{Method --- SMOTE (\textbf{S}ynthetic \textbf{M}inority \textbf{O}versampling \textbf{T}echnique)}

\TD{method --- Tomek Links, select pairs of examples that are of opposite class, near one another. \TD{Figure}}

\TD{``shows that outcome imbalance is not a problem in itself, and that imbalance correction may even worsen model performance'' -- The harm of class imbalance corrections for risk prediction models: illustration and simulation using logistic regression~\cite{Goorbergh2022TheHO}}

\subsubsection{Model}

\TD{Utility-based Regression --- penalty based on \TD{relevance function}}

\paragraph{Losses}

% TODO: does this belong here?
% generalized loss function
\TD{Cyclical Focal Loss~\cite{Smith2022CyclicalFL}}
\TD{Asymmetric Loss For Multi-Label Classification~\cite{DBLP:journals/corr/abs-2009-14119}}

\subsubsection{Post processing}

% TODO: is this where calibration belongs??
\paragraph{Calibration}

%TODO: need to get a grasp on this...

% original
\TD{On Calibration of Modern Neural Networks \cite{DBLP:journals/corr/GuoPSW17}}

% recent
\TD{Revisiting the Calibration of Modern Neural Networks~\cite{DBLP:journals/corr/abs-2106-07998}}

% other, relevant
\TD{On the Dark Side of Calibration for Modern Neural Networks~\cite{Singh2021OnTD}}

% other, popular
\TD{Simple and Scalable Predictive Uncertainty Estimation using Deep Ensembles~\cite{Lakshminarayanan2017SimpleAS}}



\subsection{Evaluation}

\r{Difficult to discuss class imbalance without discussing the importance of having appropriate metrics in place to evaluate the methods. \TD{point to example of disease test where 1 out of N are positive --- acc is very high, yet...}}

\TD{point to metrics section and specific metrics that may be useful for various scenarios}


\section{Activation Functions}

\textcolor{blue}{Activation functions are XXXXXXXX}

\subsection{Why Non-linear}

\textcolor{blue}{Non-linear is necessary XXXXXXXXXX}

\subsection{Advancements}

\textcolor{green}{TODO: From step function to ?selu}

\subsection{Popular Activation Functions}

\textcolor{blue}{Activation functions can be grouped into two main categories -- smooth and not smooth. Smooth activation functions (such as sigmoid) are differentiable at every point along the function where as the other activation functions are not differentiable at every location (relu).}

\subsubsection{Smooth Non-linear}

\textcolor{blue}{The sigmoid\index{sigmoid} activation function}

% {{{act_smooth_sigmoid}}}
\begin{figure}
\centering
\includegraphics[width=0.65\textwidth]{./sync_imgs/act/smooth/sigmoid.png}
\label{fig:act_smooth_sigmoid}
\end{figure}

% {{{act_smooth_tangent}}}
\begin{figure}
\centering
\includegraphics[width=0.65\textwidth]{./sync_imgs/act/smooth/tangent.png}
\label{fig:act_smooth_tangent}
\end{figure}

% {{{act_smooth_elu}}}
\begin{figure}
\centering
\includegraphics[width=0.65\textwidth]{./sync_imgs/act/smooth/elu.png}
\label{fig:act_smooth_elu}
\end{figure}

% {{{act_smooth_selu}}}
\begin{figure}
\centering
\includegraphics[width=0.65\textwidth]{./sync_imgs/act/smooth/selu.png}
\label{fig:act_smooth_selu}
\end{figure}

% {{{act_smooth_softplus}}}
\begin{figure}
\centering
\includegraphics[width=0.65\textwidth]{./sync_imgs/act/smooth/softplus.png}
\label{fig:act_smooth_softplus}
\end{figure}

% {{{act_smooth_softsign}}}
\begin{figure}
\centering
\includegraphics[width=0.65\textwidth]{./sync_imgs/act/smooth/softsign.png}
\label{fig:act_smooth_softsign}
\end{figure}


\subsubsection{Not Smooth Non-linear}

% {{{act_notsmooth_relu}}}
\begin{figure}
\centering
\includegraphics[width=0.65\textwidth]{./sync_imgs/act/notsmooth/relu.png}
\label{fig:act_notsmooth_relu}
\end{figure}

% {{{act_notsmooth_leakyrelu}}}
\begin{figure}
\centering
\includegraphics[width=0.65\textwidth]{./sync_imgs/act/notsmooth/leakyrelu.png}
\label{fig:act_notsmooth_leakyrelu}
\end{figure}

% {{{act_notsmooth_relu6}}}
\begin{figure}
\centering
\includegraphics[width=0.65\textwidth]{./sync_imgs/act/notsmooth/relu6.png}
\label{fig:act_notsmooth_relu6}
\end{figure}

% {{{act_notsmooth_prelu}}}
\begin{figure}
\centering
\includegraphics[width=0.65\textwidth]{./sync_imgs/act/notsmooth/prelu.png}
\label{fig:act_notsmooth_prelu}
\end{figure}




\section{Fine-Tuning Architectures}

\input{tensorflow/api_comp/profiler}

\input{tensorflow/api_comp/debugger}

\section{Distributed Training}

\chapter{Distributed Methods}

\textcolor{blue}{Achieving distributed training in two main ways, and \textcolor{red}{federated}. Implementation details are discussed in \textcolor{red}{local ref}}

\textcolor{blue}{note about synthetic gradients \textcolor{red}{local ref}}

%  model is replicated and placed on multiple workers

\subsubsection{Data Parallelism}

\textcolor{blue}{data parallelism. \textcolor{green}{TODO: figure}}

\subsubsection{Model Parallelism}

\textcolor{blue}{Model parallelism. \textcolor{green}{TODO: figure}}

\subsubsection{Federated learning}

\textcolor{blue}{Federated learning -- consensus change}



\section{Model Evaluation}

\textcolor{green}{TODO: para about using metrics.}

\textcolor{green}{Para about using tensorboard during training and tfma after training}

\textcolor{blue}{Metrics computed during training (e.g. training and validation metrics) can be visualized in tensorboard. Tensorboard displays, and continuously updates during training, these metrics graphically against global training steps (or time) and is used to determine how well your model is being trained. TFMA computes and visualizes metrics from the final (presumably after training) model. These metrics are computed only once. \textcolor{red}{TFMA exports and computes the metrics once on a saved model which contains the eval graph and additional metadata}}

\textcolor{blue}{Where tensorboard will compare models to each other over time, TFMA will compare models at only one point in time. This is better displayed in \textcolor{red}{see figure xx}.}

\textcolor{green}{TODO: figure showing the difference between graphs from tensorboard and TFMA}

\emph{Cost} is frequently used interchangeably with loss. Technically, loss refers to the error on a single example and cost is the average of the loss across the entire training set.

% page 94 of AGtext
One-versus-all \emph{OvA} (also \emph{one-versus-rest})

One-versus-one (OvO) -- train a binary classifier for every pair


\section{Metrics}

%% Confusion matrix
\begin{table}
	\centering
	\begin{tabular}{l|l|c|c|}
		\multicolumn{2}{c}{}&\multicolumn{2}{c}{Ground Truth}\\ 
		\cline{3-4}
		\multicolumn{2}{c|}{}&Positive&Negative\\ 
		\cline{2-4}
		\multirow{2}{*}{\rotatebox{90}{Pred}}& Positive & $TP$ & $FP$ \\ 
		\cline{2-4}
		& Negative & $FN$ & $TN$ \\ 
		\cline{2-4}
	\end{tabular}
	\caption{Example confusion matrix}
	\label{tab:sample_conf_matrix}
\end{table}


\begin{itemize}
	
\item \textit{Accuracy}, (Eq.~\ref{eq:accuracy}): the ratio of correct predictions to the total number of predictions.

\begin{equation}
{\frac{TP+TN}{TP+TN+FP+FN}}
\label{eq:accuracy}
\end{equation}

\item \textit{Sensitivity}, (Eq.~\ref{eq:sensitivity}): the ratio of true positives that are correctly identified.

\begin{equation}
{\frac{TP}{TP+FN}}
\label{eq:sensitivity}
\end{equation}

\item \textit{Precision}, (Eq.~\ref{eq:precision}): the ratio of positives that are, in fact, positive. If the classifier predicts positive, how often is is correct?

\begin{equation}
{\frac{TP}{TP+FP}}
\label{eq:precision}
\end{equation}

\item \textit{AUC (Area Under the Curve)}, is a single value representing the area under an ROC curve. Though generally referred to as the AUC, the term is correctly abbreviated AUROC, specifying that the curve is an ROC curve.
\end{itemize}

\input{tensorflow/api_comp/tensorboard}

\subsection{TFMA: TensorFlow Model Analysis}

\subsubsection{Key Features}

\paragraph{Sliced Metrics}

\textcolor{blue}{Typically metrics are aggregated from the entire test dataset. TFMA allows for examining specific slices from the dataset which may indicate that certain instances or groups of instances aren't predicted as well as others.}

\textcolor{green}{TODO: figure showing how this might look}

\paragraph{Full-pass Metrics}

\textcolor{blue}{Metrics that are typically computed and visualized in tensorflow (such as on tensorboard) are actually approximations computed on mini-batches and known as ``streaming metrics''.}

\textcolor{blue}{TFMA performs a ``full pass'' on the evaluation dataset, leveraging \textcolor{yellow}{Apache Beam}. There are two main advantages to this approach: i) \textcolor{red}{more accurate calculation} and ii) scaling up to massive evaluation datasets.}

\subsubsection{Implementation}

\paragraph{Overview}

\textcolor{blue}{An evaluation graph from the trainer needs to be exported and saved as a `SavedModel'. TFMA will then use this saved model to compute metrics and provide visualization tools to analyze the metrics.}

\paragraph{Code}

\textcolor{blue}{The overall process is outlined below and assumes a trained model and test set already exist. At a high level, the steps are i) Exporting the Evaluation Graph ii) Computing Metrics and iii) Visualize Metrics (in a notebook). }

\subparagraph{Exporting the Evaluation Graph}

\textcolor{green}{TODO: code sample}

\subparagraph{Computing Metrics}

\textcolor{green}{TODO: code sample}

\subparagraph{Visualize Metrics (in a notebook)}

\textcolor{green}{TODO: code sample}


% Resources:
% 1. https://medium.com/tensorflow/introducing-tensorflow-model-analysis-scaleable-sliced-and-full-pass-metrics-5cde7baf0b7b


\section{Model Persistence}

\subsection{Saver}

%%%% may belong in serving, may belong in Saver -- then ref other?
\textcolor{blue}{SavedModel is the universal serialization format for TensorFlow models. SavedModel has support for multiple metagraphs -- this is important in serving where the model is slightly different than in training (removing dropout layers), and allows storage/access of models with tags. Supports SignatureDefs -- allows the specification of nodes as input/output (also supports multiple signatureDefs for multi-headed inference (see \textcolor{red}{local ref}))}

\input{tensorflow/api_comp/hub}


\section{Deployment}

\section{Deployment with TF Serving}

\subsection{Overview}
% https://github.com/tensorflow/serving/blob/master/tensorflow_serving/g3doc/architecture_overview.md
% https://www.youtube.com/watch?v=q_IkJcPyNl0

\subsubsection{Standard Abstractions}

\textcolor{blue}{Core components with APIs}

\paragraph{ServableHandle}

\paragraph{Manager}

\textcolor{blue}{Uses the Loader to load and unload a model}

\paragraph{Loader}

\textcolor{blue}{Loader for a TensorFlow <saved\_model>}

\textcolor{red}{the loader knows how to load a model and knows how to estimate the resources such as RAM or GPU}

\textcolor{red}{The loader signals to the manager that has an `aspired version' (ready for loading). The manager will then decide when to load this new version based on the version policy plugin (see below).}

\paragraph{Source}


\subsubsection{Plugins into Abstractions}


\paragraph{File System (Source)}

\textcolor{blue}{Monitors the file system.}


\paragraph{Version Policy (Manager)}

\textcolor{blue}{Preserve availability or preserve resources. Preserving availability might be more important in a live, user facing scenario and will keep keep one model loaded, load another beside, then point client to new model -- There's no down time, but more resources are consumed (two models loaded at once).  Preserving resources might be important if using a large model on a resource constrained environment or in an internal application where some downtime is considered acceptable. Under a policy preserving resources, the current model will be unloaded and a new model will be loaded -- cost is a slight hiccup in service, but the benefit is a saving on resources (like memory) since there's only the one model loaded at a time.}

\textcolor{red}{When loading a new model, the original/old model can't be immediately unloaded/deleted in case there are pending/queued jobs. TensorFlow keeps track of these jobs via ref-counting, and only then removes that model once all the jobs have completed}


\subsubsection{FIT}

\paragraph{ServerCore}

\textcolor{blue}{declare set of models to be loaded, pass them to ServerCore, and server core returns a manager of these models with the best practices out of the box.}


\paragraph{Binaries and APIs}

\subparagraph{Predict}

% coming soon?
%\subparagraph{Regress}

%\subparagraph{Classify}

%\subparagraph{MultiInference}

\paragraph{Servables}

\textcolor{blue}{The central abstraction to TensorFlow Serving -- the underlying object that will be used for inference.}

\textcolor{blue}{Servables may be big and complex (composite inference model) to small and simple (lookup table). }

\textcolor{red}{A composite model can be represented as either multiple independent servables or a single composite servable.}

\paragraph{Loaders}

\textcolor{blue}{Statdarize the API for loading and unloading a servable i.e. manage the servable's lifecycle.}

%\paragraph{Sources}
%\textcolor{blue}{}




\subsection{Example}


\section{Other}

\input{tensorflow/api_comp/probability}

\subsection{TensorFlow Extended}

\subsection{Keras}

\subsection{Image}

\subsection{Image Augmentation}

\subsection{Edward 2.0}

\subsection{TensorFlow Lite}


% chapter
\chapter{Implementing Common Methodologies}

\section{Artificial Neural Networks}

\section{Convolutional Neural Networks}

\section{Recurrent Neural Networks}










\part{End To End Examples}

\chapter{EndToEnd}

\section{Structured}

\subsection{Linear Regression}

\subsection{KNN}

\section{Image}

\subsection{Image Classification}

\subsection{Image Segmentation}

\subsection{Adversarial Exmaples}

\subsection{AutoEncoder}


\section{TimeSeries}

\section{Text}

\subsection{Sentiment Analysis}

\section{Audio}

\subsection{Audio to Text}

\section{Recommendation Systems}

\TD{todo:}



\part{Moving Forward}

\chapter{Putting it All Together}


\part{Dump Space}


% This section is a container for ``random" bits and pices that will be fit into other sections


\textcolor{blue}{tensorflow's XLA compiler}

\textcolor{blue}{checkpoints -- allow: save/stop/resume training, resume on failure, predict from point}


%%%% may talk about in production
% Google Cloud MLE


%%%
% stats vs ML -- in ML you may keep outliers and build models for them. in ML outliers may be collapsed (capped) and in statistics they may be removed
% ML is used to learn the ``long tail'', make fine grained predictions, not just gobal averages

%% why try to stay in linear (like with feature crosses)
% NN with many layers are non-convex
% optimizing linear models is a convex problem (much easier)

% how have I not talked about transfer learning yet?

% optimizing is an NP-hard, non-convex optimization problem (coursera.need to double check)
\textcolor{blue}{L0-norm (the count of non-zero weights).}


%%%%


%%%
\textcolor{blue}{Each layer in a DNN is a composition of the previous layer. i.e. if layer 1 = f(x), then layer 1 = g(f(x)), layer three is h(g(f(x)) \textcolor{green}{TODO: create diagram}}



%%%%
\textcolor{blue}{SavedModel is the universal serializion format for TensorFlow models. SavedModel has support for multiple metagraphs -- this is important in serving where the model is slightly different than in training (removing dropout layers), and allows storage/access of models with tags. Supports SignatureDefs -- allows the specification of nodes as input/output (also supports multiple signatureDefs for multi-headed inference (see \textcolor{red}{local ref}))}


%%%% rough...
%\textcolor{red}{Multi-headed inference. Useful when using the same model for different tasks. e.g. using a model for one task, then later deciding to perform a similar task on the same data -- rather than train an entirely new model, the original model may be performing may of the same computations. }


%%%%%%
\textcolor{blue}{Pooling may not fully determine learned deformation stability -- possibly filter smoothness\cite{ruderman2018learned}}

%%%% % plus index
\textcolor{blue}{manifold learning}

%%%  preprocessing, this wasn't already somewhere? % plus index
\textcolor{blue}{whitening}

%%%
\textcolor{blue}{NFM (Non-Negative Matrix Factorization)}

%%%
\textcolor{blue}{TODO: parameter calculation -- use VGG example (conv + dense)}
\begin{table}
	\centering
	\begin{tabular}{|c|c|c|c|c|}
		\hline
		\multicolumn{5}{|c|}{\textbf{Number of VGG-16 Parameters}}     \\ \hline
		Layer & Out Shape & Weights & Bias & Total  \\ \hline
		\emph{Convolution}        & & $(in)\times(h\times w)\times(out)$ & $(out)$ & $weights+bias$    \\ \hline
		Conv3-64          & $224\times224\times64$ & $3\times(3\times3)\times64$ & $64$ & 1792    \\ \hline
		Conv3-64 (p)      & $112\times112\times64$ & $64\times(3\times3)\times64$ & $64$ & 36928    \\ \hline
		Conv3-128         & $112\times112\times128$ & $64\times(3\times3)\times128$ & $128$ & 73856     \\ \hline
		Conv3-128 (p)     & $56\times 56\times 128$ & $128\times(3\times3)\times128$ & $128$ & 147584   \\ \hline
		Conv3-256         & $56\times 56\times 256$ & $128\times(3\times3)\times256$ & $256$ & 295168   \\ \hline
		Conv3-256         & $56\times 56\times 256$ & $256\times(3\times3)\times256$ & $256$ & 590080   \\ \hline
		Conv3-256 (p)     & $28\times 28\times 256$ & $256\times(3\times3)\times256$ & $256$ & 590080   \\ \hline
		Conv3-512         & $28\times 28\times 512$ & $256\times(3\times3)\times512$ & $512$ & 1180160  \\ \hline
		Conv3-512         & $28\times 28\times 512$ & $512\times(3\times3)\times512$ & $512$ & 2359808  \\ \hline
		Conv3-512 (p)     & $14\times 14\times 512$ & $512\times(3\times3)\times512$ & $512$ & 2359808  \\ \hline
		Conv3-512         & $14\times 14\times 512$ & $512\times(3\times3)\times512$ & $512$ & 2359808  \\ \hline
		Conv3-512         & $14\times 14\times 512$ & $512\times(3\times3)\times512$ & $512$ & 2359808  \\ \hline
		Conv3-512 (p)     & $7\times 7\times 512$ & $512\times(3\times3)\times512$ & $512$ & 2359808    \\ \hline
		\emph{dense}         & & $(in)\times(num)$ & $(out)$ & $weights+bias$    \\ \hline
		fc1 (4096)        & 4096 &$(512\times7\times7)\times4096$ & $4096$ & $102764544$    \\ \hline
		fc2 (4096)        & 4096 &$(4096)\times4096$ & $4096$ & $16781312$    \\ \hline
		fc3 (1000)        & 1000 &$(4096)\times1000$ & $1000$ & $4097000$    \\ \hline
		\emph{Total} & & & & 138,357,544 \\ \hline
	\end{tabular}
	\caption{Calculation of VGG parameters. (p) denotes that the layer is followed by a pooling layer (which does not affect the parameter count)}
	\label{tab:vgg_parameter_count}
\end{table}




%%%
\textcolor{blue}{NN from scratch in appendix}

%%%
\textcolor{blue}{CNN without layers API -- in github}






\part{Research}


% TODO: I'm still not sure how/where to structure this

\section{Dense}

\TD{TODO}

\section{Convolutions}

\TD{TODO}

\section{Pooling}

\TD{TODO}

\section{Recurrent Cells}

\TD{TODO}

\section{Capsule Networks}

\TD{TODO}

\section{Attention}

\r{overview can be found here\cite{weng2018attention}}

\TD{The original attention mechanism is introduced\cite{Bahdanau2015NeuralMT}.}

% TF attention implementation (https://www.tensorflow.org/tutorials/text/nmt_with_attention)

\TD{Effective Approaches to Attention-based Neural Machine Translation \cite{DBLP:journals/corr/LuongPM15}}

\TD{Massive Exploration of Neural Machine Translation Architectures \cite{DBLP:journals/corr/BritzGLL17}}

% TODO: index for transformer
\TD{Attention Is All You Need -- Transformer network --- multi-head self-attention mechanism, key-value pairs \cite{DBLP:journals/corr/VaswaniSPUJGKP17}}

% self-attention \TD{Self-attention, less commonly intra-attention}
\TD{Long Short-Term Memory-Networks for Machine Reading \cite{DBLP:journals/corr/ChengDL16}}


%\TD{Nice table comparing mechanisms https://lilianweng.github.io/lil-log/2018/06/24/attention-attention.html}

\TD{in above post\cite{weng2018attention}: soft vs hard attention and global vs local attention}


% TODO: read https://lilianweng.github.io/lil-log/2020/04/07/the-transformer-family.html


\section{NTM (Neural Turing Machines)}
% nerual network + external memory storage
% controller + memory
\TD{Neural Turing Machines \cite{DBLP:journals/corr/GravesWD14}}



\backmatter 

% TODO: glossary

\printindex

%%%%%%%%%%%%%%%%%%%%%%%%%%%%%%%%%%%%%%%%%%% Bibliography

\bibliographystyle{unsrt}
\bibliography{TCF}


\end{document} 
