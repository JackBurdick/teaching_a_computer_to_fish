\documentclass[12pt]{book} 

% TODO: create an inline \code command for monospace font

% The document preamble 

\usepackage[superscript,biblabel]{cite}

\usepackage{makeidx}

\usepackage{subcaption} % 2x2 figure

\usepackage{amssymb} % math notation (R)

\usepackage{url}
\usepackage[dvipsnames]{xcolor} % needs to loaded before \usepackage{markdown}
% had to add --shell-escape to pdflex command
\usepackage{markdown}
%\usepackage[hashEnumerators,smartEllipses]{markdown}

\usepackage{times} % Use PS times fonts 

\usepackage{listings} % code examples
 % textcolor

\usepackage{multirow}
\usepackage{rotating}
\usepackage{titlesec} % for using paragraphs as a "subsubsubsection"

% for encoded images
%\usepackage{filecontents}
%\newcommand{\generateimage}[2]{%
%\immediate\write18{convert.cmd #1 > #2}}
\usepackage{graphicx}	% Use pdf, png, jpg, or eps§ with pdflatex; use eps in DVI mode

\usepackage{enumitem}

\makeindex 

% https://tex.stackexchange.com/questions/45711/defining-lstset-parameters-for-multiple-languages
% Copied from the listings documentation
\lstdefinestyle{numbers} {numbers=left, stepnumber=1, numberstyle=\tiny, numbersep=10pt}
\lstdefinestyle{pyinput}{backgroundcolor=\color{gray!8},frame=shadowbox,showspaces=false, basicstyle=\scriptsize }
\lstdefinestyle{pyoutput}{backgroundcolor=\color{green!8},frame=shadowbox,showspaces=false, basicstyle=\scriptsize }
\lstdefinestyle{terminal}{backgroundcolor=\color{gray!8},frame=shadowbox,showspaces=false, basicstyle=\scriptsize }

% TODO: need common code style
\def\code#1{\texttt{#1}}

% used to determine level of rough-drafted-ness, where r= most rough, rrr= presumably less rough (3 passes)
\def\r#1{\textcolor{SeaGreen}{#1}}
%SpringGreen
\def\rr#1{\textcolor{ProcessBlue}{#1}}
\def\rrr#1{\textcolor{NavyBlue}{#1}}
\def\TD#1{\textcolor{green}{#1}}

%% indexing w/it and w/o it
\def\IDI#1{{\textit{#1}}\index{#1}}
\def\ID#1{{#1}\index{#1}}

\def\ALR{\textcolor{red}{ADD LOCAL REF}}


% breaklines=true is used to wrap lines
\lstdefinestyle{pyOutStyle} {language=Bash,style=numbers,style=pyoutput,frame=none}
\lstdefinestyle{terminalBash} {language=Bash,style=numbers,style=terminal,frame=lines, breaklines=true}

% captions for equations
\usepackage{caption}
\DeclareCaptionType{equ}[][]

% for python snippets
\usepackage{pythonhighlight}

% for piecewise functions
\usepackage{amsmath}


% Details of the titlepage 
\title{Teaching a Computer to Fish} 
\author{Jack Burdick} 
\date{Last Update: \today} % Use the system date

% for using paragraphs as a "subsubsubsection"
\setcounter{secnumdepth}{4}
\setcounter{tocdepth}{4}
\titleformat{\paragraph}
{\normalfont\normalsize\bfseries}{\theparagraph}{1em}{}
\titlespacing*{\paragraph}
{0pt}{3.25ex plus 1ex minus .2ex}{1.5ex plus .2ex}

\begin{document} 
	
\frontmatter

\maketitle 

\tableofcontents

\mainmatter

\part{Background}

\chapter{Introduction}

But if you teach your computer to fish..


\section{Motivation for this text}
There are many great resources that exist.

I wanted to create the guide I wish I found.

\r{ML -- study of how programs learn from data. predictive analytics or statistical learning.}


\section{What}

\r{data and objective. ML tries to ``model'' the data as ``best'' as possible -- where the model may mean any number of things (I'll define the model as representing ``some function'' that accepts inputs and produces outputs throughout this text) and where ``best'' is user defined, typically as some quantitative value/measure. Typically, the goal is produce a model/function for data that is as generalizable as possible -- one that is capable of ``performing'' well, in the future based on some predefined objective on data the model has never seen before.}

% TODO: lot to unpack here -- maybe a highlighting scheme would be better than the quotes for user defined concepts

\r{as you may imagine, there are many other constraints that may be placed on this model. Maybe it needs to be able to ``infer'' a certain number of samples per unit time. Maybe the model must be able to execute in a resource constraint environment (where the constrained resource could be battery life, memory, etc.).}

% TODO: more


\subsection{Rule based vs ``learning''}


\subsection{High Level Overview}


\r{Arthur Samuel, essay ``Artificial Intelligence: A frontier of automation'' \cite{samuel1962artificial} -- ``ML is the study that gives computers the ability to learn without being explicitly programed.'' }

\r{Tom Mitchel -- ``A program can be said to learn from experience `E' with respect to some class of tasks `T' and performance measure `P', if its performance at tasks in `T', as measured by `P', improves with experience `E'.''}

\r{ML is everywhere.... everyday --- personalized music, show,book,product recommendations, automatic image tagging to specific tasks, detecting fraudulent credit card activity -- analyzing medical images (benign, or malignant). }

\TD{TODO: venn diagram, AI (knowledge bases -- logical inference rules (cyc))-> Machine Learning  (SVM, Logistic Regression, naive bayes) -> Representation Learning (...), Deep Learning (MLP, Deep CNN, RNN)}



\subsection{Deep Learning}

\r{Model depth --- depends on the definition of what is considered a computational step --- what level of detail is being considered. This may be either the length of the longest path through the computational graph, where each multiplication, addition, etc. are considered, or this may be described by how the concepts are related to one another, where a group of opperations may be grouped together for a single count (such as in a dense or convolutional layer). In general, we will not be overly concerned with depth or how it is described. It is only important to be aware that some individuals may have different definitions for ``deep'' than others.}

\subsubsection{Why Now?}

\r{Why is deep learning suddenly so popular?}

\r{Deep learning is nothing ``new''. It has been rebranded, i.e. gone by many different names \TD{cybernetics (1940s-1950s), connectionism (1980s-1990s), deep learning ($\approx 2006$) more prev+new examples} and its popularity has increased and decreased over time.}

\TD{``Deep Learning = new electricity'' - Ng}

\r{A few reasons that may be contributed to the recent surge in popularity may be the following:}

\begin{itemize}
	
	\item \textit{Data (a lot more of it)}: \r{increase in the collection and labeling of data} \TD{information about the increase in the collection of data}.
	
	\item \textit{Hardware (Computational Power/Price)}: \r{GPUs, TPUs, examples of price -- both clock freq, number cores, memory, bandwidth}.
	
	\item \textit{Performance Benchmarks (Kaggle challenges, etc.)}: \r{Examples like imagenet}.
	
	\item \textit{Software (algorithms)}: \r{Advances in activation functions, weight-initialization schemes, optimization schemes --- batchnormalization, residual connections, separable convolutions --- allowed for deeper models to be trained.}
	
	\item \textit{Software (Libraries)}: \r{More accessible --previously needed to have a deep understanding of C++ and/or CUDA --- python --- pytorch, theano, tensorflow, abstractions on top, lasagne, keras}. \textcolor{green}{TODO: CITE these}
	
	\item \textit{Investment}: \r{Rapid rise of machine learning investment and deployment}
	
\end{itemize}

\begin{figure}[htp]
	\centering
	\includegraphics[width=0.5\textwidth]{example-image-a}\hfil
	\caption{Cyclic figure of software, hardware, investment, benchmarks, \textcolor{green}{TODO}}
	\label{fig:cyclic_rise_of_dl_overview}
\end{figure}
\TD{TODO: figure of how software, hardware, investment, benchmarks are related to the rise of ML .}

\section{Why ML}

%TODO: is this the right location for this



\section{Notes}

\subsection{Target Audience}

\r{The target audience for this book is not well defined, largely because I am writing this collection of notes and guides for myself. However, as I continue to iterate on the ideas, I hope to make them easier to understand. The intention is to one day have a large enough collection to call a ``book'' and to publish for a larger audience.}

\subsection{How to Read this Book}
\r{Excepts sequential acess, but attempts to support semi-random access as well.}

\TD{``All models are wrong, but some are useful''. George E. P. Box}

\TD{TODO: figure of the structure of the book. I really like figure 1.6 in DL}




\include{community}

% TODO: include overview, but move specifics to the appendix
\chapter{Prerequisites}

\r{The following sections are a non-exhaustive refresher on some of the important underlying concepts underpinning later concepts. Resources to further explore and learn these concepts are shared.}

\r{But first, let's start by answering the million dollar question: what is a tensor?  {Tensors}\index{tensor} are a generalization of matrices and can have an arbitrary number of dimensions (axis). The number of axes in a tensor is also called the {rank}\index{rank} of the tensor. For example, a $224\times224\times3$ image, would be considered a rank three tensor. If the batch dimension were included, ($n\times224\times224\times3$, where $n$ is the number of samples in a batch), we would have a rank four tensor.}


\section{Math Notation}

\r{Depending on the resource, the level of formal math education required to understand a se tion may vary greatly. In order to demystify some of the resources that do not expand on the proofs and notations, below are some of the symbols and XXXXX used in math notation and their interpretation in simple text/natural language form.}

\textcolor{green}{TODO: Show symbols and examples in both equation and simple text/natural language form.}

\textcolor{blue}{$\mathbb{R}$}

\section{Calculus}

% TODO: tie this to loss functions

\TD{The Matrix Calculus You Need For Deep Learning \cite{DBLP:journals/corr/abs-1802-01528}}

\r{We won't be proving any theorems and we won't assign 45 questions due tomorrow. But, there is some basic terminology that should be revisited.}

\r{{Critical points}\index{critical points} or {stationary points}\index{stationary points} are points where the derivative is equal to zero and therefore doesn't provide any useful information about the gradient/slope (which direction and how far to move)}

\begin{figure}
	\centering
	\includegraphics[width=0.5\textwidth]{example-image-a}\hfil
	\caption{Graph of an example function and its derivative, \textcolor{green}{TODO}}
	\label{fig:calc_fn_deriv}
\end{figure}
\textcolor{green}{TODO: graph of function and it's derivative overlaid.}

\textcolor{blue}{There exist three main types of critical points:}

\begin{itemize}
	\item \textcolor{blue}{local minimum} -- 
	\item \textcolor{blue}{local maximum} -- 
	\item \textcolor{blue}{saddle point} -- 
\end{itemize}


% global + local min (strong and weak), inflection point
\begin{figure}[htp]
	\centering
	\includegraphics[width=0.3\textwidth]{example-image-a}\hfil
	\includegraphics[width=0.3\textwidth]{example-image-b}\hfil
	\includegraphics[width=0.3\textwidth]{example-image-c}
\caption{Types of critical points -- points with zero slope. From left to right, \textcolor{green}{TODO}}
\label{fig:calc_critical_points}
\end{figure}



\r{Local minimal are minimal values within a local region. It is possible, nearly certain actually, that a loss function for a given model will have many local minima. The point at which the absolute lowest value is present is considered the {global minimum}\index{global minimum}. Similarly, a value located at the absolute largest point is considered the {global maximum}\index{global maximum}. }

\begin{figure}
	\centering
	\includegraphics[width=0.8\textwidth]{./misc/critical_points.png}\hfil
	\caption{Example function with labeled local and global minima.}
	\label{fig:calc_fn_deriv}
\end{figure}

\subsection{Derivatives}

\r{The derivative of a fuction is the rate at which the value of the function changes with respect to how the a single input changes. The derivative can be thought of as the ratio between the functions output and input. A gradient is a generalization of a derivative to multi-input (multivariate) function.}

\TD{a partial derivative --- derivative with respect to one variable while all others are held constant.}


\r{Hyperplane: A subspace that consists of one dimension less than the dimensionality of the space it occupies e.g. a line in a 2D plot, or a 2D plane in a 3D environment}

\r{numerical differentiation is a process used to estimate derivatives numerically.}

\r{Two main methods for differentiating $f$ using a computational graph --- forward or reverse accumulation.}

\r{reverse accumulation\cite{linnainmaa1970representation} only requires a run to compute the complete gradient, but requires two passes through the graph (forward and reverse). \TD{more}}

\TD{importance of reverse accumulation in backpropagation\cite{rumelhart1988learning}}

\r{calculated by calculating the change over an infinitesimally small step \TD{illustration?}.}

\r{in theory, the smaller the step, the accurate the derivative estimate, however, in practice, steps that become too small may result in numerical errors}

\r{a place where the derivative is zero may also be called a stationary point}

\r{The step may be definited by the forward difference, the central diference, or the backward difference. \TD{figure?}}

% TODO: there likely is a better source for this.
\r{default step size is are the square (or cube) root of the machine precison for pointing point values\cite{kochenderfer2019algorithms} and are used as a balance between round-off error and step size error }

\r{more information on derivatives\cite{griewank2008evaluating}}

\subsection{Chain Rule}

\r{(not the chain rule of probability)}



\section{Boolean Logic}

\TD{TODO: background/overview on boolean logic and importance}

\TD{TODO: Examples}


\section{Linear Algebra}

\subsection{Overview}

\textcolor{green}{TODO: background/overview on linear algebra and importance}

\textcolor{green}{TODO: Examples}

\subsubsection{Scalars (0D tensors)}

\textcolor{green}{TODO: diagram}

\textcolor{blue}{scalar -- single value (not a vector or matrix)}

\textcolor{green}{example: a single value}


\subsubsection{Vectors (1D tensors)}

\textcolor{green}{TODO: diagram}

\textcolor{blue}{vectors -- denoted by lowercase names}

\textcolor{blue}{vector: an $n x 1$ matrix. A $12 x 1$ vector may be considered a 12 dimension vector.}

\textcolor{blue}{1-indexed or 0-indexed. In this document, we will only use 0-indexed terms}

\textcolor{green}{example: a ``list'' of features}


\subsubsection{Matrices (2D tensors)}

\textcolor{green}{TODO: diagram}

\textcolor{blue}{Matrix -- written as (rows x columns)}

\textcolor{blue}{$M_{i,j}$ means matrix entry at ($i$,$j$), or $i$th row, $j$th column}


\textcolor{blue}{Matrices -- denoted by uppercase names}

\textcolor{green}{example: a grayscale (single color channel image)}



\subsection{Matrix Arithmetic}

\textcolor{blue}{{element-wise}\index{element-wise} operations: operations that are applied independently to each entry. These types of operations lend themselves nicely to parallelization/vectorization}

\textcolor{blue}{dot product --- \textcolor{green}{TODO: visualization of dot product inputs and output}}

\subsubsection{Matrices}

\paragraph{Addition, Subtraction}

\paragraph{Multiplication, Division}

\subsubsection{Scalar}

\paragraph{Addition, Subtraction}

\paragraph{Multiplication, Division}



\section{Graph Theory}

\textcolor{green}{TODO: Examples}

\textcolor{blue}{computational graph}

\textcolor{blue}{operation}



\part{Data}
\r{Collecting data, visualizing data, processing data...}

\include{data}

\part{Foundations}

% the organization and sec structure/naming needs work here.
\chapter{Basics}

\section{Overview}

%%%%%%%%%%%%%%%%%%%%%%%% obligatory "No Free Lunch"
% TODO: there is more to the no free lunch than simply this.
\r{LIkely not possible to discuss machine learning without at least mentioning the ``No Free Lunch'' theorem, which states, roughly ``No single [model] works best across all possible scenarios''~\cite{wolpert1997no}}

% “You can’t learn from data without making assumptions”
\TD{A Lot to unpack here.}
\TD{https://peekaboo-vision.blogspot.com/2019/07/dont-cite-no-free-lunch-theorem.html}

%%%%%%%%%%%%%%%%%%%%%%%% Types of data
\r{categorical or numerical. Numerical can be discrete or continuous}

%%%%%%%%%%%%%%%%%%%%%%%% Measurement Levels
\r{qualitative or quantitative.} 

\r{Qualitative can be nominal (aren't numbers and can't be put in any order -- e.g. the seasons: spring, summer, fall, winter) or ordinal (groups and categories that follow a strict order -- e.g. difficult levels: hard, medium, or easy)}

\r{Quantitative are represented by numbers but can be interval (0 is meaningless -- e.g. temperature in C or F, where true zero is not 0) or ratio (has a true 0 -- e.g. temperature in K, weight or length)}

\section{Workflow Overview / Blue Print}


% TODO: this will need to be checked+rechecked+redone
\r{1. explore, 2. create datasets, 3. benchmark}

\begin{enumerate}[noitemsep,topsep=0pt]
	\item Problem definition
	\item Hypothesis?
	\item explore data
	\item (shuffle? representative)
	\item remove split for testing
	\item prepare data
	\item split into train/val
	\item Choose measure of success
	\item Perform baseline -- what performance is expected/realistic goal expectations/ how does a simple, well tested, classifier work?
	\item Develop model
	\begin{enumerate}[noitemsep,topsep=0pt]
		\item Can you (over)fit the training data? - more data?
		\item Fit the validation as best as possible (regularization, augmentation) (WARN: \textcolor{red}{local ref to information leak})
	\end{enumerate}
	\item OTHERS...
	\item Evaluate
\end{enumerate}


\section{Some Terms}

\emph{input variable(s)} -- predictors, independent variables, features, regressors, controlled variables, exposure variables or simply variables.
 
\emph{output variable(s)} -- response or dependent variable. May also be known as regressands, criterion variables, measured variables, responding variables, explained variables, outcome variables, experimental variables, labels.

\textcolor{blue}{Both input and output variables may take on continuous or discrete values.}

\emph{relationship} $Y = f(x) + \epsilon$ \textcolor{blue}{estimate $f$. prediction and inference}.

\textcolor{blue}{\emph{{reducible error}\index{reducible error}} -- the estimated function $\hat{f}$ will likely not be perfect, and the reducible error is the error that could potentially be corrected (perhaps by using a more appropriate learning technique to estimate $\hat{f}$).  The \emph{irreducible error} is an error that can not be corrected. The irreducible error may be larger than zero due to either \emph{unmeasured variables} \emph{e.g.} variables that were not measured or \emph{unmeasurable variation} \emph{e.g.} an individual's feelings/emotions or variation in the production of a product, or both. The irreducible error provides an upper bound on the performance of the predicted $\hat{f}$}

\textcolor{red}{inference: relationship between predictors and response}

% TODO: not sure ~exactly where this fits yet..
\textcolor{blue}{parameters -- model variables that change during training. Hyper-parameters are set before training.}

\section{Type of Learning}

%That is, there are more exciting areas of coverstaion than getting hung up on the categorizations of these types of ``learning''. 
% TODO: link FB blog post
\r{Typically, three main types of machine learning are described: supervised, unsupervised, and reinforcement. However, there exist other subareas (e.g. semi-supervised learning, self-supervised learning, \TD{more}), and then within/across these divisions, there are further subdivisions that exist (e.g. contrastive learning, ciriculum learning, \TD{more}). And in reality, these lines aren't always explicit or exact. One example has recently become popular is the definition of ``unsupervised learning'' and how the name is a bit of a miscategorization in that there are often many more training signals (labels, though not called labels) when using this paradigm than say in an binary image classifier in which there is a single label assignment per input (e.g. cat vs dog.)}

% From ML for Predictive Data Analytics
\r{Another way to group types of learning -- Information-based, similarity-based, probability-based, and error-based}

% TODO: did I come up with these definitions or find them somewhere? - 18July21
\r{however, in the intrest of consistency, the typcial definitions are defined below.}
\begin{itemize}[noitemsep,topsep=0pt]
	\item Supervised Learning (\ALR)
	\begin{itemize}[noitemsep,topsep=0pt]
		\item \r{observe input variables with corresponding output values. A program that predicts an output for an input by learning from pairs of labeled inputs and outputs. Classification \ALR and regression \ALR are subcategories of supervised learning}
	\end{itemize}
	\item Unsupervised (\ALR)
	\begin{itemize}[noitemsep,topsep=0pt]
		\item \r{observe input variables without corresponding output values and attempts to discover patterns in the data. There is no error signal to measure, rather, performance metrics report some attribute of structure discovered in the data, such as the distances within and between clusters. Determining whether the method learned ``something'' useful is inherently difficult --- since, by definition, there are no labels.}
	\end{itemize}
	\item{Reinforcement (\ALR)}
	\begin{itemize}[noitemsep,topsep=0pt]
		\item \r{Reinforcement learning does not learn from labeled pairs of inputs and outputs, rather it learns from `feedback' from decisions that are not explicitly corrected. Information is still supplied to the system as to whether the networks outputs are good or bad, but no actual desired values are given. Goal -- develop an \emph{agent} that improves it's performance based on interactions with an \emph{environment} based on a \emph{reward}}
	\end{itemize}
\end{itemize}


\subsection{Supervised vs Unsupervised}

\subsection{Classification vs Regression}

\subsubsection{Regression} 

\r{Regression, also called regression analysis \textcolor{red}{local ref?} involves predicting a continuous or quantitative output value. For example attempting to find a relationship between a given features/predictor/explanatory variables (\textit{e.g.} age, job title, zip code) and a continuous response (\textit{e.g.} an individuals outcome).}

\subsubsection{Classification} 

Classification involves predicting categorical (discrete) or qualitative output value (such as a non-numerical value). 

\r{Binary classification (\textit{e.g.} dog vs cat, true vs false) and multi-class classification (\textit{e.g.} identifying many different skin diseases).}

\subsection{Multi-class}

% TODO:
\begin{itemize}[noitemsep,topsep=0pt]
	\item \TD{macro averaged: all classes independently, averaged}
	\item \TD{micro averaged:}
	\item \TD{weighted: macro, weighted by sample frequency}
\end{itemize}


\subsection{Multi-label}
\TD{TODO: {multi-label classification}\index{multi-label classification} --- where a classifier assigns multiple labels to each instance}


% TODO: placement
\textcolor{blue}{Evaluating every output node for every example can quickly become computationally expensive. approximate version of softmax. i) candidate sampling (tf.nn.sampled\_softmax\_loss()) \textcolor{red}{calculates the output for all of the positive labels, but will only calculate the label for a random sample of negatives. Where the number of negatives sampled is a a hyperparameter} ii) noise-contrastive estimation (tf.nn.nce\_loss()) \textcolor{red}{approximates the denominator of softmax by modeling the distribution of outputs}. Typically, these two methods may be used during training, but the full softmax function will be used during inference.}


\subsubsection{Approaches: Problem transformation}

\textcolor{blue}{There are two main approaches to multi-label classification}

\textcolor{blue}{{Problem transformation}\index{Problem transformation} modify the original multi-label problem to a set of single-label classification problems.}

\paragraph{Unique set/combination of labels}

\textcolor{green}{TODO: table and example.}

\textcolor{blue}{Two main concerns with this methodology: i) increasing the number of classes is impractical and will often have very few instances and ii) the classifier can only predict combinations that were seen in the training data.}

\paragraph{Many Binary Classifiers}

% p124[112] of Mastering ML with SKL
\textcolor{blue}{Train a classifier for each label in the training set. The final prediction is the combination of all the predictions from the binary classifiers.}

\textcolor{blue}{The main concern with this approach is that the relationships between labels is ignored.}

\subsubsection{Evaluating Multi-label Classification}

\textcolor{blue}{see \textcolor{red}{local ref}.}

% page 94 of AGtext
\r{One-versus-all \emph{OvA} (also \emph{one-versus-rest}) -- $n$, separate binary classification problems, where the $n$ is the number of classes. The target class may be assigned as the positive class and `all' the `rest' of the classes may be assigned as the negative class.
}
\r{One-versus-one (OvO) -- train a binary classifier for every pair}


\textcolor{blue}{Binary classification can be extended to multi-class classification via the OvR method.}

%\subsubsection{Bayes Classifier}

\section{Training}

\TD{Difference between the loss and metric functions}

\r{Metrics are what we care about and how we, as humans, measure the performance.}

\r{Loss is a proxy for the metric}

\TD{distinction: losses are differentiable, metrics are not \TD{show the importance of this somewhere}}

\subsection{Performance}

\subsubsection{Cost, Loss Function}

\textcolor{red}{Cost and Loss functions -- I'm not sure why I didn't have this yet?}

\textcolor{blue}{A loss function is responsible for providing a measure of performance for the training data -- thus the loss function is responsible for guiding the training process.}

\textcolor{blue}{\textit{objective function}}

\textcolor{red}{example of plots - step by step}

\textcolor{red}{contour maps}

\textcolor{blue}{it is possible that a DNN graph may have multiple loss functions (where each one is responsible for a specific output class). Since the gradient descent process relies on a single scalar loss value, graphs with multiple losses will combine these losses (\textcolor{red}{via averaging})}

\textcolor{blue}{TODO: general loss plot example/figure}

\textcolor{blue}{TODO: section (maybe in another location) that shows how to "troubleshoot" a loss plot. So examples of what a noisy loss plot vs a flat loss plot might indicate. Would also be nice to expand to include examples of the loss plot of the training and validation (likely another section, but with a reference in this section pointing to it.)}

\textcolor{blue}{TODO: talk about "convexity" -- show 3D loss plot (2D for params, 3rd for loss), and explain how running/re-running a model may result in slightly different values.}

\textcolor{blue}{Loss calculation. TODO: example for linear linear regression and why sum of loss will cancel out terms. \textcolor{red}{point to local ref.}}

\r{``surrogate loss'' --- takes a classification problem and turns it to a continuous/smooth surface.}


\subsubsection{Metrics}

\subsection{Error Functions}

\TD{discussion of techniques used for training networks}

\r{the training of a network consists of a suitable error function to be minimized with respect to the parameters (weights and biases) of a network.}



\subsubsection{forward pass}

\subsubsection{backward pass}

\subsubsection{Least-squares techniques}

\r{simple error function, typically most suitable for regression problems, but could also be used for classificaiton problems, though other error functions typcially perform better for classification (\textcolor{red}{local ref})}

\paragraph{Sum-of-squares error function}

\r{potential negative --- the largest contributions are from points with the largest error. If the dataset contains many outliers (or if the distribution contains long tails), the the adjustments can be dominated by these outliers. NOTE: in this case especially, it is essential to ensure there are no mislabeled data, as these can have dramatic effects. Techniques used to address this problem are known as robust statistics \TD{local ref -- create subsec for this == \textcolor{red}{see (Huber 1981)}}}

% see p.211 of NN by bishop (p.226 on tablet)
\textcolor{red}{input-dependent variance}

\begin{figure}[htp]
	\centering
	\includegraphics[width=0.4\textwidth]{example-image-a}\hfil
	\includegraphics[width=0.4\textwidth]{example-image-b}\hfil
	\caption{ \TD{figure of how a model is affected by outliers. left: the data does not have any outliers and a linear line fits the data fairly well. right: a single outlier dramatically alters the linear line}}
	\label{fig:basics_error_fn_sumofsquares_outlier}
\end{figure}

\r{can be used for regression or classification (though typically for regression)}

 
\paragraph{Normal Equation}

%TODO: this belongs elsewhere -- maybe appendix?

\TD{pseudo inverse, moore penrose}

\TD{can't fit in memory --> gradient decent}

\r{find an explicity solution to the function}

\paragraph{Singular Value Decomposition (SVD)}
% TODO: read: https://twitter.com/WomenInStat/status/1285610321747611653

\r{Singular Value Decomposition (SVD) --- technique used to help solve problems like ``near degeneracies'' (\textcolor{green}{TODO: unsure of this --- also see Press et al. 1992 for an introduction})}


\paragraph{Gradient Descent}

\r{repeatedly choosing and moving toward a descent direction until convergence} \TD{local descent may be another name}. \TD{there are many schemes for choosing the size of the step, discussed further in \ALR.}

%\r{\textcolor{green}{TODO: is this the first time I've talked about this?}. Finding the weight values of a sum-of-squares error function can be found explicitly in terms of the \TD{pseudo inverse of a matrix} (if a linear network).}
% However, if a non-linear activation fuction is used, then the closed form of a solution is no longer possible ? % what is the source for this?

\r{If a derivative of the activation function is differentiable, the derivatives of the error function with respect to the weight parameters can then be evaluated. The derivatives can then be used by gradient-based optimization algorithms to find the minimum of the error function \TD{local ref - sec on optimizers}}

\r{update the parameters one iteration at a time --- sequential, pattern-based, update}

\TD{equation}


\subsubsection{Global vs Local Minima}

\ALR to calculus section

% TODO: this needs to be placed elsewhere
\textcolor{blue}{show how a simple problem may get "stuck" in a local mimima (plot decision boundary for 2D feature space) and show how it changes with different initialization (use different random seeds). }

\textcolor{blue}{{inappropriate minima}\index{inappropriate minima} -- don't reflect the relationship between features and output and/or don't generalize well.}

\r{a local minimum will have a zero derivative, as it is not possible for a nonzero derivative to be a mimima.  However, just because a point has a nonzero derivative, it does not mean that it is a minima, it could also be local maxima or an inflection point. A positive second derivative would indicate that the point is a local minima.}

\section{Quality of Fit}

%% regression example

\subsection{Regression Example}

% TODO: need to think about placement of these and how to organize the general case.

\textcolor{blue}{squared will handle the issue with positive and negative error terms canceling each other out as well as penalizing terms more when they are larger}

\textcolor{blue}{Mean Squared Error.$\hat{f}(x_i)$ is the prediction that $\hat{f}$ produces for the $i$th sample (\ref{eq:MSE_def}). The output will be small for predicted values that are similar to the ground truth}

\begin{equation}
{MSE = \frac{1}{n}\sum_{i=1}^{n}(y_i - \hat{f}(x_i))^2}
\label{eq:MSE_def}
\end{equation}

\r{The MSE may be hard to interpret since the error is squared. \textcolor{red}{example}}

\r{Root mean squared error, which is simply the root of the MSE. (\ref{eq:RMSE_def})}

\begin{equation}
{RMSE = \sqrt{\frac{1}{n}\sum_{i=1}^{n}(y_i - \hat{f}(x_i))^2}}
\label{eq:RMSE_def}
\end{equation}

\textcolor{blue}{"problem" with this loss function is that it does not follow the intuition that really bad predictions should be penalized more harshly than predictions that are just "a little bad"}

% TODO: again, need to devise a better way to organize this section (loss functions)
\textcolor{red}{Cross entropy or log loss. \textcolor{red}{CITE} (related to Shannon's information theory \textcolor{red}{CITE})}

% TODO, include in losses?: penealize highly confident, highly incorrect
\TD{Log loss: def + interpretation}


%% classification example

\subsection{Classification Example}

\textcolor{blue}{The proportion of mistakes that are made.}

\begin{equation}
{error\_rate = \frac{1}{n}\sum_{i=1}^{n}(y_i \ne \hat{y_i})}
\label{eq:class_error_rate_def}
\end{equation}

\textcolor{blue}{$\hat{y_i}$ is the predicted classification label for the $i$th observation using our predictor/model $\hat{f}$ and $y_i$ is the ground truth label}

\section{Describing Learners}

\subsection{Parametric and non-parametric}

\subsubsection{parametric}

\rr{parametric models are models that learn a fixed number of parameters that are able to classify new data points without requiring the original dataset anymore. First, a function form is selected (linear, polynomial, etc.), then the coefficients for the function are learned from the training data.}

\textcolor{green}{TODO: example \r{predicting the income of an individual $income \approx \beta_0 + \beta_1 \times education_{yrs} + \beta_2 \times experience_{yrs}$ --- assuming a linear relationship between response and two predictors}}

\textcolor{green}{TODO: plot of example -- }

\begin{figure}[htp]
	\centering
	\includegraphics[width=0.5\textwidth]{example-image-a}\hfil
	\caption{Figure example of assumed linear model and datapoints \textcolor{green}{TODO}}
	\label{fig:basics_para_assume_linear}
\end{figure}



\textcolor{blue}{Examples of parametric models may be simple artificial neural networks, naive bayes, logistic regression, etc.}

\subsubsection{nonparametric}

%% unsure about this! 
\textcolor{red}{Nonparametric models are not models without parameters, rather they are models were the number of parameters are not fixed, they may grow with the number of training instances}

\textcolor{blue}{May be useful when little is known about the underlying relationship in the data and there is an abundance of data.}

\textcolor{blue}{An Example of a nonparametric model may be k-Nearest neighbors -- where the model does not assume anything about the form of the mapping function and makes predictions based on the k most similar training instances.}

\subsubsection{parametric vs nonparametric}

\r{An advantage of a nonparametric approach may be that the model does not make any explicit assumptions about the best fitting model thus avoiding limiting the model to a functional form $f$ that may not be similar to the true $f$ --- for example, using a linear model for a model that is cleary \textcolor{red}{parametric} in form. Typically, since a nonparametric approach is not limited to an explicit number of parameters, a larger amount of data is required to obtain an accurate estimate of $f$.}

\textcolor{blue}{A disadvantage to this type of approach is that the computational complexity for classifying new samples grow linearly with the number of samples in the training set.}



\subsection{Eager vs Lazy Learners}

\textcolor{green}{TODO: Eager vs Lazy overview}
\textcolor{blue}{Training an eager learner is often more computationally expensive, but typically prediction with the resulting model is inexpensive.}

\subsubsection{Eager Learners}

\textcolor{blue}{Eager learners estimate the parameters of a model that generalize to a training set --- build an input-independent model}

\subsubsection{Lazy Learners}

\r{Also known as Instance-based Learners}

\r{do not spend time training, but may predict responses slowly (relatively) compared to eager learners}

\r{Lazy learners store the training dataset with little to no processing.}


\subsection{Generative vs Discriminative Models}

\TD{TODO: Generative vs Discriminative models overview}


\subsubsection{Discriminative Models}

\r{learn a decision boundary that is used to \textit{discriminate} between classes. There exist both probabilistic and non-probabilistic discriminative models.}

\paragraph{Probabilistic Discriminative}

\r{Probabilistic discriminative models learn to estimate the conditional probability i.e. which class is most probable given the input features.}

\paragraph{Non-probabilistic Discriminative}

\r{Non-probabilistic discriminative models directly map features to classes.}

\subsubsection{Generative Models}

% see p129[117] of Mastering ML w/SKL
\TD{TODO: Generative Models --- ``do not learn a decision boundary, rather, they model the joint probability distribution of the features and classes i.e. they model how the classes generate features. Then, using Bayes' theorem, they are able to estimate the conditional probability of a class given the features.''}

\r{must be probabilistic, not deterministic and also some degree of randomness (otherwise the same output would be generated each time).}

\TD{TODO: will need to expand into a much larger section and talk about some types of generative models}

\TD{TODO: ``important'' examples --- GPT2, StyleGAN}

\TD{TODO: use cases -- direct and indirect. music, art, game design, simulations for RL (paper\cite{ha2018world})}

\r{\TD{Generative Deep Learning} makes a point that (paraphrasing) ``categorizing data is not enough, we should try to understand how and why the data came to existence in the first place.''}


% see p130[118] of Mastering ML w/SKL
\r{One advantage of generative models is that they can be used to generate new examples of data}

\subsection{Strong vs Weak Learners}

\TD{TODO: Strong vs Weak learners (classifier, predictor, etc.) overview}
%\textcolor{blue}{}

\begin{itemize}[noitemsep,topsep=0pt]
	\item Strong Learners
	\begin{itemize}[noitemsep,topsep=0pt]
		\item \r{Strong Learners are models that are arbitrarily better than weak learners.}
	\end{itemize}
	\item Weak Learners
	\begin{itemize}[noitemsep,topsep=0pt]
		\item \r{Models (typically simple models) that perform only slightly better than random chance. Typically used in ensemble methods (discussed in more detail in \ALR)}
	\end{itemize}
\end{itemize}

\section{Online Learning}

% See p.246 of Understanding Machine learning
\textcolor{blue}{difference to \textcolor{red}{PAC learning?}}


\section{Kernel Methods}
\label{sec:kernel_trick}

\r{Adding non-linear features to data in attempt linearly separate the data.}

 %Adding non-linear features is a powerful method for allowing linear methods to separate non-linear data. However, which features, combinations of features, and types of features is often not easily known. And adding may of these features may become computationally limiting

\r{Transform the training data onto a higher dimensional feature space}

% see p177[165] of mastering ML with SKL

\r{Choosing an appropriate kernel can be challenging}

% computing the distance (scalar products) of data points for the expanded feature representation --- but doesn't compute the expansion

% see p180[168] of mastering ML w/SKL for more on kernels
\r{Some commonly used kernels include polynomial, RBF, sigmoid, Guassian, and linear kernels}

\r{commonly used in SVMs (see \textcolor{red}{local ref}), the kernel trick can be used with any model that can be expressed in terms of the dot product of two feature vectors.}

\r{the mapping function is not fully computed due to the kernel trick}

\subsection{Kernel Trick}

\subsection{Kernels}

\r{There are many kernel functions. Choosing the ``best'' kernel will depend on the current problem.}

\textcolor{red}{kernel functions are continuous and symmetric}

\textcolor{red}{TODO: put these into context and/or give an example}

\textcolor{blue}{the word kernels is representative of a weighting function \textcolor{red}{or weighted sum or integral}}


% https://data-flair.training/blogs/svm-kernel-functions/
% http://crsouza.com/2010/03/17/kernel-functions-for-machine-learning-applications/

\textcolor{red}{TODO: automatic kernel selection}

\subsubsection{Polynomial}

\r{polynomial computes all possible polynomials of the original features up to a certain degree}

\textcolor{blue}{parameters: slope ($\alpha$), constant $c$, polynomial degree $d$}
\begin{equation}
{k(x, y) = (\alpha x^T y + c)^d}
\label{eq:kernel_polynomial_eq}
\end{equation}
% TODO: double check

\subsubsection{Gaussian / RBF (Radial Basis Function)}

\textcolor{blue}{circles/hypersphere}
\textcolor{red}{infinite-dimensional feature space}

\begin{equation}
{k(x, y) = exp(- \gamma || x_1 - x_2 || ^2 ) }
\label{eq:kernel_guassian_rbf_eq}
\end{equation}

\textcolor{blue}{$|| x_1 - x_2 ||$ represents the euclidean distance and $\gamma$ represents the parameter that controls the width of the Gaussian kernel (the inverse width of the Gaussian kernel). $\gamma$ controls how far the influence of a single training instance reaches --- high values correspond to a limited reach (typically result in lower complexity) and low values correspond to a far reach (typically result in higher complexity).}

\textcolor{green}{Would be nice to have a figure with low - medium - high values for the hyperparameters and the outcome}


\subsection{Less Common Kernels}



%\subsubsection{Laplace RBF}

%\subsubsection{Hyperbolic Tangent}

%\subsubsection{ANOVA Radial Basis Kernel}

%\subsubsection{Sigmoid}

%%%%%%%%%%%%%%%%%%% Hyper-parameters
\section{Hyper-Parameters}


\subsection{Parameters: "tuning knobs"}

\subsubsection{Learning Rate}
\label{hp_learning_rate}

\TD{TODO: Learning rate overview}

% TODO: Learning rate practical advice

% TODO: figure showing cost vs iteration for a LR that is too small, just right, and too large

% TODO: Learning rate figure showing how if the learning rate is too high, you'll likely see the cost diverage when plotted vs iterations

% TODO: schedules

\r{In general, if the LR is too small, convergence (with something like gradient descent) may be slow.  If LR is too large, then convergence may not occur and the reduction in error may oscillate wildly or may even diverge.}

\TD{The large learning rate phase of deep learning: the catapult mechanism \cite{Lewkowycz2020TheLL}}

\paragraph{Schedule}

\r{Rather than keep the same learning rate during all of training, the learning rate is adjusted during training according to a ``schedule''.}

\paragraph{Descriminative/Differential}

\r{FastAI -- ``discriminative'' however, typically shows up as ``differential'' learning rate. Rather than use the same learning rate for all layers/components, different layers/components use different learning rates.}

\paragraph{research}

% TODO: this section may not belong here - may belong in an "advanced section"

%%%% learning rates
\textcolor{blue}{cyclic learning rate~\cite{smith2017cyclical}}

\textcolor{blue}{sgdr: stochastic gradient descent with restarts~\cite{loshchilov2016sgdr} (SGDR). The learning rate is decreased from the max value along a curve (cosine, shown in Eq.\ref{eq:sgdr_def}, where $n_{max}^i$ and $n_{min}^i$ are ranges for the learning rate, $T_i$ represents epochs, $T_{cur}$ is how many epochs have been performed since the last restart). The authors also suggest making each next cycle longer than the previous cycle by a constant $T_mul$ may be beneficial.}

\r{LR annealing, Cosine anealing ($1/2$ cosine curve)}

% \TD{`differential learning rate'/different learning rates at different levels of the network} blog: https://blog.slavv.com/differential-learning-rates-59eff5209a4f

\begin{equation}
{n_t = n_{min}^i + 1/2(n_{max}^i - n_{min}^i)(1 + cos(\frac{T_{cur}}{T_i}\pi))}
\label{eq:sgdr_def}
\end{equation}

\subsubsection{Batch size}

\textcolor{green}{TODO: batch size overview}

\r{anywhere from a single instance to the entire training set size.}

\textcolor{blue}{optimal batch size is problem dependent}

\textcolor{blue}{TODO: notebook and plots showing how the smoothness is affected when comparing batch sizes of 1 vs 10 vs 20 etc.}

% related to shuffling - the gradients are computed on a batch and so a batch should be representative of the data

%%%%% small batch size
\r{small minibatch sizes (between 2 and 32) may be better than large batch sizes~\cite{masters2018revisiting}.}

\r{``generalization gap'' may not be due to large mini-batches, rather, due to the number of updates made to the system~\cite{hoffer2017train}}

%%%% minibatch
\r{Batch training is almost always slower to converge than on-line/mini-batch training, which is likely due to the fact that on-line/mini-batches learning will follow the error surface, allowing for larger learning rates, and thus faster convergence~\cite{wilson2003general}.}

% incrementing batchsize over time
\r{Increasing the batch size may achieve similar benefits to decaying the learning rate ~\cite{smith2017don} -- which could lead to use of larger batch sizes, reducing the number of parameter updates and therefore reducing training time.}

\r{minibatches use the hardware more efficiently}


\subsection{Hyper-Parameter Optimization}

% \r{opinion: perfomed last to eek out extra performance}



\subsubsection{Coordinate Descent}

All hyper-parameters remain fixed, except for the hyper-parameter of interest. The hyper-parameter of interest is then adjusted such that the validation error is minimized.

\r{bayesian $>$ random $>$ grid}

\subsubsection{Grid Search}

\textcolor{blue}{{Grid search}\index{Grid search} Exhaustive search that trains+evaluates a model for each combination of specified hyperparameter configurations and combinations defined by a Cartesian product of the sets of possible values for each hyperparameter.}

\subsubsection{Randomized Search}

\r{{Randomized search}\index{Randomized search} }

\r{TODO: figure demonstrating difference between grid and randomized search}

\TD{TODO: grid vs random search figure}

\subsubsection{Other Methods: Automated / Model-based Methods}

\textcolor{blue}{See \textcolor{red}{local ref? --- advanced methods and research}}

\paragraph{Bayesian Methods}

\TD{todo:}




%%%%%%%%%%%%%%%%%%%%%%%% Optimizers

\chapter{Estimating Model Parameters}

\textcolor{green}{Iterative Estimation vs Calculation}

% http://mathworld.wolfram.com/NormalEquation.html
% https://eli.thegreenplace.net/2014/derivation-of-the-normal-equation-for-linear-regression
\textcolor{green}{TODO: Normal Equation}

% TODO: non-invertable (singular or degenerate) matrix

% common causes (not verified) - 1) redundant features 2) too many features - more features than samples. solutions may be to delete features

\section{Initialization}

\TD{initialization --- how we define/set the initial values of parameters}

\TD{largely focused on neuralnetworks initialization}

\TD{TODO: initialization methods and importance}

\TD{figure showing the importance of initialization strategies for different architectures after \textit{n} layers}

\r{motivated partially by reducing the possibility of exploding or vanishing gradients.}

\TD{AutoInit: Analytic Signal-Preserving Weight Initialization for Neural Networks \cite{Bingham2021AutoInitAS}}

\TD{basic idea: initialize with small random values, typically from uniform or gaussian --- more advanced: hueristics based on characteristics --- motivation: }


\subsection{Parameter types (the initialization of)}

\TD{TODO: different types of parameters may benefit from different strategies}

\paragraph{Weights}

\TD{TODO: fully connected, convolution}

Break symmetry -- two things:
\begin{itemize}[noitemsep,topsep=0pt]
	\item \r{Non-zero}
	\item \r{some diversity}
\end{itemize}

\paragraph{Biases}

% HUGO talk
\TD{initializing with negative values may encourage sparsity?}

\TD{fan in and fan out}

\subsection{Normal Vs Uniform}


\subsection{Strategies}

% TODO: Nice write up: https://machinelearningmastery.com/weight-initialization-for-deep-learning-neural-networks/
% also possibly useful: https://machinelearningmastery.com/why-initialize-a-neural-network-with-random-weights/
\TD{write up\cite{brownlee2021WeightInit}}

\TD{TODO: strategies overview}


\TD{fixup initialization \cite{zhang2019fixup}}

\TD{LeCun \cite{lecun2012efficient}}

\subsubsection{Glorot or Xavier}

\TD{\cite{glorot2010understanding}}

\r{xavier: derived based on linear activations (which isn't true for modern architectures)}



\subsubsection{he}

\TD{Kaiming initialization \cite{he2015delving}}

\subsubsection{Implementation}



\chapter{Optimization}

\section{Parameterized}
\label{subsec:optimization}

\TD{This may need it's own chapter!}

\r{Estimate the values of the model's parameters that minimize the value of the cost function based on the data it observes}

\r{"turning a loss function into a search strategy"}

% this may belong elsewhere
% alternatives to gradient descent 
% conjugate gradient
% BFGS
% L-BFGS
% pro: faster, don't need to pick the LR con: more complex
% line search algorithm

\TD{Error surface definition} --- \r{the error surface may include flat region which, in high-dimensional spaces is considered a saddle point.}


\r{choosing:}

\begin{itemize}[noitemsep,topsep=0pt]
	\item Step direction
	\begin{itemize}[noitemsep,topsep=0pt]
		\item \r{.}
	\end{itemize}
	\item Step size
	\begin{itemize}[noitemsep,topsep=0pt]
		\item \r{.}
	\end{itemize}
\end{itemize}

\r{Simply stepping in the direction of the steepest descent for a given location (or batch) is not always the best strategy for convergence. \TD{a figure of this would be nice}}



\subsection{Descent Direction Methods}

\TD{overview of descent direction methods.}

% see C4 of algorithms for optimization


\subsection{First-order}

\r{first-order methods rely on the first derivative (gradient) of the objective function to select the direction to descend.}

\r{The value and gradient for a location can help guide the direction to step, but this first order information does not directly guide the step size.}

\subsubsection{Gradient Descent}
% not sure this belongs right here

\r{Gradient descent refers to descending the gradient of the objective function and is an optimization algorithm that can be used to estimate the local minimum of a function}

\r{Iteratively updates the model parameters by calculating the partial derivatives of the cost function at each step during training}

\r{Gradient descent is only guaranteed to find the local minimum of the cost function.}

\r{simultaneous update.}


\TD{First Order (\ALR), Second Order (\ALR) optimizers. Second-order approximations are based on the Hessian (\ALR) or the objective function and are capable of informing not only the direction to step in, but also the step size.}



\paragraph{Batch Gradient Descent}

\r{batch gradient descent --- taking a step (update the weights) opposite (down) the gradient calculated from the entire training set}

\r{Batch gradient descent is deterministic --- will produce the same paramter values if the same dataset is used multiple times.}

\r{single static error surface}


\paragraph{Stochastic Gradient Descent}

\r{Stochastic Gradient Descent (sometimes called iterative or on-line gradient descent) --- rather than update the weights based on the sum of the accumulated errors, the weights are updated for each training sample}

\r{Stochastic gradient descent is deterministic --- may produce the different parameter values if the same dataset is used multiple times. May not minimize the cost function as well as gradient descent but the approximation is often ``close enough''. One potential downside is that if the approximation of the error surface is not ``good enough'' minimization could take a, relatively speaking, long time.}

\r{rather than the single static error surface, the error surface is now dynamic as it is being estimated during every iteration with respect to only one training example.}


\paragraph{Mini-batch Gradient Descent}

\r{mini-batch gradient descent --- compromise between batch and stochastic gradient descent where the gradient is calculated over a subset of training data (minibatch). The minibatch size then acts as another hyper-parameter.}

\r{Since the gradient is calculated on a single example, the error surface will appear noisier than if it was calculated over a batch or the entire training set.}

\r{When using stochastic gradient descent, it is important to shuffle the data after each epoch.}


% when looking at specific optimizers, http://ruder.io/optimizing-gradient-descent/ was a useful resource

% TODO: these are really subcategories/improvements of gradient descent

\subsubsection{Conjugate Gradient}

% gd can perform poorly in narrow valleys -- orthoganal steps

\TD{include? paragraph?}

\subsubsection{Momentum Descent}

Momentum~\cite{qian1999momentum}

\r{gradient descent can take a long time to traverse flat surfaces}

\r{momentum is intuitively what it sounds like. -- can imagine a ball rolling down a hill where it gains speed as it travels down the slope.}

\TD{figure of momentum}

\TD{figure of gradient descent with and without momentum side by side on the same surfaces.}

\subsubsection{Nesterov Momentum Descent}

% ok. blast probably isn't the best choice here..
\r{Nesterov accelerated gradient (NAG). One issue with momentum can be that the momentum may be ``too strong'' and blast through the bottom and climb the other side}

\r{modifies the momentum values -- \TD{original paper}}

\TD{explain how Nesterov works}

\TD{figure of momentum vs Nesterov}



\subsubsection{Adagrad Descent}
% see p.77 of optimization

\r{Adagrad~\cite{duchi2011adaptive}, \textcolor{red}{will assign frequently occurring features low learning rates}}

\r{\textit{Ada}ptive sub\textit{grad}ient method (\textit{adagrad})}

\r{apdapts a learning rate for each component. Nesterov and Momentum use the same learning rate for each component}

\r{one problem is that the effective learning rate decreases}

\paragraph{Adagrad Extensions: (RMSProp, Adadelta, Adam)}

\TD{extension of adagrad to overcome the decreasing learning rate ...}

\subparagraph{RMSProp}

\r{unpublished -- from Geoff Hinton's lecture for a coursera course}

% ``if small and not v___ let's take bigger jumps'' -took this note

\subparagraph{Adadelta}

\TD{explanation}

\r{Adadelta~\cite{zeiler2012adadelta}, expands on AdaGrad by avoiding reducing the learning rate to zero.}


\subparagraph{Adam}
% not 100% sure this belongs here

\r{Adaptive Moment Estimation (Adam)~\cite{kingma2014adam}}

\r{Adaptive moment estimation method (adam)}

\r{similar to both RMSProp and Adadelta in that it stores the exponentially decaying squared gradient}

\r{also uses an exponentially decaying gradient like momentum}

\r{RMSProp plus momentum}

% TODO:
\TD{bias correction step (bias caused by initializing the gradient to zero?)}

\subsubsection{AdaMax}

\TD{TODO:}

\subsubsection{Hypergradient Descent}

% see p.80 of optimizers
\TD{overview}

\r{the derivative of the learning rate may be useful. A hyperparameter gradient, (Hypergradient) is what it sounds like, a derivative taken with respect to a hyperparameter.}

\r{applies gradient descent to the learning rate}

\TD{paper\cite{baydin2017online}}


\subsubsection{FTRL}

% TODO: https://medium.com/@dhirajreddy13/factorization-machines-and-follow-the-regression-leader-for-dummies-7657652dce69
\TD{``follow the regularized leader'' -- TODO}



\subsection{Nadam}

\r{Nadam (Nesterov-accelerated Adaptive Moment Estimation)~\cite{dozat2016incorporating}}


\subsubsection{AMSGrad}

\r{paper\cite{reddi2019convergence}}


\subsection{To Include}

\TD{RAdam --- On the Variance of the Adaptive Learning Rate and Beyond \cite{DBLP:journals/corr/abs-1908-03265}}

\TD{``SGDP and AdamP: get rid of the radial component, or the norm-increasing direction, at each optimizer step'' --- Slowing Down the Weight Norm Increase in Momentum-based Optimizers \cite{DBLP:journals/corr/abs-2006-08217}}

\TD{``propose maintaining only the per-row and per-column sums of these moving averages, and estimating the per-parameter second moments based on these sums'' --- Adafactor: Adaptive Learning Rates with Sublinear Memory Cost \cite{DBLP:journals/corr/abs-1804-04235}}

\TD{NovoGrad --- Stochastic Gradient Methods with Layer-wise Adaptive Moments for Training of Deep Networks \cite{DBLP:journals/corr/abs-1905-11286}}


%
\TD{Lookahead Optimizer: k steps forwar \cite{DBLP:journals/corr/abs-1907-08610}}


\subsection{second-order}

\r{use the second derivative (in univariate optimization) or the Hessian (in multivariate optimization) to help guide the direction and step size of descent methods.}

\r{The second order information can be used to speed up convergence since it also helps determine the step size}


% TODO: read this paper for more missing citations
\TD{ADAHESSIAN: \cite{DBLP:journals/corr/abs-2006-00719}}

\subsubsection{Newton's Method}

\subsubsection{Secant Method}

\subsubsection{Quasi-Newton Method}

\r{approximate Newton's method when second-order information is not directly available.}

\section{Non-Parameterized}

\subsection{Direct methods}

\r{may also be called zero-oder, black box, pattern search, or derivative-free methods.}

\subsection{Stochastic methods}

\subsection{Population methods}

\subsection{Further optimization information}


\subsection{Parallelizing and distributing SGD}




\chapter{Losses}


\section{losses}

% TODO: not sure where this section belongs
% TODO: It would maybe make sense to have an appendix section that covers the common losses

% potentially ``irrelvant''/~uninterpretible in value to the researcher

% did I already write this somewhere
\r{often we use a loss as a proxy for our performance metric. Most often because the performance metric is not differentiable \textit{e.g.} accuracy results in a binary output. We may also have a different loss function from our performance measure because we wish to penalize/constrain the model during training in a way that is seperate from how we report results \textit{e.g.} using MSE and MAE}

\subsection{fit somewhere}

\subsubsection{Contrastive Losses}

\TD{link to secion on self-supervised learning}


\subsection{Discrete}

\r{classification}


\subsubsection{Cross-Entropy}

%TODO: https://colah.github.io/posts/2015-09-Visual-Information/

\TD{A mathematical theory of communication \cite{shannon1948mathematical}}


\r{cross entropy --- difference between two probability distributions. similar, but not the same as KL-divergence}

\r{entropy, information theory --- the average length of bits necessary to encode a distribution of events}

\r{if cross entropy is perfect, it will be equal to the entropy of the distribution itself (the intrinsic iunpredictability)}

\r{binary cross entropy (in the context of loss functions, may be called/interchanged with logistic loss / log loss, even though they are not the exact same.)}

% https://machinelearningmastery.com/cross-entropy-for-machine-learning/
\r{KL-divergence is the measure of ``\textbf{extra bits}'' need to encode an event from $q$, instead of $p$, where as cross-entropy is the \textbf{total number} of bits}

\r{cross entropy of itself will be the entropy}

\r{NOTE: cross entropy is not symmetric,\textit{i.e.}calculating the cross entropy of one distribution $p$ from another $q$, is different from the cross entropy of $q$ from $p$}


 %  + \textrm{}(\textrm{}) \\



\begin{equation}
	\begin{split}
		\textrm{cross-entropy(preds, targets) } & =  \textrm{entropy} (\textrm{preds}) + \textrm{kl\_divergence}(\textrm{preds, targets})\\
		& = -(\textrm{sum}(\textrm{pred}) \times \log (\textrm{pred}) )+ KL(\textrm{preds, targets})\\
		& =  -\sum_{i=1}^{n}p(x_i)\log p(x_i)+ KL(\textrm{preds, targets}) \\
		& =  -\sum_{i=1}^{n}p(x_i)\log p(x_i)+ sum(\textrm{preds} * \log ( \frac{\textrm{preds}}{\textrm{targets}} ) \\
		& =  -\sum_{i=1}^{n}p(x_i)\log p(x_i)+ \sum_{i=1}^{n}p(x_i)\log \frac{p(x_i)}{q(x_i)} 
	\end{split}
\end{equation}

\r{sometimes (more often) the entropy equation is the negative sum, but it can also be written as the following positive case (using the identity $ \log ( \frac{1}{a} )  = - \log (a) $)}

\begin{equation}
	\begin{split}
		\textrm{entropy} (\textrm{preds})  & =  -(\textrm{sum}(\textrm{pred}) \times \log (\textrm{pred}) )\\
		& = - \sum_{i=1}^{n}p(x_i)\log p(x_i) \\
		& =   \sum_{i=1}^{n}p(x_i)\log ( \frac{1}{p(x_i) } )
	\end{split}
\end{equation}


\r{if using $log_{10}$, the units are ``bits'', if using $log_2$, the units are ``nats''}

\r{binary cross-entropy referes to cross-entropy of two classes, whereas categrorical cross-entropy referes to the cross-entropy of multiple ($n$) classes (where $n > 2$)}

% label smoothing
\subsection{label smoothing}
\TD{Label smoothing}
\TD{When Does Label Smoothing Help? \cite{DBLP:journals/corr/abs-1906-02629}}
\TD{Regularizing Neural Networks by Penalizing Confident Output Distributions \cite{DBLP:journals/corr/PereyraTCKH17}}



\subsection{Continuous}

\r{training objective -- continuous}

\subsubsection{Losses}

\TD{ELBO (Evidence Lower BOund)}

% TODO: index
\TD{Squared logarithmic error (SLE) and Mean SLE (MSLE)}
\TD{Root Mean Squared logarithmic error (RMSLE) and Mean RMSLE (RMSLE)}
\TD{Mean Absolute Percentage Error (MAPE)}


\subsection{Distribution}





\input{./nested/basics/genetic_algorithms}

%%%%%%%%%%%%%%%%%%%%%%%% Evaluation
\chapter{Evaluation}


\r{Importance of dataset partitioning \textcolor{red}{local ref?}}

% \textcolor{blue}{The best performance measure will vary depending on the task. For instance, in a medical setting, it may be life threating to classify an event as ``healthy'' when the patient is not healthy.}

\r{A performance measure is used to capture, empirically, how well a prediction made by the model aligns with the expected, ground truth, value.}

\r{Evaluation metrics allow for intuitive explaination of the results to those who may be non/less-technical}

\subsection{Creating a Test Set}

% rough para
\r{The most important rule regarding evaluating models, is to ensure that the data used to evaluate the model has never been used before to influence the during training or selection -- this means it was not used during training to update the parameters and it was not used to influence which models are `best' (like a validation set may be used for)}

\r{The performance of a model on a test set may be indicative of how well the model can generalize to unseen data. (This assumes your data sample is representative of the data population)}

\r{Hold-out test set -- created by randomly sampling the dataset. Again, it is important to emphasize that the instances in the test set are never used in the training process and are instead reserved for use only during the evaluation phase.}

\r{peeking\index{peeking}, is an issue that arises when part or all of the test set is included in the training set. This means the model has already seen the data on which the model will be evaluated and so it is possible, probable, that the model will produce high evaluation scores, which will likely translate to an overoptimistic estimation of the models performance when used in production.}

\r{Evaluating the performance of a model can be challenging and will vary depending on the task. For instance, accuracy may not always be the best measure of performance -- consider a medical setting in which sensitivity may be more important since a false negative may be life threatening where as a false positive may only require additional observation.}

\r{When comparing various models, it may be challenging to rank them on a single performance measure. \TD{TODO: more.}}

\subsection{Qualitative Evaluation}

\r{generalization is a measure of how well the system preforms on previously unseen data. generalization error.}



\subsubsection{(Over$|$Under)fitting and Capacity}

\r{{Model capacity}\index{model capacity} helps control how likely a model is to overfit or underfit. Where a model with low capacity may have difficulty fitting a a training set and a model with high capacity may ``overfit'' the data by essentially memorizing the training data.}

\r{Model capcity is closely related to model complexity and the models {hypothesis space}index{hypothesis space} (The set of functions available to the learning algorithm --- \textcolor{green}{TODO: expand - for example a linear vs polynomial model})}

\TD{TODO: figure showing training and validation error and 1) optimal capacity, 2) under and overfitting region 3)generalization gap, 4) capacity}

\paragraph{Overfitting}

\r{Arguably, the most important consideration/challenge}

\r{Overfitting\index{Overfitting} refers to a case in which a model fits the training data very well (maybe ``too'' well) but does not fit validation/test set. If a model is overfitting, it is said to have a high variance and is analogous to memorizing the training set.}

\r{Overfitting can arise from modeling data with too many parameters/too complex of a model.}

\r{learning ``particularities in the training set''}

\TD{TODO: figure showing an example of overfitting}

% addressing overfitting: 1) reduce number of features (manual selection or w/model selection algor) 2) regularization

\TD{worth pointing out that even if the loss hasn't gone to zero/even if the model hasn't memorized ``everything'', it is still possible to have memorized \textit{some} samples.}

\begin{figure}[htp]
	\centering
	\includegraphics[width=0.3\textwidth]{example-image-a}\hfil
	\includegraphics[width=0.3\textwidth]{example-image-b}\hfil
	\includegraphics[width=0.3\textwidth]{example-image-c}\hfil
	\caption{Figure example showing the same 2d dataset and an underfitting, overfitting, and ``good'' fitting. \textcolor{green}{TODO} circles=training, x=test -- include scores for each.. slight curve' under=linear, over=extreme poly, good=``smooth''}
	\label{fig:basics_eval_fitting_examples}
\end{figure}

\paragraph{Underfitting}

\r{Underfitting\index{Underfitting} refers to a case in which a model does not fit the training data well. If a model is underfitting, it is said to have a high bias}

\r{Underfitting can arise from modeling data with too few parameters/too simple of a model.}

\TD{TODO: figure showing an example of underfitting}

\subparagraph{Solution}

\TD{method for better optimization and increasing model capacity: greedy layer-wise --- unsupervised pre-training}

\r{better optimization --- use better optimization methods \ALR}


\subsubsection{Bias Variance Trade-off}

\r{Two fundamental causes of prediction error in a model -- the bias and the variance.}

\paragraph{Variance}
\r{variance\index{Variance} refers to the amount the model would change (consistency or variability) if it was re-trained/estimated multiple using a different subsets of the training data set. A model that has high variance is sensitive to randomness in the training data}

\r{A model with high variance may be described as highly flexible and will likely overfit the data.}


\paragraph{Bias}
\r{Bias\index{Bias} refers to the amount of error that is introduced by approximating a problem with a model that is simpler than the (complex) problem}

\r{A model with high bias will produce similar errors for instances regardless of the training data that is used to train the model -- the model is more strongly ``biased'' to its own assumptions of the relationship (as defined by the model), than the relationship the data may be indicating. A model with high bias may also be described as inflexible and will likely underfit the data.}


% not word-for-word, but example adapted from p35 of ISL
\textcolor{red}{For example, linear regression assumes a linear relationship between the features and labels. However, it is unlikely that a true linear relationship exists and so using linear regression to model this type of particular problem will likely introduce some bias.}

\paragraph{Trade-Off}

% TODO: see page 34 of ISL for eq and explaination here

\r{In general, as a more ``flexible'' model is used, the variance will increase and the bias will decrease.}

\r{One reason to choose a more restrictive model is that they are often more interpretable.}

\begin{figure}[htp]
	\centering
	\includegraphics[width=0.4\textwidth]{example-image-a}\hfil
	\includegraphics[width=0.4\textwidth]{example-image-b}\hfil
	\caption{\TD{side by side figure: a: complex vs simple training on trainnig data (nth poly vs linear), b: same models on test data}}
	\label{fig:basics_eval_tradeoff_examples}
\end{figure}


% see page 36 of ISL
\r{It is easy to obtain a model with low bias but high variance (\emph{e.g.} drawing a squiggly line through every training observation) and it is easy to obtain a model with low variance but high bias (\emph{e.g.} drawing a straight line approximating every training observation) but it is difficult to obtain a model that has both low variance and low bias.}

\textcolor{blue}{It should be noted that in a real world example, it may not be possible to explicitly calculate the test error, bias, or variance.}



\begin{figure}[htp]
	\centering
	\includegraphics[width=0.3\textwidth]{example-image-a}\hfil
	\includegraphics[width=0.3\textwidth]{example-image-b}\hfil
	\includegraphics[width=0.3\textwidth]{example-image-c}\hfil
	\caption{\TD{same 2D dataset with 3 layers and the hidden layer in a has few nodes, b: normal amount of nodes, and c: many nodes} \r{illistrative that the number of connections and complexity increases the chances for overfitting also increases}}
	\label{fig:basics_eval_nodesinhidden}
\end{figure}


\begin{figure}[htp]
	\centering
	\includegraphics[width=0.2\textwidth]{example-image-a}\hfil
	\includegraphics[width=0.2\textwidth]{example-image-b}\hfil
	\includegraphics[width=0.2\textwidth]{example-image-c}\hfil
	\includegraphics[width=0.2\textwidth]{example-image-a}\hfil
	\caption{\TD{same 2D dataset with one, two, three and four hidden layers}}
	\label{fig:basics_eval_numlayers}
\end{figure}

\r{observing a direct trade-off between overfitting and model complexity.}

\r{When we talk about deep learning, we're talking about deep and powerful models that are attempting to solve complex problems that are prone to overfitting and thus usually employ additional countermeasures, such as regularization, to help prevent overfitting.}


\TD{TODO: para about using regularization here/finding the right balance \textcolor{red}{local ref to regularization?}}

\section{Evalution beyond aggregated score}

\TD{Slided Evaluation}


%%%%%%%%%%% Metrics (subsec nested under sec.Eval)
\section{Qualitative Evalutation: Performance Metrics}

% NOTE: the term for this section should be ``performance metrics'' that are metrics related to model performance

% performance metrics are typcially directly related to business goals

% TODO: this para needs to be merged with the prev section and moved to where it is decided it best fits
\emph{Cost} is frequently used interchangeably with loss. Technically, loss refers to the error on a single example and cost is the average of the loss across the entire training set.


\TD{Need to rethink this definition and placement.}
\r{here I'm defining (perhaps inappropriately) metics as not differentiable --- at least without any clever manipulations.  That is all ``losses'' (found in \TD{section}) could be considered metrics, but it is unlikely that the metrics in this section would be used as a loss function (again, without any clever manipulation \TD{ref more later}).}

% % https://towardsdatascience.com/evaluating-text-output-in-nlp-bleu-at-your-own-risk-e8609665a213
\TD{BLEU Score\cite{papineni2002bleu}}

\TD{case study on the importance of metrics -- disease example.}


\subsection{discrete}

\r{classification}


\subsubsection{Common Metrics}

\subsubsection{Confusion Matrix}
\textcolor{blue}{A confusion matrix (sometimes referred to as a table of confusion, or contingency table) XXXXXXXX}

%% Confusion matrix
\begin{table}
	\centering
	\begin{tabular}{l|l|c|c|}
		\multicolumn{2}{c}{}&\multicolumn{2}{c}{Ground Truth}\\ 
		\cline{3-4}
		\multicolumn{2}{c|}{}&Positive&Negative\\ 
		\cline{2-4}
		\multirow{2}{*}{\rotatebox{90}{Pred}}& Positive & $TP$ & $FP$ \\ 
		\cline{2-4}
		& Negative & $FN$ & $TN$ \\ 
		\cline{2-4}
	\end{tabular}
	\caption{Example confusion matrix}
	\label{tab:sample_conf_matrix}
\end{table}

\textcolor{blue}{From the confusion matrix:}

% TODO: index type-II and type-II
\begin{itemize}[noitemsep,topsep=0pt]
	\item \textit{TP (True Positive)}: ``hit'', correct positive prediction. The ground truth is positive and the prediction is positive.
	
	\item \textit{TN (True Negative)}: correct rejection. The ground truth is negative and the prediction is negative.
	
	\item \textit{FP (False Positive)}: False alarm or Type-I error\index{Type I error}. The ground truth is negative, but the prediction is positive.
	
	\item \textit{FN (False Negative)}: Miss or Type-II error\index{Type II error}. The ground truth is positive, but the prediction is negative.
\end{itemize}

\subsubsection{Classification Metrics}

\textcolor{blue}{The below measures of performance are calculated with the indicated equation with values obtained from the confusion matrix XXXXXXXX}


% TODO: these may belong in an appendix
\begin{itemize}[noitemsep,topsep=0pt]
	
%%%%%%%%%%%%%%%%%%%%%%%%%%%%%%%%%%%%%%%%%%%%%%%%%%%%%
\item \textit{Accuracy (ACC)}, (Eq.~\ref{eq:accuracy}): the ratio of correct predictions to the total number of predictions. \textcolor{blue}{this is typically the ``go to metric'', however, accuracy may give a false sense of XXXXX and is particularly not very informative if dealing with skewed (unbalanced data) --- see example in \textcolor{red}{local ref?}}

\begin{equation}
{\frac{TP+TN}{TP+TN+FP+FN}}
\label{eq:accuracy}
\end{equation}

%%%%%%%%%%%%%%%%%%%%%%%%%%%%%%%%%%%%%%%%%%%%%%%%%%%%%
\item \textit{Misclassification rate}, (Eq.~\ref{eq:misclassification_def}): \textcolor{blue}{the ``opposite'' of accuracy}.

\begin{equation}
{\frac{FP+FN}{TP+TN+FP+FN}}
\label{eq:misclassification_def}
\end{equation}


%%%%%%%%%%%%%%%%%%%%%%%%%%%%%%%%%%%%%%%%%%%%%%%%%%%%%
\item \textit{Sensitivity (recall, hit rate, true positive rate (TPR))}, (Eq.~\ref{eq:sensitivity}): the ratio of true positives that are correctly identified.

\begin{equation}
{\frac{TP}{TP+FN}}
\label{eq:sensitivity}
\end{equation}

%%%%%%%%%%%%%%%%%%%%%%%%%%%%%%%%%%%%%%%%%%%%%%%%%%%%%
\item \textit{Specificity (true negative rate (TNR))}, (Eq.~\ref{eq:specificity}): \textcolor{blue}{XXXXXXXXXX}.

\begin{equation}
{\frac{TN}{TN+FP}}
\label{eq:specificity}
\end{equation}

%%%%%%%%%%%%%%%%%%%%%%%%%%%%%%%%%%%%%%%%%%%%%%%%%%%%%
\item \textit{Precision (positive predictive value (PPV))}, (Eq.~\ref{eq:precision}): the ratio of positives that are, in fact, positive. If the classifier predicts positive, how often is is correct?

\begin{equation}
{\frac{TP}{TP+FP}}
\label{eq:precision}
\end{equation}

%%%%%%%%%%%%%%%%%%%%%%%%%%%%%%%%%%%%%%%%%%%%%%%%%%%%%
\item \textit{Negative Predictive Value (NPV)}, (Eq.~\ref{eq:npv}): \textcolor{blue}{XXXXXXXXXX}.

\begin{equation}
{\frac{TN}{TN+FN}}
\label{eq:npv}
\end{equation}

%%%%%%%%%%%%%%%%%%%%%%%%%%%%%%%%%%%%%%%%%%%%%%%%%%%%%
\item \textit{Miss Rate (False Negative Rate (FNR))}, (Eq.~\ref{eq:miss_rate}): \textcolor{blue}{XXXXXXXXXX}.

\begin{equation}
{\frac{FN}{FN+TP}}
\label{eq:miss_rate}
\end{equation}

%%%%%%%%%%%%%%%%%%%%%%%%%%%%%%%%%%%%%%%%%%%%%%%%%%%%%
\item \textit{False Positive Rate (FPR) (Fall-Out, false alarm rate)}, (Eq.~\ref{eq:fall_out}): \textcolor{blue}{XXXXXXXXXX}.

\begin{equation}
{\frac{FP}{FP+TN}}
\label{eq:fall_out}
\end{equation}

%%%%%%%%%%%%%%%%%%%%%%%%%%%%%%%%%%%%%%%%%%%%%%%%%%%%%
\item \textit{False Discovery Rate (FDR)}, (Eq.~\ref{eq:false_discovery}): \textcolor{blue}{XXXXXXXXXX}.

\begin{equation}
{\frac{FP}{FP+TP}}
\label{eq:false_discovery}
\end{equation}

%%%%%%%%%%%%%%%%%%%%%%%%%%%%%%%%%%%%%%%%%%%%%%%%%%%%%
\item \textit{False Omission Rate (FOR)}, (Eq.~\ref{eq:false_omission}): \textcolor{blue}{XXXXXXXXXX}.

\begin{equation}
{\frac{FN}{FN+TN}}
\label{eq:false_omission}
\end{equation}

%%%%%%%%%%%%%%%%%%%%%%%%%%%%%%%%%%%%%%%%%%%%%%%%%%%%%
% TODO: define harmonic mean somewhere
\item \textit{F-1 Score}, (Eq.~\ref{eq:f1_metric}): \textcolor{blue}{F1 is the \textcolor{red}{harmonic mean} of precision and sensitivity XXXXXXXXXX. The F1 score will penalize classifiers more as the difference between the precision and sensitivity increases.}.

\TD{harmonic mean is like taking average, but places emphasis on the lower number}

% F1 = 2 * (precision * recall) / (precision + recall)
% https://scikit-learn.org/stable/modules/generated/sklearn.metrics.f1_score.html

\begin{equation}
{\frac{2TP}{2TP+FP+FN}}
\label{eq:f1_metric}
\end{equation}

%%%%%%%%%%%%%%%%%%%%%%%%%%%%%%%%%%%%%%%%%%%%%%%%%%%%%
\item \textit{Matthews Correlation Coefficient (MCC)}, (Eq.~\ref{eq:mcc_metric}): \textcolor{blue}{MCC is  an alternative to the F1 score for evaluating binary classifiers. MCC is useful even when the ratio of class in the data is severely imbalanced. The output is $[-1,1]$, where 1 would be considered a perfect prediction and -1 an imperfect and 0 being random.}.

\begin{equation}
{\frac{TP \times TN - FP \times FN}{\sqrt{(TP + FP)(TP + FN)(TN + FP)(TN + FN)}}}
\label{eq:mcc_metric}
\end{equation}

%%%%%%%%%%%%%%%%%%%%%%%%%%%%%%%%%%%%%%%%%%%%%%%%%%%%%
\item \textit{Informedness (Bookmaker Informedness (BM))}, (Eq.~\ref{eq:informed_metric}): \textcolor{blue}{Informedness is the XXXXXXXXXX}.

\begin{equation}
{\frac{TP}{TP+FN}+\frac{TN}{TN+FP}-1}
\label{eq:informed_metric}
\end{equation}

%%%%%%%%%%%%%%%%%%%%%%%%%%%%%%%%%%%%%%%%%%%%%%%%%%%%%
\item \textit{Markedness (MK)}, (Eq.~\ref{eq:markedness_metric}): \textcolor{blue}{Markedness is the XXXXXXXXXX}.

\begin{equation}
{\frac{TP}{TP+FP}+\frac{TN}{TN+FN}-1}
\label{eq:markedness_metric}
\end{equation}


\end{itemize}

\subsubsection{AUC (Area Under the Curve)}

\TD{ROC (Receiver Operating Characteristics) curve --- predicting the probability of a binary outcome}
%TODO: index all metrics

\TD{create visualization for two distributions and create the ROC curve figure}

\begin{figure}[htp]
	\centering
	\includegraphics[width=0.3\textwidth]{example-image-a}\hfil
	\includegraphics[width=0.3\textwidth]{example-image-a}\hfil
	\includegraphics[width=0.3\textwidth]{example-image-a}\hfil
	\caption{\TD{AUC dists}}
	\label{fig:auc_dist}
\end{figure}

\begin{figure}[htp]
	\centering
	\includegraphics[width=0.3\textwidth]{example-image-a}\hfil
	\includegraphics[width=0.3\textwidth]{example-image-a}\hfil
	\includegraphics[width=0.3\textwidth]{example-image-a}\hfil
	\caption{\TD{three AUC curves}}
	\label{fig:auroc_curves}
\end{figure}

\TD{when considering a multi-class (e.g. $n$ class) model, $n$ curves can be produced using a OvR or OvA strategy}

\r{plot of ``positive rates'' --- that is where the False positive rate is on the x-axis and the true positive rate is on the y-axis}

False Positive Rate (FPR) (Fall-Out, false alarm rate)), (Eq.~\ref{eq:fall_out}) [x-axis] vs true positive rate (TPR) (sensitivity, recall, hit rate), (Eq.~\ref{eq:sensitivity}) [y-axis]

\begin{equation}
\textmd{FPR} \textmd{ vs } \textmd{TPR} =	{\frac{FP}{FP+TN}} \textmd{vs} {\frac{TP}{TP+FN}} 
	\label{eq:roc}
\end{equation}

\r{AUC is a single value representing the area under an ROC curve. Though generally referred to as the AUC, the term is correctly abbreviated AUROC, specifying that the curve is an ROC curve. The larger the auROC, the better. Useful metric for summarizing how the model is performing at different thresholds}

\TD{select the best threshold from ROC curve that gives the desired balance between false positives and false negatives.}

\TD{interpretation}
\begin{itemize}[noitemsep,topsep=0pt]
	\item $0$: --- The model is exceptionally bad (but if you flip the output, is it actually exceptionally good), likely something is wrong
	\item $0 - 0.5$: --- If no mistakes are made, the mode is doing worse than random guessing.
	\item $0.5$: --- The model is making random guesses
	\item $0.5 - 1$: --- interpretation here is highly specific to the problem being worked on. The model is doing better than random guessing but less than perfect.
	\item $1$: --- The inverse of the value $0$, the model is exceptionally good, likely something is wrong
\end{itemize}

\r{It's worth noting, that really, the distance from 0.5 is desired. That is a model that produces a $0.1$ is potentially better than a model that produces a $0.6$, that is because if the prediction was ``flipped'', the $0.1$ value would be $0.9$.}

\r{often used as an important metric for binary classification tasks, though isn't necessarily ``the'' metric of interest (\TD{see sec: ref --- sensitivity/precision may be more important})}


\subsubsection{Precision-Recall curve}
% TODO: read https://stats.stackexchange.com/questions/7207/roc-vs-precision-and-recall-curves

\TD{Diagram}

\r{choice of the threshold to use moving forward}

% https://machinelearningmastery.com/roc-curves-and-precision-recall-curves-for-classification-in-python/
\r{auROC curves may be more informative when there is roughly the same number of classes, whereas PR curves may be more informative when there is a large class imbalance.\cite{davis2006relationship} \TD{EXPLAIN WHY}}

\r{auROC \ALR used}

% TODO: continuous probability rank score.

\subsection{continuous}

\TD{does this section even belong here --- wouldn't this, by default, belong in the discrete section?}

\r{It is important to note that regression performance metrics must ignore the direction of the error, otherwise the positive and negative errors would cancel each other out and the overall score would appear artificially optimistic \textcolor{blue}{see local figure}. This is typically corrected for by either taking the absolute value or the square of the value. An important consideration will be how severely outliers should be penalized, as a squared component will result in a larger penalization than an absolute value.}

\textcolor{green}{todo: Figure showing $\pm$errors and how direction is important}

\subsubsection{Common Metrics}

 \subsubsection{Confusion Matrix}
\textcolor{blue}{A confusion matrix (sometimes referred to as a table of confusion, or contingency table) XXXXXXXX}

%% Confusion matrix
\begin{table}
	\centering
	\begin{tabular}{l|l|c|c|}
		\multicolumn{2}{c}{}&\multicolumn{2}{c}{Ground Truth}\\ 
		\cline{3-4}
		\multicolumn{2}{c|}{}&Positive&Negative\\ 
		\cline{2-4}
		\multirow{2}{*}{\rotatebox{90}{Pred}}& Positive & $TP$ & $FP$ \\ 
		\cline{2-4}
		& Negative & $FN$ & $TN$ \\ 
		\cline{2-4}
	\end{tabular}
	\caption{Example confusion matrix}
	\label{tab:sample_conf_matrix}
\end{table}

\textcolor{blue}{From the confusion matrix:}

% TODO: index type-II and type-II
\begin{itemize}[noitemsep,topsep=0pt]
	\item \textit{TP (True Positive)}: ``hit'', correct positive prediction. The ground truth is positive and the prediction is positive.
	
	\item \textit{TN (True Negative)}: correct rejection. The ground truth is negative and the prediction is negative.
	
	\item \textit{FP (False Positive)}: False alarm or Type-I error\index{Type I error}. The ground truth is negative, but the prediction is positive.
	
	\item \textit{FN (False Negative)}: Miss or Type-II error\index{Type II error}. The ground truth is positive, but the prediction is negative.
\end{itemize}

\subsubsection{Classification Metrics}

\textcolor{blue}{The below measures of performance are calculated with the indicated equation with values obtained from the confusion matrix XXXXXXXX}


% TODO: these may belong in an appendix
\begin{itemize}[noitemsep,topsep=0pt]
	
%%%%%%%%%%%%%%%%%%%%%%%%%%%%%%%%%%%%%%%%%%%%%%%%%%%%%
\item \textit{Accuracy (ACC)}, (Eq.~\ref{eq:accuracy}): the ratio of correct predictions to the total number of predictions. \textcolor{blue}{this is typically the ``go to metric'', however, accuracy may give a false sense of XXXXX and is particularly not very informative if dealing with skewed (unbalanced data) --- see example in \textcolor{red}{local ref?}}

\begin{equation}
{\frac{TP+TN}{TP+TN+FP+FN}}
\label{eq:accuracy}
\end{equation}

%%%%%%%%%%%%%%%%%%%%%%%%%%%%%%%%%%%%%%%%%%%%%%%%%%%%%
\item \textit{Misclassification rate}, (Eq.~\ref{eq:misclassification_def}): \textcolor{blue}{the ``opposite'' of accuracy}.

\begin{equation}
{\frac{FP+FN}{TP+TN+FP+FN}}
\label{eq:misclassification_def}
\end{equation}


%%%%%%%%%%%%%%%%%%%%%%%%%%%%%%%%%%%%%%%%%%%%%%%%%%%%%
\item \textit{Sensitivity (recall, hit rate, true positive rate (TPR))}, (Eq.~\ref{eq:sensitivity}): the ratio of true positives that are correctly identified.

\begin{equation}
{\frac{TP}{TP+FN}}
\label{eq:sensitivity}
\end{equation}

%%%%%%%%%%%%%%%%%%%%%%%%%%%%%%%%%%%%%%%%%%%%%%%%%%%%%
\item \textit{Specificity (true negative rate (TNR))}, (Eq.~\ref{eq:specificity}): \textcolor{blue}{XXXXXXXXXX}.

\begin{equation}
{\frac{TN}{TN+FP}}
\label{eq:specificity}
\end{equation}

%%%%%%%%%%%%%%%%%%%%%%%%%%%%%%%%%%%%%%%%%%%%%%%%%%%%%
\item \textit{Precision (positive predictive value (PPV))}, (Eq.~\ref{eq:precision}): the ratio of positives that are, in fact, positive. If the classifier predicts positive, how often is is correct?

\begin{equation}
{\frac{TP}{TP+FP}}
\label{eq:precision}
\end{equation}

%%%%%%%%%%%%%%%%%%%%%%%%%%%%%%%%%%%%%%%%%%%%%%%%%%%%%
\item \textit{Negative Predictive Value (NPV)}, (Eq.~\ref{eq:npv}): \textcolor{blue}{XXXXXXXXXX}.

\begin{equation}
{\frac{TN}{TN+FN}}
\label{eq:npv}
\end{equation}

%%%%%%%%%%%%%%%%%%%%%%%%%%%%%%%%%%%%%%%%%%%%%%%%%%%%%
\item \textit{Miss Rate (False Negative Rate (FNR))}, (Eq.~\ref{eq:miss_rate}): \textcolor{blue}{XXXXXXXXXX}.

\begin{equation}
{\frac{FN}{FN+TP}}
\label{eq:miss_rate}
\end{equation}

%%%%%%%%%%%%%%%%%%%%%%%%%%%%%%%%%%%%%%%%%%%%%%%%%%%%%
\item \textit{False Positive Rate (FPR) (Fall-Out, false alarm rate)}, (Eq.~\ref{eq:fall_out}): \textcolor{blue}{XXXXXXXXXX}.

\begin{equation}
{\frac{FP}{FP+TN}}
\label{eq:fall_out}
\end{equation}

%%%%%%%%%%%%%%%%%%%%%%%%%%%%%%%%%%%%%%%%%%%%%%%%%%%%%
\item \textit{False Discovery Rate (FDR)}, (Eq.~\ref{eq:false_discovery}): \textcolor{blue}{XXXXXXXXXX}.

\begin{equation}
{\frac{FP}{FP+TP}}
\label{eq:false_discovery}
\end{equation}

%%%%%%%%%%%%%%%%%%%%%%%%%%%%%%%%%%%%%%%%%%%%%%%%%%%%%
\item \textit{False Omission Rate (FOR)}, (Eq.~\ref{eq:false_omission}): \textcolor{blue}{XXXXXXXXXX}.

\begin{equation}
{\frac{FN}{FN+TN}}
\label{eq:false_omission}
\end{equation}

%%%%%%%%%%%%%%%%%%%%%%%%%%%%%%%%%%%%%%%%%%%%%%%%%%%%%
% TODO: define harmonic mean somewhere
\item \textit{F-1 Score}, (Eq.~\ref{eq:f1_metric}): \textcolor{blue}{F1 is the \textcolor{red}{harmonic mean} of precision and sensitivity XXXXXXXXXX. The F1 score will penalize classifiers more as the difference between the precision and sensitivity increases.}.

\TD{harmonic mean is like taking average, but places emphasis on the lower number}

% F1 = 2 * (precision * recall) / (precision + recall)
% https://scikit-learn.org/stable/modules/generated/sklearn.metrics.f1_score.html

\begin{equation}
{\frac{2TP}{2TP+FP+FN}}
\label{eq:f1_metric}
\end{equation}

%%%%%%%%%%%%%%%%%%%%%%%%%%%%%%%%%%%%%%%%%%%%%%%%%%%%%
\item \textit{Matthews Correlation Coefficient (MCC)}, (Eq.~\ref{eq:mcc_metric}): \textcolor{blue}{MCC is  an alternative to the F1 score for evaluating binary classifiers. MCC is useful even when the ratio of class in the data is severely imbalanced. The output is $[-1,1]$, where 1 would be considered a perfect prediction and -1 an imperfect and 0 being random.}.

\begin{equation}
{\frac{TP \times TN - FP \times FN}{\sqrt{(TP + FP)(TP + FN)(TN + FP)(TN + FN)}}}
\label{eq:mcc_metric}
\end{equation}

%%%%%%%%%%%%%%%%%%%%%%%%%%%%%%%%%%%%%%%%%%%%%%%%%%%%%
\item \textit{Informedness (Bookmaker Informedness (BM))}, (Eq.~\ref{eq:informed_metric}): \textcolor{blue}{Informedness is the XXXXXXXXXX}.

\begin{equation}
{\frac{TP}{TP+FN}+\frac{TN}{TN+FP}-1}
\label{eq:informed_metric}
\end{equation}

%%%%%%%%%%%%%%%%%%%%%%%%%%%%%%%%%%%%%%%%%%%%%%%%%%%%%
\item \textit{Markedness (MK)}, (Eq.~\ref{eq:markedness_metric}): \textcolor{blue}{Markedness is the XXXXXXXXXX}.

\begin{equation}
{\frac{TP}{TP+FP}+\frac{TN}{TN+FN}-1}
\label{eq:markedness_metric}
\end{equation}


\end{itemize}

\subsubsection{AUC (Area Under the Curve)}

\TD{ROC (Receiver Operating Characteristics) curve --- predicting the probability of a binary outcome}
%TODO: index all metrics

\TD{create visualization for two distributions and create the ROC curve figure}

\begin{figure}[htp]
	\centering
	\includegraphics[width=0.3\textwidth]{example-image-a}\hfil
	\includegraphics[width=0.3\textwidth]{example-image-a}\hfil
	\includegraphics[width=0.3\textwidth]{example-image-a}\hfil
	\caption{\TD{AUC dists}}
	\label{fig:auc_dist}
\end{figure}

\begin{figure}[htp]
	\centering
	\includegraphics[width=0.3\textwidth]{example-image-a}\hfil
	\includegraphics[width=0.3\textwidth]{example-image-a}\hfil
	\includegraphics[width=0.3\textwidth]{example-image-a}\hfil
	\caption{\TD{three AUC curves}}
	\label{fig:auroc_curves}
\end{figure}

\TD{when considering a multi-class (e.g. $n$ class) model, $n$ curves can be produced using a OvR or OvA strategy}

\r{plot of ``positive rates'' --- that is where the False positive rate is on the x-axis and the true positive rate is on the y-axis}

False Positive Rate (FPR) (Fall-Out, false alarm rate)), (Eq.~\ref{eq:fall_out}) [x-axis] vs true positive rate (TPR) (sensitivity, recall, hit rate), (Eq.~\ref{eq:sensitivity}) [y-axis]

\begin{equation}
\textmd{FPR} \textmd{ vs } \textmd{TPR} =	{\frac{FP}{FP+TN}} \textmd{vs} {\frac{TP}{TP+FN}} 
	\label{eq:roc}
\end{equation}

\r{AUC is a single value representing the area under an ROC curve. Though generally referred to as the AUC, the term is correctly abbreviated AUROC, specifying that the curve is an ROC curve. The larger the auROC, the better. Useful metric for summarizing how the model is performing at different thresholds}

\TD{select the best threshold from ROC curve that gives the desired balance between false positives and false negatives.}

\TD{interpretation}
\begin{itemize}[noitemsep,topsep=0pt]
	\item $0$: --- The model is exceptionally bad (but if you flip the output, is it actually exceptionally good), likely something is wrong
	\item $0 - 0.5$: --- If no mistakes are made, the mode is doing worse than random guessing.
	\item $0.5$: --- The model is making random guesses
	\item $0.5 - 1$: --- interpretation here is highly specific to the problem being worked on. The model is doing better than random guessing but less than perfect.
	\item $1$: --- The inverse of the value $0$, the model is exceptionally good, likely something is wrong
\end{itemize}

\r{It's worth noting, that really, the distance from 0.5 is desired. That is a model that produces a $0.1$ is potentially better than a model that produces a $0.6$, that is because if the prediction was ``flipped'', the $0.1$ value would be $0.9$.}

\r{often used as an important metric for binary classification tasks, though isn't necessarily ``the'' metric of interest (\TD{see sec: ref --- sensitivity/precision may be more important})}


\subsubsection{Precision-Recall curve}
% TODO: read https://stats.stackexchange.com/questions/7207/roc-vs-precision-and-recall-curves

\TD{Diagram}

\r{choice of the threshold to use moving forward}

% https://machinelearningmastery.com/roc-curves-and-precision-recall-curves-for-classification-in-python/
\r{auROC curves may be more informative when there is roughly the same number of classes, whereas PR curves may be more informative when there is a large class imbalance.\cite{davis2006relationship} \TD{EXPLAIN WHY}}

\r{auROC \ALR used}



\subsubsection{Additional Metrics}


\paragraph{Linear Evaluation}

\TD{TODO: overview of linear evaluation metrics}

% TODO: index
\r{Coefficient of Determination (R-squared ($R^2$)) quantifies how close the data is to a \textcolor{red}{hyperplane} -- a line in a 2-Dimensional space.}

\TD{Note that it is possible for R-squared to be negative.}


% TODO: really?
\textcolor{red}{Several methods exist to calculate R-squared}

% p42(30) of Mastering ML w/scikit
\textcolor{blue}{Pearson product-moment correlation coefficient (PPMCC), or {Pearson's R}\index{Pearson's R} results in a positive number between 0 and 1.}

\textcolor{blue}{NOTE: R-squared is particularly sensitive to outliers.}

\textcolor{blue}{R-squared can spuriously increase when features are added}

\paragraph{Distance Metrics}

\textcolor{blue}{There are four basic requirements for the distance metric:}

\begin{itemize}
	\item Non-negativity: the value must be greater or equal to 0
	\item Identity: if the distance metric between $a$ and $b$ is zero, the two values must be at the same location
	\item Symmetry: the distance metric from $a$ to $b$ must be the same as the distance metric from $b$ to $a$
	\item Triangular inequality: metric($a$,$b$) $\le$ metric($a$,$c$) $+$ metric($b$,$c$)
\end{itemize}

\textcolor{blue}{When calculating the nearest neighbors the terms \textit{distance} and \textit{similarity} may be used interchangeably -- it is important to keep in mind that though they are the ``same'', they are different terms in that the lowest value for distance is ``best'' and the highest value for similarity is ``best''.}

\textcolor{blue}{The default distance metric is the Euclidean distance}

\textcolor{blue}{both the Euclidean and Manhattan distances are special cases of the Minkowski distance}

% see p184 of FofMLforpred data analytics
\textcolor{green}{TODO: more about the Minkowski distance def here}

\textcolor{blue}{Minkowski-based Euclidean distance -- a straight line between two points (Eq~\ref{eq:euclidean_distance_def})}

\begin{equation}
{\sqrt{\sum_{i=1}^{m}{{(a[i] - b[i])}^2}}}
\label{eq:euclidean_distance_def}
\end{equation}


\textcolor{blue}{Manhattan distance (Eq.~\ref{eq:manhattan_distance_def}) -- may also be called the taxi-cab distance, since it is similar to how a driver would have to drive from one point to another on a grid based road system (\textit{e.g.} like Manhattan).}

\begin{equation}
{\sum_{i=1}^{m}{abs(a[i] - b[i])}}
\label{eq:manhattan_distance_def}
\end{equation}

\textcolor{blue}{When implementing a nearest neighbor using Euclidean distance, the feature space is partitioned into {Voronoi tessellation}\index{Voronoi tessellation}. New points are assigned to a {Voronoi region}\index{Voronoi region}.}

% see p214 of FofMLforpred data analytics
\textcolor{green}{TODO: More about other similarity measures}




\paragraph{Multi-label Classification}

\textit{Intersection over union (IOU)}, (Eq.~\ref{eq:iou_def}): \r{intersection over union, with perfect overlap, the value is equal to 1, with no overlap the value is 0.}.

\TD{show examples}

\begin{equation}
	{\textrm{IOU} = \frac{\hat{y} \cap y)}{\hat{y} \cup y}}
	\label{eq:iou_def}
\end{equation}


\TD{include additional metrics like JI, DC, others}

\TD{Jaccard Similarity}

% p125[113] of Mastering ML with SKL
\TD{Hamming Loss}





\subsubsection{Choosing the ``right'' metrics}

\textcolor{blue}{TODO: paras on choosing the right metrics -- need to consider balance, others}





%%%%%%%%%%%%%%%%%%%%%%%% Regularization
\chapter{Improving Generalizability}

\r{The methods shown in the upcoming sections aim to reduce overfitting. That is, these methods aim to prevent the model from becoming too specialized to the training dataset in hopes that it will generalize to data that it has not specifically seen during training (e.g from the ``test'' set).}

\r{By implementing some of these methods (e.g. reducing the model capacity), the model often has less ability to model the training set as well as it might otherwise be able to. This is ok, high performance on the test set is the ultimate goal.}

%  some of the methods aren't used before they are necessary \TD{section on determining overfitting}

% TODO: index overfitting
\r{overfitting: a practical definition may include observing the training loss to improve while the validation loss degrades. \TD{possibly mention \\cite{Nakkiran2020DeepDD}}}

\r{Overfitting --- too complex --- Occam's razor --- hypothesis with the fewest assumptions is best}

\r{A specific instance of improving generalization might be accounting for imblance. Either in the labels or in the features.  Section \ref{app_data_imbalance} discusses this topic and strategies in more detail.}

\r{Typicaly types of modifications that are made to improve generalization.}

\begin{itemize}[noitemsep,topsep=0pt]
	\item Data
	\begin{itemize}[noitemsep,topsep=0pt]
		\item Increase ammount of data
		\item Augmentation
		\item Sampling
	\end{itemize}
	\item Architecture --- Reduce complexity of model e.g. applying parameter constraints, and/or reduce overall number of parameters
	\begin{itemize}[noitemsep,topsep=0pt]
		\item Reduce complexity/number of parameters
		\item Ensembling
		\item Constraints
		\begin{itemize}[noitemsep,topsep=0pt]
			\item Directly on parameters
			\item Through additional losses/tasks
		\end{itemize}
	\end{itemize}
	\item Training Pattern
	\begin{itemize}[noitemsep,topsep=0pt]
		\item Early stopping
		\item Stochastic Behavior
	\end{itemize}
\end{itemize}


\section{Data}

\subsection{Data Collection}

\r{Arguably the best way to increase generalizability of a model is to train the model on more data. However, as readers may already be aware, this is not always easy. Collecting more data may not be time/cost effective, or even possible.}

\r{``free'' data in that the ``cost'' is minor computation}

\subsubsection{Data Labeling}

%TODO: later sections likely belong in an appendix

\r{Labeling unlabed data}

\begin{itemize}[noitemsep,topsep=0pt]
	\item semi-supervised
	\item active learning
	\item weak supervision
\end{itemize}

\paragraph{Semi-supervised}

\TD{label propagation}

\TD{Book~\cite{chapelle2010semi}}

\TD{using GANs: Improved Techniques for Training GANs~\cite{DBLP:journals/corr/SalimansGZCRC16}}

\TD{Temporal Ensembling for Semi-Supervised Learning~\cite{DBLP:journals/corr/LaineA16}}

\paragraph{Active Learning}

\TD{A Survey of Deep Active Learning~\cite{DBLP:journals/corr/abs-2009-00236}}

\TD{intelligently sample data. Select instances that would be most informative for training}

\TD{Intelligent sampling could use a few different methods}

\TD{life cycle could include: taking unlabeled data, using the active learning sampler to pick instances, using a human annotator for these points, then using this new labeled set for or in addition to the current training set for training}

\begin{itemize}[noitemsep,topsep=0pt]
	\item Margin Sampling
	\item Cluster Based Sampling
	\item Query-by-committee
	\item Region-based Sampling
\end{itemize}

\subparagraph{Margin Sampling}

\r{Select instances that are nearest to the decision boundary (margin) e.g. the most uncertain and train on these points}

\subparagraph{Cluster Based Sampling}

\r{sample from the well formed clusters}

\subparagraph{Query-by-Committee}

\r{train and ensemble of models and sample from the data points that the models disagree on.}

\subparagraph{Region-based Sampling}

\r{Run several algorithms (from above) on different portions of the space}


\paragraph{Weak Supervision}

\TD{Weak supervision: https://ai.stanford.edu/blog/weak-supervision/}

\TD{Snorkel: Rapid Training Data Creation with Weak Supervision~\cite{DBLP:journals/corr/abs-1711-10160}}


\subsection{Augmentation}

\r{Dataset augmentation is \textcolor{green}{TODO}}

\r{adds examples that are similar to real}

\TD{Usupervised data augmentation: UDA}

\r{Please note, augmentation must be done responsibly. For example, if performing digit recognition, it would not be wise to perform rotational or flip transformations on the data since, depending on the specific data, a 6, rotated 180 or flipped vertically may now appear as a 9.}


\r{invariances in the data}

\r{For specific techniques, see~\ref{app_aug_techniques}}

\TD{Beyond improving generalization, augmentation may be used in other contexts as well, such as in helping quantify uncertainty -- \TD{see ref ---\TD{Augmenting the test set. A simple augmentation (horizontal filliping) was performed on the test set in \cite{simonyan2014very} -- where the prediction of the original and augmented images are averaged to obtain the final output score.} }}


\subsection{Sampling}

\r{The line between the techniques described here and ``augmentation'' might be a little blurred, in that sampling might technically be considered a augmentation technique (and I'm not even sure ``sampling'' is the appropriate title). But the intended distinction is that in augmentation, we are diliberately altering something (e.g. the input data) and in sampling, we are altering the number of times an architecture sees a particular instance in a training dataset.}

\TD{see appendix section for methods}



\section{Architecture}

\section{Training Pattern}

\subsection{Early Stopping}

\r{see p.243 of DL, papers Bishop 1995 and Sjoberg and Ljung 1995}

% TODO: note about regularization --- the smaller the value, the stronger the regularization.


\subsection{Stochastic Behavior}

\subsubsection{Dropout}

\r{``Dropout'' as a node in a computational graph may be considered an architectural structure change, but the method itself affects the training pattern in possibly not obvious ways. }

% TODO: explain dropout

\r{Dropout -- ref original paper (Hinton? -- intuitive, inspired by bank -- that defrauding the bank would require cooperation between employees to defraud the bank \TD{cite})}.

\r{Dropout (proposed in ``Improving Neural Networks by Preventing Co-Adaption of Feature Dectors''~\cite{DBLP:journals/corr/abs-1207-0580}, and popularized by Nitish et.al in ``Dropout: a Simple Way to Prevent Nerual Networks from Overfitting''~\cite{JMLR:v15:srivastava14a}}

\r{It is important to note that dropout is only present during training. i.e. dropout does not occur during test/evaluation if using dropout in the ``standard way''. However dropout is occassionally used for evaluation in attempt to quantify model uncertainty \TD{CITATION}}

\r{keeps a neuron active by a hyperparameterized probability.}

\r{used in any/all neurons in the network (other than the output neruons).}

\r{think about where dropout is used. That is when you use dropout at any given nueron the upstream paths transversing that particular neuron are also affected (in this case, ``turned off''), as well downstream connections (but often only modified, not entirely turned off since they often still have other inputs) }

\r{Forces the network to learn mappings even in the absence of all the information, that is the network is forced to consider the values of other values and can't rely on a smaller number of values or groups of values. Said another way, the network is prevented from becoming too dependent on certain inputs or features.}

\r{In this way, dropout can be thought of as sort of an ensembling method. When dropout is in use during training, each loop technically produces a different network that is then trained for the given task. During the next loop, a different network is used. As Aurélien Géron~\cite{geron2019hands} describes, if you train for 10,000 training steps (where dropout is used), you will have likely (almost certainly) trained 10,000 different neural networks. It's true that each network is not indpendant (they share weights), but they are different. More generally, a network with $N$ activations with dropout present, there exist $2^N$ possible networks ($2$ since each activation/neuron/value can have either an `on` or `off` state.) and thus, the use of all of these networks at once can be considered an ensembling of sorts.}

\TD{create figure of this ensemble of many networks.}


% TODO: find recent paper I saw mentioned on twitter.... (4July) it may be in my pocket

\begin{figure}[htp]
	\centering
	\includegraphics[width=0.3\textwidth]{example-image-a}\hfil
	\includegraphics[width=0.3\textwidth]{example-image-b}\hfil
	\includegraphics[width=0.3\textwidth]{example-image-c}\hfil
	\caption{\TD{Graph of an example function including dropout. three separate training iterations and how the network changes}}
	\label{fig:regularization_dropout_overview_training}
\end{figure}

\begin{figure}[htp]
	\centering
	\includegraphics[width=0.3\textwidth]{example-image-a}\hfil
	\caption{\TD{Same graph during test --- no dropout applied}}
	\label{fig:regularization_dropout_overview_test}
\end{figure}

\r{It is worth pointing out that since dropout is only applied at training time, comparing the loss curve of training and inference (validation splits) will be a bit misleading since the full ensemble network is used for calculating the validation loss/metrics and only the component \TD{is there a better word than this?} networks are used for the training set.}

\r{Additionally, if you run the training set through multiple times, you may find slightly different results. Again, this is because while dropout is on, you'll find that a slightly different network is used. \TD{This idea can be exploited at inference time to get uncertainty estimates.}}

\r{some important notes about the implementation. The outputs at test time should be equivalent to their expected outputs at training time (which is altered due to the application of dropout).}

\r{Couple solutions}
\begin{itemize}[noitemsep,topsep=0pt]
	\item scale the outputs during inference
	\item
\end{itemize}

\r{One potential solution to this problem is to scale the outputs during inference in a way that compensates for the dropout probability.  For example, if the dropout rate was set to $0.5$, then it would become necessary to halve the neurons outputs at test time in order to keep the expected output the neurons have learned during training.  However, this may not be ideal in practice since it would require scaling all the neuron outputs at test time (where performance is often critical and more important).}

\r{at test time, multiply the values by the expectation, not the on/off mask}

\r{Another, perhaps more desirable solution, would be to use \IDI{inverted dropout}. The cs231n~\cite{cs231n} course provides a concise explaination and example code on this topic.}

\r{This applies the same principal as outlined above, only the scaling occurs at training time rather that at test time. That is, during training, any neuron whose activation was not turned off, has the output divided by the dropout rate before being propagation to the next layer.  This way, at test time, no scaling is required.}

% helps learn ``multiple paths''/simulates ensembles
\TD{link to ensemble section}

\subsubsection{Others}

\TD{``during training, for each mini-batch, randomly drop a subset of layers and bypass them with the identity function'' --- Deep Networks with Stochastic Depth \cite{DBLP:journals/corr/HuangSLSW16}}

\TD{DropConnect~\cite{wan2013regularization} is similar to dropout, except that individual weights are disabled, not entire individual nodes and can be considered a generalization of dropout.}

\TD{figure showing difference}

% `drop block''?
\TD{investigate more structured dropout.}


\TD{structured --- ``contiguous region of a feature map are dropped together'' DropBlock  \cite{DBLP:journals/corr/abs-1810-12890}}


\TD{alpha dropout\cite{DBLP:journals/corr/KlambauerUMH17}}



\subsection{Parameter Regularization}

\r{Collection of techniques used to help generalize a model -- which may help prevent overfitting. Typically regularization penalizes complexity of a model.}


% TODO: figure of loss plot showing a steep training and shallow+divergent val/test loss

\r{imposes a penalty on the parameters}

\r{Helps prevent the model from memorizing noise in the training data.}

\r{Discourages the learned mapping/function/model from becoming too complex}


\subsubsection{Types of Regularization}

\textcolor{blue}{Regularization is an active area of research.}

% more information on L1/L2 http://www.chioka.in/differences-between-l1-and-l2-as-loss-function-and-regularization/

\begin{itemize}[noitemsep,topsep=0pt]
	\item Early Stopping (implementation: \textcolor{red}{local ref})
	\item Parameter Norm Penalties (implementation: \textcolor{red}{local ref})
	\begin{itemize}[noitemsep,topsep=0pt]
		\item L1 (Lasso) Regularization
		\item L2 (Ridge) Regularization
		\item Elastic Nets
	\end{itemize}
	\item Dataset Augmentation (implementation: \textcolor{red}{local ref})
	\item Noise Robustness
	\item Sparse Representations
	\item Dropout (implementation: \textcolor{red}{local ref})
	\item Ensemble methods (implementation: \textcolor{red}{local ref})
	\item Adversarial Training
\end{itemize}



\subsubsection{Parameter Norm Penalties}

\r{key difference is the penalty term}

\TD{TODO: DIGRAM OF L2 + L1 + elastic nets}

\paragraph{L2 Regularization}

\TD{TODO: DIAGRAM OF L2}

\r{L2, ({Ridge regression}\index{Ridge regression}) may also be known as {Tikhonov regularization}\index{Tikhonov regularization}}

\r{penalizes model parameters that become too large. Will force most of the parameters to be small, but still non-zero}

\r{square of the absolute value of the coefficient}

\begin{figure}[htp]
	\centering
	\includegraphics[width=0.3\textwidth]{example-image-a}\hfil
	\includegraphics[width=0.3\textwidth]{example-image-b}\hfil
	\includegraphics[width=0.3\textwidth]{example-image-c}\hfil\\
	\medskip
	\includegraphics[width=0.3\textwidth]{example-image-a}\hfil
	\includegraphics[width=0.3\textwidth]{example-image-b}\hfil
	\includegraphics[width=0.3\textwidth]{example-image-c}\hfil
	\caption{\TD{Top: NN output decision boundary on 2D dataset Bottom: weight params distribution from tensorboard... from LtoR = same arch with varying degrees of L2 regularization (0.01, 0.1 and 1.0)}}
	\label{fig:basics_regularization_l2_example}
\end{figure}


% p91(71) of mastering ML w SKL says "when lambda is equal to zero, ridge regression is equal to linear regression"

\paragraph{L1 Regularization}

\TD{TODO: DIAGRAM OF L1}

\r{LASSO (\textbf{L}east \textbf{A}bsolute \textbf{S}hrinkage and \textbf{S}election \textbf{O}perator) --- produces sparse parameters. This will force coefficients to zero and cause the model to depend on a small subset of the features.}

\r{absolute value of the weight coefficient}

\r{use only a small subset of the input features and can become resistant to noisy inputs.}

\r{It could be argued that using L1 regularization may help to make a model more interpretable, by using less (presumably more important/relevant) features when making predictions.}

\r{The use of L1 regularization for feature selection}


\paragraph{Elastic Net Regularization}

\r{Linearly combines the $L^1$ (feature selection) and $L^2$ (generalizability) penalties used by both LASSO and ridge regression. The cost is having two parameters (as opposed to just one when using either L1 or L2).}

\TD{TODO: figure}.



\subsection{Ensemble Methods}

\r{see \textcolor{red}{local ref} for more information on ensemble basics and see \textcolor{red}{local ref} for implementation details.}

% TODO: find Breiman 1994 paper referenced in p249 of Deep Learning
\r{As described in \textcolor{red}{local ref} ensemble methods act as a form of regularization by combining several different models \TD{Breiman 1994}. This often improves generalizability since the included models will often make independent, different, errors on the data.}

\subsection{Adversarial Training}



\subsection{Transfer Learning}

%TODO: read this survey
\TD{A Survey on Deep Transfer Learning \cite{DBLP:journals/corr/abs-1808-01974}}

\TD{How transferable are features in deep neural networks? \cite{DBLP:journals/corr/YosinskiCBL14}}
\TD{CNN Features off-the-shelf: an Astounding Baseline for Recognition \cite{DBLP:journals/corr/RazavianASC14}}

% TODO: haven't read this one (I don't think), but looks relevant
\TD{Learning and transferring mid-level image representations using convolutional neural networks\cite{oquab2014learning}}
\TD{Pay attention to features, transfer learn faster CNNs\cite{wang2019pay}}

% TODO: is this talked about anywhere else? this is probably the best place for it.

\TD{TODO: transfer learning, using -- explanation}

\TD{tool that may sometimes be efficient way of getting to potentially more accurate approximations, faster. \TD{citations}}

% TODO: index
\r{using parameters or pre-trained components from a model/task for a new model/task.  In practice, this often amounts to running inputs through a network that has been previously trained, and obtaining ``embeddings'' from this model (sometimes at an abitrary layer in the network), and then using these ``embeddings'' as input to train an additional model on the desired task. The process of adapting these components to a new model/task is called fine-tuning}


\begin{figure}[htp]
	\centering
	\includegraphics[width=0.5\textwidth]{example-image-a}\hfil
	\caption{Figure example layer hierarchy and where/when to transfer/freeze params -- this will be 1-2 figures and include many sub-figures \textcolor{green}{TODO}}
	\label{fig:transfer_learning_subfigs_a}
\end{figure}

\textcolor{green}{{freezing}\index{freezing} parameters or a layer means preventing the parameters from being updated during training. This is often controlled by a parameter called ``trainable''.}

% In relation to transfer learning and freezing, mention the difficulty of propagating updates though a large network

\TD{Scaling Laws for Transfer \cite{DBLP:journals/corr/abs-2102-01293}}

\r{One difficulty of fine tuning is knowing where and by how much to either freeze or learn. That is should you freeze the first $n\%$ of the network, why not $m\%$?. Maybe you should leave the entire network trainable? But if the entire network is trianable, the previously learned (and presumably useful features), may be erased by the updates. Aside from selecting where to make the distinction, the main method used to combat these issues is to modify the learning rate. There are two core methods to adjusting the learning rate to address these issues.}

\begin{itemize}[noitemsep,topsep=0pt]
	\item Learning rate schedule
	\item Layer-wise learning rates
\end{itemize}

\TD{These methods are described in more detail in section ~\ref{hp_learning_rate}}


\TD{Adversarially robust transfer learning \cite{DBLP:journals/corr/abs-1905-08232}}

% TODO: check this paper out
\TD{DT-LET: Deep Transfer Learning by Exploring where to Transfer \cite{Lin2020DTLETDT}}

\subsubsection{Potential downsides of TL}

\TD{biases, attacks}

\TD{A Target-Agnostic Attack on Deep Models: Exploiting Security Vulnerabilities of Transfer Learning \cite{DBLP:journals/corr/abs-1904-04334}}

% TODO: this likely does not belong here...
\subsection{Normalization}

% TODO: Read this
\TD{Evolving Normalization-Activation Layers \cite{DBLP:journals/corr/abs-2004-02967}}

\TD{TODO: overview para + importance}

\TD{TODO: figure showing differences}

\paragraph{Instance normalization}

\r{see section in preprocessing \textcolor{red}{local ref?}}

\paragraph{Layer normalization}

\TD{Layer Normalization \cite{Ba2016LayerN}}

\paragraph{Batch normalization}

% TODO: Read this
\TD{Training BatchNorm and Only BatchNorm: On the Expressive Power of Random Features in CNNs \cite{DBLP:journals/corr/abs-2003-00152}}

\TD{Show / explain}

\TD{Batch Normalization: Accelerating Deep Network Training by Reducing	Internal Covariate Shift \cite{DBLP:journals/corr/IoffeS15}}

\r{similar to dropout \ALR, the behavior of batch norm is different at training time and inference time.}

\r{normalizes values across a batch of data. Where the normalization is controlled by two learned parameters. The ``center'' and ``scale''.}

\r{Standard implementation is to calculate the population values using an exponential moving average (EMA).}

%TODO: here!

\TD{{Rethinking "Batch" in BatchNorm}~\cite{Wu2021RethinkingI} concludes that using EMA as the method for calculating the population statistics is not ideal. They show that during the early epochs, the xxxxxxx.}

\r{An adaptive re-parameterization.}

\r{reduce sensitivity to hyperparameterization.}

\TD{TODO: transfer learning considerations --- will likely have to unfreeze these params}

% HUGO talk
\r{``making the optimization easier''. batch norm is not effective in RNNs -- more so layer norm}

\r{seems to help when both under and over fitting.}

\r{order, up for debate and often described as either pre-activation operation, then activation, then batch norm, or pre-activation operation, then batch norm, then activation.}

\r{$\gamma$ and $\beta$ parameters that are learned parameters. These params could effectively undo the normalization caused (if ``learned'' to do so.)}


\begin{enumerate}[noitemsep,topsep=0pt]
	\item batch statistics
	\begin{itemize}[noitemsep,topsep=0pt]
		\item mean
		\item variance
	\end{itemize}
	\item normalize the pre-activation
	\item $\gamma$ and $\beta$ --- learned rescalling
\end{enumerate}

\TD{Rethinking "Batch" in BatchNorm \cite{DBLP:journals/corr/abs-2105-07576}}

% TODO: haven't read this paper yet 9Oct21 (I don't think...)
\TD{How Does Batch Normalization Help Optimization? \cite{Santurkar2018HowDB}}


% Graham Taylor talk
\begin{itemize}[noitemsep,topsep=0pt]
	\item turn down other regularization
	\item fixes first and second moments which may suppress information in these moments.
\end{itemize}

\TD{work related to adversarial spheres. --- with batch norm, the result was more reflective of the batch, not the entire dataset (which makes sense, right?)}


\paragraph{Group normalization}

\TD{Group Normalization \cite{DBLP:journals/corr/abs-1803-08494}}


\section{Output regularization}

\r{confidence penalty on predictions that are extrememly confident\cite{pereyra2017regularizing}. Originally an RL idea to promote expoloration. In SL, we would prefer fast convergence i) anneal confidence penalty ii) only penalize at a certain confidence threshold (lower entropy threshold). Intuitive (or not), can improve generalization.}

%TODO:
\r{label smoothing\cite{szegedy2016rethinking}}

\r{Adding label noise\cite{xie2016disturblabel}}

\r{smooth labels -- either via a ``teacher model''\cite{hinton2015distilling} or using it's own distribution\cite{reed2014training}}

\r{virtual adversarial training\cite{miyato2018virtual}}


%%%%%%%%%%%%%%%%%%%%%%%% Distributed
\input{./nested/basics/distributed}

\input{./nested/basics/federated}

\part{Algorithms}

\chapter{Foundational Methods}

%% Maybe (Foundational Methods --- supervised)

%%%%%%%%%%%%%%%%%%%%%%%%%%%%%% Regression
\section{Regression}

\textcolor{blue}{Three cases of the {generalized linear model}\index{generalized linear model}, simple, multiple, and polynomial linear regression.}

\textcolor{blue}{TODO: define generalized linear model}

\subsection{Simple Linear Regression}

% C2 of Mastering ML
\textcolor{blue}{Model a \emph{linear} relationship between a response variable and a feature representing an explanatory variable. The relationship is modeled with a linear surface called a hyperplane \ALR.}

\textcolor{blue}{Simple linear regression consists of two total dimensions (a dimension for the response variable and another for the explanatory) -- the hyperplane, as explained above, has one dimension (line)}

\textcolor{blue}{May also be called univariate regression (one variable).}

\textcolor{red}{convex loss function}

\textcolor{red}{history: ``method of least squares'', Legendre and Gauss, astronomy --- 1936 Fisher proposed ``linear discriminant analysis) --- 1940s (various authors - logistic regression) --- 1970s (Nelder and Wedderburn ``generalized linear models'' == entire class of statistical learning methods) (TODO: verify, find papers, ISLRp6) --- 1980s (Breinman, Friedman, Olshen, and Stone) == ``classification and regression trees''. 1986 (Hastie and Tibshirani == ``generalized additive models'' == non-linear extensions to generalized linear models)}

\begin{equation}
{Y \approx \beta_0 + \beta_1 x}
\label{eq:slr_ex}
\end{equation}

\textcolor{blue}{$\approx$ can be read as ``\emph{is approximately modeled as}''. $Y$ is a quantitative response (output/prediction) and $X$ predictor variable(input/feature). $\beta_0$ and $\beta_1$ are two unknown constants representing the intercept and slope, respectively. These unknown values that determine the behavior of the model are known as the model \emph{parameters} or \emph{coefficients}}

\subsubsection{OLS}

\TD{weighted sum of features plus a bias}


\begin{equation}
	\begin{split}
		\hat{y} & = \theta + \theta_1 x_1 + \theta_2 x_2 + ... + \theta_n + x_n  \\
		  \textrm{pred} & = \textrm{bias} + \textrm{feature weight} * \textrm{feature value}
	\end{split}
\end{equation}

\textcolor{blue}{{Ordinary Lease Squares (OLS)}\index{Ordinary Lease Squares (OLS)}, or {Linear Least Squares}\index{Linear Least Squares} is a method for estimating the parameters for a simple linear regression model.}

\textcolor{blue}{Solving OLS for simple linear regression ($y=\beta_0 + \beta_1 x$).}

\textcolor{blue}{First we'll solve for the slope $\beta_1$, where $\beta_1$ is can be found using Eq.\ref{eq:slr_ols_slope}.}

\begin{equation}
{\beta_1 =  \frac{cov(x,y)}{var(x)}}
\label{eq:slr_ols_slope}
\end{equation}

%% TODO: JACK -- these two def var and covar need to be moved

\textcolor{blue}{Variance (Eq.\ref{eq:variance_def}) is the measure of how far the set of values are spread apart -- if all the numbers in a set were equal, their variance would be zero.}

\begin{equation}
{var(x) = \frac{\sum_{i=1}^{n}(x_i - \hat{x})^2}{n-1}}
\label{eq:variance_def}
\end{equation}


\textcolor{blue}{Covariance (Eq.\ref{eq:covariance_def}) is the measure of how much two variable change together -- if two variables increase together, their covariance is positive}

\begin{equation}
{cov(x) = \frac{\sum_{i=1}^{n}(x_i - \hat{x})(y_i - \hat{y})}{n-1}}
\label{eq:covariance_def}
\end{equation}

\textcolor{blue}{After solving for $\beta_1$, $\beta_0$ can be found by rearranging the original equation and \textcolor{red}{substituting in the means of $x$ and $y$}($y=\beta_0 + \beta_1 X$) to become Eq.\ref{eq:slr_ols_intercept}}

\begin{equation}
{\beta_0 =  \bar{y} - \beta_1 \bar{x}}
\label{eq:slr_ols_intercept}
\end{equation}

\subsubsection{Cost}

\textcolor{blue}{Cost or loss function (See \textcolor{red}{local ref?}) is used to define and quantitatively measure the error of the model -- the differences between the predicted and ground truth values. The differences between the training is called the residuals\index{residuals} or training errors where as the differences observed between the test predictions and ground truths are called the prediction or test errors.}

\textcolor{blue}{A common measure of the models fitness may be the {residual sum of squares (RSS)}\index{residual sum of squares (RSS)} (Eq.\ref{eq:rss_def}, where $y_i$ is the observed value and $f(x_i)$ is the predicted value)}

\begin{equation}
{\sum_{i=1}^{n}{(y_i - f(x_i))^2}}
\label{eq:rss_def}
\end{equation}

% see p62 of ISL for more

\subsubsection{Evaluation}

\textcolor{blue}{Several methods exist for measuring the models predictive capability (see \textcolor{red}{local ref?} for more details.)}


\subsection{Multiple Linear Regression}

\textcolor{blue}{Using $n$ predictors:}

\textcolor{blue}{generalization of simple linear regression. Uses multiple features to predict the response variable}

\textcolor{blue}{linear regression with multiple variables may be called "multivariate linear regression"}



\begin{equation}
{Y \approx \beta_0 + \beta_1 X_1 + \beta_2 X_2 + \cdots + \beta_n X_n}
\label{eq:mlr_ex}
\end{equation}


\subsection{Polynomial Regression}

% TODO: this needs to be written more clearly

\textcolor{blue}{Special case of multiple linear regression, models a linear relationship between a response variable and polynomial feature terms}

\textcolor{blue}{a linear model that, using polynomial feature terms, can model non-linear relationships.}

\textcolor{blue}{It is important to note that when representing features as polynomials, feature scaling becomes increasingly important. e.g. if a feature is on a 0-100 scale and the feature is cubed, the value is now on a 0-1000000 scale.}

\textcolor{blue}{Quadratic regression (second-order polynomial) shown in equation (EQ\ref{eq:quad_regression_def})}

\begin{equation}
{Y \approx \alpha + \beta_1 X + \beta_2 X^2}
\label{eq:quad_regression_def}
\end{equation}










%%%%%%%%%%%%%%%%%%%%%%%%%%%%%% Logistic Regression
\section{Logistic Regression}

\r{Despite the `regression' bit in the name, logistic regression (logit regression) is a classification model}

\r{Similar to linear regression \ALR, logistic regression computes the weightedf sum of the input features plus a bias term. However, rather than output the result directly, a logistic of the result is output. The logistic, is sigmoid \ALR that outputs a value between 0 and 1}

\r{estimates the probability that an instance $x$ belongs in a class}


\r{odds, or odds ratio\index{odds ratio} (Eq.~\ref{eq:odds_ratio}), where $p$ is representative of the probability of a positive (event we aim to predict) event and is defined as the probability of the event occuring divided by the probability of the event not occuring (see Eq~\ref{eq:odds_ratio}).  As an example, if the probability of an event happening it $10\%$, then the odds of the event happening are $\frac{0.10}{1-0.10} = {1:9}$}

\begin{equation}
{\frac{p}{1-p}}
\label{eq:odds_ratio}
\end{equation}

\r{A logit\index{logit} is the log of the odds of the event happening. (Eq.~\ref{eq:logit_def}) (log-odds)}

\begin{equation}
{logit(p)=\log{\frac{p}{1-p}}}
\label{eq:logit_def}
\end{equation}


\begin{figure}[htp]
	\centering
	    \includegraphics[width=0.33\textwidth]{example-image-a}\hfil
		\includegraphics[width=0.33\textwidth]{example-image-b}\hfil
	\caption{\TD{The logit value is on the range $-inf$ to $inf$ and sigmoid is on the range $0$ to $1$}}
	\label{fig:logit_vs_sigmoid}
\end{figure}

\r{logistic function (sigmoid function) (Eq.~\ref{eq:sigmoid_def}) -- the inverse of a logit function and corresponds to the probability that a certain sample belongs to a particular, positive, class. If the response variable value meets or exceeds the {discrimination threshold}\index{discrimination threshold}, the positive class is predicted. As described later, the \ALR{} softmax function is used to extend to multi-class}

\begin{equation}
{S(x)={\frac{1}{1+e^{-x}}}={\frac{e^x}{e^x+1}}}
\label{eq:sigmoid_def}
\end{equation}

% TODO: placement/link around sigmoid func
\textcolor{red}{regularization is important in logistic regression since the activation function will never reach zero and attempting to do so (e.g. longer training) can lead to weights being driven to $-inf$ or $+inf$. also, near the asymptotes, the gradient is quite small}

\begin{equation}
cost =	\left\{
	\begin{array}{ll}
		-\log (\hat{p}) & \textrm{if }  y = 1 \\
		-\log (1 - \hat{p}) & \textrm{if }  y = 0 \\
	\end{array} 
	\right.
\end{equation}


\begin{equation}
	\begin{split}
		\textrm{log loss} & =  \textrm{avg over all instances} ( \textrm{cost} ) \\
		& =  \frac{1}{m} \sum_{i=1}^{m} ( \textrm{cost} ) \\
		& =  \frac{1}{m} \sum_{i=1}^{m}(   (\textrm{target}) \times \textrm{cost}_ \textrm{true} +  (1 - \textrm{target}) \times \textrm{cost}_ \textrm{false} ) \\
		& =  \frac{1}{m} \sum_{i=1}^{m}( y^{(i)} \log ({\hat{p}}^{(i)}) +(1-y^{(i)}) \log (1 - \hat{p}^{(i)}) )
	\end{split}
\end{equation}

\r{unlike linear regression, there is no presently known closed form equation for computing the parameters that minimizes the cost function.}

\r{the equation for the partial derivatives is the same as for linear regression, only with the addition of the sigmoid}

\begin{equation}
	\begin{split}
		 \textrm{derivative}_ \textrm{partial} & =  \textrm{avg over all instances} ( \textrm{error}_\textrm{pred} *  \textrm{feature}) \\
		& =  \textrm{avg}((\sigma ( \textrm{pred}) -  \textrm{target}) *  \textrm{feature}) \\
		& = \frac{1}{m} \sum_{i=1}^{m}(\sigma ( \textrm{pred}) -  \textrm{target}) *  \textrm{feature} \\
		& = \frac{1}{m} \sum_{i=1}^{m}(\sigma ( \theta^T x^{(i)}) -  y^{(i)})) *  x_j 
	\end{split}
\end{equation}

\subsection{Softmax Regression}

%TODO: index

\r{Softmax regression which may also be called Multinomial Logistic Regression, creates multi-class prediction by predicting one class from $n$ classes.}

\TD{the softmax function effectively drives small values to/near zero and pushes large values toward 1 -- where the sum of all values is equal to 1.}

\r{Let's pretend we want to make predictions over multiple classes}

\r{The softmax function (may also be called the normalized exponential)}

\r{The cost function is similar to above, now only averaging over each class}

\begin{equation}
	\begin{split}
		\textrm{cross entropy cost} & =  \textrm{avg}_\textrm{instance}  \textrm{avg}_\textrm{class}( \textrm{cost} ) \\
		& =  \frac{1}{m} \sum_{i=1}^{m}  \frac{1}{k} \sum_{i=1}^{k}  ( \textrm{cost} ) \\
		& =  \frac{1}{m} \sum_{i=1}^{m}  \frac{1}{k} \sum_{i=1}^{k}  ( (\textrm{p. instance belongs to class k}) \log ({\hat{p}}^{(i)}_k) ) \\
		& =  \frac{1}{m} \sum_{i=1}^{m}  \frac{1}{k} \sum_{i=1}^{k}  ( y^{(i)}_k \log ({\hat{p}}^{(i)}_k) ) 
	\end{split}
\end{equation}

\r{in the equation above, it's worth noting that when $k$ is equal to $2$, the cost function is equivalent to The Logistic Regression cost function \ALR}


\r{\ALR cross entropy}


%%%%%%%%%%%%%%%%%%%%%%%%%%%%%% KNN
\input{./foundations/nearest_neighbor}

%%%%%%%%%%%%%%%%%%%%%%%%%%%%%% Support Vector Machines
\input{./foundations/svm}

%%%%%%%%%%%%%%%%%%%%%%%%%%%%%% Naive Bayes
\input{./foundations/naive_bayes}

%%%%%%%%%%%%%%%%%%%%%%%%%%%%%% Decision Trees
\input{./foundations/decision_trees}


\chapter{Artificial Neural Networks}

\textcolor{blue}{If a perceptron is analogous to a single neuron, an artificial neural network (either feedforward or feedback) would be analogous to a brain.}

\r{powerful and general framework for representing non-linear mappings (function approximation) from input features to outputs, where the form of the mapping is controlled by adjustable parameters (weights and biases). Determining the values for these parameters is the ``learning'' or training.}

%%%%%%%%%%%%%%%%%%%%%%%%%%%%%% perceptron
\input{./foundations/perceptron}

%%%%%%%%%%%%%%%%%%%%%%%%%%%%%% overview
\section{Artificial Neural Networks (ANN)}

\textcolor{blue}{Principals and basic feed forward networks}

\r{The most computationally expensive component is calculating the gradient of the loss function with respect to the parameters of the network}

% see page 233 of Understanding Machine Learning
\r{Artificial neural networks are {universal approximators}\index{universal approximators} -- \textcolor{red}{expand}}

\r{universal approximation theorem \textcolor{green}{(Hornik 1989, Cybenko, 1989)}. Regardless of the function that is attempted to being learned, a large MLP will be able to \textbf{represent} this function. However, it is not guaranteed that the large MLP, despite being a universal \textcolor{red}{approximator} capable of representing the function, is able to \textit{learn} the function}


\subsection{Multi-layer Perceptron}

\r{surprisingly/dangerously robust to bugs}

\r{Not a single multi-layer perceptron with multiple layers, rather it is a network composed of multiple layers of perceptrons. Multi-layer perceptrons, through use of sucessive transformations (multipel layers of adaptive weights) address some of the limitations presented with a single layer perceptron \TD{local ref}.  MLPs, even composed of just two layers, are capable of approximating any continuous functional mapping --- the restriction being that the network must be feed-forward (described in \TD{local ref}) ensuring the outouts are possible to calcualted from as explicit functions of the inputs.}

\r{universal approximator\cite{hornik1991approximation}}

\r{justification for deeper networks --- can be exponentially more compact.}

\subsection{Architecture}

\r{{input layer}\index{input layer}, {hidden layer}\index{hidden layer}, {output layer}\index{output layer}}

\r{the input layer is not counted in the number of layers in a network}

\begin{figure}[htp]
	\centering
	\includegraphics[width=0.5\textwidth]{example-image-a}\hfil
	\caption{\TD{TODO: diagram of neural network showing layers}}
	\label{fig:foundations_ann_overview}
\end{figure}


\subsection{Components}

\begin{figure}[htp]
	\centering
	\includegraphics[width=0.5\textwidth]{example-image-a}\hfil
	\caption{\TD{TODO: labeled diagram of nodes (weights and biases), connections, activation functions}}
	\label{fig:foundations_ann_overview}
\end{figure}


\subsubsection{Nodes / units}

\paragraph{Initialization}

\TD{TODO: initialization methods and for different layers}


\subsubsection{Activation Function}

\TD{TODO: I think this is where I'll talk about activation functions}

%% need for non-linearity
\r{If all the activation functions in the hidden layers of the network were to be linear then it is possible to create a equivalent network without the hidden units. This is due to the principle that the composition of successive linear transformations is itself a linear transformation \TD{show + detail more clearly}. \textcolor{red}{the activation functions of the hidden and output layers may be different.}}

\TD{TODO: step function to sigmoid function -- smoothed version of the step function -- can understand how an input changes the output.}

\r{When considering networks only consisting of threshold activations, we run into the {credit assignment}~\index{credit assignment problem} during training. That is we have no way of determining which of the hidden units is more/less responsible for the incorrect output.  A solution to this issue is to use differentiable activation functions, this then allows for the activation of the output to become differentiable functions of both the input variables and the parameters (weights and biases).}

%% TODO: placement
\r{A sigmoidal hidden unit can be used to approximate a hidden linear unit by scaling the input parameters (weights and biases) to be very small such that the values are small and lie on the linear part of the sigmoidal curve near the origin. Similarly a step function may be approximated by scaling the input parameters (weights and biases) to be very large such that the values are either in the activated or not activated state. Nearly any continuous functional mapping can be represetned by a network consisting of two layers of sigmoidal hidden units.  A network consisting of three or more sigmoidal hidden units can approximate any smooth mapping \TD{Lapedes and Farber 1988}}

\TD{local ref to a more in depth discussion of activation functions.}

%% this likely doesn't belong here
\begin{figure}[htp]
	\centering
	\includegraphics[width=0.3\textwidth]{example-image-a}\hfil
	\includegraphics[width=0.3\textwidth]{example-image-b}\hfil
	\includegraphics[width=0.3\textwidth]{example-image-c}\hfil
	\caption{\TD{TODO: three images of possible decision boundaries created by NN with threshold act.fn and 1,2, and 3 layers. one is a single linear hyperplane, 2 is a non convex and 3 is a disjoint}}
	\label{fig:foundations_ann_layers_decision_region}
\end{figure}

\r{networks having three or more layers of weights can create non-convex and disjoint decision regions. \TD{see Huang and Lippmann 1988 for examples of 2 layers.}. Networks with two layers are not capable of creating arbitrary decision boundries \TD{Gibson and Cownan 1990, Blum and Li, 1991} (also see \TD{fig ref}). However, if the activation function is converted to a sigmoidal activation, it is possible to arbitrarily closely approximate an given decision boundry.}

%%%%%%%%%%%%%%%%%%

\textcolor{blue}{Activation functions are XXXXXXXX}

\subsubsection{Why Non-linear}

\textcolor{blue}{Non-linear is necessary XXXXXXXXXX}


\subsubsection{Popular Activation Functions}

\r{Activation functions can be grouped into two main categories -- smooth and not smooth. Smooth activation functions (such as sigmoid) are differentiable at every point along the function where as the other activation functions are not differentiable at every location (relu).}

% history
%differentiable everywhere, monotonic, and smooth.


\r{linear (see above), }

\textcolor{blue}{ReLu, better because \textcolor{red}{help prevent saturation}, but still have problems \textcolor{red}{can "die" at 0.} }

\textcolor{blue}{ELU fuctions. they prevent the "dying" problem by being \textcolor{red}{non-zero} but their main drawback is that they are more computationally expensive due to the calculation of the exponent.}

\paragraph{Smooth Non-linear}

\subparagraph{Sigmoid}

\textcolor{blue}{The sigmoid\index{sigmoid} activation function.}

\textcolor{blue}{calibrated probability estimate}


% {{{act_smooth_sigmoid}}}
\begin{figure}
	\centering
	\includegraphics[width=0.65\textwidth]{./sync_imgs/act/smooth/sigmoid.png}
	\label{fig:act_smooth_sigmoid}
\end{figure}

% {{{act_smooth_tangent}}}
\begin{figure}
	\centering
	\includegraphics[width=0.65\textwidth]{./sync_imgs/act/smooth/tangent.png}
	\label{fig:act_smooth_tangent}
\end{figure}

\subparagraph{ELU}

\textcolor{blue}{\textcolor{red}{CITE}. Smooth, monotonic, and non-zero in the negative portion of the input. The main drawback is that they are more computationally expensive (due to calculating the exponential)}


\begin{equation}
{
	ELU = f(x) = \left\{
	\begin{array}{ll}
	\alpha(e^x - 1) x & \quad $for$ \ x < 0 \\
	x & \quad $for$ \ x \ge 0
	\end{array}
	\right.
}
\label{eq:act_elu_def}
\end{equation}


% {{{act_smooth_elu}}}
\begin{figure}
	\centering
	\includegraphics[width=0.65\textwidth]{./sync_imgs/act/smooth/elu.png}
	\label{fig:act_smooth_elu}
\end{figure}

% {{{act_smooth_selu}}}
\begin{figure}
	\centering
	\includegraphics[width=0.65\textwidth]{./sync_imgs/act/smooth/selu.png}
	\label{fig:act_smooth_selu}
\end{figure}


\subparagraph{Softplus}

\textcolor{blue}{continuous and differentiable at zero. However, due to the natural log and exponential function, there is added computation compared to th ReLU.}

% typcially discouraged in practice since ReLU achieves similar results and is less computationally expensive

\begin{equation}
{
	Softplus = f(x) = \ln{(1+e^x)}
}
\label{eq:act_softplus_def}
\end{equation}


% {{{act_smooth_softplus}}}
\begin{figure}
	\centering
	\includegraphics[width=0.65\textwidth]{./sync_imgs/act/smooth/softplus.png}
	\label{fig:act_smooth_softplus}
\end{figure}

% {{{act_smooth_softsign}}}
\begin{figure}
	\centering
	\includegraphics[width=0.65\textwidth]{./sync_imgs/act/smooth/softsign.png}
	\label{fig:act_smooth_softsign}
\end{figure}


\paragraph{Not Smooth Non-linear}

\subparagraph{ReLU}

\begin{equation}
{
	ReLU = f(x) = \left\{
	\begin{array}{ll}
	0 & \quad $for$ \ x < 0 \\
	x & \quad $for$ \ x \ge 0
	\end{array}
	\right.
}
\label{eq:act_relu_def}
\end{equation}

% {{{act_notsmooth_relu}}}
\begin{figure}
	\centering
	\includegraphics[width=0.65\textwidth]{./sync_imgs/act/notsmooth/relu.png}
	\label{fig:act_notsmooth_relu}
\end{figure}

\subparagraph{Leaky ReLU}

\textcolor{blue}{The Leaky ReLU (Eq~\ref{eq:act_leaky_relu_def}) was designed in attempt to address the dying ReLU issue \textcolor{red}{CITE}. Rather than simply outputting a zero in the negative range, the Leaky ReLU will will have a small non-zero slope (user specified) -- allowing weight updating and training to continue.}

\textcolor{green}{TODO: randomized Leaky ReLU \textcolor{red}{cite} --- $\alpha$ (from PReLU) is sampled from a uniform distribution randomly. The net-effect could be considered similar to drop out since, technically, there is a different network for each value of $\alpha$, resulting in an ensemble of sorts. At test time, the values for $\alpha$ are averaged.}

\begin{equ}[!ht]
	\begin{equation}
	{
		Leaky ReLU = f(x) = \left\{
		\begin{array}{ll}
		N x & \quad $for$ \ x < 0 \\
		x & \quad $for$ \ x \ge 0
		\end{array}
		\right.
	}
	\label{eq:act_leaky_relu_def}
	\end{equation}
	\caption{where $N$ is a constant. $N$ is typically set to 0.01}
\end{equ}

% {{{act_notsmooth_leakyrelu}}}
\begin{figure}
	\centering
	\includegraphics[width=0.65\textwidth]{./sync_imgs/act/notsmooth/leakyrelu.png}
	\label{fig:act_notsmooth_leakyrelu}
\end{figure}

\subparagraph{ReLU6}

\textcolor{blue}{In general, this function is referred to as a {ReLUN}\index{ReLUN} function, where $N$ is some constant. However, in practice, $6$, was determined to be the optimal value.\textcolor{red}{CITE}. \textcolor{red}{This capped value, may help learn the sparse values sooner.} By having the upper limit bounded, the prepare the network for a fixed point precision for inference --- if the upper limit is unbounded, then you may loose too many bits to \textcolor{red}{Q} portion of the fixed point number.}


\textcolor{blue}{Similar to the ReLU fuction, only the output is capped to six in the positive domain.}

\begin{equation}
{
	ReLU6 = f(x) = min{(max{(0,x)},6)}
}
\label{eq:act_ReLU6_def}
\end{equation}

% {{{act_notsmooth_relu6}}}
\begin{figure}
	\centering
	\includegraphics[width=0.65\textwidth]{./sync_imgs/act/notsmooth/relu6.png}
	\label{fig:act_notsmooth_relu6}
\end{figure}

\subparagraph{PReLU}

\begin{equ}[!ht]
	\begin{equation}
	{
		PReLU = f(x) = \left\{
		\begin{array}{ll}
		\alpha x & \quad $for$ \ x < 0 \\
		x & \quad $for$ \ x \ge 0
		\end{array}
		\right.
	}
	\label{eq:act_prelu_def}
	\end{equation}
	\caption{where $\alpha$ is a parameterized --- a learned parameter from training.}
\end{equ}

\r{Parametric Rectified Linear Unit (PReLU) \cite{he2015delving}}

\textcolor{blue}{$\alpha$, rather than being hard coded, is determined during training by the data. The logic being that the value would be more optimal than we could set \textcolor{red}{CITE}}

% {{{act_notsmooth_prelu}}}
\begin{figure}
	\centering
	\includegraphics[width=0.65\textwidth]{./sync_imgs/act/notsmooth/prelu.png}
	\label{fig:act_notsmooth_prelu}
\end{figure}


\TD{Self-Normalizing Neural Networks \cite{DBLP:journals/corr/KlambauerUMH17}}




%%%%%%%%%%%%%%%%


\subsection{Characterization}

\subsubsection{Types: Feed-forward vs Feedback}

\textcolor{blue}{Feed-forward --- Directed acyclic graph of artificial neurons. Feedback contain feedback connections that are fed back into itself. When feedforward are include these feedback connections, they become considered recurrent neural networks.}

\paragraph{Feed-forward}

\r{``general framework for representing non-linear functional mappings between a set of input variables and a set of output variables''}

\subparagraph{Layered networks}

\begin{figure}[htp]
	\centering
	\includegraphics[width=0.5\textwidth]{example-image-a}\hfil
	\caption{\TD{TODO: layered network diagram}}
	\label{fig:foundations_ann_layered_network}
\end{figure}

\r{Whereas a single layer network is composed of linear combination of input variables, that are then, transformed by a non-linear activation function, more general functions are creating layered networks that are composed of successive layers of processing units (adaptive weights) with connections running from every unit in one layer to every unit in the next.}

\subparagraph{General topologies}

\begin{figure}[htp]
	\centering
	\includegraphics[width=0.5\textwidth]{example-image-b}\hfil
	\caption{\TD{TODO: general topology}}
	\label{fig:foundations_ann_general_topology}
\end{figure}

\r{general topologies}

\paragraph{Feedback}

\subsubsection{Terminology}

\r{Considered \textit{networks} since they are typically composed of many different functions --- creating a ``network''.}

\r{Considered \textit{neural} since they are \textbf{loosely} inspired by neuroscience.}

\r{layer --- a layer may be considered a group of units that act in parallel. The layer will extract representations from the input, that are (in theory) more useful to the specific task.  Chaining together these layers results in a form of progressive \IDI{data distillation}.}

\r{Visible and Hidden Layers. Visible layers are called visible since they contain variables that are ``visible'', where as the hidden layers extract increasingly abstract features -- hidden since their values are not given in the raw data, but rather an output from a previous layer.}



\subsection{Learning: Backpropagation}

% see p196[184] of Mastering ML w/SKL
\TD{TODO: whoooo, this is going to be a big one. understand how each component contributes to the error and adjust accordingly.}

\r{popularized by \TD{Rumelhard, Hinton and Williams (1986)}, but similar ideas were discussed earlier by \TD{Werbos 1974}, and \TD{Parker 1985}}

\r{error backpropagation is used for evaluating the dervivatives of an error function with respect to the parameters (weights and biases) of the network}

\r{Iterative algorithm consisting of two main components --- the forward, then reverse, pass.}

% see p.141(156) - 146(161) of NNbishop
\TD{MORE}

\TD{Example}

\subsubsection{Forward pass}

\r{In the forward pass inputs are propagated through the network and derivatives of the error functuion, with respect to the parameters (weights and biases) are evaluated. Propagation o ferrors through the network, calculating the derivatives, can be applied to may different error functions.}

\r{it becomes important to use a computationally efficient method for evaluating these derivatives \TD{local ref}}

\r{During this stage is when the errors are propagated through the network.}

\subsubsection{Backward pass}

\r{In the Backward pass, the previously calcuated derivatives are used to compute the adjustments to the parameters --- propagated in reverse through the network (from cost function to input layer) and each node is updated -- \TD{TODO: expand}.}

\r{backpropagation\cite{alber2018backprop}}

\r{Many optimization schemes \TD{local ref} may be used to adjust the parameters by using the calculated derivatives from the forward pass.}

\r{The calculated derivatives are used by the majority of training algorithms}

%% p116 of neural networks, p131 on tablet
\TD{Hessian matrix is a matrix containing the second derivative of a function. The second derivative provides information about the curvature of the function.}

% see p197-201[180] of Mastering ML w/SKL
\TD{TODO: figure showing sample calculation}


% See p207 of DL



\r{\IDI{symbolic differentiation} --- compute a gradient function for the chain (chain rule) mapping parameter values to gradient values}

\subsubsection{Back-propagation efficiency}

%% see para in p146(161) of bishop NN

\subsubsection{Chain Rule}

\r{See \textcolor{red}{local ref to math prereq section}}

\TD{TODO: chain rule}

\r{Backpropagation is typically used with an optimization algorithm (see \textcolor{red}{local ref?})}

\subsection{Autodiff}

\TD{Autodifferentiation (\IDI{autodiff})}

\begin{itemize}[noitemsep,topsep=0pt]
	\item Manual differentiation
	\item Finite difference approximation
	\item Autodiff
	\begin{itemize}[noitemsep,topsep=0pt]
		\item Forward-mode autodiff
		\item Reverse-mode autodiff
	\end{itemize}
\end{itemize}

\TD{automatic differentiation may be used to estimate derivatives numerically as the derivative is not always known.}

\r{Tensorflow uses \IDI{reverse-mode differentiation}.  Calculate the contribution that each parameter had on the loss value}

\subsubsection{Manual differentiation}

\r{calculus ``by hand''}

\subsubsection{Finite difference approximation}

\r{using infinitely close points on a function to calculate the slope of line passing through these points.}

%TODO: very nice code example in hands on ML that is worth including here

\TD{specific equation, Newton's difference quotient}

\subsubsection{Forward-mode autodiff}

% TODO: same flaw as above, more accurate, but requires just as many passes -- which is ``unfeasible'' for large scale NN

\TD{dual numbers}

\subsubsection{Reverse-mode autodiff}

% TODO: only two passes needed

%%%%%%%%%%%%%%%%%%%%%%%%%%%%%% feedforward
\input{./foundations/feedforward}

%%%%%%%%%%%%%%%%%%%%%%%%%%%%%% feedback
\section{Feedback or Recurrent}

\textcolor{green}{TODO: Overview}

\r{RNNs or ``\textit{\textbf{r}}ecurrent \textit{\textbf{n}}eural \textit{\textbf{n}}etworks'' are used for a variety of purposes but are typically designed with sequences of data as an input in mind. They are similar in concept to a standard/feed-forward netowrk, with the major distinction being that they also have connections that point ``backwards'' i.e. they have connections that feed into themselves.}

\r{Are capable fo working on sequences of arbitrary lengths, rather than fixed-sized inputs}

\r{universal approximator~\cite{doya1993universality}}

\r{``superiority of gated models over vanilla RNN models is almost exclusively driven by trainability''~\cite{Collins2017CapacityAT}, `` Our results suggest that, contrary to common
	belief, the capacity of RNNs to remember their input history is not a practical limiting factor on their
	performance.'' ~\cite{Collins2017CapacityAT}}

% TODO: I'm not sure where the other citation for this is... but somewhere...
\TD{systematically removing pieces of an LSTM~\cite{DBLP:journals/corr/GreffSKSS15}}

\TD{An empirical exploration of recurrent network architectures~\cite{jozefowicz2015empirical}}
\TD{``Thus we recommend adding a bias of 1 to the forget gate of every LSTM in every application''~\cite{jozefowicz2015empirical}}
\TD{``if there are [RNN] architectures that are much better than the LSTM, then they are not trivial to find''~\cite{jozefowicz2015empirical}}


\subsection{Foundation}

\r{An example of an RNN diagram is shown in \TD{fig}. However, this representation is misleading since it does not show ``every'' connection in the model --- most notably, the recurrent connections.  RNNs may also be often represented in diagrams as ``unrolled'' (\TD{fig}). The unrolled RNN is easier to visualize how these recurrent connections are included.  This makes it easier to understand how each timestep is dependent on not only the current input (at the particular time step), but also dependent on ``all'' previous time steps. It is often stated that at a certain timestep (n), the output has ``memory'' since it is a function of all the previous time steps.}


\footnotetext{the term ``all'' is emphasized here since it is the goal to include information from all previous time steps. This is true in theory, however, this is not always the case in practice. This is discussed further in \ALR{}}

\subsection{Simple RNN and Recurrent Neuron}

\TD{Diagram of the inside of a RNN neuron}


\subparagraph{Overview}

\TD{todo}


\section{Common Problems}

Two well known main problems with RNNs.

\begin{enumerate}[noitemsep,topsep=0pt]
	\item Maintaining states are expensive
	\item Vanishing and/or exploding gradients
\end{enumerate}

\TD{hardware acceleration}

\subsection{Maintaining States}


\subsection{Addressing Vanishing and Exploding Gradients}

\r{Propagating signals through a long/deep network without loosing (vanishing gradient) or overamplifying the signal (exploding gradient) is difficult.  There have been a few advances to address this issue.}

\begin{enumerate}[noitemsep,topsep=0pt]
	\item Architecture (different cell types, memory schemes)
	\item Initialization Strategies
	\item Activation Function
\end{enumerate}



\chapter{Common Operations/Components}

% TODO: I'm still not sure how/where to structure this

\section{Dense}

\TD{TODO}

\section{Convolutions}

% this reads strangely --> DNN on an image may not take advantage of the ``stationarity'' (statistics) of an image.

\r{When using a standard dense layer, all inputs are treated independently. However, adjacent pixels, on average, are highly highly correlated. For example, if there is a texture in the image, a similar pattern of pixels may occur repeatedly. Convolutions architecturally build in an implicit spatial structure to consider these spatial.}

% TODO: I'm not sure how I'm going to structure these yet or where I'll be placing them

% TODO: https://arxiv.org/abs/1904.11486
% https://www.youtube.com/watch?v=HjewNBZz00w


\TD{LeNet-5 \cite{lecun1998gradient}}

\r{Convolutions are built upon a lie -- that is we refer to the opperation as a convolution, yet it is in fact a cross-correlation operation since we don't rotate the kernel 180$\deg$. However, it is convention to refer to the operation as a convolution. For more, please see section \ref{conv_vs_cross}}

\r{translational invariance --- a property that relates to how a systems decisions are insensitive to the location of a features within an input. That is, if we're looking for an object or feature, our system shouldn't change if the object is in different locations within the input}

\TD{``Filter factorization'' (not the exact same definition of mathematical factorization)-- one $5\times5$ filter vs $2$ $3\times3$ filters stacked.  in the $5\times5$ there are $5\times5 = 25$ parameters, in the $3\times3$, there are $3\times3 \times 2 = 18$ learnable parameters, resulting in a ``cheaper'' operation.}

\TD{Neocognitron -- CNN paper prior to ``CNN''\cite{fukushima1982neocognitron}}

% Survey on CNNs
% TODO: a lot here -- good read
\TD{A Survey of the Recent Architectures of Deep Convolutional Neural Networks \cite{DBLP:journals/corr/abs-1901-06032}}


\TD{Squeeze-and-Excitation Networks \cite{DBLP:journals/corr/abs-1709-01507}}


% Graham Taylor
\r{weighted averaging operation in time or space}


\r{translation equivariant --- }

\TD{BlurPool --- ``fix is anti-aliasing by low-pass filtering before downsampling'' ---Making Convolutional Networks Shift-Invariant Again \cite{DBLP:journals/corr/abs-1904-11486}}


\r{spatial hierarchies --- \TD{TODO: figure raw data, abstract edges+, then more distinct images, then closer output to the output, then the final label}}


\r{typcially a feature extraction phase (consisting of convolutional and pooling layers) followed by a classifier block (dense layers).}

%%%% popular layer types
\textcolor{green}{TODO: feature maps, (height, width, and depth (also called channels axis)). Stride, filter size, depth. talk about parameters}

\r{The output feature map (every dimension in the depth axis is a feature/filter) --- after a convolution operation the depth of a layer is no longer representative of a color channel (like RGB), it is now representative of a feature extracted by the convolutional operation, these are called filters.}

\TD{Strided Convolution\cite{springenberg2014striving}}

\TD{Dilated Convolution --- `atrous' convolution. (famously used by wavenet), which is convenient in time series analysis.}

\r{weight tieing}


\textcolor{green}{TODO: figure}

\begin{figure}[htp]
	\centering
	\includegraphics[width=0.5\textwidth]{example-image-a}\hfil
	\caption{Figure example of convolution operation on 2d image \textcolor{green}{TODO}}
	\label{fig:conv_2d_example_calc}
\end{figure}

\begin{figure}[htp]
	\centering
	\includegraphics[width=0.5\textwidth]{example-image-b}\hfil
	\caption{Figure example of convolution operation on 3d image \textcolor{green}{TODO}}
	\label{fig:conv_2d_depth_example_calc}
\end{figure}

\textcolor{green}{TODO: examples of how different filter values and strides can effect the output dimensions.}




\section{Pooling}

\TD{TODO: examples of max vs average pooling}

%%%%%% research
\textcolor{blue}{Pooling may not fully determine learned deformation stability -- possibly filter smoothness\cite{ruderman2018learned}}

\r{downsampling}

\r{Why? importance of reducing the number of params.}

\TD{L2-pooling}

\TD{L2-pooling over the features or channels.}

\TD{additional --- learned/parameterized pooling}

\begin{figure}[htp]
	\centering
	\includegraphics[width=0.5\textwidth]{example-image-a}\hfil
	\caption{Figure example of max pooling operation on 2d image \textcolor{green}{TODO: I want this figure to be basic 2d}}
	\label{fig:pooling_max_2d_ex_a}
\end{figure}

\begin{figure}[htp]
	\centering
	\includegraphics[width=0.5\textwidth]{example-image-b}\hfil
	\caption{Figure example of average pooling operation on 3d image \textcolor{green}{TODO: I want this figure to be 3d}}
	\label{fig:pooling_avg_3d_ex_a}
\end{figure}


\r{may be better to use convolutional layers in place of the pooling layers\cite{springenberg2014striving}}

\section{Recurrent Cells}

% TODO: read this
% Recurrent / Echo state networks / ESN
\TD{The ``echo state'' approach to analysing and training recurrent neural networks-with an erratum note \cite{jaeger2001echo}}
\TD{Deep Echo State Network (DeepESN): A \cite{DBLP:journals/corr/abs-1712-04323}}

\subsection{Cell Advancements}

\subsubsection{LSTM}

% TODO: Nice overview of LSTMs: https://colah.github.io/posts/2015-08-Understanding-LSTMs/

Introduced in 1997 %\cite{hochreiter1997long}

\r{detect long term dependencies in sequence}

\r{two state vectors, short and long term}

\r{Main motivation: learning what to store in the long-term state and what to ``forget''.}

\r{at each time step, some information is ``stored'' and some information is ``forgotten''.}

\paragraph{variants}

\TD{Depth-Gated LSTM \cite{DBLP:journals/corr/YaoCVDD15}}

\TD{A Clockwork RNN \cite{DBLP:journals/corr/KoutnikGGS14}}

\TD{LSTM: A Search Space Odyssey \cite{DBLP:journals/corr/GreffSKSS15} --- survey of LSTM variants --- all variants are essentially equal.}


\paragraph{other directions}

% interesting paper on ``grid LSTMs'' -- not sure why they never become popular
\TD{Grid Long Short-Term Memory \cite{Kalchbrenner2016GridLS}}

\paragraph{Fully Connected Layers}


\begin{enumerate}[noitemsep,topsep=0pt]
	\item Main
	\item \textit{Gate Controllers}
	\begin{enumerate}[noitemsep,topsep=0pt]
		\item Forget
		\item Input
		\item Output
	\end{enumerate}
\end{enumerate}

\r{The gate controllers use a logistic activation fuction (output a range from 0 to 1). This output is then fed through an element-wise multiplication function and thus if the value is $0$, the gate is ``closed'', and $1$ if the gate is ``open''.}

\r{These gates are able to potentially:}

\begin{enumerate}[noitemsep,topsep=0pt]
	\item Recognize an important input
	\item Store the important input in a long-term state ()
	\item Preserve the information for as long as it's needed
	\item Extract the important information when needed
\end{enumerate}


\subparagraph{Main}

\begin{figure}
	\centering
	\includegraphics[width=0.5\textwidth]{example-image-a}\hfil
	\caption{\TD{Main Layer DIAGRAM}}
	%\label{}
\end{figure}

\r{This allows for the same basic functionality as a ``standard'' RNN cell --- however, the output, rather than being only sent to the next cell, is now partially stored in the long-term state.}


\subparagraph{Forget}

\r{Determines which part of the long-term state is forgotten/erased.}

\begin{figure}
	\centering
	\includegraphics[width=0.5\textwidth]{example-image-a}\hfil
	\caption{\TD{Forget Layer DIAGRAM}}
	%\label{}
\end{figure}



\subparagraph{Input}

\r{Determines which part of the output from the \textbf{main layer} are kept in the long-term state.}

\begin{figure}
	\centering
	\includegraphics[width=0.5\textwidth]{example-image-a}\hfil
	\caption{\TD{Input Layer DIAGRAM}}
	%\label{}
\end{figure}

\subparagraph{Output}

\r{Determines which part of the long term state is ``relevant'' (read and output).}

\begin{figure}
	\centering
	\includegraphics[width=0.5\textwidth]{example-image-a}\hfil
	\caption{\TD{Output Layer DIAGRAM}}
	%\label{}
\end{figure}


\paragraph{Other}

\subparagraph{Peephole Connections}

\r{In basic LSTM cells, the gate controller can only look at the input and previous short-term state. Peephole connections, proposed in 2000 \TD{cite gers2000recurrent} add an extra connection that allows for the gate controller to also see information from the long term state as well. }

\r{The previous long-term state also becomes an input to the forget and input gate. The current long-term state becomes an intput to the output gate.}



\subsubsection{GRU}

\r{The GRU (gated recurrent unit) is a varient of the LSTM cell \TD{cite - cho2014learning}. The main modifications include:}

\begin{itemize}[noitemsep,topsep=0pt]
	\item Both state vectors are merged into one state vector
	\item A single gate controller determines the \textbf{Forget} and \textbf{Input} gate
	\begin{itemize}[noitemsep,topsep=0pt]
		\item If the gate output is a 1, the input is open and the forget gate is closed. If the gate output is 0, the input gate is closed and the forget gate is open
	\end{itemize}
	\item \r{The output gate is removed and a new controller exists that controls which part of ht previous state will be ``shown'' to the main layer}. At each timestep the full state vector is output.
\end{itemize}

\subsection{Notes -- add}

\r{A recent paper \TD{greff2017lstm}, compares three LSTM variants and makes three main observations:}

\begin{itemize}[noitemsep,topsep=0pt]
	\item no significant architecture improvements over LSTMs
	\item forget gate and the output activation function are the most critical components
	\item \TD{hyperparams...}
\end{itemize}




\section{Capsule Networks}

% TODO: capsule networks
\TD{Dynamic Routing Between Capsules \cite{DBLP:journals/corr/abs-1710-09829}}

\section{Attention}

\r{``An attention function can be described as mapping a query and a set of key-value pairs to an output,
	where the query, keys, values, and output are all vectors. The output is computed as a weighted sum
	of the values, where the weight assigned to each value is computed by a compatibility function of the
	query with the corresponding key.'' \cite{DBLP:journals/corr/VaswaniSPUJGKP17}}

\TD{Self-attention Does Not Need $O(n^{2})$ Memory~\cite{Rabe2021SelfattentionDN}}

%TODO: another blog to checkout https://distill.pub/2016/augmented-rnns/

\r{overview can be found here\cite{weng2018attention}}


\TD{The original attention mechanism is introduced\cite{Bahdanau2015NeuralMT}.}

% TF attention implementation (https://www.tensorflow.org/tutorials/text/nmt_with_attention)

\TD{Effective Approaches to Attention-based Neural Machine Translation \cite{DBLP:journals/corr/LuongPM15}}

\TD{Massive Exploration of Neural Machine Translation Architectures \cite{DBLP:journals/corr/BritzGLL17}}

% TODO: index for transformer
% 'self-attention'
\TD{Attention Is All You Need -- Transformer network --- multi-head self-attention mechanism, key-value pairs \cite{DBLP:journals/corr/VaswaniSPUJGKP17}}

% self-attention \TD{Self-attention, less commonly intra-attention}
\TD{Long Short-Term Memory-Networks for Machine Reading \cite{DBLP:journals/corr/ChengDL16}}


%\TD{Nice table comparing mechanisms https://lilianweng.github.io/lil-log/2018/06/24/attention-attention.html}

\TD{in above post\cite{weng2018attention}: soft vs hard attention and global vs local attention}

% ``heads learn redundant key/query projections'' --> share
% https://github.com/epfml/collaborative-attention
\TD{Multi-Head Attention: Collaborate Instead of Concatenate \cite{Cordonnier2020MultiHeadAC}}

% soft vs hard and global vs local

\TD{Describes two variants: a ``hard'' stochastic attention mechanism (trainable via ``maximizing an approximate variational lower bound'' or REINFORCE) and a ``soft'' deterministic attention mechanism(trainable by standard back-propagation) \cite{DBLP:journals/corr/XuBKCCSZB15}. Soft attention --- scores to all entities (is differenetiable but expensive) and hard attention --- only selects one entity (non-differentiable (and complicated, reinforcement learning), but requires less computation at inference)}


% TODO: does this make sense?
\TD{Non-linear projection for K,Q, and V~\cite{DBLP:journals/corr/abs-2111-10017}}


\subsubsection{Scoring Functions}

% TODO: https://lilianweng.github.io/lil-log/2018/06/24/attention-attention.html#summary
\TD{table from \cite{weng2018attention}}


\subsection{Self-Attention}

\r{sometimes refered to as ``intra-attention''\cite{DBLP:journals/corr/VaswaniSPUJGKP17}. Keys, queries and values are all derived from the same sequence. \TD{Self-attention transforms a sequence to create a representation of itself.}}



\subsection{transformers}

% possibly useful: http://nlp.seas.harvard.edu/2018/04/03/attention.html

\TD{survey of recent transformer architectures \TD{Efficient Transformers: A Survey \cite{Tay2020EfficientTA}}}


% Factorized Attention to self-attention
\TD{Generating Long Sequences with Sparse Transformers \cite{DBLP:journals/corr/abs-1904-10509}}

% include reccurence:  "enables learning dependency beyond a fixed length" + "relative position encodings"
\TD{Transformer-XL: Attentive Language Models Beyond a Fixed-Length Context \cite{DBLP:journals/corr/abs-1901-02860}}

% extends DBLP:journals/corr/abs-1901-02860 -- 
% https://github.com/guolinke/TUPE
\TD{Compressive Transformers for Long-Range Sequence Modeling \cite{Rae2020CompressiveTF}}

% linear attention
\TD{Transformers are RNNs: Fast Autoregressive Transformers with Linear Attention \cite{Katharopoulos2020TransformersAR}}

% 
\TD{Transformer with Untied Positional Encoding (TUPE) --- Rethinking Positional Encoding in Language Pre-training \cite{Ke2020RethinkingPE}}



\TD{Reformer: The Efficient Transformer \cite{Kitaev2020ReformerTE}}


% TODO: top-down attention
% related to self-attention
% https://twitter.com/thomaskipf/status/1277570203665170432
\TD{Object-Centric Learning with Slot Attention \cite{Locatello2020ObjectCentricLW}}

\TD{Recurrent Independent Mechanisms \cite{Goyal2019RecurrentIM}}

% DETR -- also object detection
\TD{End-to-End Object Detection with Transformers \cite{Carion2020EndtoEndOD}}


% TODO: read https://lilianweng.github.io/lil-log/2020/04/07/the-transformer-family.html


\subsection{Positional Encodings}

\TD{Positional embedding and positional encoding tend to be used interchangably. However, typically an encoding means ``fixed'' while an embedding means ``learned'' or ``trainable''.}

% TODO: example of how word order matters (not is a good example)

\r{Attention/transformers view the inputs as sets, that is there is no order associated with each input. All information enters the attention block at once. This is in contrast to something like a recurrent model, in which the order of the inputs is implicit.}

\r{trade off: potentially faster (remove the dependancy of doing operations sequentially) and can also possibly help capture longer range dependancies (without additional complexity e.g. skip connections)}

\r{(re)introducing order to the input by including additional information -- the ``positional embedding''.}

\r{NOTE: Great blog posts on this subject~\cite{kazemnejad_2021, kernes_2021, kernes_2021B}}

\subsubsection{Positional Encoding Value}

\r{why not add linear/progressive value signifying order?}

\r{This would be called an aboslute positional embedding}

\r{Include index information [0, n], where n is the length of the sentance (minus 1). This could lead to magnitude issues. Where the singal from the word embeddings is ``washed out'' by the positional embedding.  Another consideration is that (may or may not be an issue depending on the application) is that you'd like to ensure you have the largest sequence in the training set that you expect to see in evaluation set. For example, if you only see sequences of length $25$ in the training data and then see a sequence of length $32$ during inference. The model will be unsure what to do with values $25 - 31$ (zero indexing). Depending on how you include the positional embedding (e.g. additive or concat), the model may misinterpret the values or be largely/entirely unsure what to make of these previously unseen values.}
	
	
\r{To address this you could either increase the magnitude of the word embeddings or normalize/scale the positional embedding.}

\r{However, niether are ideal.}

\r{Increasing the magnitude of the word embeddings would possibly work, though you may consider issues with exploding values in the network, but you'd still have a similar issue to what would happen if you normalized the positional embedding. }

\r{That is, the normalized positional embeddings may encode different information when the sentances are longer or shorter -- the delta between words in a 5 word sentance vs a 20 word sentance doesn't have a consistent meaning}

% NOTE: haven't read this yet (I don't think, though the link is purple...)
\TD{Self-Attention with Relative Position Representations~\cite{DBLP:journals/corr/abs-1803-02155}}

\r{Ideally the embedding would be able to account for all the issues we discussed.}

\begin{itemize}[noitemsep,topsep=0pt]
	\item consistent delta between each position
		\begin{itemize}[noitemsep,topsep=0pt]
			\item regardless of sequence length, if an instance is one instance away from another, the positional encoding should be the same e.g. in a length four sequence the positional encoding should be the same from instances $1$ and $2$ as it is for instances $19$ and $20$ in a length $22$ sequence.
		\end{itemize}
	\item generalize to sequence lengths unseen in training
\end{itemize}

\r{additionally, we'd prefer to have each instance in the sequence be unique. That is the positional encoding for one instance shouldn't be the same as another in the same sequence (e.g. two words in a sentance).}

\paragraph{Positional Encoding Value(s)}

\r{Rather than use a single value, a possible solution is to use an array of values.}

\TD{Relative positional encoding (rather than absolute).}


\TD{What if we were to use a binary array to represent each location?}

\TD{issue with binary}

\TD{}


\TD{CAPE: Encoding Relative Positions with Continuous Augmented Positional Embeddings~\cite{DBLP:journals/corr/abs-2106-03143}}

% NOTE: possibly relevant: https://aclanthology.org/2021.emnlp-main.266.pdf

\paragraph{Sinusoidal}

\TD{include figure with multiple frequencies and points on the x and y axis leading to embeddings}

\subsection{Positional Embeddings (learned ``encodings'')}

% possibly useful: https://theaisummer.com/positional-embeddings/

\TD{Learning to Encode Position for Transformer with Continuous Dynamical Model~\cite{DBLP:journals/corr/abs-2003-09229}}


\TD{What Do Position Embeddings Learn? An Empirical Study of Pre-Trained Language Model Positional Encoding~\cite{DBLP:journals/corr/abs-2010-04903}}

\subsubsection{Including Positional Embeddings}

% someones thoughts on  additive vs concat: https://www.reddit.com/r/MachineLearning/comments/cttefo/d_positional_encoding_in_transformer/exs7d08/

\paragraph{Additive}

\TD{saves memory (over concatenation -- less dimensions)}

\TD{figure}

\paragraph{Concatenation}

\TD{figure}


\section{MLP-Mixer}

\r{MLPs that are used to ``mix'' tokens (spatial) and ``mix'' channels (features)}

% possible blog: https://wandb.ai/wandb_fc/pytorch-image-models/reports/Is-MLP-Mixer-a-CNN-in-Disguise---Vmlldzo4NDE1MTU

% MLP resurgence
\TD{Do You Even Need Attention? A \cite{DBLP:journals/corr/abs-2105-02723}}

\TD{gMLP (Pay Attention to MLPs) \cite{DBLP:journals/corr/abs-2105-08050}}

\TD{MLP-Mixer: An all-MLP Architecture for Vision \cite{DBLP:journals/corr/abs-2105-01601}}

\TD{RepMLP: Re-parameterizing Convolutions into Fully-connected Layers for Image Recognition \cite{DBLP:journals/corr/abs-2105-01883}}

\TD{ResMLP: Feedforward networks for image classification with data-efficient training \cite{DBLP:journals/corr/abs-2105-03404}}
Conncurrent papers released looking to replace attention with MLPs.

\TD{Do You Even Need Attention? A Stack of Feed-Forward Layers Does Surprisingly Well on ImageNet \cite{MelasKyriazi2021DoYE}}




\section{Mixture of Experts (MoE)}

\TD{Breaking down a problem (task) into multiple sub-problems (sub-tasks), training and expert in each sub-problem, then learning a meta/gating model that routes information to a specific expert and combines outputs}

% Divide and conquer vs meta-learning approach


\TD{High level steps}
\begin{itemize}[noitemsep,topsep=0pt]
	\item Decompose task into subtasks
	\item Learn ``expert'' for each subtask 
	\item Decide which expert to use (gating model or gating expert)
	\item Combine outputs as needed (pool/aggregate/select)
\end{itemize}

\TD{``20 years MoE''~\cite{yuksel2012twenty}}

\TD{Outrageously Large Neural Networks: The Sparsely-Gated Mixture-of-Experts Layer~\cite{shazeer2017outrageously}}



\chapter{Applied Neural Networks}

% TODO: I'm still not sure how/where to structure this

\section{Dense}

\TD{TODO}

\section{Convolutions}

% this reads strangely --> DNN on an image may not take advantage of the ``stationarity'' (statistics) of an image.

\r{When using a standard dense layer, all inputs are treated independently. However, adjacent pixels, on average, are highly highly correlated. For example, if there is a texture in the image, a similar pattern of pixels may occur repeatedly. Convolutions architecturally build in an implicit spatial structure to consider these spatial.}

% TODO: I'm not sure how I'm going to structure these yet or where I'll be placing them

% TODO: https://arxiv.org/abs/1904.11486
% https://www.youtube.com/watch?v=HjewNBZz00w


\TD{LeNet-5 \cite{lecun1998gradient}}

\r{Convolutions are built upon a lie -- that is we refer to the opperation as a convolution, yet it is in fact a cross-correlation operation since we don't rotate the kernel 180$\deg$. However, it is convention to refer to the operation as a convolution. For more, please see section \ref{conv_vs_cross}}

\r{translational invariance --- a property that relates to how a systems decisions are insensitive to the location of a features within an input. That is, if we're looking for an object or feature, our system shouldn't change if the object is in different locations within the input}

\TD{``Filter factorization'' (not the exact same definition of mathematical factorization)-- one $5\times5$ filter vs $2$ $3\times3$ filters stacked.  in the $5\times5$ there are $5\times5 = 25$ parameters, in the $3\times3$, there are $3\times3 \times 2 = 18$ learnable parameters, resulting in a ``cheaper'' operation.}

\TD{Neocognitron -- CNN paper prior to ``CNN''\cite{fukushima1982neocognitron}}

% Survey on CNNs
% TODO: a lot here -- good read
\TD{A Survey of the Recent Architectures of Deep Convolutional Neural Networks \cite{DBLP:journals/corr/abs-1901-06032}}


\TD{Squeeze-and-Excitation Networks \cite{DBLP:journals/corr/abs-1709-01507}}


% Graham Taylor
\r{weighted averaging operation in time or space}


\r{translation equivariant --- }

\TD{BlurPool --- ``fix is anti-aliasing by low-pass filtering before downsampling'' ---Making Convolutional Networks Shift-Invariant Again \cite{DBLP:journals/corr/abs-1904-11486}}


\r{spatial hierarchies --- \TD{TODO: figure raw data, abstract edges+, then more distinct images, then closer output to the output, then the final label}}


\r{typcially a feature extraction phase (consisting of convolutional and pooling layers) followed by a classifier block (dense layers).}

%%%% popular layer types
\textcolor{green}{TODO: feature maps, (height, width, and depth (also called channels axis)). Stride, filter size, depth. talk about parameters}

\r{The output feature map (every dimension in the depth axis is a feature/filter) --- after a convolution operation the depth of a layer is no longer representative of a color channel (like RGB), it is now representative of a feature extracted by the convolutional operation, these are called filters.}

\TD{Strided Convolution\cite{springenberg2014striving}}

\TD{Dilated Convolution --- `atrous' convolution. (famously used by wavenet), which is convenient in time series analysis.}

\r{weight tieing}


\textcolor{green}{TODO: figure}

\begin{figure}[htp]
	\centering
	\includegraphics[width=0.5\textwidth]{example-image-a}\hfil
	\caption{Figure example of convolution operation on 2d image \textcolor{green}{TODO}}
	\label{fig:conv_2d_example_calc}
\end{figure}

\begin{figure}[htp]
	\centering
	\includegraphics[width=0.5\textwidth]{example-image-b}\hfil
	\caption{Figure example of convolution operation on 3d image \textcolor{green}{TODO}}
	\label{fig:conv_2d_depth_example_calc}
\end{figure}

\textcolor{green}{TODO: examples of how different filter values and strides can effect the output dimensions.}




\section{Pooling}

\TD{TODO: examples of max vs average pooling}

%%%%%% research
\textcolor{blue}{Pooling may not fully determine learned deformation stability -- possibly filter smoothness\cite{ruderman2018learned}}

\r{downsampling}

\r{Why? importance of reducing the number of params.}

\TD{L2-pooling}

\TD{L2-pooling over the features or channels.}

\TD{additional --- learned/parameterized pooling}

\begin{figure}[htp]
	\centering
	\includegraphics[width=0.5\textwidth]{example-image-a}\hfil
	\caption{Figure example of max pooling operation on 2d image \textcolor{green}{TODO: I want this figure to be basic 2d}}
	\label{fig:pooling_max_2d_ex_a}
\end{figure}

\begin{figure}[htp]
	\centering
	\includegraphics[width=0.5\textwidth]{example-image-b}\hfil
	\caption{Figure example of average pooling operation on 3d image \textcolor{green}{TODO: I want this figure to be 3d}}
	\label{fig:pooling_avg_3d_ex_a}
\end{figure}


\r{may be better to use convolutional layers in place of the pooling layers\cite{springenberg2014striving}}

\section{Recurrent Cells}

% TODO: read this
% Recurrent / Echo state networks / ESN
\TD{The ``echo state'' approach to analysing and training recurrent neural networks-with an erratum note \cite{jaeger2001echo}}
\TD{Deep Echo State Network (DeepESN): A \cite{DBLP:journals/corr/abs-1712-04323}}

\subsection{Cell Advancements}

\subsubsection{LSTM}

% TODO: Nice overview of LSTMs: https://colah.github.io/posts/2015-08-Understanding-LSTMs/

Introduced in 1997 %\cite{hochreiter1997long}

\r{detect long term dependencies in sequence}

\r{two state vectors, short and long term}

\r{Main motivation: learning what to store in the long-term state and what to ``forget''.}

\r{at each time step, some information is ``stored'' and some information is ``forgotten''.}

\paragraph{variants}

\TD{Depth-Gated LSTM \cite{DBLP:journals/corr/YaoCVDD15}}

\TD{A Clockwork RNN \cite{DBLP:journals/corr/KoutnikGGS14}}

\TD{LSTM: A Search Space Odyssey \cite{DBLP:journals/corr/GreffSKSS15} --- survey of LSTM variants --- all variants are essentially equal.}


\paragraph{other directions}

% interesting paper on ``grid LSTMs'' -- not sure why they never become popular
\TD{Grid Long Short-Term Memory \cite{Kalchbrenner2016GridLS}}

\paragraph{Fully Connected Layers}


\begin{enumerate}[noitemsep,topsep=0pt]
	\item Main
	\item \textit{Gate Controllers}
	\begin{enumerate}[noitemsep,topsep=0pt]
		\item Forget
		\item Input
		\item Output
	\end{enumerate}
\end{enumerate}

\r{The gate controllers use a logistic activation fuction (output a range from 0 to 1). This output is then fed through an element-wise multiplication function and thus if the value is $0$, the gate is ``closed'', and $1$ if the gate is ``open''.}

\r{These gates are able to potentially:}

\begin{enumerate}[noitemsep,topsep=0pt]
	\item Recognize an important input
	\item Store the important input in a long-term state ()
	\item Preserve the information for as long as it's needed
	\item Extract the important information when needed
\end{enumerate}


\subparagraph{Main}

\begin{figure}
	\centering
	\includegraphics[width=0.5\textwidth]{example-image-a}\hfil
	\caption{\TD{Main Layer DIAGRAM}}
	%\label{}
\end{figure}

\r{This allows for the same basic functionality as a ``standard'' RNN cell --- however, the output, rather than being only sent to the next cell, is now partially stored in the long-term state.}


\subparagraph{Forget}

\r{Determines which part of the long-term state is forgotten/erased.}

\begin{figure}
	\centering
	\includegraphics[width=0.5\textwidth]{example-image-a}\hfil
	\caption{\TD{Forget Layer DIAGRAM}}
	%\label{}
\end{figure}



\subparagraph{Input}

\r{Determines which part of the output from the \textbf{main layer} are kept in the long-term state.}

\begin{figure}
	\centering
	\includegraphics[width=0.5\textwidth]{example-image-a}\hfil
	\caption{\TD{Input Layer DIAGRAM}}
	%\label{}
\end{figure}

\subparagraph{Output}

\r{Determines which part of the long term state is ``relevant'' (read and output).}

\begin{figure}
	\centering
	\includegraphics[width=0.5\textwidth]{example-image-a}\hfil
	\caption{\TD{Output Layer DIAGRAM}}
	%\label{}
\end{figure}


\paragraph{Other}

\subparagraph{Peephole Connections}

\r{In basic LSTM cells, the gate controller can only look at the input and previous short-term state. Peephole connections, proposed in 2000 \TD{cite gers2000recurrent} add an extra connection that allows for the gate controller to also see information from the long term state as well. }

\r{The previous long-term state also becomes an input to the forget and input gate. The current long-term state becomes an intput to the output gate.}



\subsubsection{GRU}

\r{The GRU (gated recurrent unit) is a varient of the LSTM cell \TD{cite - cho2014learning}. The main modifications include:}

\begin{itemize}[noitemsep,topsep=0pt]
	\item Both state vectors are merged into one state vector
	\item A single gate controller determines the \textbf{Forget} and \textbf{Input} gate
	\begin{itemize}[noitemsep,topsep=0pt]
		\item If the gate output is a 1, the input is open and the forget gate is closed. If the gate output is 0, the input gate is closed and the forget gate is open
	\end{itemize}
	\item \r{The output gate is removed and a new controller exists that controls which part of ht previous state will be ``shown'' to the main layer}. At each timestep the full state vector is output.
\end{itemize}

\subsection{Notes -- add}

\r{A recent paper \TD{greff2017lstm}, compares three LSTM variants and makes three main observations:}

\begin{itemize}[noitemsep,topsep=0pt]
	\item no significant architecture improvements over LSTMs
	\item forget gate and the output activation function are the most critical components
	\item \TD{hyperparams...}
\end{itemize}




\section{Capsule Networks}

% TODO: capsule networks
\TD{Dynamic Routing Between Capsules \cite{DBLP:journals/corr/abs-1710-09829}}

\section{Attention}

\r{``An attention function can be described as mapping a query and a set of key-value pairs to an output,
	where the query, keys, values, and output are all vectors. The output is computed as a weighted sum
	of the values, where the weight assigned to each value is computed by a compatibility function of the
	query with the corresponding key.'' \cite{DBLP:journals/corr/VaswaniSPUJGKP17}}

\TD{Self-attention Does Not Need $O(n^{2})$ Memory~\cite{Rabe2021SelfattentionDN}}

%TODO: another blog to checkout https://distill.pub/2016/augmented-rnns/

\r{overview can be found here\cite{weng2018attention}}


\TD{The original attention mechanism is introduced\cite{Bahdanau2015NeuralMT}.}

% TF attention implementation (https://www.tensorflow.org/tutorials/text/nmt_with_attention)

\TD{Effective Approaches to Attention-based Neural Machine Translation \cite{DBLP:journals/corr/LuongPM15}}

\TD{Massive Exploration of Neural Machine Translation Architectures \cite{DBLP:journals/corr/BritzGLL17}}

% TODO: index for transformer
% 'self-attention'
\TD{Attention Is All You Need -- Transformer network --- multi-head self-attention mechanism, key-value pairs \cite{DBLP:journals/corr/VaswaniSPUJGKP17}}

% self-attention \TD{Self-attention, less commonly intra-attention}
\TD{Long Short-Term Memory-Networks for Machine Reading \cite{DBLP:journals/corr/ChengDL16}}


%\TD{Nice table comparing mechanisms https://lilianweng.github.io/lil-log/2018/06/24/attention-attention.html}

\TD{in above post\cite{weng2018attention}: soft vs hard attention and global vs local attention}

% ``heads learn redundant key/query projections'' --> share
% https://github.com/epfml/collaborative-attention
\TD{Multi-Head Attention: Collaborate Instead of Concatenate \cite{Cordonnier2020MultiHeadAC}}

% soft vs hard and global vs local

\TD{Describes two variants: a ``hard'' stochastic attention mechanism (trainable via ``maximizing an approximate variational lower bound'' or REINFORCE) and a ``soft'' deterministic attention mechanism(trainable by standard back-propagation) \cite{DBLP:journals/corr/XuBKCCSZB15}. Soft attention --- scores to all entities (is differenetiable but expensive) and hard attention --- only selects one entity (non-differentiable (and complicated, reinforcement learning), but requires less computation at inference)}


% TODO: does this make sense?
\TD{Non-linear projection for K,Q, and V~\cite{DBLP:journals/corr/abs-2111-10017}}


\subsubsection{Scoring Functions}

% TODO: https://lilianweng.github.io/lil-log/2018/06/24/attention-attention.html#summary
\TD{table from \cite{weng2018attention}}


\subsection{Self-Attention}

\r{sometimes refered to as ``intra-attention''\cite{DBLP:journals/corr/VaswaniSPUJGKP17}. Keys, queries and values are all derived from the same sequence. \TD{Self-attention transforms a sequence to create a representation of itself.}}



\subsection{transformers}

% possibly useful: http://nlp.seas.harvard.edu/2018/04/03/attention.html

\TD{survey of recent transformer architectures \TD{Efficient Transformers: A Survey \cite{Tay2020EfficientTA}}}


% Factorized Attention to self-attention
\TD{Generating Long Sequences with Sparse Transformers \cite{DBLP:journals/corr/abs-1904-10509}}

% include reccurence:  "enables learning dependency beyond a fixed length" + "relative position encodings"
\TD{Transformer-XL: Attentive Language Models Beyond a Fixed-Length Context \cite{DBLP:journals/corr/abs-1901-02860}}

% extends DBLP:journals/corr/abs-1901-02860 -- 
% https://github.com/guolinke/TUPE
\TD{Compressive Transformers for Long-Range Sequence Modeling \cite{Rae2020CompressiveTF}}

% linear attention
\TD{Transformers are RNNs: Fast Autoregressive Transformers with Linear Attention \cite{Katharopoulos2020TransformersAR}}

% 
\TD{Transformer with Untied Positional Encoding (TUPE) --- Rethinking Positional Encoding in Language Pre-training \cite{Ke2020RethinkingPE}}



\TD{Reformer: The Efficient Transformer \cite{Kitaev2020ReformerTE}}


% TODO: top-down attention
% related to self-attention
% https://twitter.com/thomaskipf/status/1277570203665170432
\TD{Object-Centric Learning with Slot Attention \cite{Locatello2020ObjectCentricLW}}

\TD{Recurrent Independent Mechanisms \cite{Goyal2019RecurrentIM}}

% DETR -- also object detection
\TD{End-to-End Object Detection with Transformers \cite{Carion2020EndtoEndOD}}


% TODO: read https://lilianweng.github.io/lil-log/2020/04/07/the-transformer-family.html


\subsection{Positional Encodings}

\TD{Positional embedding and positional encoding tend to be used interchangably. However, typically an encoding means ``fixed'' while an embedding means ``learned'' or ``trainable''.}

% TODO: example of how word order matters (not is a good example)

\r{Attention/transformers view the inputs as sets, that is there is no order associated with each input. All information enters the attention block at once. This is in contrast to something like a recurrent model, in which the order of the inputs is implicit.}

\r{trade off: potentially faster (remove the dependancy of doing operations sequentially) and can also possibly help capture longer range dependancies (without additional complexity e.g. skip connections)}

\r{(re)introducing order to the input by including additional information -- the ``positional embedding''.}

\r{NOTE: Great blog posts on this subject~\cite{kazemnejad_2021, kernes_2021, kernes_2021B}}

\subsubsection{Positional Encoding Value}

\r{why not add linear/progressive value signifying order?}

\r{This would be called an aboslute positional embedding}

\r{Include index information [0, n], where n is the length of the sentance (minus 1). This could lead to magnitude issues. Where the singal from the word embeddings is ``washed out'' by the positional embedding.  Another consideration is that (may or may not be an issue depending on the application) is that you'd like to ensure you have the largest sequence in the training set that you expect to see in evaluation set. For example, if you only see sequences of length $25$ in the training data and then see a sequence of length $32$ during inference. The model will be unsure what to do with values $25 - 31$ (zero indexing). Depending on how you include the positional embedding (e.g. additive or concat), the model may misinterpret the values or be largely/entirely unsure what to make of these previously unseen values.}
	
	
\r{To address this you could either increase the magnitude of the word embeddings or normalize/scale the positional embedding.}

\r{However, niether are ideal.}

\r{Increasing the magnitude of the word embeddings would possibly work, though you may consider issues with exploding values in the network, but you'd still have a similar issue to what would happen if you normalized the positional embedding. }

\r{That is, the normalized positional embeddings may encode different information when the sentances are longer or shorter -- the delta between words in a 5 word sentance vs a 20 word sentance doesn't have a consistent meaning}

% NOTE: haven't read this yet (I don't think, though the link is purple...)
\TD{Self-Attention with Relative Position Representations~\cite{DBLP:journals/corr/abs-1803-02155}}

\r{Ideally the embedding would be able to account for all the issues we discussed.}

\begin{itemize}[noitemsep,topsep=0pt]
	\item consistent delta between each position
		\begin{itemize}[noitemsep,topsep=0pt]
			\item regardless of sequence length, if an instance is one instance away from another, the positional encoding should be the same e.g. in a length four sequence the positional encoding should be the same from instances $1$ and $2$ as it is for instances $19$ and $20$ in a length $22$ sequence.
		\end{itemize}
	\item generalize to sequence lengths unseen in training
\end{itemize}

\r{additionally, we'd prefer to have each instance in the sequence be unique. That is the positional encoding for one instance shouldn't be the same as another in the same sequence (e.g. two words in a sentance).}

\paragraph{Positional Encoding Value(s)}

\r{Rather than use a single value, a possible solution is to use an array of values.}

\TD{Relative positional encoding (rather than absolute).}


\TD{What if we were to use a binary array to represent each location?}

\TD{issue with binary}

\TD{}


\TD{CAPE: Encoding Relative Positions with Continuous Augmented Positional Embeddings~\cite{DBLP:journals/corr/abs-2106-03143}}

% NOTE: possibly relevant: https://aclanthology.org/2021.emnlp-main.266.pdf

\paragraph{Sinusoidal}

\TD{include figure with multiple frequencies and points on the x and y axis leading to embeddings}

\subsection{Positional Embeddings (learned ``encodings'')}

% possibly useful: https://theaisummer.com/positional-embeddings/

\TD{Learning to Encode Position for Transformer with Continuous Dynamical Model~\cite{DBLP:journals/corr/abs-2003-09229}}


\TD{What Do Position Embeddings Learn? An Empirical Study of Pre-Trained Language Model Positional Encoding~\cite{DBLP:journals/corr/abs-2010-04903}}

\subsubsection{Including Positional Embeddings}

% someones thoughts on  additive vs concat: https://www.reddit.com/r/MachineLearning/comments/cttefo/d_positional_encoding_in_transformer/exs7d08/

\paragraph{Additive}

\TD{saves memory (over concatenation -- less dimensions)}

\TD{figure}

\paragraph{Concatenation}

\TD{figure}


\section{MLP-Mixer}

\r{MLPs that are used to ``mix'' tokens (spatial) and ``mix'' channels (features)}

% possible blog: https://wandb.ai/wandb_fc/pytorch-image-models/reports/Is-MLP-Mixer-a-CNN-in-Disguise---Vmlldzo4NDE1MTU

% MLP resurgence
\TD{Do You Even Need Attention? A \cite{DBLP:journals/corr/abs-2105-02723}}

\TD{gMLP (Pay Attention to MLPs) \cite{DBLP:journals/corr/abs-2105-08050}}

\TD{MLP-Mixer: An all-MLP Architecture for Vision \cite{DBLP:journals/corr/abs-2105-01601}}

\TD{RepMLP: Re-parameterizing Convolutions into Fully-connected Layers for Image Recognition \cite{DBLP:journals/corr/abs-2105-01883}}

\TD{ResMLP: Feedforward networks for image classification with data-efficient training \cite{DBLP:journals/corr/abs-2105-03404}}
Conncurrent papers released looking to replace attention with MLPs.

\TD{Do You Even Need Attention? A Stack of Feed-Forward Layers Does Surprisingly Well on ImageNet \cite{MelasKyriazi2021DoYE}}




\section{Mixture of Experts (MoE)}

\TD{Breaking down a problem (task) into multiple sub-problems (sub-tasks), training and expert in each sub-problem, then learning a meta/gating model that routes information to a specific expert and combines outputs}

% Divide and conquer vs meta-learning approach


\TD{High level steps}
\begin{itemize}[noitemsep,topsep=0pt]
	\item Decompose task into subtasks
	\item Learn ``expert'' for each subtask 
	\item Decide which expert to use (gating model or gating expert)
	\item Combine outputs as needed (pool/aggregate/select)
\end{itemize}

\TD{``20 years MoE''~\cite{yuksel2012twenty}}

\TD{Outrageously Large Neural Networks: The Sparsely-Gated Mixture-of-Experts Layer~\cite{shazeer2017outrageously}}




\chapter{Unsupervised}

\r{The reality is that data is not always labeled.}

\r{unsupervised learning is capable of finding hidden patterns in the underlying structure of hte data.}

\r{attempts to represent data with increaingly fewer parameters}

\r{Discovering hidden structures or patterns in unlabeled training data.}

\TD{Neighborhood-Based Methods \ref{nearest_neighbors} \r{lazy learners  -- learn how to label new instances based on proximity to existing instances}}

% TODO: placement / may need to rename+restructure sections
\textcolor{blue}{unsupervised methods may be commonly used in two main settings:}
\begin{enumerate}[noitemsep,topsep=0pt]
	\item Data Exploration
	\begin{itemize}[noitemsep,topsep=0pt]
		\item Visualization (clustering \textcolor{red}{local ref})
	\end{itemize}
	\item Preprocessing (e.g. prior to a supervised method): unsupervised pretraining may be considered a form of regularization
	\begin{itemize}[noitemsep,topsep=0pt]
		\item Compressing (dimensionality reduction \textcolor{red}{local ref})
		\item Creating new/different representations
	\end{itemize}
\end{enumerate}

\r{regularization, feature engineering, detecting outliers -- also used for detecting how different new (incoming) training data is from the current distribution.}

\r{popular applications --- anomaly detection, group segmentation, preprocessing (dimensionality reduction)}

\subsection{TODO}

\TD{evaluating unsupervised learning systems are harder to evaluate than supervised learning systems.}

\r{evaluating supervised methods is often subjective. An approach to skirt this is issue may be to label the test set manually and then use any desired supervised learning metric.}

%%%%%%%%%%%%%%%%%%%%%%%%%%%%%% clustering
\input{./foundations/unsupervised/clustering}

%%%%%%%%%%%%%%%%%%%%%%%%%%%%%% Dimensionality Reduction
\input{./foundations/unsupervised/dimensionality_reduction}


% TODO: format / other
%\input{./foundations/unsupervised/applications}


\chapter{Semi-supervised}

\input{./foundations/semi_supervised}


\chapter{Common Architectures}
%TODO: should this also include a ``history''?


\TD{Overview}

% TODO: I'm still not sure if I should divide this up by ``types'' of architectures or types of problems and the architectures used to solve them.. maybe both? unsure...


Images and Videos
\begin{itemize}[noitemsep,topsep=0pt]
	\item Classification
	\item Segmentation
	\begin{itemize}[noitemsep,topsep=0pt]
		\item Semantic segmentation (where we don't differentiate between instances)
		\item Instance segmentation
	\end{itemize}
	\item Object detection
\end{itemize}


\subsection{Spatial Data}

\r{types of problems -- spatial to spatial (1:1), spatial to sequence, spatial to value (single or multiple)}

\subsubsection{Convolutional Approaches}



\subsubsection{Image Classification}
% Graham Taylor talk
\r{alexnet, network-in-network, inception/google lenet (end with $1 \times 1 \times N_{classes}$ into a global average pooling layer), vggnet}

% TODO: I'm not sure where this belongs
\TD{$1 \times 1 \times$ convolution explaination --- useful for adjusting the number of features in the feature dimension, either up or down.}

\TD{Inception module --- rather than manually choosing proper\``best'' filter size, ``let the model choose''.}

\r{Resnet (\TD{Deep Residual Learning for Image Recognition \cite{DBLP:journals/corr/HeZRS15}}, similar to highway networks) --- skip connections ``residual modual/block'' (dynamically adjusting depth) (others in this time: fractal nets, stochastic XXXX nets, where skip connections are shortcut paths)}

\r{DenseNet --- where these skip connections are taken to an extreme. also, concatenation, not summation}

% TODO: not sure where this belongs yet
\paragraph{skip connections}


\r{three popular explainations (as described in \cite{lakshmanan2021practical}).}
\begin{itemize}[noitemsep,topsep=0pt]
	\item Addition opperation makes the task ``easier''
	\begin{itemize}[noitemsep,topsep=0pt]
		\item Detailed by the authors of the original paper. That it is easier to predict the ``residue''/delta between the input and output rather than the solely the desired output.
	\end{itemize}
	\item Residual connections make the network effectively shallower 
	\begin{itemize}[noitemsep,topsep=0pt]
		\item Described in \TD{Residual Networks are Exponential Ensembles of Relatively Shallow Networks \cite{DBLP:journals/corr/VeitWB16}. The connections effectively create an ensemble of shallower networks. The network then learns to select the best path/subnetwork for each instance.}
	\end{itemize}
	\item Loss topological ``smoothing'' (TODO: change term)
	\begin{itemize}[noitemsep,topsep=0pt]
		\item \TD{show figure. \TD{Visualizing the Loss Landscape of Neural Nets \cite{DBLP:journals/corr/abs-1712-09913}}}
	\end{itemize}
\end{itemize}



\TD{SqueezeNet \cite{DBLP:journals/corr/IandolaMAHDK16} ``fire modules'', contraction and expansion}

\r{Squeeze and Excitation network --- adaptive recalibration of feature maps. \cite{DBLP:journals/corr/abs-1709-01507} }

% fit into an architecture ``timeline''
\TD{ResNeXt\cite{xie2017aggregated}}


\subsection{Object Detection}

\r{sliding window and use each window as input to a classifier. But the problem is that we need to apply the classifier to a huge number of locations and scales and aspect ratios. A potential solution to this problem is to use region proposals.}

\r{region proposals were introduced by R-CNN \TD{cite}, where conventional methods were used to propose regions for the CNN to classify. It's important to note that most regions are not square may need to be warped to be insert to the CNN}

\r{Bbox regression is also used --- how much to offset the region proposal.}


\r{fast R-CNN proposed regions by using the feature maps from a layer from a CNN that was previously trained (VGG or Resnet, for example). after identifying the regions, then crop-resize, then CNN over each region, then get output (class + bbox regression.)}

\r{faster r-cnn. Insert a region proposal network (RPN) to predict proposals from features --- in this architecture, there are actually four losses to jointly train 1) object/net RPN, 2) box/net RPN, 3) Class prediction, and 4) final bounding box score.}


% TODO: YOLO object detection
\TD{You Only Look Once: Unifie \cite{DBLP:journals/corr/RedmonDGF15}}
% TODO more object detection
\TD{Mask R-CNN \cite{DBLP:journals/corr/HeGDG17}}
\TD{Faster R-CNN: \cite{DBLP:journals/corr/RenHG015}}
\TD{SSD: \cite{DBLP:journals/corr/LiuAESR15}}

% 
\TD{Focal Loss (RetinaNet) \cite{DBLP:journals/corr/abs-1708-02002}}


\subsection{Segmentation}

\r{fully convolution --- dowsample to upsample: 1) efficency when using convs on smaller dimensional inputs 2) latent representation}

\r{CNN and RPN --- nice results \TD{cite}}

\r{mask R-CNN --- also used for pose prediction}

% code: https://github.com/facebookresearch/detectron2/tree/master/projects/PointRend
% blog: https://ai.facebook.com/blog/using-a-classical-rendering-technique-to-push-state-of-the-art-for-image-segmentation/
\TD{PointRend: Image Segmentation as Rendering \cite{Kirillov2019PointRendIS}}


\section{Text: Natural Language Processing}

\r{bag of words -- RNN -- RNN augmented (LSTM, GRU, etc) -- attention -- transformers }


\section{Structured data: Tablular}


\section{Other Common Architectures}




% chapter: model compression
\chapter{Model ``Compression'' or ``Distillation''}
% this section will likely be moved around
motivation

\TD{Distilling the Knowledge in a Neural Network \cite{Hinton2015DistillingTK}}

\begin{itemize}[noitemsep,topsep=0pt]
	\item Model (output) performance
	\begin{itemize}[noitemsep,topsep=0pt]
		\item regularization (through reduction of number of parameters)
	\end{itemize}
	\item Model performance
	\begin{itemize}[noitemsep,topsep=0pt]
		\item memory/storage --- by creating physically smaller networks
		\item energy --- computation, latency, which supports edge depolyments
	\end{itemize}
\end{itemize}



\section{Quantization}


\r{reduces the precision of parameters in a model}

\r{latency, memory}

\r{some fixed-point accellerators become available in edge settings}

\subsection{When}

\subsubsection{Initialization}

\TD{Pruning Neural Networks at Initialization: Why are We Missing the Mark? \cite{Frankle2020PruningNN}}

\TD{Robust Pruning at Initialization \cite{Hayou2021RobustPA}}

\subsubsection{during training}

\subsubsection{post-training}

\r{calibration data}

\section{Weight Pruning}

%TODO: https://github.com/he-y/Awesome-Pruning -- repo of recent pruning work

% intial skeleton influence by:
\TD{The initial structure and citations were heavily influenced by \cite{lange2020_lottery_ticket_hypothesis}}

\TD{SNIP: Single-shot Network Pruning based on Connection Sensitivity \cite{DBLP:journals/corr/abs-1810-02340}}

\r{reduces the overall number of parameters}

\r{in practice, this often refers to setting the parameters of a particular ``node'' to zero and making it untrainable during training.}

\TD{Comparing Rewinding and Fine-tuning in Neural Network Pruning \cite{Renda2020ComparingRA} --- Learning rate rewinding}

\TD{Deconstructing Lottery Tickets: Zero \cite{DBLP:journals/corr/abs-1905-01067} --- SuperMasks}

\TD{The Early Phase of Neural Network Training \cite{Frankle2020TheEP}}

% `` identify highly sparse trainable subnetworks at initialization, without ever training, or indeed without ever looking at the data''
\TD{Synaptic Flow Pruning (SynFlow) --- Pruning neural networks without any data by iteratively conserving synaptic flow \cite{Tanaka2020PruningNN}}

% code: https://github.com/varungohil/Generalizing-Lottery-Tickets
\TD{One ticket to win them all: generalizing lottery ticket initializations across datasets and optimizers \cite{Morcos2019OneTT}}



\subsubsection{What to remove}



\paragraph{relationship of nodes to nodes pruned}

\subparagraph{unstructured}

\r{no consideration for relationship between pruned weights}

\subparagraph{structured}

\r{prunes ``groups'' of weights.}

\paragraph{relationship of nodes to architecture}

\subparagraph{local}

\r{enforcment of a percent of weights pruned at each layer}

\subparagraph{global}

\r{total percent of weights are pruned, no restriction layer wise.}

\subsubsection{Deciding what to remove}

\r{common thought: large magnitude parameters are ``more important'' and thus should be removed less --- which is counterintuitive to concepts such as l2 regularization which penalize large magnitude parameters.}

\TD{other techniques}

\subsubsection{When to prune}

\begin{itemize}[noitemsep,topsep=0pt]
	\item Before training
	\item During training
	\item After training
\end{itemize}

\subsubsection{Pruning mechanism}

\begin{itemize}[noitemsep,topsep=0pt]
	\item One shot (prune all at once)
	\item Iterative
\end{itemize}

\section{Topology}

\r{more efficient model topology}

\subsection{Distillation}

% TODO: these are popular citations in the space, I haven't read many of them yet. putting them here for a start when I come back to the topic
\TD{Like What You Like: Knowledge Distill via Neuron Selectivity Transfer \cite{DBLP:journals/corr/HuangW17a}}
\TD{FitNets: Hints for Thin Deep Nets \cite{Romero2015FitNetsHF}}
\TD{Similarity-Preserving Knowledge Distillation \cite{DBLP:journals/corr/abs-1907-09682}}
\TD{Correlation Congruence for Knowledge Distillation \cite{DBLP:journals/corr/abs-1904-01802}}
\TD{A gift from knowledge distillation: Fast optimization, network minimization and transfer learning\cite{yim2017gift}}
\TD{Relational Knowledge Distillation \cite{DBLP:journals/corr/abs-1904-05068}}
\TD{Paraphrasing Complex Network: Network Compression via Factor Transfer \cite{DBLP:journals/corr/abs-1802-04977}}
\TD{Contrastive Representation Distillation \cite{DBLP:journals/corr/abs-1910-10699}}


\subsubsection{Student Teacher}

\TD{not sure if this belongs as its own section or not}

\TD{Knowledge Distillation and Student-Teacher Learning for Visual Intelligence: A Review and New Outlooks \cite{Wang2020KnowledgeDA}}

\TD{Self-Training with Weak Supervision~\cite{DBLP:journals/corr/abs-2104-05514}}



\paragraph{Knowledge Transfer}

% TODO: I'm unclear on the distinction here, is knowlege transfer the transfer of knowledge (like transfer learning but student/teacher?) and knowledge distillation the process of distilling that knowledge?

%% knowledge transfer -- but really distillation is kt? unclear here. need to read more
\TD{Probabilistic Knowledge Transfer for Deep Representation Learning \cite{DBLP:journals/corr/abs-1803-10837}}
\TD{Knowledge Transfer via Distillation of Activation Boundaries Formed by Hidden Neurons \cite{DBLP:journals/corr/abs-1811-03233}}
\TD{Variational Information Distillation for Knowledge Transfer \cite{DBLP:journals/corr/abs-1904-05835}}

\subsection{Tensor Decomposition}
% Not sure what this is -- from coursera


% TODO: this fits somewhere in this chapter, but not necessarily `here'
\subsection{The Lottery Ticket Hypothesis}

% RigL -- training sparse networks
\TD{Rigging the Lottery: Making All Tickets Winners \cite{Evci2019RiggingTL}}


\TD{Original paper --- The Lottery Ticket Hypothesis: Training Pruned Neural Networks \cite{DBLP:journals/corr/abs-1803-03635} --- iterative pruning. Dense network at initialization contains a number of ``winning tickets''}


\TD{Stabilizing the Lottery Ticket Hypothesis \cite{DBLP:journals/corr/abs-1903-01611}}

\TD{IMP with rewind}

% CODE: https://github.com/RICE-EIC/Early-Bird-Tickets
\TD{Drawing early-bird tickets: Towards more efficient training of deep networks \cite{You2020DrawingET}}

\TD{Playing the lottery with rewards and multiple languages: lottery tickets in RL and NLP \cite{Yu2020PlayingTL}}


\section{Sparsity}

% TODO: not sure where this section belongs quite yet

\TD{Training Neural Networks with Fixed Sparse Masks~\cite{DBLP:journals/corr/abs-2111-09839}}



% External memory
\chapter{External Memory}


\section{Neural Turning Machines}

% neural network + external memory storage
% controller + memory
\TD{Neural Turing Machines \cite{DBLP:journals/corr/GravesWD14}}


\r{inspired by Turing's ``automatic machines''  \cite{turing1936computable} (later described as a ``turing machine'' \cite{church1937turing})}




\section{Other}

\TD{MANN}

% external memory
\r{external memory \cite{santoro2016meta}}

%TODO: I can't rememver the significance of this paper based on the title
\TD{Meta Networks\cite{munkhdalai2017meta}}

% TODO: this doesn't seem like the appropriate place for this paper
\r{Conditional neural processes \cite{garnelo2018conditional}}

% chapter: training dynamics
\input{./foundations/training_dynamics}

\input{./foundations/adversarial_examples}


\chapter{Graph Neural Networks}

\TD{A Comprehensive Survey on Graph Neural Networks \cite{DBLP:journals/corr/abs-1901-00596}}


\chapter{Generative}

\r{Nice resource: "Generative Deep Learning"\cite{foster2019generative} and \TD{An Introduction to Deep Generative Modeling \cite{DBLP:journals/corr/abs-2103-05180}}}

\section{Generative Adversarial Networks (GANs)}

\TD{Generative Adversarial Networks \cite{Goodfellow2014GenerativeAN}}

\TD{NIPS 2016 Tutorial: Generative Adversarial Networks \cite{DBLP:journals/corr/Goodfellow17}}

\TD{``Inverse PM -- semi-famous interaction, stemming from this review describing/inquiring about the relation to https://web.archive.org/web/20160411075236/http://media.nips.cc/nipsbooks/nipspapers/paper\_files/nips27/reviews/1384.html '' "predictability minimisation" or PM\cite{schmidhuber1992learning}}

% TODO: how as this not been done yet? I think I've written some pretty nice text on this before
\r{Discriminator and generator}

\r{Tries to learn the underlying structure of the data}

%TODO:

\TD{Self-Attention Generative Adversarial Networks \cite{Zhang2019SelfAttentionGA} uses attention\cite{DBLP:journals/corr/abs-1711-07971}}

\TD{Unsupervised Representation Learning with Deep Convolutional Generative Adversarial Networks \cite{Radford2015UnsupervisedRL}}

\TD{Mode collapse}

% TODO: wgan
\TD{Wasserstein GAN \cite{Arjovsky2017WassersteinG}}
\TD{Improved Training of Wasserstein GANs \cite{DBLP:journals/corr/GulrajaniAADC17}}
% TODO: DCGAN
\TD{Unsupervised Representation Learning with Deep Convolutional Generative Adversarial Networks \cite{Radford2016UnsupervisedRL}}

\section{Variational AutoEncoder}

% TODO: I've written on this somewhere.... 9Oct21

\TD{Auto-Encoding Variational Bayes \cite{Kingma2014AutoEncodingVB}}

\TD{Tutorial on Variational Autoencoders \cite{Doersch2016TutorialOV}}

% consistency in latent space w/without augementations in VAE
\TD{Consistency Regularization for Variational Auto-Encoders \cite{DBLP:journals/corr/abs-2105-14859}}

\chapter{Curriculum Learning}

% TODO: Curriculum Learning
\TD{Curriculum learning~\cite{Bengio2009CurriculumL}}

\TD{A Comprehensive Survey on Curriculum Learning~\cite{DBLP:journals/corr/abs-2010-13166}}

\TD{General Cyclical Training of Neural Networks~\cite{Smith2022GeneralCT}}
 % ML algorithm foundations

\part{Architectures}

% TODO: I'm still not sure how/where to structure this

\section{Dense}

\TD{TODO}

\section{Convolutions}

% this reads strangely --> DNN on an image may not take advantage of the ``stationarity'' (statistics) of an image.

\r{When using a standard dense layer, all inputs are treated independently. However, adjacent pixels, on average, are highly highly correlated. For example, if there is a texture in the image, a similar pattern of pixels may occur repeatedly. Convolutions architecturally build in an implicit spatial structure to consider these spatial.}

% TODO: I'm not sure how I'm going to structure these yet or where I'll be placing them

% TODO: https://arxiv.org/abs/1904.11486
% https://www.youtube.com/watch?v=HjewNBZz00w


\TD{LeNet-5 \cite{lecun1998gradient}}

\r{Convolutions are built upon a lie -- that is we refer to the opperation as a convolution, yet it is in fact a cross-correlation operation since we don't rotate the kernel 180$\deg$. However, it is convention to refer to the operation as a convolution. For more, please see section \ref{conv_vs_cross}}

\r{translational invariance --- a property that relates to how a systems decisions are insensitive to the location of a features within an input. That is, if we're looking for an object or feature, our system shouldn't change if the object is in different locations within the input}

\TD{``Filter factorization'' (not the exact same definition of mathematical factorization)-- one $5\times5$ filter vs $2$ $3\times3$ filters stacked.  in the $5\times5$ there are $5\times5 = 25$ parameters, in the $3\times3$, there are $3\times3 \times 2 = 18$ learnable parameters, resulting in a ``cheaper'' operation.}

\TD{Neocognitron -- CNN paper prior to ``CNN''\cite{fukushima1982neocognitron}}

% Survey on CNNs
% TODO: a lot here -- good read
\TD{A Survey of the Recent Architectures of Deep Convolutional Neural Networks \cite{DBLP:journals/corr/abs-1901-06032}}


\TD{Squeeze-and-Excitation Networks \cite{DBLP:journals/corr/abs-1709-01507}}


% Graham Taylor
\r{weighted averaging operation in time or space}


\r{translation equivariant --- }

\TD{BlurPool --- ``fix is anti-aliasing by low-pass filtering before downsampling'' ---Making Convolutional Networks Shift-Invariant Again \cite{DBLP:journals/corr/abs-1904-11486}}


\r{spatial hierarchies --- \TD{TODO: figure raw data, abstract edges+, then more distinct images, then closer output to the output, then the final label}}


\r{typcially a feature extraction phase (consisting of convolutional and pooling layers) followed by a classifier block (dense layers).}

%%%% popular layer types
\textcolor{green}{TODO: feature maps, (height, width, and depth (also called channels axis)). Stride, filter size, depth. talk about parameters}

\r{The output feature map (every dimension in the depth axis is a feature/filter) --- after a convolution operation the depth of a layer is no longer representative of a color channel (like RGB), it is now representative of a feature extracted by the convolutional operation, these are called filters.}

\TD{Strided Convolution\cite{springenberg2014striving}}

\TD{Dilated Convolution --- `atrous' convolution. (famously used by wavenet), which is convenient in time series analysis.}

\r{weight tieing}


\textcolor{green}{TODO: figure}

\begin{figure}[htp]
	\centering
	\includegraphics[width=0.5\textwidth]{example-image-a}\hfil
	\caption{Figure example of convolution operation on 2d image \textcolor{green}{TODO}}
	\label{fig:conv_2d_example_calc}
\end{figure}

\begin{figure}[htp]
	\centering
	\includegraphics[width=0.5\textwidth]{example-image-b}\hfil
	\caption{Figure example of convolution operation on 3d image \textcolor{green}{TODO}}
	\label{fig:conv_2d_depth_example_calc}
\end{figure}

\textcolor{green}{TODO: examples of how different filter values and strides can effect the output dimensions.}




\section{Pooling}

\TD{TODO: examples of max vs average pooling}

%%%%%% research
\textcolor{blue}{Pooling may not fully determine learned deformation stability -- possibly filter smoothness\cite{ruderman2018learned}}

\r{downsampling}

\r{Why? importance of reducing the number of params.}

\TD{L2-pooling}

\TD{L2-pooling over the features or channels.}

\TD{additional --- learned/parameterized pooling}

\begin{figure}[htp]
	\centering
	\includegraphics[width=0.5\textwidth]{example-image-a}\hfil
	\caption{Figure example of max pooling operation on 2d image \textcolor{green}{TODO: I want this figure to be basic 2d}}
	\label{fig:pooling_max_2d_ex_a}
\end{figure}

\begin{figure}[htp]
	\centering
	\includegraphics[width=0.5\textwidth]{example-image-b}\hfil
	\caption{Figure example of average pooling operation on 3d image \textcolor{green}{TODO: I want this figure to be 3d}}
	\label{fig:pooling_avg_3d_ex_a}
\end{figure}


\r{may be better to use convolutional layers in place of the pooling layers\cite{springenberg2014striving}}

\section{Recurrent Cells}

% TODO: read this
% Recurrent / Echo state networks / ESN
\TD{The ``echo state'' approach to analysing and training recurrent neural networks-with an erratum note \cite{jaeger2001echo}}
\TD{Deep Echo State Network (DeepESN): A \cite{DBLP:journals/corr/abs-1712-04323}}

\subsection{Cell Advancements}

\subsubsection{LSTM}

% TODO: Nice overview of LSTMs: https://colah.github.io/posts/2015-08-Understanding-LSTMs/

Introduced in 1997 %\cite{hochreiter1997long}

\r{detect long term dependencies in sequence}

\r{two state vectors, short and long term}

\r{Main motivation: learning what to store in the long-term state and what to ``forget''.}

\r{at each time step, some information is ``stored'' and some information is ``forgotten''.}

\paragraph{variants}

\TD{Depth-Gated LSTM \cite{DBLP:journals/corr/YaoCVDD15}}

\TD{A Clockwork RNN \cite{DBLP:journals/corr/KoutnikGGS14}}

\TD{LSTM: A Search Space Odyssey \cite{DBLP:journals/corr/GreffSKSS15} --- survey of LSTM variants --- all variants are essentially equal.}


\paragraph{other directions}

% interesting paper on ``grid LSTMs'' -- not sure why they never become popular
\TD{Grid Long Short-Term Memory \cite{Kalchbrenner2016GridLS}}

\paragraph{Fully Connected Layers}


\begin{enumerate}[noitemsep,topsep=0pt]
	\item Main
	\item \textit{Gate Controllers}
	\begin{enumerate}[noitemsep,topsep=0pt]
		\item Forget
		\item Input
		\item Output
	\end{enumerate}
\end{enumerate}

\r{The gate controllers use a logistic activation fuction (output a range from 0 to 1). This output is then fed through an element-wise multiplication function and thus if the value is $0$, the gate is ``closed'', and $1$ if the gate is ``open''.}

\r{These gates are able to potentially:}

\begin{enumerate}[noitemsep,topsep=0pt]
	\item Recognize an important input
	\item Store the important input in a long-term state ()
	\item Preserve the information for as long as it's needed
	\item Extract the important information when needed
\end{enumerate}


\subparagraph{Main}

\begin{figure}
	\centering
	\includegraphics[width=0.5\textwidth]{example-image-a}\hfil
	\caption{\TD{Main Layer DIAGRAM}}
	%\label{}
\end{figure}

\r{This allows for the same basic functionality as a ``standard'' RNN cell --- however, the output, rather than being only sent to the next cell, is now partially stored in the long-term state.}


\subparagraph{Forget}

\r{Determines which part of the long-term state is forgotten/erased.}

\begin{figure}
	\centering
	\includegraphics[width=0.5\textwidth]{example-image-a}\hfil
	\caption{\TD{Forget Layer DIAGRAM}}
	%\label{}
\end{figure}



\subparagraph{Input}

\r{Determines which part of the output from the \textbf{main layer} are kept in the long-term state.}

\begin{figure}
	\centering
	\includegraphics[width=0.5\textwidth]{example-image-a}\hfil
	\caption{\TD{Input Layer DIAGRAM}}
	%\label{}
\end{figure}

\subparagraph{Output}

\r{Determines which part of the long term state is ``relevant'' (read and output).}

\begin{figure}
	\centering
	\includegraphics[width=0.5\textwidth]{example-image-a}\hfil
	\caption{\TD{Output Layer DIAGRAM}}
	%\label{}
\end{figure}


\paragraph{Other}

\subparagraph{Peephole Connections}

\r{In basic LSTM cells, the gate controller can only look at the input and previous short-term state. Peephole connections, proposed in 2000 \TD{cite gers2000recurrent} add an extra connection that allows for the gate controller to also see information from the long term state as well. }

\r{The previous long-term state also becomes an input to the forget and input gate. The current long-term state becomes an intput to the output gate.}



\subsubsection{GRU}

\r{The GRU (gated recurrent unit) is a varient of the LSTM cell \TD{cite - cho2014learning}. The main modifications include:}

\begin{itemize}[noitemsep,topsep=0pt]
	\item Both state vectors are merged into one state vector
	\item A single gate controller determines the \textbf{Forget} and \textbf{Input} gate
	\begin{itemize}[noitemsep,topsep=0pt]
		\item If the gate output is a 1, the input is open and the forget gate is closed. If the gate output is 0, the input gate is closed and the forget gate is open
	\end{itemize}
	\item \r{The output gate is removed and a new controller exists that controls which part of ht previous state will be ``shown'' to the main layer}. At each timestep the full state vector is output.
\end{itemize}

\subsection{Notes -- add}

\r{A recent paper \TD{greff2017lstm}, compares three LSTM variants and makes three main observations:}

\begin{itemize}[noitemsep,topsep=0pt]
	\item no significant architecture improvements over LSTMs
	\item forget gate and the output activation function are the most critical components
	\item \TD{hyperparams...}
\end{itemize}




\section{Capsule Networks}

% TODO: capsule networks
\TD{Dynamic Routing Between Capsules \cite{DBLP:journals/corr/abs-1710-09829}}

\section{Attention}

\r{``An attention function can be described as mapping a query and a set of key-value pairs to an output,
	where the query, keys, values, and output are all vectors. The output is computed as a weighted sum
	of the values, where the weight assigned to each value is computed by a compatibility function of the
	query with the corresponding key.'' \cite{DBLP:journals/corr/VaswaniSPUJGKP17}}

\TD{Self-attention Does Not Need $O(n^{2})$ Memory~\cite{Rabe2021SelfattentionDN}}

%TODO: another blog to checkout https://distill.pub/2016/augmented-rnns/

\r{overview can be found here\cite{weng2018attention}}


\TD{The original attention mechanism is introduced\cite{Bahdanau2015NeuralMT}.}

% TF attention implementation (https://www.tensorflow.org/tutorials/text/nmt_with_attention)

\TD{Effective Approaches to Attention-based Neural Machine Translation \cite{DBLP:journals/corr/LuongPM15}}

\TD{Massive Exploration of Neural Machine Translation Architectures \cite{DBLP:journals/corr/BritzGLL17}}

% TODO: index for transformer
% 'self-attention'
\TD{Attention Is All You Need -- Transformer network --- multi-head self-attention mechanism, key-value pairs \cite{DBLP:journals/corr/VaswaniSPUJGKP17}}

% self-attention \TD{Self-attention, less commonly intra-attention}
\TD{Long Short-Term Memory-Networks for Machine Reading \cite{DBLP:journals/corr/ChengDL16}}


%\TD{Nice table comparing mechanisms https://lilianweng.github.io/lil-log/2018/06/24/attention-attention.html}

\TD{in above post\cite{weng2018attention}: soft vs hard attention and global vs local attention}

% ``heads learn redundant key/query projections'' --> share
% https://github.com/epfml/collaborative-attention
\TD{Multi-Head Attention: Collaborate Instead of Concatenate \cite{Cordonnier2020MultiHeadAC}}

% soft vs hard and global vs local

\TD{Describes two variants: a ``hard'' stochastic attention mechanism (trainable via ``maximizing an approximate variational lower bound'' or REINFORCE) and a ``soft'' deterministic attention mechanism(trainable by standard back-propagation) \cite{DBLP:journals/corr/XuBKCCSZB15}. Soft attention --- scores to all entities (is differenetiable but expensive) and hard attention --- only selects one entity (non-differentiable (and complicated, reinforcement learning), but requires less computation at inference)}


% TODO: does this make sense?
\TD{Non-linear projection for K,Q, and V~\cite{DBLP:journals/corr/abs-2111-10017}}


\subsubsection{Scoring Functions}

% TODO: https://lilianweng.github.io/lil-log/2018/06/24/attention-attention.html#summary
\TD{table from \cite{weng2018attention}}


\subsection{Self-Attention}

\r{sometimes refered to as ``intra-attention''\cite{DBLP:journals/corr/VaswaniSPUJGKP17}. Keys, queries and values are all derived from the same sequence. \TD{Self-attention transforms a sequence to create a representation of itself.}}



\subsection{transformers}

% possibly useful: http://nlp.seas.harvard.edu/2018/04/03/attention.html

\TD{survey of recent transformer architectures \TD{Efficient Transformers: A Survey \cite{Tay2020EfficientTA}}}


% Factorized Attention to self-attention
\TD{Generating Long Sequences with Sparse Transformers \cite{DBLP:journals/corr/abs-1904-10509}}

% include reccurence:  "enables learning dependency beyond a fixed length" + "relative position encodings"
\TD{Transformer-XL: Attentive Language Models Beyond a Fixed-Length Context \cite{DBLP:journals/corr/abs-1901-02860}}

% extends DBLP:journals/corr/abs-1901-02860 -- 
% https://github.com/guolinke/TUPE
\TD{Compressive Transformers for Long-Range Sequence Modeling \cite{Rae2020CompressiveTF}}

% linear attention
\TD{Transformers are RNNs: Fast Autoregressive Transformers with Linear Attention \cite{Katharopoulos2020TransformersAR}}

% 
\TD{Transformer with Untied Positional Encoding (TUPE) --- Rethinking Positional Encoding in Language Pre-training \cite{Ke2020RethinkingPE}}



\TD{Reformer: The Efficient Transformer \cite{Kitaev2020ReformerTE}}


% TODO: top-down attention
% related to self-attention
% https://twitter.com/thomaskipf/status/1277570203665170432
\TD{Object-Centric Learning with Slot Attention \cite{Locatello2020ObjectCentricLW}}

\TD{Recurrent Independent Mechanisms \cite{Goyal2019RecurrentIM}}

% DETR -- also object detection
\TD{End-to-End Object Detection with Transformers \cite{Carion2020EndtoEndOD}}


% TODO: read https://lilianweng.github.io/lil-log/2020/04/07/the-transformer-family.html


\subsection{Positional Encodings}

\TD{Positional embedding and positional encoding tend to be used interchangably. However, typically an encoding means ``fixed'' while an embedding means ``learned'' or ``trainable''.}

% TODO: example of how word order matters (not is a good example)

\r{Attention/transformers view the inputs as sets, that is there is no order associated with each input. All information enters the attention block at once. This is in contrast to something like a recurrent model, in which the order of the inputs is implicit.}

\r{trade off: potentially faster (remove the dependancy of doing operations sequentially) and can also possibly help capture longer range dependancies (without additional complexity e.g. skip connections)}

\r{(re)introducing order to the input by including additional information -- the ``positional embedding''.}

\r{NOTE: Great blog posts on this subject~\cite{kazemnejad_2021, kernes_2021, kernes_2021B}}

\subsubsection{Positional Encoding Value}

\r{why not add linear/progressive value signifying order?}

\r{This would be called an aboslute positional embedding}

\r{Include index information [0, n], where n is the length of the sentance (minus 1). This could lead to magnitude issues. Where the singal from the word embeddings is ``washed out'' by the positional embedding.  Another consideration is that (may or may not be an issue depending on the application) is that you'd like to ensure you have the largest sequence in the training set that you expect to see in evaluation set. For example, if you only see sequences of length $25$ in the training data and then see a sequence of length $32$ during inference. The model will be unsure what to do with values $25 - 31$ (zero indexing). Depending on how you include the positional embedding (e.g. additive or concat), the model may misinterpret the values or be largely/entirely unsure what to make of these previously unseen values.}
	
	
\r{To address this you could either increase the magnitude of the word embeddings or normalize/scale the positional embedding.}

\r{However, niether are ideal.}

\r{Increasing the magnitude of the word embeddings would possibly work, though you may consider issues with exploding values in the network, but you'd still have a similar issue to what would happen if you normalized the positional embedding. }

\r{That is, the normalized positional embeddings may encode different information when the sentances are longer or shorter -- the delta between words in a 5 word sentance vs a 20 word sentance doesn't have a consistent meaning}

% NOTE: haven't read this yet (I don't think, though the link is purple...)
\TD{Self-Attention with Relative Position Representations~\cite{DBLP:journals/corr/abs-1803-02155}}

\r{Ideally the embedding would be able to account for all the issues we discussed.}

\begin{itemize}[noitemsep,topsep=0pt]
	\item consistent delta between each position
		\begin{itemize}[noitemsep,topsep=0pt]
			\item regardless of sequence length, if an instance is one instance away from another, the positional encoding should be the same e.g. in a length four sequence the positional encoding should be the same from instances $1$ and $2$ as it is for instances $19$ and $20$ in a length $22$ sequence.
		\end{itemize}
	\item generalize to sequence lengths unseen in training
\end{itemize}

\r{additionally, we'd prefer to have each instance in the sequence be unique. That is the positional encoding for one instance shouldn't be the same as another in the same sequence (e.g. two words in a sentance).}

\paragraph{Positional Encoding Value(s)}

\r{Rather than use a single value, a possible solution is to use an array of values.}

\TD{Relative positional encoding (rather than absolute).}


\TD{What if we were to use a binary array to represent each location?}

\TD{issue with binary}

\TD{}


\TD{CAPE: Encoding Relative Positions with Continuous Augmented Positional Embeddings~\cite{DBLP:journals/corr/abs-2106-03143}}

% NOTE: possibly relevant: https://aclanthology.org/2021.emnlp-main.266.pdf

\paragraph{Sinusoidal}

\TD{include figure with multiple frequencies and points on the x and y axis leading to embeddings}

\subsection{Positional Embeddings (learned ``encodings'')}

% possibly useful: https://theaisummer.com/positional-embeddings/

\TD{Learning to Encode Position for Transformer with Continuous Dynamical Model~\cite{DBLP:journals/corr/abs-2003-09229}}


\TD{What Do Position Embeddings Learn? An Empirical Study of Pre-Trained Language Model Positional Encoding~\cite{DBLP:journals/corr/abs-2010-04903}}

\subsubsection{Including Positional Embeddings}

% someones thoughts on  additive vs concat: https://www.reddit.com/r/MachineLearning/comments/cttefo/d_positional_encoding_in_transformer/exs7d08/

\paragraph{Additive}

\TD{saves memory (over concatenation -- less dimensions)}

\TD{figure}

\paragraph{Concatenation}

\TD{figure}


\section{MLP-Mixer}

\r{MLPs that are used to ``mix'' tokens (spatial) and ``mix'' channels (features)}

% possible blog: https://wandb.ai/wandb_fc/pytorch-image-models/reports/Is-MLP-Mixer-a-CNN-in-Disguise---Vmlldzo4NDE1MTU

% MLP resurgence
\TD{Do You Even Need Attention? A \cite{DBLP:journals/corr/abs-2105-02723}}

\TD{gMLP (Pay Attention to MLPs) \cite{DBLP:journals/corr/abs-2105-08050}}

\TD{MLP-Mixer: An all-MLP Architecture for Vision \cite{DBLP:journals/corr/abs-2105-01601}}

\TD{RepMLP: Re-parameterizing Convolutions into Fully-connected Layers for Image Recognition \cite{DBLP:journals/corr/abs-2105-01883}}

\TD{ResMLP: Feedforward networks for image classification with data-efficient training \cite{DBLP:journals/corr/abs-2105-03404}}
Conncurrent papers released looking to replace attention with MLPs.

\TD{Do You Even Need Attention? A Stack of Feed-Forward Layers Does Surprisingly Well on ImageNet \cite{MelasKyriazi2021DoYE}}




\section{Mixture of Experts (MoE)}

\TD{Breaking down a problem (task) into multiple sub-problems (sub-tasks), training and expert in each sub-problem, then learning a meta/gating model that routes information to a specific expert and combines outputs}

% Divide and conquer vs meta-learning approach


\TD{High level steps}
\begin{itemize}[noitemsep,topsep=0pt]
	\item Decompose task into subtasks
	\item Learn ``expert'' for each subtask 
	\item Decide which expert to use (gating model or gating expert)
	\item Combine outputs as needed (pool/aggregate/select)
\end{itemize}

\TD{``20 years MoE''~\cite{yuksel2012twenty}}

\TD{Outrageously Large Neural Networks: The Sparsely-Gated Mixture-of-Experts Layer~\cite{shazeer2017outrageously}}



\chapter{Attention}

\r{``An attention function can be described as mapping a query and a set of key-value pairs to an output,
	where the query, keys, values, and output are all vectors. The output is computed as a weighted sum
	of the values, where the weight assigned to each value is computed by a compatibility function of the
	query with the corresponding key.'' \cite{DBLP:journals/corr/VaswaniSPUJGKP17}}

\TD{Self-attention Does Not Need $O(n^{2})$ Memory~\cite{Rabe2021SelfattentionDN}}

%TODO: another blog to checkout https://distill.pub/2016/augmented-rnns/

\r{overview can be found here\cite{weng2018attention}}


\TD{RNNsearch~\cite{Bahdanau2015NeuralMT}. \TD{MOTIVATION -- rather than ``RNNencdec''}}

% TF attention implementation (https://www.tensorflow.org/tutorials/text/nmt_with_attention)



\TD{Massive Exploration of Neural Machine Translation Architectures \cite{DBLP:journals/corr/BritzGLL17}}

% TODO: index for transformer
% 'self-attention'
\TD{Attention Is All You Need -- Transformer network --- multi-head self-attention mechanism, key-value pairs \cite{DBLP:journals/corr/VaswaniSPUJGKP17}}

% self-attention \TD{Self-attention, less commonly intra-attention}
\TD{Long Short-Term Memory-Networks for Machine Reading \cite{DBLP:journals/corr/ChengDL16}}


%\TD{Nice table comparing mechanisms https://lilianweng.github.io/lil-log/2018/06/24/attention-attention.html}

\TD{in above post\cite{weng2018attention}: soft vs hard attention and global vs local attention}

% ``heads learn redundant key/query projections'' --> share
% https://github.com/epfml/collaborative-attention
\TD{Multi-Head Attention: Collaborate Instead of Concatenate \cite{Cordonnier2020MultiHeadAC}}

% soft vs hard and global vs local

\TD{Describes two variants: a ``hard'' stochastic attention mechanism (trainable via ``maximizing an approximate variational lower bound'' or REINFORCE) and a ``soft'' deterministic attention mechanism(trainable by standard back-propagation) \cite{DBLP:journals/corr/XuBKCCSZB15}. Soft attention --- scores to all entities (is differenetiable but expensive) and hard attention --- only selects one entity (non-differentiable (and complicated, reinforcement learning), but requires less computation at inference)}


% TODO: does this make sense?
\TD{Non-linear projection for K,Q, and V~\cite{DBLP:journals/corr/abs-2111-10017}}


\section{Compatibility/Scoring/Attention Functions}

% TODO: https://lilianweng.github.io/lil-log/2018/06/24/attention-attention.html#summary
\TD{table from \cite{weng2018attention}}


\subsection{Additive}

\TD{NEURAL MACHINE TRANSLATION BY JOINTLY LEARNING TO ALIGN AND TRANSLATE\cite{Bahdanau2015NeuralMT}}

\subsection{Dot-Product}

\TD{Effective Approaches to Attention-based Neural Machine Translation \cite{DBLP:journals/corr/LuongPM15}}

\r{exponential to promote sparsity}

\subsection{Scaled Dot-Product}

\TD{Attention Is All You Need -- Transformer network --- multi-head self-attention mechanism, key-value pairs \cite{DBLP:journals/corr/VaswaniSPUJGKP17}}


\subsection{Other}

% TODO: self attention
\r{sometimes refered to as ``intra-attention''\cite{DBLP:journals/corr/VaswaniSPUJGKP17}. Keys, queries and values are all derived from the same sequence. \TD{Self-attention transforms a sequence to create a representation of itself.}}



\section{softmax}

\TD{Softermax: Hardware/Software Co-Design of an Efficient Softmax for Transformers~\cite{DBLP:journals/corr/abs-2103-09301}}

% possibly helpful: https://towardsdatascience.com/google-deepminds-rfa-approximating-softmax-attention-mechanism-in-transformers-d685345bbc18
\TD{Random Feature Attention~\cite{DBLP:journals/corr/abs-2103-02143}}

\TD{Mixtape: Breaking the Softmax Bottleneck Efficiently~\cite{Yang2019MixtapeBT}}

\TD{cosFormer: Rethinking Softmax in Attention~\cite{Qin2022cosFormerRS}}

\TD{SOFT:~\cite{DBLP:journals/corr/abs-2110-11945}}

\section{Reducing Complexity}

% TODO: not quite sure where linear attention belongs yet
\TD{Transformers are RNNs: Fast Autoregressive Transformers with Linear
	Attention~\cite{DBLP:journals/corr/abs-2006-16236}}

\subsection{Low-Rank}

\subsection{Compression}

\subsubsection{Memory (keys and Values)}

\subsubsection{Query}

\subsection{Structural Sparsity}

\TD{Generating Long Sequences with Sparse Transformers~\cite{DBLP:journals/corr/abs-1904-10509} shows that the attention matrix is commonly sparse after training.}

\subsubsection{Global}

\begin{figure}[htp]
	\centering
	\includegraphics[width=0.3\textwidth]{example-image-a}\hfil
	\caption{\TD{global attention}}
	\label{fig:attn_global}
\end{figure}

\TD{a selection of nodes that can attend to the whole sequence and that can be attended by the whole sequence}

\paragraph{External}

\paragraph{Internal}


\subsubsection{Local}

\r{sometimes called sliding window attention}

\begin{figure}[htp]
	\centering
	\includegraphics[width=0.3\textwidth]{example-image-a}\hfil
	\includegraphics[width=0.3\textwidth]{example-image-b}\hfil
	\includegraphics[width=0.3\textwidth]{example-image-c}\hfil
	\caption{\TD{band attention (band, dilated, block local)}}
	\label{fig:attn_diagonal}
\end{figure}

\paragraph{Band}

\paragraph{Dialated}

\paragraph{Block Local}

\subsubsection{Random}

\begin{figure}[htp]
	\centering
	\includegraphics[width=0.3\textwidth]{example-image-a}\hfil
	\caption{\TD{global attention}}
	\label{fig:attn_random}
\end{figure}

\subsubsection{Combination}

% TODO: examples



\chapter{Multi-Headed Attention}

\r{rather than a single head}

\TD{Nothing guarantees that different heads attend to different positions or capture distinct features}




\chapter{Positional Encodings}

\TD{Positional embedding and positional encoding tend to be used interchangably. However, typically an encoding means ``fixed'' while an embedding means ``learned'' or ``trainable''.}

% TODO: example of how word order matters (not is a good example)

\r{Attention/transformers view the inputs as sets, that is there is no order associated with each input. All information enters the attention block at once. This is in contrast to something like a recurrent model, in which the order of the inputs is implicit.}

\r{trade off: potentially faster (remove the dependancy of doing operations sequentially) and can also possibly help capture longer range dependancies (without additional complexity e.g. skip connections)}

\r{(re)introducing order to the input by including additional information -- the ``positional embedding''.}

\r{NOTE: Great blog posts on this subject~\cite{kazemnejad_2021, kernes_2021, kernes_2021B}}

\section{Positional Values}

\r{why not add linear/progressive value signifying order?}

\r{This would be called an aboslute positional embedding}

\r{Include index information [0, n], where n is the length of the sentence (minus 1). This could lead to magnitude issues. Where the singal from the word embeddings is ``washed out'' by the positional embedding.  Another consideration is that (may or may not be an issue depending on the application) is that you'd like to ensure you have the largest sequence in the training set that you expect to see in evaluation set. For example, if you only see sequences of length $25$ in the training data and then see a sequence of length $32$ during inference. The model will be unsure what to do with values $25 - 31$ (zero indexing). Depending on how you include the positional embedding (e.g. additive or concat), the model may misinterpret the values or be largely/entirely unsure what to make of these previously unseen values.}


\r{To address this you could either increase the magnitude of the word embeddings or normalize/scale the positional embedding.}

\r{However, niether are ideal.}

\r{Increasing the magnitude of the word embeddings would possibly work, though you may consider issues with exploding values in the network, but you'd still have a similar issue to what would happen if you normalized the positional embedding. }

\r{That is, the normalized positional embeddings may encode different information when the sentences are longer or shorter -- the delta between words in a 5 word sentance vs a 20 word sentance doesn't have a consistent meaning}

% NOTE: haven't read this yet (I don't think, though the link is purple...)
\TD{Self-Attention with Relative Position Representations~\cite{DBLP:journals/corr/abs-1803-02155}}

\r{Ideally the embedding would be able to account for all the issues we discussed.}

\begin{itemize}[noitemsep,topsep=0pt]
	\item consistent delta between each position
	\begin{itemize}[noitemsep,topsep=0pt]
		\item regardless of sequence length, if an instance is one instance away from another, the positional encoding should be the same e.g. in a length four sequence the positional encoding should be the same from instances $1$ and $2$ as it is for instances $19$ and $20$ in a length $22$ sequence.
	\end{itemize}
	\item generalize to sequence lengths unseen in training
\end{itemize}

\r{additionally, we'd prefer to have each instance in the sequence be unique. That is the positional encoding for one instance shouldn't be the same as another in the same sequence (e.g. two words in a sentance).}

\paragraph{Positional Encoding Value(s)}

\r{Rather than use a single value, a possible solution is to use an array of values.}

\TD{Relative positional encoding (rather than absolute).}


\TD{What if we were to use a binary array to represent each location?}

\TD{issue with binary}

\subsection{Fixed vs Learned and Relative vs Absolute}

\r{positional information can be included as either absolute or relative.}

\r{Additionally, the included positional information can either be fixed or learned.}

\subsection{Absolute}

\subsubsection{Fixed}

\paragraph{Sinusoidal}

\TD{include figure with multiple frequencies and points on the x and y axis leading to embeddings}

\TD{Sinusoidal used in orginal~\cite{DBLP:journals/corr/VaswaniSPUJGKP17}}

\subsubsection{Learned}

\TD{BERT:~\cite{DBLP:journals/corr/abs-1810-04805}}

% fixed and learned
\TD{Fixed and learned, uses sinusoidal, but learns the frequency, On Position Embeddings in BERT~\cite{Wang2021OnPE}}

\TD{FLOATER --- Learning to Encode Position for Transformer with Continuous Dynamical Model~\cite{DBLP:journals/corr/abs-2003-09229}}

\subsection{Relative}

% TODO: read - https://www.youtube.com/watch?v=7XHucAvHNKs

\TD{Represent positional relationship between tokens, rather than sequence as a whole}

\TD{Included in the keys and values}

\TD{offset between key and query?}

\TD{InDIGO --- Insertion-based Decoding with Automatically Inferred Generation Order~\cite{Gu2019InsertionbasedDW}}

\TD{Music Transformer~\cite{Huang2018MusicT}}

\TD{Exploring the Limits of Transfer Learning with a Unified Text-to-Text Transformer~\cite{Raffel2020ExploringTL}}

\TD{Transformer-XL: Attentive Language Models beyond a Fixed-Length Context~\cite{Dai2019TransformerXLAL}}

\TD{DeBERTa: Decoding-enhanced BERT with Disentangled Attention~\cite{He2021DeBERTaDB}}


\subsubsection{Fixed}

\subsubsection{Learned}

\TD{relative~\cite{DBLP:journals/corr/abs-1803-02155}}


\subsubsection{Use in Linear Transformers}

\TD{whole attention matrix is needed for classical relative positional encodings}

\TD{Relative Positional Encoding for Transformers with Linear Complexity~\cite{DBLP:journals/corr/abs-2105-08399}}

\section{Hybrid approaches}

\TD{ \textbf{T}ransformer with \textbf{U}ntied \textbf{P}ositional \textbf{E}ncoding (TUPE) --- Rethinking Positional Encoding in Language Pre-training~\cite{Ke2021RethinkingPE}}

\TD{RoPE --- RoFormer: Enhanced Transformer with Rotary Position Embedding~\cite{Su2021RoFormerET}}





%%%%%%%%%%%%%%%%%%%%%
\section{TO INCLUDE}

\TD{What Do Position Embeddings Learn? An Empirical Study of Pre-Trained Language Model Positional Encoding~\cite{DBLP:journals/corr/abs-2010-04903}}


\TD{CAPE: Encoding Relative Positions with Continuous Augmented Positional Embeddings~\cite{DBLP:journals/corr/abs-2106-03143}}

% NOTE: possibly relevant: https://aclanthology.org/2021.emnlp-main.266.pdf


% possibly useful: https://theaisummer.com/positional-embeddings/


% Unclear, are these positional embeddings learned?
\TD{Convolutional Sequence to Sequence Learning~\cite{DBLP:journals/corr/GehringAGYD17}}




\section{Including Positional Information}

% someones thoughts on  additive vs concat: https://www.reddit.com/r/MachineLearning/comments/cttefo/d_positional_encoding_in_transformer/exs7d08/

\subsection{Additive}

\TD{saves memory (over concatenation -- less dimensions)}

\TD{figure}

\subsection{Concatenation}

\TD{figure}


\subsection{Multiple inclusions}

\TD{It's possible the information from the positional values lose their significance after each layer.}

\TD{The following papers include positional information at each layer.}

\TD{learned per-layer positional embedding --- Character-Level Language Modeling with Deeper Self-Attention~\cite{DBLP:journals/corr/abs-1808-04444}}

\TD{also per-layer addition of a positional embedding~\cite{Guo2019LowRankAL, DBLP:journals/corr/abs-2003-09229}}



\section{Beyond One Dimension}


\subsection{Two Dimensional}

\TD{Extend sinusoidal to 2D~\cite{DBLP:journals/corr/abs-1802-05751}}

\TD{An Intriguing Failing of Convolutional Neural Networks and the CoordConv
	Solution~\cite{DBLP:journals/corr/abs-1807-03247}}

\TD{Attention Augmented Convolutional Networks~\cite{DBLP:journals/corr/abs-1904-09925}}


\chapter{Transformer}

% TODO: section is being redone

% TODO: read https://lilianweng.github.io/lil-log/2020/04/07/the-transformer-family.html

% possibly useful: http://nlp.seas.harvard.edu/2018/04/03/attention.html

\TD{A Survey of Transformers~\cite{DBLP:journals/corr/abs-2106-04554}}

\TD{survey of recent transformer architectures \TD{Efficient Transformers: A Survey \cite{Tay2020EfficientTA}}}


% Factorized Attention to self-attention
\TD{Generating Long Sequences with Sparse Transformers \cite{DBLP:journals/corr/abs-1904-10509}}

% include reccurence:  "enables learning dependency beyond a fixed length" + "relative position encodings"
\TD{Transformer-XL: Attentive Language Models Beyond a Fixed-Length Context \cite{DBLP:journals/corr/abs-1901-02860}}

% extends DBLP:journals/corr/abs-1901-02860 -- 
% https://github.com/guolinke/TUPE
\TD{Compressive Transformers for Long-Range Sequence Modeling \cite{Rae2020CompressiveTF}}

% linear attention
\TD{Transformers are RNNs: Fast Autoregressive Transformers with Linear Attention \cite{Katharopoulos2020TransformersAR}}

% 
\TD{Transformer with Untied Positional Encoding (TUPE) --- Rethinking Positional Encoding in Language Pre-training \cite{Ke2020RethinkingPE}}



\TD{Reformer: The Efficient Transformer \cite{Kitaev2020ReformerTE}}


\TD{Rethinking Attention with Performers~\cite{DBLP:journals/corr/abs-2009-14794}}

\TD{LambdaNetworks: Modeling Long-Range Interactions Without Attention~\cite{DBLP:journals/corr/abs-2102-08602}}


% TODO: top-down attention
% related to self-attention
% https://twitter.com/thomaskipf/status/1277570203665170432
\TD{Object-Centric Learning with Slot Attention \cite{Locatello2020ObjectCentricLW}}

\TD{Recurrent Independent Mechanisms \cite{Goyal2019RecurrentIM}}

% DETR -- also object detection
\TD{End-to-End Object Detection with Transformers \cite{Carion2020EndtoEndOD}}



\section{Normalization}




\subsection{Normalization Function}

\TD{LayerNormalization \TD{Add reference}}

\TD{AdaNorm, no learnable parameters~\cite{DBLP:journals/corr/abs-1911-07013}}

\TD{PowerNorm~\cite{DBLP:journals/corr/abs-2003-07845}.}

\TD{ReZero~\cite{DBLP:journals/corr/abs-2003-04887}}

\TD{DeepNorm~\cite{Wang2022DeepNetST}}



\subsection{Normalization Placement}

\TD{Figure}

\subsubsection{Post-LN}

\TD{Original placement, after the residual connections~\cite{DBLP:journals/corr/VaswaniSPUJGKP17}}

\subsubsection{Pre-LN}

\TD{Possible first use~\cite{DBLP:journals/corr/abs-1803-07416}, as reported~\cite{DBLP:journals/corr/abs-2106-04554}}





\section{Positional Transformation}

\TD{Feed forward network}

\TD{Attention is Not All You Need: Pure Attention Loses Rank Doubly Exponentially with Depth~\cite{DBLP:journals/corr/abs-2103-03404}}


\subsection{Replacement}

\TD{Use of Mixture-of-Experts \TD{\ALR}}

\TD{GShard: Scaling Giant Models with Conditional Computation and Automatic
	Sharding~\cite{DBLP:journals/corr/abs-2006-16668}}

\TD{Switch Transformers: Scaling to Trillion Parameter Models with Simple
	and Efficient Sparsity~\cite{DBLP:journals/corr/abs-2101-03961}}


\subsection{non-linearity}

\TD{GLU~\cite{DBLP:journals/corr/abs-2002-05202}}




\part{Ensembling}

\include{ensembles}

\chapter{Term dump}

\textcolor{green}{Terms that are important but haven't been placed in the document yet}



\emph{Curse of Dimensionality} --- \r{phenomenon where the feature space becomes increasingly sparse as the number of dimensions/features is increased -- trade off between the density of instances in the feature space and the number of dimensions (number of descriptive features) \r{including too many features can paradoxically, lead to worsening of performance.} \textcolor{green}{TODO: mentioned in paper Bellman 1961}}. \r{The more data and features we have, the ``better'' we should be able to find hidden structures and patterns in the data. However, more data and features, also means training will become more difficult (there are more dimensions and and dimension relationships to explore).}

\emph{dummy variable or dummy coding} --- $0$ or $1$, constrast coding $-1$ or $1$ -- creates diff in diff between, either 1 or two units apart.



\emph{$e$} --- \textcolor{blue}{Euler's numbers}

\subsection{Distributions}



\emph{epoch} --- \textcolor{blue}{a complete pass through the entire training set. Every sample in the training set has been seen by the model.} 

\emph{vector} --- \textcolor{blue}{direction, and magnitude (length)}

\emph{eigenvalue} and \emph{eigenvector} --- \textcolor{blue}{`eigen' is a german word for ``belonging to'' or ``particular to''.} 

\TD{alternating least squares (ALS) - fix one then other, quadratic}



\chapter{Self-Supervised Learning (SSL)}

\r{``rebranded unsupervised''?}

\r{learn from data, even without labels present}

\TD{Evaluating Self-Supervised Pretraining Without Using Labels \cite{Reed2020EvaluatingSP}}

%% contrastive losses

% TODO: this paper likely belongs elsewhere
\TD{Siamese networks \cite{bromley1994signature}}

\TD{The triplet loss: (FaceNet: A Unified Embedding for Face Recognition and Clustering) \cite{DBLP:journals/corr/SchroffKP15}}

\TD{Beyond triplet loss: a deep quadruplet network for person re-identification \cite{DBLP:journals/corr/ChenCZH17}}

\TD{Debiased Contrastive Learning \cite{Chuang2020DebiasedCL}}

\TD{Contrastive Representation Learning: A \cite{DBLP:journals/corr/abs-2010-05113}}

\TD{Bootstrap Your Own Latent: A New Approach to Self-Supervised Learning \cite{Grill2020BootstrapYO}}

% NOTE: I don't understand why RAFT hasn't received more attention, this seems like a very good idea
\TD{Run Away From your Teacher: Understanding BYOL by a Novel Self-Supervised Approach \cite{Shi2020RunAF}}

\TD{SWAV paper ---  Unsupervised Learning of Visual Features by Contrasting Cluster Assignments \cite{Caron2020UnsupervisedLO}}

\TD{SimSiam paper [stop-gradient] --- Exploring Simple Siamese Representation Learning \cite{Chen2020ExploringSS}}

\TD{Barlow Twins: Self-Supervised Learning via Redundancy Reduction [cross-correlation matrix loss] \cite{Zbontar2021BarlowTS}}

\TD{Momentum Contrast for Unsupervised Visual Representation Learning \cite{DBLP:journals/corr/abs-1911-05722}}

\TD{SimCLR --- A Simple Framework for Contrastive Learning of Visual Representations [https://github.com/google-research/simclr] \cite{Chen2020ASF}}

\TD{Moco V2 --- Improved Baselines with Momentum Contrastive Learning \cite{Chen2020ImprovedBW}}

\TD{SimCLRv2 --- Big Self-Supervised Models are Strong Semi-Supervised Learners [https://github.com/google-research/simclr] \cite{Chen2020BigSM}}

% TODO: this title is rough
\part{Multiple Tasks and Datasets}

\chapter{Mutiple objectives}

% TODO: Not sure what the title should be
\TD{This part is a grouping of methodologies for mixing information from mutiple sources or tasks}

\TD{Break down section into}

\begin{itemize}[noitemsep,topsep=0pt]
	\item Transfer Learning
	\item Multi-Task Learning
	\item Meta-Learning
\end{itemize}

\TD{diagram + explaination  of each}



\chapter{Transfer Learning}

\TD{What is being transferred in transfer learning?~\cite{DBLP:journals/corr/abs-2008-11687}}

%TODO: read this survey
\TD{A Survey on Deep Transfer Learning \cite{DBLP:journals/corr/abs-1808-01974}}

\TD{A Comprehensive Survey on Transfer Learning~\cite{DBLP:journals/corr/abs-1911-02685}}

\TD{How transferable are features in deep neural networks? \cite{DBLP:journals/corr/YosinskiCBL14}}
\TD{CNN Features off-the-shelf: an Astounding Baseline for Recognition \cite{DBLP:journals/corr/RazavianASC14}}

% TODO: haven't read this one (I don't think), but looks relevant
\TD{Learning and transferring mid-level image representations using convolutional neural networks\cite{oquab2014learning}}
\TD{Pay attention to features, transfer learn faster CNNs\cite{wang2019pay}}

% TODO: is this talked about anywhere else? this is probably the best place for it.

\TD{TODO: transfer learning, using -- explanation}

\TD{tool that may sometimes be efficient way of getting to potentially more accurate approximations, faster. \TD{citations}}

% TODO: index
\r{using parameters or pre-trained components from a model/task for a new model/task.  In practice, this often amounts to running inputs through a network that has been previously trained, and obtaining ``embeddings'' from this model (sometimes at an abitrary layer in the network), and then using these ``embeddings'' as input to train an additional model on the desired task. The process of adapting these components to a new model/task is called fine-tuning}


\TD{Head2Toe: Utilizing Intermediate Representations for Better Transfer Learning~\cite{Evci2022Head2ToeUI}}


\begin{figure}[htp]
	\centering
	\includegraphics[width=0.5\textwidth]{example-image-a}\hfil
	\caption{Figure example layer hierarchy and where/when to transfer/freeze params -- this will be 1-2 figures and include many sub-figures \textcolor{green}{TODO}}
	\label{fig:transfer_learning_subfigs_a}
\end{figure}

\textcolor{green}{{freezing}\index{freezing} parameters or a layer means preventing the parameters from being updated during training. This is often controlled by a parameter called ``trainable''.}

% In relation to transfer learning and freezing, mention the difficulty of propagating updates though a large network

\TD{Scaling Laws for Transfer \cite{DBLP:journals/corr/abs-2102-01293}}

\r{One difficulty of fine tuning is knowing where and by how much to either freeze or learn. That is should you freeze the first $n\%$ of the network, why not $m\%$?. Maybe you should leave the entire network trainable? But if the entire network is trianable, the previously learned (and presumably useful features), may be erased by the updates. Aside from selecting where to make the distinction, the main method used to combat these issues is to modify the learning rate. There are two core methods to adjusting the learning rate to address these issues.}

\begin{itemize}[noitemsep,topsep=0pt]
	\item Learning rate schedule
	\item Layer-wise learning rates
\end{itemize}

\TD{These methods are described in more detail in section ~\ref{hp_learning_rate}}


\TD{Adversarially robust transfer learning \cite{DBLP:journals/corr/abs-1905-08232}}

% TODO: check this paper out
\TD{DT-LET: Deep Transfer Learning by Exploring where to Transfer \cite{Lin2020DTLETDT}}

\subsubsection{Potential downsides of TL}

\TD{biases, attacks}

\TD{A Target-Agnostic Attack on Deep Models: Exploiting Security Vulnerabilities of Transfer Learning \cite{DBLP:journals/corr/abs-1904-04334}}


\include{multitask}

\chapter{Meta Learning}


% TODO: I'm still not sure how/where to structure this

\section{Dense}

\TD{TODO}

\section{Convolutions}

% this reads strangely --> DNN on an image may not take advantage of the ``stationarity'' (statistics) of an image.

\r{When using a standard dense layer, all inputs are treated independently. However, adjacent pixels, on average, are highly highly correlated. For example, if there is a texture in the image, a similar pattern of pixels may occur repeatedly. Convolutions architecturally build in an implicit spatial structure to consider these spatial.}

% TODO: I'm not sure how I'm going to structure these yet or where I'll be placing them

% TODO: https://arxiv.org/abs/1904.11486
% https://www.youtube.com/watch?v=HjewNBZz00w


\TD{LeNet-5 \cite{lecun1998gradient}}

\r{Convolutions are built upon a lie -- that is we refer to the opperation as a convolution, yet it is in fact a cross-correlation operation since we don't rotate the kernel 180$\deg$. However, it is convention to refer to the operation as a convolution. For more, please see section \ref{conv_vs_cross}}

\r{translational invariance --- a property that relates to how a systems decisions are insensitive to the location of a features within an input. That is, if we're looking for an object or feature, our system shouldn't change if the object is in different locations within the input}

\TD{``Filter factorization'' (not the exact same definition of mathematical factorization)-- one $5\times5$ filter vs $2$ $3\times3$ filters stacked.  in the $5\times5$ there are $5\times5 = 25$ parameters, in the $3\times3$, there are $3\times3 \times 2 = 18$ learnable parameters, resulting in a ``cheaper'' operation.}

\TD{Neocognitron -- CNN paper prior to ``CNN''\cite{fukushima1982neocognitron}}

% Survey on CNNs
% TODO: a lot here -- good read
\TD{A Survey of the Recent Architectures of Deep Convolutional Neural Networks \cite{DBLP:journals/corr/abs-1901-06032}}


\TD{Squeeze-and-Excitation Networks \cite{DBLP:journals/corr/abs-1709-01507}}


% Graham Taylor
\r{weighted averaging operation in time or space}


\r{translation equivariant --- }

\TD{BlurPool --- ``fix is anti-aliasing by low-pass filtering before downsampling'' ---Making Convolutional Networks Shift-Invariant Again \cite{DBLP:journals/corr/abs-1904-11486}}


\r{spatial hierarchies --- \TD{TODO: figure raw data, abstract edges+, then more distinct images, then closer output to the output, then the final label}}


\r{typcially a feature extraction phase (consisting of convolutional and pooling layers) followed by a classifier block (dense layers).}

%%%% popular layer types
\textcolor{green}{TODO: feature maps, (height, width, and depth (also called channels axis)). Stride, filter size, depth. talk about parameters}

\r{The output feature map (every dimension in the depth axis is a feature/filter) --- after a convolution operation the depth of a layer is no longer representative of a color channel (like RGB), it is now representative of a feature extracted by the convolutional operation, these are called filters.}

\TD{Strided Convolution\cite{springenberg2014striving}}

\TD{Dilated Convolution --- `atrous' convolution. (famously used by wavenet), which is convenient in time series analysis.}

\r{weight tieing}


\textcolor{green}{TODO: figure}

\begin{figure}[htp]
	\centering
	\includegraphics[width=0.5\textwidth]{example-image-a}\hfil
	\caption{Figure example of convolution operation on 2d image \textcolor{green}{TODO}}
	\label{fig:conv_2d_example_calc}
\end{figure}

\begin{figure}[htp]
	\centering
	\includegraphics[width=0.5\textwidth]{example-image-b}\hfil
	\caption{Figure example of convolution operation on 3d image \textcolor{green}{TODO}}
	\label{fig:conv_2d_depth_example_calc}
\end{figure}

\textcolor{green}{TODO: examples of how different filter values and strides can effect the output dimensions.}




\section{Pooling}

\TD{TODO: examples of max vs average pooling}

%%%%%% research
\textcolor{blue}{Pooling may not fully determine learned deformation stability -- possibly filter smoothness\cite{ruderman2018learned}}

\r{downsampling}

\r{Why? importance of reducing the number of params.}

\TD{L2-pooling}

\TD{L2-pooling over the features or channels.}

\TD{additional --- learned/parameterized pooling}

\begin{figure}[htp]
	\centering
	\includegraphics[width=0.5\textwidth]{example-image-a}\hfil
	\caption{Figure example of max pooling operation on 2d image \textcolor{green}{TODO: I want this figure to be basic 2d}}
	\label{fig:pooling_max_2d_ex_a}
\end{figure}

\begin{figure}[htp]
	\centering
	\includegraphics[width=0.5\textwidth]{example-image-b}\hfil
	\caption{Figure example of average pooling operation on 3d image \textcolor{green}{TODO: I want this figure to be 3d}}
	\label{fig:pooling_avg_3d_ex_a}
\end{figure}


\r{may be better to use convolutional layers in place of the pooling layers\cite{springenberg2014striving}}

\section{Recurrent Cells}

% TODO: read this
% Recurrent / Echo state networks / ESN
\TD{The ``echo state'' approach to analysing and training recurrent neural networks-with an erratum note \cite{jaeger2001echo}}
\TD{Deep Echo State Network (DeepESN): A \cite{DBLP:journals/corr/abs-1712-04323}}

\subsection{Cell Advancements}

\subsubsection{LSTM}

% TODO: Nice overview of LSTMs: https://colah.github.io/posts/2015-08-Understanding-LSTMs/

Introduced in 1997 %\cite{hochreiter1997long}

\r{detect long term dependencies in sequence}

\r{two state vectors, short and long term}

\r{Main motivation: learning what to store in the long-term state and what to ``forget''.}

\r{at each time step, some information is ``stored'' and some information is ``forgotten''.}

\paragraph{variants}

\TD{Depth-Gated LSTM \cite{DBLP:journals/corr/YaoCVDD15}}

\TD{A Clockwork RNN \cite{DBLP:journals/corr/KoutnikGGS14}}

\TD{LSTM: A Search Space Odyssey \cite{DBLP:journals/corr/GreffSKSS15} --- survey of LSTM variants --- all variants are essentially equal.}


\paragraph{other directions}

% interesting paper on ``grid LSTMs'' -- not sure why they never become popular
\TD{Grid Long Short-Term Memory \cite{Kalchbrenner2016GridLS}}

\paragraph{Fully Connected Layers}


\begin{enumerate}[noitemsep,topsep=0pt]
	\item Main
	\item \textit{Gate Controllers}
	\begin{enumerate}[noitemsep,topsep=0pt]
		\item Forget
		\item Input
		\item Output
	\end{enumerate}
\end{enumerate}

\r{The gate controllers use a logistic activation fuction (output a range from 0 to 1). This output is then fed through an element-wise multiplication function and thus if the value is $0$, the gate is ``closed'', and $1$ if the gate is ``open''.}

\r{These gates are able to potentially:}

\begin{enumerate}[noitemsep,topsep=0pt]
	\item Recognize an important input
	\item Store the important input in a long-term state ()
	\item Preserve the information for as long as it's needed
	\item Extract the important information when needed
\end{enumerate}


\subparagraph{Main}

\begin{figure}
	\centering
	\includegraphics[width=0.5\textwidth]{example-image-a}\hfil
	\caption{\TD{Main Layer DIAGRAM}}
	%\label{}
\end{figure}

\r{This allows for the same basic functionality as a ``standard'' RNN cell --- however, the output, rather than being only sent to the next cell, is now partially stored in the long-term state.}


\subparagraph{Forget}

\r{Determines which part of the long-term state is forgotten/erased.}

\begin{figure}
	\centering
	\includegraphics[width=0.5\textwidth]{example-image-a}\hfil
	\caption{\TD{Forget Layer DIAGRAM}}
	%\label{}
\end{figure}



\subparagraph{Input}

\r{Determines which part of the output from the \textbf{main layer} are kept in the long-term state.}

\begin{figure}
	\centering
	\includegraphics[width=0.5\textwidth]{example-image-a}\hfil
	\caption{\TD{Input Layer DIAGRAM}}
	%\label{}
\end{figure}

\subparagraph{Output}

\r{Determines which part of the long term state is ``relevant'' (read and output).}

\begin{figure}
	\centering
	\includegraphics[width=0.5\textwidth]{example-image-a}\hfil
	\caption{\TD{Output Layer DIAGRAM}}
	%\label{}
\end{figure}


\paragraph{Other}

\subparagraph{Peephole Connections}

\r{In basic LSTM cells, the gate controller can only look at the input and previous short-term state. Peephole connections, proposed in 2000 \TD{cite gers2000recurrent} add an extra connection that allows for the gate controller to also see information from the long term state as well. }

\r{The previous long-term state also becomes an input to the forget and input gate. The current long-term state becomes an intput to the output gate.}



\subsubsection{GRU}

\r{The GRU (gated recurrent unit) is a varient of the LSTM cell \TD{cite - cho2014learning}. The main modifications include:}

\begin{itemize}[noitemsep,topsep=0pt]
	\item Both state vectors are merged into one state vector
	\item A single gate controller determines the \textbf{Forget} and \textbf{Input} gate
	\begin{itemize}[noitemsep,topsep=0pt]
		\item If the gate output is a 1, the input is open and the forget gate is closed. If the gate output is 0, the input gate is closed and the forget gate is open
	\end{itemize}
	\item \r{The output gate is removed and a new controller exists that controls which part of ht previous state will be ``shown'' to the main layer}. At each timestep the full state vector is output.
\end{itemize}

\subsection{Notes -- add}

\r{A recent paper \TD{greff2017lstm}, compares three LSTM variants and makes three main observations:}

\begin{itemize}[noitemsep,topsep=0pt]
	\item no significant architecture improvements over LSTMs
	\item forget gate and the output activation function are the most critical components
	\item \TD{hyperparams...}
\end{itemize}




\section{Capsule Networks}

% TODO: capsule networks
\TD{Dynamic Routing Between Capsules \cite{DBLP:journals/corr/abs-1710-09829}}

\section{Attention}

\r{``An attention function can be described as mapping a query and a set of key-value pairs to an output,
	where the query, keys, values, and output are all vectors. The output is computed as a weighted sum
	of the values, where the weight assigned to each value is computed by a compatibility function of the
	query with the corresponding key.'' \cite{DBLP:journals/corr/VaswaniSPUJGKP17}}

\TD{Self-attention Does Not Need $O(n^{2})$ Memory~\cite{Rabe2021SelfattentionDN}}

%TODO: another blog to checkout https://distill.pub/2016/augmented-rnns/

\r{overview can be found here\cite{weng2018attention}}


\TD{The original attention mechanism is introduced\cite{Bahdanau2015NeuralMT}.}

% TF attention implementation (https://www.tensorflow.org/tutorials/text/nmt_with_attention)

\TD{Effective Approaches to Attention-based Neural Machine Translation \cite{DBLP:journals/corr/LuongPM15}}

\TD{Massive Exploration of Neural Machine Translation Architectures \cite{DBLP:journals/corr/BritzGLL17}}

% TODO: index for transformer
% 'self-attention'
\TD{Attention Is All You Need -- Transformer network --- multi-head self-attention mechanism, key-value pairs \cite{DBLP:journals/corr/VaswaniSPUJGKP17}}

% self-attention \TD{Self-attention, less commonly intra-attention}
\TD{Long Short-Term Memory-Networks for Machine Reading \cite{DBLP:journals/corr/ChengDL16}}


%\TD{Nice table comparing mechanisms https://lilianweng.github.io/lil-log/2018/06/24/attention-attention.html}

\TD{in above post\cite{weng2018attention}: soft vs hard attention and global vs local attention}

% ``heads learn redundant key/query projections'' --> share
% https://github.com/epfml/collaborative-attention
\TD{Multi-Head Attention: Collaborate Instead of Concatenate \cite{Cordonnier2020MultiHeadAC}}

% soft vs hard and global vs local

\TD{Describes two variants: a ``hard'' stochastic attention mechanism (trainable via ``maximizing an approximate variational lower bound'' or REINFORCE) and a ``soft'' deterministic attention mechanism(trainable by standard back-propagation) \cite{DBLP:journals/corr/XuBKCCSZB15}. Soft attention --- scores to all entities (is differenetiable but expensive) and hard attention --- only selects one entity (non-differentiable (and complicated, reinforcement learning), but requires less computation at inference)}


% TODO: does this make sense?
\TD{Non-linear projection for K,Q, and V~\cite{DBLP:journals/corr/abs-2111-10017}}


\subsubsection{Scoring Functions}

% TODO: https://lilianweng.github.io/lil-log/2018/06/24/attention-attention.html#summary
\TD{table from \cite{weng2018attention}}


\subsection{Self-Attention}

\r{sometimes refered to as ``intra-attention''\cite{DBLP:journals/corr/VaswaniSPUJGKP17}. Keys, queries and values are all derived from the same sequence. \TD{Self-attention transforms a sequence to create a representation of itself.}}



\subsection{transformers}

% possibly useful: http://nlp.seas.harvard.edu/2018/04/03/attention.html

\TD{survey of recent transformer architectures \TD{Efficient Transformers: A Survey \cite{Tay2020EfficientTA}}}


% Factorized Attention to self-attention
\TD{Generating Long Sequences with Sparse Transformers \cite{DBLP:journals/corr/abs-1904-10509}}

% include reccurence:  "enables learning dependency beyond a fixed length" + "relative position encodings"
\TD{Transformer-XL: Attentive Language Models Beyond a Fixed-Length Context \cite{DBLP:journals/corr/abs-1901-02860}}

% extends DBLP:journals/corr/abs-1901-02860 -- 
% https://github.com/guolinke/TUPE
\TD{Compressive Transformers for Long-Range Sequence Modeling \cite{Rae2020CompressiveTF}}

% linear attention
\TD{Transformers are RNNs: Fast Autoregressive Transformers with Linear Attention \cite{Katharopoulos2020TransformersAR}}

% 
\TD{Transformer with Untied Positional Encoding (TUPE) --- Rethinking Positional Encoding in Language Pre-training \cite{Ke2020RethinkingPE}}



\TD{Reformer: The Efficient Transformer \cite{Kitaev2020ReformerTE}}


% TODO: top-down attention
% related to self-attention
% https://twitter.com/thomaskipf/status/1277570203665170432
\TD{Object-Centric Learning with Slot Attention \cite{Locatello2020ObjectCentricLW}}

\TD{Recurrent Independent Mechanisms \cite{Goyal2019RecurrentIM}}

% DETR -- also object detection
\TD{End-to-End Object Detection with Transformers \cite{Carion2020EndtoEndOD}}


% TODO: read https://lilianweng.github.io/lil-log/2020/04/07/the-transformer-family.html


\subsection{Positional Encodings}

\TD{Positional embedding and positional encoding tend to be used interchangably. However, typically an encoding means ``fixed'' while an embedding means ``learned'' or ``trainable''.}

% TODO: example of how word order matters (not is a good example)

\r{Attention/transformers view the inputs as sets, that is there is no order associated with each input. All information enters the attention block at once. This is in contrast to something like a recurrent model, in which the order of the inputs is implicit.}

\r{trade off: potentially faster (remove the dependancy of doing operations sequentially) and can also possibly help capture longer range dependancies (without additional complexity e.g. skip connections)}

\r{(re)introducing order to the input by including additional information -- the ``positional embedding''.}

\r{NOTE: Great blog posts on this subject~\cite{kazemnejad_2021, kernes_2021, kernes_2021B}}

\subsubsection{Positional Encoding Value}

\r{why not add linear/progressive value signifying order?}

\r{This would be called an aboslute positional embedding}

\r{Include index information [0, n], where n is the length of the sentance (minus 1). This could lead to magnitude issues. Where the singal from the word embeddings is ``washed out'' by the positional embedding.  Another consideration is that (may or may not be an issue depending on the application) is that you'd like to ensure you have the largest sequence in the training set that you expect to see in evaluation set. For example, if you only see sequences of length $25$ in the training data and then see a sequence of length $32$ during inference. The model will be unsure what to do with values $25 - 31$ (zero indexing). Depending on how you include the positional embedding (e.g. additive or concat), the model may misinterpret the values or be largely/entirely unsure what to make of these previously unseen values.}
	
	
\r{To address this you could either increase the magnitude of the word embeddings or normalize/scale the positional embedding.}

\r{However, niether are ideal.}

\r{Increasing the magnitude of the word embeddings would possibly work, though you may consider issues with exploding values in the network, but you'd still have a similar issue to what would happen if you normalized the positional embedding. }

\r{That is, the normalized positional embeddings may encode different information when the sentances are longer or shorter -- the delta between words in a 5 word sentance vs a 20 word sentance doesn't have a consistent meaning}

% NOTE: haven't read this yet (I don't think, though the link is purple...)
\TD{Self-Attention with Relative Position Representations~\cite{DBLP:journals/corr/abs-1803-02155}}

\r{Ideally the embedding would be able to account for all the issues we discussed.}

\begin{itemize}[noitemsep,topsep=0pt]
	\item consistent delta between each position
		\begin{itemize}[noitemsep,topsep=0pt]
			\item regardless of sequence length, if an instance is one instance away from another, the positional encoding should be the same e.g. in a length four sequence the positional encoding should be the same from instances $1$ and $2$ as it is for instances $19$ and $20$ in a length $22$ sequence.
		\end{itemize}
	\item generalize to sequence lengths unseen in training
\end{itemize}

\r{additionally, we'd prefer to have each instance in the sequence be unique. That is the positional encoding for one instance shouldn't be the same as another in the same sequence (e.g. two words in a sentance).}

\paragraph{Positional Encoding Value(s)}

\r{Rather than use a single value, a possible solution is to use an array of values.}

\TD{Relative positional encoding (rather than absolute).}


\TD{What if we were to use a binary array to represent each location?}

\TD{issue with binary}

\TD{}


\TD{CAPE: Encoding Relative Positions with Continuous Augmented Positional Embeddings~\cite{DBLP:journals/corr/abs-2106-03143}}

% NOTE: possibly relevant: https://aclanthology.org/2021.emnlp-main.266.pdf

\paragraph{Sinusoidal}

\TD{include figure with multiple frequencies and points on the x and y axis leading to embeddings}

\subsection{Positional Embeddings (learned ``encodings'')}

% possibly useful: https://theaisummer.com/positional-embeddings/

\TD{Learning to Encode Position for Transformer with Continuous Dynamical Model~\cite{DBLP:journals/corr/abs-2003-09229}}


\TD{What Do Position Embeddings Learn? An Empirical Study of Pre-Trained Language Model Positional Encoding~\cite{DBLP:journals/corr/abs-2010-04903}}

\subsubsection{Including Positional Embeddings}

% someones thoughts on  additive vs concat: https://www.reddit.com/r/MachineLearning/comments/cttefo/d_positional_encoding_in_transformer/exs7d08/

\paragraph{Additive}

\TD{saves memory (over concatenation -- less dimensions)}

\TD{figure}

\paragraph{Concatenation}

\TD{figure}


\section{MLP-Mixer}

\r{MLPs that are used to ``mix'' tokens (spatial) and ``mix'' channels (features)}

% possible blog: https://wandb.ai/wandb_fc/pytorch-image-models/reports/Is-MLP-Mixer-a-CNN-in-Disguise---Vmlldzo4NDE1MTU

% MLP resurgence
\TD{Do You Even Need Attention? A \cite{DBLP:journals/corr/abs-2105-02723}}

\TD{gMLP (Pay Attention to MLPs) \cite{DBLP:journals/corr/abs-2105-08050}}

\TD{MLP-Mixer: An all-MLP Architecture for Vision \cite{DBLP:journals/corr/abs-2105-01601}}

\TD{RepMLP: Re-parameterizing Convolutions into Fully-connected Layers for Image Recognition \cite{DBLP:journals/corr/abs-2105-01883}}

\TD{ResMLP: Feedforward networks for image classification with data-efficient training \cite{DBLP:journals/corr/abs-2105-03404}}
Conncurrent papers released looking to replace attention with MLPs.

\TD{Do You Even Need Attention? A Stack of Feed-Forward Layers Does Surprisingly Well on ImageNet \cite{MelasKyriazi2021DoYE}}




\section{Mixture of Experts (MoE)}

\TD{Breaking down a problem (task) into multiple sub-problems (sub-tasks), training and expert in each sub-problem, then learning a meta/gating model that routes information to a specific expert and combines outputs}

% Divide and conquer vs meta-learning approach


\TD{High level steps}
\begin{itemize}[noitemsep,topsep=0pt]
	\item Decompose task into subtasks
	\item Learn ``expert'' for each subtask 
	\item Decide which expert to use (gating model or gating expert)
	\item Combine outputs as needed (pool/aggregate/select)
\end{itemize}

\TD{``20 years MoE''~\cite{yuksel2012twenty}}

\TD{Outrageously Large Neural Networks: The Sparsely-Gated Mixture-of-Experts Layer~\cite{shazeer2017outrageously}}



\chapter{Continual learning}

\r{\IDI{continual learning}~\cite{ring1994continual} or \IDI{lifelong learning}~\cite{thrun1998lifelong}, where a model may continue to learn new tasks/data over time.}

\TD{Continual Lifelong Learning with Neural Networks: \cite{DBLP:journals/corr/abs-1802-07569}}

% TODO: read this, interesting idea...
\TD{Continual Learning with Self-Organizing Maps \cite{DBLP:journals/corr/abs-1904-09330}}

\TD{A holistic View of Continual Learning with Deep Neural Networks: Forgotten Lessons and the Bridge to Active and Open World Learning \cite{Mundt2020AWV}}


\section{Catastrophic Forgetting}

\r{\IDI{catastrophic forgetting}~\cite{MCCLOSKEY1989109} -- ``forgetting'' or previous knowledge.  May be with respect to previously learned tasks or distributions of data.}

% provides a overview of recent methods, though I'm not personally convinced w/their results
\TD{Wide Neural Networks Forget Less Catastrophically~\cite{DBLP:journals/corr/abs-2110-11526}}

\subsection{Overcoming catastrophic forgetting}

\subsubsection{Regularization-based}

\r{retain ``important'' parameters}

\subsubsection{Expansion-based}

\r{increase capacity/new modules for new tasks}

\subsubsection{Replay-based Methods}

\r{store portions of data that are relevant for certain tasks}


\part{Interpretability}

\include{interpretability}


\part{Model Releases: Using+Monitoring Models}


\chapter{Experimental Design}

\section{Overview}

\TD{\ALR stats section where needed}


\TD{Assumption here is that metrics and business usecase have been defined --- which should not be underestimated}
\TD{Design}
\begin{itemize}[noitemsep,topsep=0pt]
	\item Power analysis --- how big does the sample need to be \ALR
	\item Sample selection (\textit{e.g.} is your sample a user, time of day, group of users, type of order, filters needed? etc) \ALR
	\item $<$Experiment$>$
	\item Experimental Analysis \ALR
\end{itemize}

\TD{A/B tests and analysis on experiments is a noisy process \TD{link to book}}

\r{To increase the statistical power, the most straightforward methods may be to increase the amount of data. This may include running the experiment for longer or including more samples. However, due to a number of constraints, this may not always be possible.}

\r{However, there are other approaches that may be used}

\begin{itemize}[noitemsep,topsep=0pt]
	\item Stratification \ALR
	\item Control variates \ALR
\end{itemize}


\chapter{Model Reporting}

\TD{Model Cards for Model Reporting \cite{DBLP:journals/corr/abs-1810-03993}}

\chapter{Deployment}


\chapter{Monitoring}

% TODO: TFX

\chapter{Papers to read}



% from coursera course
% TODO: https://towardsdatascience.com/machine-learning-in-production-why-you-should-care-about-data-and-concept-drift-d96d0bc907fb
% TODO: https://christophergs.com/machine%20learning/2020/03/14/how-to-monitor-machine-learning-models/
\TD{Towards ML Engineering \cite{DBLP:journals/corr/abs-2010-02013}}
\TD{Challenges in Deploying Machine Learning: a Survey of Case Studies \cite{DBLP:journals/corr/abs-2011-09926}}
% This includes the famous figure of the ML block being very small
\TD{Hidden technical debt in machine learning system \cite{sculley2015hidden}}




\part{Ethics}

\include{ethics}

\part{My Opinions: Moving Forward}

\include{opinions}


\part{Applications}

\include{examples}

\include{ideas}

\part{Brief Reference}

\include{bloopers}



\textcolor{blue}{Brief overview of how all packages and environments work together.}

%%%%%%%%%%%% Environment
\chapter{Environment}

\textcolor{green}{TODO: Diagram of an Overview of Environment}

\textcolor{blue}{The following sections are quick overviews of various components. Relevant, common, and useful information (in the context of AI/ML/DL) is provided but not expanded upon. Recommended references for further reading are included in each section. I would not advise ``reading'' this chapter as it is more of a reference and meant to be skipped around.}


\textcolor{blue}{Popular cloud providers offering GPUs:}

\begin{itemize}
	\item \emph{AWS}~\cite{cloudHW_amazon_aws}
	\item \emph{GoogleCompute (GCP)}~\cite{cloudHW_google_cloud}
	\item \emph{Microsoft Azure}~\cite{cloudHW_micro_azure} 
	\item \emph{Nvidia}~\cite{cloudHW_nvidia_cloud}
\end{itemize}

%%%%%%%%%%%%%%%%%%%%%%%% Common
\chapter{Common Libraries}

\TD{TODO: Overview of Environment and libraries that are described}

% numpy
% images: opencv
% NLP: nltk
% data acquisition: scrapy, beautiful soup
% other: apache_beam
% database: sql, mongo
% analyzing: pandas
% visualizing: matplotlib, D3
% predicting: scikitlearn
% data reproducibility: pachyderm
% others: regular expressions, tangent, markdown
% other libs; clustering algorithms --- fastcluster, hdbscan, tslearn




\part{appendix}

\chapter{TODO:}


\section{Time Zones}

\r{Dealing with time zones are a colossal pain and should be avoided at all costs.}

\section{Types of missingness}

\TD{Data is rarely ever truly missing at random. Often the result of some unknown systematic error.}

% NOTE: I find this confusing, needs to be beefed up.
\r{Base types of missing data \cite{allison2001missing}}
\begin{itemize}[noitemsep,topsep=0pt]
	\item Missing at Random (MAR)
	\item Missing Completely at Random
	\item Ignorable (structurally missing)
	\item Nonignorable (missing not at random)
\end{itemize}

\paragraph{Missing at Random (MAR)}

\r{Can predict via other given data}

\paragraph{Missing Completely at Random}

\r{Completely unrelated to any given data}

\paragraph{ Ignorable (structurally missing)}

\r{Missing for a logical reason}

\paragraph{Nonignorable (missing not at random)}

\TD{UNSURE}



\section{Imputing Missing Values}
\label{appendix:imputing_missing_values}

\r{fill missing values}

\subsection{Imputation}

\r{Based on observations from the entire dataset}

% static
\TD{mean}
\TD{median}


\subsection{Interpolation}

\r{Based on observations from neighboring datapoints.  Interpolation can be viewed as a subset of imputation.}

% dynamic
\TD{linear}
\TD{Advanced}
\TD{LForward or reverse fill. Note that technically a reverse is a ``lookahead''}
\TD{Moving/Rolling values (\textit{e.g.} average or median)}

\TD{NOTE: averages could be either geometric or exponentially weighted moving averages (giving more weight to more recent data)}

% other
\TD{predicting}

\section{Anomaly Detection}

\r{difficult to evaluate -- often onnly evaluated on ``known'' anomalies.}

\subsection{methods}

\subsubsection{PCA}

\TD{use of dimensionality methods \ALR to reduce the feature space and look for anomalous instances.}

\r{sometimes PCA is ``chained'' i.e. PCA may be performed on an already reduced dataset}

\r{reconstruction error largely depends on the number of components kept --- balance in the number of components to keep: too few and the original data may not be constructed well enough and if there are too many they all may be proproduced}

\r{sparse could also be }

\TD{example}


\section{labeling data}

\TD{dimensionality reduction followed by clustering}

\section{Group Segmentation}

\TD{clustering}

\section{Time Series Concepts}

\TD{Stationarity --- the level of stability. Can we expect to predict the future given the past?}

\TD{Self-correlation --- }

\TD{Spurious correlations --- relationships that appear to be causal, but are not}







\part{Dump Space}

% This section is a container for ``random" bits and pices that will be fit into other sections


\textcolor{blue}{tensorflow's XLA compiler}

\textcolor{blue}{checkpoints -- allow: save/stop/resume training, resume on failure, predict from point}


%%%% may talk about in production
% Google Cloud MLE


%%%
% stats vs ML -- in ML you may keep outliers and build models for them. in ML outliers may be collapsed (capped) and in statistics they may be removed
% ML is used to learn the ``long tail'', make fine grained predictions, not just gobal averages

%% why try to stay in linear (like with feature crosses)
% NN with many layers are non-convex
% optimizing linear models is a convex problem (much easier)

% how have I not talked about transfer learning yet?

% optimizing is an NP-hard, non-convex optimization problem (coursera.need to double check)
\textcolor{blue}{L0-norm (the count of non-zero weights).}


%%%%


%%%
\textcolor{blue}{Each layer in a DNN is a composition of the previous layer. i.e. if layer 1 = f(x), then layer 1 = g(f(x)), layer three is h(g(f(x)) \textcolor{green}{TODO: create diagram}}



%%%% rough...
%\textcolor{red}{Multi-headed inference. Useful when using the same model for different tasks. e.g. using a model for one task, then later deciding to perform a similar task on the same data -- rather than train an entirely new model, the original model may be performing may of the same computations. }



%%%  preprocessing, this wasn't already somewhere? % plus index
\textcolor{blue}{whitening}

%%% unsupervised method
\textcolor{blue}{NFM (Non-Negative Matrix Factorization)}

%%%
\textcolor{blue}{TODO: parameter calculation -- use VGG example (conv + dense)}
\begin{table}
	\centering
	\begin{tabular}{|c|c|c|c|c|}
		\hline
		\multicolumn{5}{|c|}{\textbf{Number of VGG-16 Parameters}}     \\ \hline
		Layer & Out Shape & Weights & Bias & Total  \\ \hline
		\emph{Convolution}        & & $(in)\times(h\times w)\times(out)$ & $(out)$ & $weights+bias$    \\ \hline
		Conv3-64          & $224\times224\times64$ & $3\times(3\times3)\times64$ & $64$ & 1792    \\ \hline
		Conv3-64 (p)      & $112\times112\times64$ & $64\times(3\times3)\times64$ & $64$ & 36928    \\ \hline
		Conv3-128         & $112\times112\times128$ & $64\times(3\times3)\times128$ & $128$ & 73856     \\ \hline
		Conv3-128 (p)     & $56\times 56\times 128$ & $128\times(3\times3)\times128$ & $128$ & 147584   \\ \hline
		Conv3-256         & $56\times 56\times 256$ & $128\times(3\times3)\times256$ & $256$ & 295168   \\ \hline
		Conv3-256         & $56\times 56\times 256$ & $256\times(3\times3)\times256$ & $256$ & 590080   \\ \hline
		Conv3-256 (p)     & $28\times 28\times 256$ & $256\times(3\times3)\times256$ & $256$ & 590080   \\ \hline
		Conv3-512         & $28\times 28\times 512$ & $256\times(3\times3)\times512$ & $512$ & 1180160  \\ \hline
		Conv3-512         & $28\times 28\times 512$ & $512\times(3\times3)\times512$ & $512$ & 2359808  \\ \hline
		Conv3-512 (p)     & $14\times 14\times 512$ & $512\times(3\times3)\times512$ & $512$ & 2359808  \\ \hline
		Conv3-512         & $14\times 14\times 512$ & $512\times(3\times3)\times512$ & $512$ & 2359808  \\ \hline
		Conv3-512         & $14\times 14\times 512$ & $512\times(3\times3)\times512$ & $512$ & 2359808  \\ \hline
		Conv3-512 (p)     & $7\times 7\times 512$ & $512\times(3\times3)\times512$ & $512$ & 2359808    \\ \hline
		\emph{dense}         & & $(in)\times(num)$ & $(out)$ & $weights+bias$    \\ \hline
		fc1 (4096)        & 4096 &$(512\times7\times7)\times4096$ & $4096$ & $102764544$    \\ \hline
		fc2 (4096)        & 4096 &$(4096)\times4096$ & $4096$ & $16781312$    \\ \hline
		fc3 (1000)        & 1000 &$(4096)\times1000$ & $1000$ & $4097000$    \\ \hline
		\emph{Total} & & & & 138,357,544 \\ \hline
	\end{tabular}
	\caption{Calculation of VGG parameters. (p) denotes that the layer is followed by a pooling layer (which does not affect the parameter count)}
	\label{tab:vgg_parameter_count}
\end{table}




%%%
\textcolor{green}{TODO: NN from scratch in appendix}

%%%
\textcolor{green}{TODO: CNN without layers API -- in github}

%%% interactive
\textcolor{blue}{The TensorFlow playground~\cite{tf_playground} (\textcolor{green}{TODO: screen shot}).  There are two other notable projects --- deeplearn.js~\cite{deeplearnjs} and ConvNetJS~\cite{convnet_js}.}


%%% distributed
\textcolor{blue}{Distributed TensorFlow Section}


%%% CUDA
\textcolor{blue}{CUDA (Compute Unified Device Architecture) library --- computation}
\textcolor{blue}{cuDNN (CUDA Deep Neural Network) library --- library of GPU-accelerated DNN primitives (activations, normalization, convolutions, pooling)}


%% model quantization
\textcolor{blue}{drop the parameter float precision from 32 bits (\code{tf.float32}) to 16 bits (\code{tf.bfloat16})}

%%%%
\textcolor{blue}{performance metrics --- Bayes error}

\textcolor{blue}{performance metrics --- Ability to perform on new, previously unseen data, generalization error (test error)}


\textcolor{green}{TODO: figure similar to DL pipeline in my thesis and similar to figure 1.5 in DL book. DL automates feature engineering}


\textcolor{green}{MNIST: described by Geoffrey Hinton as ``the drosophilia'' of machine learning. --- fruitflies are often used in biology as controlled laboratory experiments. Also considered the ``hello world'' of deep learning. The dataset is created from grayscale images ($28 \times 28 \times 1$) and has 10 labels 0-9.  There exist 60,000 training images and 10,000 test images. Created by the National Institute of Standards and Technology.}


\textcolor{green}{Two camps of statistics: frequentist estimators and bayesian inference.}


% TODO: figure -- [tensorflow ] -> cudnn -> cuda -> GPU | -> cuda -> GPU | -> [lib] -> CPU


%%
\textcolor{blue}{nonstationary problems --- unsolvable --- need data that makes sense for the problem. Predicting rentals of snowmobiles doesn't make as much sense if you only have data from summer. predicting cloth purchases during summer doesn't make sense if you only have data from winter.}


%%%
\textcolor{green}{Visualizing the output of a CNN}

\begin{enumerate}
	\item Filters: \r{help understand what visual patter a filter is receptive to \textcolor{green}{TODO: lots of examples and refs to implementations}}
	\item Intermediate outputs: \r{help understand the hierarchy of what is important to the classification task \textcolor{green}{TODO: lots of examples and refs to implementations}}
	\item Heatmaps of activation: \r{Help understand what parts (and by relatively how much) of an image were in its classification. \textcolor{green}{TODO: lots of examples and refs to implementations}}
\end{enumerate}


%%
\r{NOTE: somewhere about terms 'higher' and 'lower' levels of an architecture and how they're meaningless/interchangable and depend on context e.g. a diagram vs concept of lower=being closer to raw input.}

%%
\r{dataset == a ``sample'' in statistics}

%%
\r{decision boundary -- separation of classes}

%%

\r{prior knowledge --- additional knowledge about the desired form of a solution that is not obvious in the training data.  Inclusion of prior knowledge may influence the design of a solution, usually though preprocessing.}


\r{scale invariance --- a property that relates to how a systems decisions are insensitive to a uniform resizing of a feature within an input }

%%
\r{Intrisic dimensionality --- }


%%
\r{predictions on new data --- good ``generalization'' -- error/performance metrics}

%%
\r{controlling the complexity of a model = controlling the number of parameters. controlling the \textit{effective complexity} is optimizing the generalization performance of the model (using a penalty term (regularization)).}


\r{radial basis function --- }


% page 16 or neural networks (p31 on tablet)
\r{\textcolor{green}{(Barron 1993)} -- \textcolor{red}{``neural networks offer a dramatic advantage for function approximation in spaces of many dimensions''} --- (+) efficient scaling with dimensionality --- (-) now a non-linear optimiation problem = computationally intensive and may include many minimal in the error function}


%%
\r{\textit{joint probability} --- $x_a$ and $x_b$. \textit{conditional probability} --- P of $x_b$, given $x_a$}

%% TODO: stylegan
\TD{https://arxiv.org/abs/1812.04948}


%% BERT - GPT2

%%
\TD{\cite{smith2018disciplined}}


%%
\r{read this https://ai.googleblog.com/2019/06/innovations-in-graph-representation.html}

%% curiousity and "...allow the agent to create rewards for itself..."
\r{- (link: https://www.frontiersin.org/articles/10.3389/neuro.12.006.2007/full)
	- https://pathak22.github.io/noreward-rl/
	- (link: https://arxiv.org/abs/1705.05363) arxiv.org/abs/1705.05363
	- (link: https://arxiv.org/abs/1810.02274) arxiv.org/abs/1810.02274)
	- https://ai.googleblog.com/2018/10/curiosity-and-procrastination-in.html}


%%
\TD{Kohonen networks (self-organizing maps) \TD{Improving Self-Organizing Maps with Unsupervised Feature Extraction \cite{Khacef2020ImprovingSM}}}

%%
% p.20 of Neural Networks 
\r{``the outputs of neural networks can be interpreted as (approximations to) posterior probabilities''} \r{two stages in a classification process 1) \textit{inference} --- data is used to determine the values for the posterior probablities 2) \textit{decision making} --- the probabilities are used to make new decisions}


%%
\textcolor{green}{TODO: Bayesian vs frequentist --- see page 21 of neural networks (p36 on tablet)}

%%
\r{sigmoid == `S-shaped' == the logistic form of sigmoid maps $(-\infty, \infty)$ to $(0, 1)$}

%%
\r{linearly seperable --- where a dataset can be correctly seperated by a linear (hyperplanar) decision boundry}
\r{non linearly separable --- two-dimensional excllusive OR problem. --- generalized to d-dimensions when it is knowns as the d-bit parity problem.}

%%
\r{small changes in inputs, ideally, should not generally lead to dramatic changes in the outputs --- leading to a mapping that is represented relatively smoothly.}

%%
\r{colinear --- \textcolor{green}{TODO}}

%%
\TD{robbins-monro procedure}

%%
\r{\TD{perceptron convergence theorem} --- ``for any linearly separable data set, the learning rule (see fig 3.68 in NN - p100), is guaranteed to find a solution in a finite number of steps.'')}

\TD{the ``pocket algorithm'' --- finding solutions to problems which are not linearly separable.}

%%
\r{``projection pursuit regression'' --- similar to a two-layer feed-forward neural network --- typcially the parameters, rather than being all updated simultaneously such as in a nerual network ,are optimized cyclically in groups. Training takes place for one hidden unit at a time}
% another framework for non-linear regression
\r{``generalized additive models \TD{Hastie and Tibshirani, 1990} --- restrictive class of models since it does not allow for interactions between the input variables. === a generalization that does allow for interactions is the ``multi-variate adaptive regression splies (MARS) \TD{(Friedman, 1991)}}


%%
% this is pretty interesting... checkout p136 of NN_bishop, p152 tablet
\TD{Kolmogorov's Theorem}

%%
% p148(163) of NN Bishop
\TD{Jacobian Matrix}
% p150(165) of NN Bishop
\TD{Hessian Matrix}
% diagonal approximation
% outer product approximation
% inverse hessian
% finite differences
% exact evaluation of the hessian
\TD{Radial Basis Functions}


%% Look into
\TD{Max norm constraints --- enforce an absolute upper bound.}


%% Unsupervised pretraining - 2006, Hinton
\TD{greedy layer-wise unsupervised pretraining. greedy since each portion of the network is trained independently. Typically, today, layers are trained jointly using backpropagation}



\subsection{Restricted Boltzmann Machines}
\r{Restricted Boltzmann machines (RBMs) -- no output layer}

\TD{Deep Belief Networks (DBN) --- RBMs linked together to form multistage neural netowrk. Each RBM generates a representation of the data that the subsequent layer builds upon.  --- may be used as feature detectors}

\TD{neural network paper\cite{hansen1990neural}}

\TD{Bracketing: in the context of optimizers? may be work mentioning?}

%%
\r{code resuse and \code{tf.reuse}}

%%
\r{trying to learn the activation function -- however, this is partially already done by using multiple layers}

%%
\r{``learning is cast as an optimization problem -- minimize the loss with regularization''}

%% debugging
\TD{practical advice --- 1) looking at the norm of the outputs? 2) looking at the gradients at different layers and making sure they're relatively in the same magnitude i.e. you wouldn't want very small gradients at one level and super large gradients in another layer. 3) finite-difference approximation of the gradient (``gradient checking''). forward pass + epsilon vs forward pass - epsilon should be roughly the same.  $\frac{f(x+\epsilon) - f(x-\epsilon)}{2\epsilon}$ 4) ensure overfitting is possible on a small dataset, this should reach ``perfect'' (at least for classification on say a sample size of 50)}



%% Hugo talk
\r{relu may act as a gate, in which values are on or off}

%% increasing generalization
\r{dropout on the input layer of an autoencoder}


%%
\r{previously thought that the neural networkworks were difficult to train because they often get stuck in local minima --- however, this is now thought to be untrue and that many of the local minima points are saddle points. \TD{CITATION?}}

\TD{flat minima generalize better --- 1997 paper Hock}


%% padding -- how has this not been written about yet?
\TD{padding --- padding examples --- half(same), full / valid, no padding, reflect padding.}


\TD{mixed effect is a combination of fixed and random effects}

% in relation to imagery/spatial data
\r{local contrast normalization}

%%% upsampling
\r{\begin{itemize}[noitemsep,topsep=0pt]
		\item nearest neighbors
		\item ``bed of nails'' - copy to one position
		\item ``max unpooling'' -- remember which position the value came from
		\item learnable (transpose convolution) --- fractional stride -- ratio of in to out, stride of 1 on input and n on output.
	\end{itemize}
}


\TD{In vs Out of memory data}


% Resources:
% 1. https://medium.com/tensorflow/introducing-tensorflow-model-analysis-scaleable-sliced-and-full-pass-metrics-5cde7baf0b7b
\r{A \textit{dynamic placer} algorithm is presented in the {TensorFlow whitepaper}~\cite{abadi2016tensorflow_device_placement} that is capable of automatically distributing operations across devices. This algorithm takes into consideration  estimates of the sizes of input and output tensors, computation time for each node, \textcolor{green}{OTHERS - TODO:read entire paper}.}

% TODO: see slides I downloaded by Andrew Zisserman
\r{self-supervised learning}


%% shuffling
\r{include shuffling information}

\r{multinomial distribution}

\r{``probabalistic matrix factorization''}

\r{``affine functions''}

\TD{Outrageously Large Neural Networks: The Sparsely-Gated Mixture-of-Experts Layer~\cite{shazeer2017outrageously}}


\TD{``weight norm'' or weight normalization \cite{salimans2016weight}}


\r{Show a graph of traditional ML, small - large DL and how in data limited scenarios the ordering is difficult (but need to include something about effort) and tends to skew toward ML, but in large data, tends to scew towards large DL}

\TD{trend in moving from intermediate representations (hand engineered features) and toward fully learned/raw data to predictions.}



\subsection{Greek letters}
\TD{basic Greek letters}


\subsection{Basic Log Math}
\TD{include basic log math}

\subsection{Basic Matrix Math}
\TD{include basic log math}

\r{matrix decomposition}

\subsection{Basic Linear Algebra}
\TD{basic LA}


\subsection{Representation Learning}

% p.4 of DL
\r{Representation learning: learn the representation from input to output, not just the mapping.}
\r{form low(er) dimensional features}
\r{latent space}


\r{snapshot ensembles \cite{huang2017snapshot} - obtain parameters at each lowest point and ensemble}


\r{entity embeddings \cite{guo2016entity}}

\subsection{others}


% few shot learning
\r{Protoypical networks \cite{snell2017prototypical}}


% what it is, etc.
\TD{Metric learning}

% TODO: possibly useful: https://towardsdatascience.com/metric-learning-loss-functions-5b67b3da99a5
% another possibly useful link: https://github.com/KevinMusgrave/pytorch-metric-learning
\TD{metric learning survey \cite{kaya2019deep}}


% https://weightagnostic.github.io/
\TD{Weight Agnostic Neural Networks \cite{DBLP:journals/corr/abs-1906-04358}}


\r{optimization trajectories of neural networks \TD{On the Spectral Bias of Neural Networks \cite{Rahaman2019OnTS}} \TD{Exact solutions to the nonlinear dynamics of learning in deep linear neural networks \cite{Saxe2013ExactST}}}


%%%%%%%%%%%%%%%%%%%%%

%
\r{variable length encoding. ``prefix property'' --- no codeword should be the prefix of another codeword}



% TODO:
\TD{Energy Functions}

% use of softmax and normalization
% https://stackoverflow.com/questions/17187507/why-use-softmax-as-opposed-to-standard-normalization

\section{research to include}



%% 
\TD{synthetic petri dish -- inner and outer loop \cite{rawal2020synthetic}}


% use style gan to create augmented data.
\TD{DermGAN: Synthetic Generation of Clinical Skin Images with Pathology \cite{Ghorbani2019DermGANSG}}

\TD{Visual attention \cite{DBLP:journals/corr/XuBKCCSZB15}}

% TODO: index for ptr-nets / pointer networks
\TD{Pointer Networks (Ptr-Nets) \cite{Vinyals2015PointerN}}

\TD{Non-local Neural Networks \cite{DBLP:journals/corr/abs-1711-07971}}

% augmented AI
\TD{Does the Whole Exceed its Parts? The Effect of AI Explanations on Complementary Team Performance \cite{Bansal2020DoesTW}}


% blog: https://ai.googleblog.com/2020/06/spinenet-novel-architecture-for-object.html 
\TD{SpineNet: Learning Scale-Permuted Backbone for Recognition and Localization \cite{Du2019SpineNetLS}}


% TODO: https://arxiv.org/abs/2006.16668 --- tomorrow?

% https://twitter.com/zacharynado/status/1276252197915942927
% ~``improve calibration on dataset shift''
\TD{Evaluating Prediction-Time Batch Normalization for Robustness under Covariate Shift \cite{Nado2020EvaluatingPB}}

% 1000s of tasks with little forgetting: https://twitter.com/Mitchnw/status/1278711255977492482
\TD{Supermasks in Superposition \cite{Wortsman2020SupermasksIS}}


% no normalization or skip connections - image classification, https://github.com/HaozhiQi/ISONet
\TD{Deep Isometric Learning for Visual Recognition \cite{Qi2020DeepIL}}


% training on synthetic data
\TD{Synthetic Data for Deep Learning \cite{Nikolenko2019SyntheticDF}}


% Sparsely-Gated MoE > 600B -- significant: 1T weights, but issues --
\TD{GShard: Scaling Giant Models with Conditional Computation and Automatic Sharding \cite{Lepikhin2020GShardSG}}


% 
\TD{Bag of Tricks for Image Classification with Convolutional Neural Networks \cite{DBLP:journals/corr/abs-1812-01187}}




% TODO: group convolutions: https://colah.github.io/posts/2014-12-Groups-Convolution/


% TODO: create section for Test-Time Augmentation (TTA)
% TODO: index TTA
\TD{Greedy Policy Search: A Simple Baseline for Learnable Test-Time Augmentation \cite{Molchanov2020GreedyPS} (TTA)}

\TD{Training independent subnetworks for robust prediction \cite{DBLP:journals/corr/abs-2010-06610} --- MIMO, achieve the benefits of ensembled neural networks ``without the multiple forward passes''.}


% NOTE: possible paper on hyper parameter optimization
\TD{Bayesian Optimization for Selecting Efficient Machine Learning Models \cite{Wang2020BayesianOF}}

% 
\TD{The Hardware Lottery \cite{Hooker2020TheHL}}



\TD{pretraining with \textit{random} images, i.e. not Imagenet --- Self-supervised Pretraining of Visual Features in the Wild \cite{Goyal2021SelfsupervisedPO}}

\TD{RegNet --- Designing Network Design Spaces \cite{Radosavovic2020DesigningND}}



\TD{Neural Processes \cite{DBLP:journals/corr/abs-1807-01622}}

\TD{Why AI is Harder Than We Think \cite{DBLP:journals/corr/abs-2104-12871} ``everyone hurridly changes the names of their research projects to something else. This condition is called `AI winter' '' four fallacies}




\TD{Torch.manual\_seed(3407) is all you need: On the influence of random seeds in deep learning architectures for computer vision \cite{Picard2021Torchmanual} ``even if the variance is not very large, it is surprisingly easy to find an outlier that performs much better or much worse than the average''. Shows that pretraining reduces variance, but it is still present. Limitations worth noting, but likely won't chage the high level message (HP are finiky, results can vary)}

\TD{Data and its (dis)contents: A survey of dataset development and use in machine learning research \cite{Paullada2020DataAI} ``Survey the many concerns raised about the way we collect and use data in machine learning and advocate that a more cautious and thorough understanding of data is necessary to address several of the practical and ethical issues of the field'' Bring up legal issues of creating datasets (Montreal data license)}



% TODO: ~improve reliability by multiple individual submodels
\TD{Can You Trust Your Model's Uncertainty? Evaluating Predictive Uncertainty Under Dataset Shift \cite{Ovadia2019CanYT}}


% TODO: ~show that MoEs and ensembles complementary features and work well together
\TD{Sparse MoEs meet Efficient Ensembles \cite{Allingham2021SparseMM}}


% TODO: DenseNet
\TD{Densely Connected Convolutional Networks \cite{DBLP:journals/corr/HuangLW16a}}



% TODO: justification of DL concepts mathematically (haven't read this yet...)
\TD{Mathematics of Deep Learning \cite{DBLP:journals/corr/abs-1712-04741}}


% TODO: these likely should be cited together
% TODO: MultiModel
\TD{One Model To Learn Them All \cite{DBLP:journals/corr/KaiserGSVPJU17}}
\TD{Perceiver: General Perception with Iterative Attention \cite{DBLP:journals/corr/abs-2103-03206}}
\TD{Perceiver IO: \cite{DBLP:journals/corr/abs-2107-14795}}
% Multi modal multi task learning model
\TD{OmniNet: \cite{DBLP:journals/corr/abs-1907-07804}}




% TODO: neural ODE
\TD{Neural Ordinary Differential Equations \cite{DBLP:journals/corr/abs-1806-07366}}

\TD{BinaryNet: Training Deep Neural Networks with Weights and Activations Constrained to +1 or -1 \cite{DBLP:journals/corr/CourbariauxB16}}

% TODO: there should be a "Additional papers to be aware of" section

% TODO: CycleGan and cycle consistency
% I've written about this before, but I'm unsure where it is...
\TD{Unpaired Image-to-Image Translation using Cycle-Consistent Adversarial Networks \cite{DBLP:journals/corr/ZhuPIE17}}


% TODO: read https://openai.com/blog/deep-double-descent/
\TD{Deep Double Descent: Where Bigger Models and More Data Hurt \cite{Nakkiran2020DeepDD}}

\TD{GROKKING: GENERALIZATION BEYOND OVERFIT-TING ON SMALL ALGORITHMIC DATASETS \cite{power2021grokking}}


% tf lattice
\TD{Monotonic Calibrated Interpolated Look-Up Tables \cite{DBLP:journals/corr/GuptaCPVCMM15}}


%% vision transformers
\TD{An Image is Worth 16x16 Words: Transformers for Image Recognition at Scale \cite{DBLP:journals/corr/abs-2010-11929}}
% TODO: still need to read
\TD{Vision Transformers are Robust Learners \cite{DBLP:journals/corr/abs-2105-07581}}






% student teacher, transfering attention from one network to another, transfer learning?
\TD{Paying More Attention to Attention: Improving the Performance of Convolutional Neural Networks via Attention Transfer \cite{DBLP:journals/corr/ZagoruykoK16a}}


% possibly self-supervised?
\TD{Objects that Sound \cite{DBLP:journals/corr/abs-1712-06651}}

% optimization, gradient clipping, batch norm
\TD{Adaptive Gradient Clipping  --- High-Performance Large-Scale Image Recognition Without Normalization \cite{DBLP:journals/corr/abs-2102-06171}}


% TODO: to consider
\TD{Carbon Emissions and Large Neural Network Training \cite{DBLP:journals/corr/abs-2104-10350}}


% vision transformers and patches
\TD{Aggregating Nested Transformers \cite{DBLP:journals/corr/abs-2105-12723}}
\TD{Swin Transformer: Hierarchical Vision Transformer using Shifted Windows \cite{DBLP:journals/corr/abs-2103-14030}}



\TD{Robust fine-tuning of zero-shot models \cite{Wortsman2021RobustFO}}


\TD{Learning Disentangled Representations with Semi-Supervised Deep Generative Models \cite{Narayanaswamy2017LearningDR}}


% I thought I already included this...
\TD{Grokking: Generalization Beyond Overfitting on Small Algorithmic Datasets~\cite{Power2022GrokkingGB}}

\TD{Deep Double Descent: Where Bigger Models and More Data Hurt \cite{DBLP:journals/corr/abs-1912-02292}}

\TD{Toward Trustworthy AI \cite{DBLP:journals/corr/abs-2004-07213}}

% weak supervision
% https://ai.stanford.edu/blog/weak-supervision/


\TD{Second-Order Neural ODE~\cite{DBLP:journals/corr/abs-2109-14158}}


% leslie smith tweet to paper
% https://twitter.com/lnsmith613/status/1480666855048159235?s=20
\TD{Machine Learning Application Development: Practitioners' Insights~\cite{Rahman2021MachineLA}}


% conv nets strike back??
% manual fine tuning >?  automl?
\TD{A ConvNet for the 2020s~\cite{Liu2022ACF}}

% TODO:
\TD{GOPHER: Scaling Language Models: Method~\cite{DBLP:journals/corr/abs-2112-11446}}
\TD{RETRO: Improving language models by retrieving from trillions of tokens~\cite{DBLP:journals/corr/abs-2112-04426}}

%%%%%
% reinforcement learning = sequence modeling problem ?
\TD{Decision Transformer: Reinforcement Learning via Sequence Modeling~\cite{DBLP:journals/corr/abs-2106-01345}}
\TD{Reinforcement Learning as One Big Sequence Modeling Problem~\cite{DBLP:journals/corr/abs-2106-02039}}
%%%%%



%
\TD{Deep Information Propagation~\cite{Schoenholz2017DeepIP}}

% self supervised learning
\TD{FixMatch: Simplifying Semi-Supervised Learning with Consistency and Confidence~\cite{DBLP:journals/corr/abs-2001-07685}}


% TODO: synthetic gradients --- how I have I not included these?? they're my favorite
\TD{Gradients without Backpropagation~\cite{Baydin2022GradientsWB}}
\TD{Decoupled Neural Interfaces using Synthetic Gradients~\cite{DBLP:journals/corr/JaderbergCOVGK16}}


%%%%%%%%%%%%%%%%%%%%%%%%%%%%%%%%%%%%%%%%%%%%
\TD{importance of early stages in training neural networks.}
\TD{Time Matters in Regularizing Deep Networks: Weight Decay and Data  Augmentation Affect Early Learning Dynamic~\cite{DBLP:journals/corr/abs-1905-13277}}
\TD{The Early Phase of Neural Network Training~\cite{DBLP:journals/corr/abs-2002-10365}}
%%%%%%%%%%%%%%%%%%%%%%%%%%%%%%%%%%%%%%%%%%%%


\part{Appendix}

\chapter{Improving Generalizability}

\r{The methods shown in the upcoming sections aim to reduce overfitting. That is, these methods aim to prevent the model from becoming too specialized to the training dataset in hopes that it will generalize to data that it has not specifically seen during training (e.g from the ``test'' set).}

\r{By implementing some of these methods (e.g. reducing the model capacity), the model often has less ability to model the training set as well as it might otherwise be able to. This is ok, high performance on the test set is the ultimate goal.}

%  some of the methods aren't used before they are necessary \TD{section on determining overfitting}

% TODO: index overfitting
\r{overfitting: a practical definition may include observing the training loss to improve while the validation loss degrades. \TD{possibly mention \\cite{Nakkiran2020DeepDD}}}

\r{Overfitting --- too complex --- Occam's razor --- hypothesis with the fewest assumptions is best}

\r{A specific instance of improving generalization might be accounting for imblance. Either in the labels or in the features.  Section \ref{app_data_imbalance} discusses this topic and strategies in more detail.}

\r{Typicaly types of modifications that are made to improve generalization.}

\begin{itemize}[noitemsep,topsep=0pt]
	\item Data
	\begin{itemize}[noitemsep,topsep=0pt]
		\item Increase ammount of data
		\item Augmentation
		\item Sampling
	\end{itemize}
	\item Architecture --- Reduce complexity of model e.g. applying parameter constraints, and/or reduce overall number of parameters
	\begin{itemize}[noitemsep,topsep=0pt]
		\item Reduce complexity/number of parameters
		\item Ensembling
		\item Constraints
		\begin{itemize}[noitemsep,topsep=0pt]
			\item Directly on parameters
			\item Through additional losses/tasks
		\end{itemize}
	\end{itemize}
	\item Training Pattern
	\begin{itemize}[noitemsep,topsep=0pt]
		\item Early stopping
		\item Stochastic Behavior
	\end{itemize}
\end{itemize}


\section{Data}

\subsection{Data Collection}

\r{Arguably the best way to increase generalizability of a model is to train the model on more data. However, as readers may already be aware, this is not always easy. Collecting more data may not be time/cost effective, or even possible.}

\r{``free'' data in that the ``cost'' is minor computation}

\subsubsection{Data Labeling}

%TODO: later sections likely belong in an appendix

\r{Labeling unlabed data}

\begin{itemize}[noitemsep,topsep=0pt]
	\item semi-supervised
	\item active learning
	\item weak supervision
\end{itemize}

\paragraph{Semi-supervised}

\TD{label propagation}

\TD{Book~\cite{chapelle2010semi}}

\TD{using GANs: Improved Techniques for Training GANs~\cite{DBLP:journals/corr/SalimansGZCRC16}}

\TD{Temporal Ensembling for Semi-Supervised Learning~\cite{DBLP:journals/corr/LaineA16}}

\paragraph{Active Learning}

\TD{A Survey of Deep Active Learning~\cite{DBLP:journals/corr/abs-2009-00236}}

\TD{intelligently sample data. Select instances that would be most informative for training}

\TD{Intelligent sampling could use a few different methods}

\TD{life cycle could include: taking unlabeled data, using the active learning sampler to pick instances, using a human annotator for these points, then using this new labeled set for or in addition to the current training set for training}

\begin{itemize}[noitemsep,topsep=0pt]
	\item Margin Sampling
	\item Cluster Based Sampling
	\item Query-by-committee
	\item Region-based Sampling
\end{itemize}

\subparagraph{Margin Sampling}

\r{Select instances that are nearest to the decision boundary (margin) e.g. the most uncertain and train on these points}

\subparagraph{Cluster Based Sampling}

\r{sample from the well formed clusters}

\subparagraph{Query-by-Committee}

\r{train and ensemble of models and sample from the data points that the models disagree on.}

\subparagraph{Region-based Sampling}

\r{Run several algorithms (from above) on different portions of the space}


\paragraph{Weak Supervision}

\TD{Weak supervision: https://ai.stanford.edu/blog/weak-supervision/}

\TD{Snorkel: Rapid Training Data Creation with Weak Supervision~\cite{DBLP:journals/corr/abs-1711-10160}}


\subsection{Augmentation}

\r{Dataset augmentation is \textcolor{green}{TODO}}

\r{adds examples that are similar to real}

\TD{Usupervised data augmentation: UDA}

\r{Please note, augmentation must be done responsibly. For example, if performing digit recognition, it would not be wise to perform rotational or flip transformations on the data since, depending on the specific data, a 6, rotated 180 or flipped vertically may now appear as a 9.}


\r{invariances in the data}

\r{For specific techniques, see~\ref{app_aug_techniques}}

\TD{Beyond improving generalization, augmentation may be used in other contexts as well, such as in helping quantify uncertainty -- \TD{see ref ---\TD{Augmenting the test set. A simple augmentation (horizontal filliping) was performed on the test set in \cite{simonyan2014very} -- where the prediction of the original and augmented images are averaged to obtain the final output score.} }}


\subsection{Sampling}

\r{The line between the techniques described here and ``augmentation'' might be a little blurred, in that sampling might technically be considered a augmentation technique (and I'm not even sure ``sampling'' is the appropriate title). But the intended distinction is that in augmentation, we are diliberately altering something (e.g. the input data) and in sampling, we are altering the number of times an architecture sees a particular instance in a training dataset.}

\TD{see appendix section for methods}



\section{Architecture}

\section{Training Pattern}

\subsection{Early Stopping}

\r{see p.243 of DL, papers Bishop 1995 and Sjoberg and Ljung 1995}

% TODO: note about regularization --- the smaller the value, the stronger the regularization.


\subsection{Stochastic Behavior}

\subsubsection{Dropout}

\r{``Dropout'' as a node in a computational graph may be considered an architectural structure change, but the method itself affects the training pattern in possibly not obvious ways. }

% TODO: explain dropout

\r{Dropout -- ref original paper (Hinton? -- intuitive, inspired by bank -- that defrauding the bank would require cooperation between employees to defraud the bank \TD{cite})}.

\r{Dropout (proposed in ``Improving Neural Networks by Preventing Co-Adaption of Feature Dectors''~\cite{DBLP:journals/corr/abs-1207-0580}, and popularized by Nitish et.al in ``Dropout: a Simple Way to Prevent Nerual Networks from Overfitting''~\cite{JMLR:v15:srivastava14a}}

\r{It is important to note that dropout is only present during training. i.e. dropout does not occur during test/evaluation if using dropout in the ``standard way''. However dropout is occassionally used for evaluation in attempt to quantify model uncertainty \TD{CITATION}}

\r{keeps a neuron active by a hyperparameterized probability.}

\r{used in any/all neurons in the network (other than the output neruons).}

\r{think about where dropout is used. That is when you use dropout at any given nueron the upstream paths transversing that particular neuron are also affected (in this case, ``turned off''), as well downstream connections (but often only modified, not entirely turned off since they often still have other inputs) }

\r{Forces the network to learn mappings even in the absence of all the information, that is the network is forced to consider the values of other values and can't rely on a smaller number of values or groups of values. Said another way, the network is prevented from becoming too dependent on certain inputs or features.}

\r{In this way, dropout can be thought of as sort of an ensembling method. When dropout is in use during training, each loop technically produces a different network that is then trained for the given task. During the next loop, a different network is used. As Aurélien Géron~\cite{geron2019hands} describes, if you train for 10,000 training steps (where dropout is used), you will have likely (almost certainly) trained 10,000 different neural networks. It's true that each network is not indpendant (they share weights), but they are different. More generally, a network with $N$ activations with dropout present, there exist $2^N$ possible networks ($2$ since each activation/neuron/value can have either an `on` or `off` state.) and thus, the use of all of these networks at once can be considered an ensembling of sorts.}

\TD{create figure of this ensemble of many networks.}


% TODO: find recent paper I saw mentioned on twitter.... (4July) it may be in my pocket

\begin{figure}[htp]
	\centering
	\includegraphics[width=0.3\textwidth]{example-image-a}\hfil
	\includegraphics[width=0.3\textwidth]{example-image-b}\hfil
	\includegraphics[width=0.3\textwidth]{example-image-c}\hfil
	\caption{\TD{Graph of an example function including dropout. three separate training iterations and how the network changes}}
	\label{fig:regularization_dropout_overview_training}
\end{figure}

\begin{figure}[htp]
	\centering
	\includegraphics[width=0.3\textwidth]{example-image-a}\hfil
	\caption{\TD{Same graph during test --- no dropout applied}}
	\label{fig:regularization_dropout_overview_test}
\end{figure}

\r{It is worth pointing out that since dropout is only applied at training time, comparing the loss curve of training and inference (validation splits) will be a bit misleading since the full ensemble network is used for calculating the validation loss/metrics and only the component \TD{is there a better word than this?} networks are used for the training set.}

\r{Additionally, if you run the training set through multiple times, you may find slightly different results. Again, this is because while dropout is on, you'll find that a slightly different network is used. \TD{This idea can be exploited at inference time to get uncertainty estimates.}}

\r{some important notes about the implementation. The outputs at test time should be equivalent to their expected outputs at training time (which is altered due to the application of dropout).}

\r{Couple solutions}
\begin{itemize}[noitemsep,topsep=0pt]
	\item scale the outputs during inference
	\item
\end{itemize}

\r{One potential solution to this problem is to scale the outputs during inference in a way that compensates for the dropout probability.  For example, if the dropout rate was set to $0.5$, then it would become necessary to halve the neurons outputs at test time in order to keep the expected output the neurons have learned during training.  However, this may not be ideal in practice since it would require scaling all the neuron outputs at test time (where performance is often critical and more important).}

\r{at test time, multiply the values by the expectation, not the on/off mask}

\r{Another, perhaps more desirable solution, would be to use \IDI{inverted dropout}. The cs231n~\cite{cs231n} course provides a concise explaination and example code on this topic.}

\r{This applies the same principal as outlined above, only the scaling occurs at training time rather that at test time. That is, during training, any neuron whose activation was not turned off, has the output divided by the dropout rate before being propagation to the next layer.  This way, at test time, no scaling is required.}

% helps learn ``multiple paths''/simulates ensembles
\TD{link to ensemble section}

\subsubsection{Others}

\TD{``during training, for each mini-batch, randomly drop a subset of layers and bypass them with the identity function'' --- Deep Networks with Stochastic Depth \cite{DBLP:journals/corr/HuangSLSW16}}

\TD{DropConnect~\cite{wan2013regularization} is similar to dropout, except that individual weights are disabled, not entire individual nodes and can be considered a generalization of dropout.}

\TD{figure showing difference}

% `drop block''?
\TD{investigate more structured dropout.}


\TD{structured --- ``contiguous region of a feature map are dropped together'' DropBlock  \cite{DBLP:journals/corr/abs-1810-12890}}


\TD{alpha dropout\cite{DBLP:journals/corr/KlambauerUMH17}}



\subsection{Parameter Regularization}

\r{Collection of techniques used to help generalize a model -- which may help prevent overfitting. Typically regularization penalizes complexity of a model.}


% TODO: figure of loss plot showing a steep training and shallow+divergent val/test loss

\r{imposes a penalty on the parameters}

\r{Helps prevent the model from memorizing noise in the training data.}

\r{Discourages the learned mapping/function/model from becoming too complex}


\subsubsection{Types of Regularization}

\textcolor{blue}{Regularization is an active area of research.}

% more information on L1/L2 http://www.chioka.in/differences-between-l1-and-l2-as-loss-function-and-regularization/

\begin{itemize}[noitemsep,topsep=0pt]
	\item Early Stopping (implementation: \textcolor{red}{local ref})
	\item Parameter Norm Penalties (implementation: \textcolor{red}{local ref})
	\begin{itemize}[noitemsep,topsep=0pt]
		\item L1 (Lasso) Regularization
		\item L2 (Ridge) Regularization
		\item Elastic Nets
	\end{itemize}
	\item Dataset Augmentation (implementation: \textcolor{red}{local ref})
	\item Noise Robustness
	\item Sparse Representations
	\item Dropout (implementation: \textcolor{red}{local ref})
	\item Ensemble methods (implementation: \textcolor{red}{local ref})
	\item Adversarial Training
\end{itemize}



\subsubsection{Parameter Norm Penalties}

\r{key difference is the penalty term}

\TD{TODO: DIGRAM OF L2 + L1 + elastic nets}

\paragraph{L2 Regularization}

\TD{TODO: DIAGRAM OF L2}

\r{L2, ({Ridge regression}\index{Ridge regression}) may also be known as {Tikhonov regularization}\index{Tikhonov regularization}}

\r{penalizes model parameters that become too large. Will force most of the parameters to be small, but still non-zero}

\r{square of the absolute value of the coefficient}

\begin{figure}[htp]
	\centering
	\includegraphics[width=0.3\textwidth]{example-image-a}\hfil
	\includegraphics[width=0.3\textwidth]{example-image-b}\hfil
	\includegraphics[width=0.3\textwidth]{example-image-c}\hfil\\
	\medskip
	\includegraphics[width=0.3\textwidth]{example-image-a}\hfil
	\includegraphics[width=0.3\textwidth]{example-image-b}\hfil
	\includegraphics[width=0.3\textwidth]{example-image-c}\hfil
	\caption{\TD{Top: NN output decision boundary on 2D dataset Bottom: weight params distribution from tensorboard... from LtoR = same arch with varying degrees of L2 regularization (0.01, 0.1 and 1.0)}}
	\label{fig:basics_regularization_l2_example}
\end{figure}


% p91(71) of mastering ML w SKL says "when lambda is equal to zero, ridge regression is equal to linear regression"

\paragraph{L1 Regularization}

\TD{TODO: DIAGRAM OF L1}

\r{LASSO (\textbf{L}east \textbf{A}bsolute \textbf{S}hrinkage and \textbf{S}election \textbf{O}perator) --- produces sparse parameters. This will force coefficients to zero and cause the model to depend on a small subset of the features.}

\r{absolute value of the weight coefficient}

\r{use only a small subset of the input features and can become resistant to noisy inputs.}

\r{It could be argued that using L1 regularization may help to make a model more interpretable, by using less (presumably more important/relevant) features when making predictions.}

\r{The use of L1 regularization for feature selection}


\paragraph{Elastic Net Regularization}

\r{Linearly combines the $L^1$ (feature selection) and $L^2$ (generalizability) penalties used by both LASSO and ridge regression. The cost is having two parameters (as opposed to just one when using either L1 or L2).}

\TD{TODO: figure}.



\subsection{Ensemble Methods}

\r{see \textcolor{red}{local ref} for more information on ensemble basics and see \textcolor{red}{local ref} for implementation details.}

% TODO: find Breiman 1994 paper referenced in p249 of Deep Learning
\r{As described in \textcolor{red}{local ref} ensemble methods act as a form of regularization by combining several different models \TD{Breiman 1994}. This often improves generalizability since the included models will often make independent, different, errors on the data.}

\subsection{Adversarial Training}



\subsection{Transfer Learning}

%TODO: read this survey
\TD{A Survey on Deep Transfer Learning \cite{DBLP:journals/corr/abs-1808-01974}}

\TD{How transferable are features in deep neural networks? \cite{DBLP:journals/corr/YosinskiCBL14}}
\TD{CNN Features off-the-shelf: an Astounding Baseline for Recognition \cite{DBLP:journals/corr/RazavianASC14}}

% TODO: haven't read this one (I don't think), but looks relevant
\TD{Learning and transferring mid-level image representations using convolutional neural networks\cite{oquab2014learning}}
\TD{Pay attention to features, transfer learn faster CNNs\cite{wang2019pay}}

% TODO: is this talked about anywhere else? this is probably the best place for it.

\TD{TODO: transfer learning, using -- explanation}

\TD{tool that may sometimes be efficient way of getting to potentially more accurate approximations, faster. \TD{citations}}

% TODO: index
\r{using parameters or pre-trained components from a model/task for a new model/task.  In practice, this often amounts to running inputs through a network that has been previously trained, and obtaining ``embeddings'' from this model (sometimes at an abitrary layer in the network), and then using these ``embeddings'' as input to train an additional model on the desired task. The process of adapting these components to a new model/task is called fine-tuning}


\begin{figure}[htp]
	\centering
	\includegraphics[width=0.5\textwidth]{example-image-a}\hfil
	\caption{Figure example layer hierarchy and where/when to transfer/freeze params -- this will be 1-2 figures and include many sub-figures \textcolor{green}{TODO}}
	\label{fig:transfer_learning_subfigs_a}
\end{figure}

\textcolor{green}{{freezing}\index{freezing} parameters or a layer means preventing the parameters from being updated during training. This is often controlled by a parameter called ``trainable''.}

% In relation to transfer learning and freezing, mention the difficulty of propagating updates though a large network

\TD{Scaling Laws for Transfer \cite{DBLP:journals/corr/abs-2102-01293}}

\r{One difficulty of fine tuning is knowing where and by how much to either freeze or learn. That is should you freeze the first $n\%$ of the network, why not $m\%$?. Maybe you should leave the entire network trainable? But if the entire network is trianable, the previously learned (and presumably useful features), may be erased by the updates. Aside from selecting where to make the distinction, the main method used to combat these issues is to modify the learning rate. There are two core methods to adjusting the learning rate to address these issues.}

\begin{itemize}[noitemsep,topsep=0pt]
	\item Learning rate schedule
	\item Layer-wise learning rates
\end{itemize}

\TD{These methods are described in more detail in section ~\ref{hp_learning_rate}}


\TD{Adversarially robust transfer learning \cite{DBLP:journals/corr/abs-1905-08232}}

% TODO: check this paper out
\TD{DT-LET: Deep Transfer Learning by Exploring where to Transfer \cite{Lin2020DTLETDT}}

\subsubsection{Potential downsides of TL}

\TD{biases, attacks}

\TD{A Target-Agnostic Attack on Deep Models: Exploiting Security Vulnerabilities of Transfer Learning \cite{DBLP:journals/corr/abs-1904-04334}}

% TODO: this likely does not belong here...
\subsection{Normalization}

% TODO: Read this
\TD{Evolving Normalization-Activation Layers \cite{DBLP:journals/corr/abs-2004-02967}}

\TD{TODO: overview para + importance}

\TD{TODO: figure showing differences}

\paragraph{Instance normalization}

\r{see section in preprocessing \textcolor{red}{local ref?}}

\paragraph{Layer normalization}

\TD{Layer Normalization \cite{Ba2016LayerN}}

\paragraph{Batch normalization}

% TODO: Read this
\TD{Training BatchNorm and Only BatchNorm: On the Expressive Power of Random Features in CNNs \cite{DBLP:journals/corr/abs-2003-00152}}

\TD{Show / explain}

\TD{Batch Normalization: Accelerating Deep Network Training by Reducing	Internal Covariate Shift \cite{DBLP:journals/corr/IoffeS15}}

\r{similar to dropout \ALR, the behavior of batch norm is different at training time and inference time.}

\r{normalizes values across a batch of data. Where the normalization is controlled by two learned parameters. The ``center'' and ``scale''.}

\r{Standard implementation is to calculate the population values using an exponential moving average (EMA).}

%TODO: here!

\TD{{Rethinking "Batch" in BatchNorm}~\cite{Wu2021RethinkingI} concludes that using EMA as the method for calculating the population statistics is not ideal. They show that during the early epochs, the xxxxxxx.}

\r{An adaptive re-parameterization.}

\r{reduce sensitivity to hyperparameterization.}

\TD{TODO: transfer learning considerations --- will likely have to unfreeze these params}

% HUGO talk
\r{``making the optimization easier''. batch norm is not effective in RNNs -- more so layer norm}

\r{seems to help when both under and over fitting.}

\r{order, up for debate and often described as either pre-activation operation, then activation, then batch norm, or pre-activation operation, then batch norm, then activation.}

\r{$\gamma$ and $\beta$ parameters that are learned parameters. These params could effectively undo the normalization caused (if ``learned'' to do so.)}


\begin{enumerate}[noitemsep,topsep=0pt]
	\item batch statistics
	\begin{itemize}[noitemsep,topsep=0pt]
		\item mean
		\item variance
	\end{itemize}
	\item normalize the pre-activation
	\item $\gamma$ and $\beta$ --- learned rescalling
\end{enumerate}

\TD{Rethinking "Batch" in BatchNorm \cite{DBLP:journals/corr/abs-2105-07576}}

% TODO: haven't read this paper yet 9Oct21 (I don't think...)
\TD{How Does Batch Normalization Help Optimization? \cite{Santurkar2018HowDB}}


% Graham Taylor talk
\begin{itemize}[noitemsep,topsep=0pt]
	\item turn down other regularization
	\item fixes first and second moments which may suppress information in these moments.
\end{itemize}

\TD{work related to adversarial spheres. --- with batch norm, the result was more reflective of the batch, not the entire dataset (which makes sense, right?)}


\paragraph{Group normalization}

\TD{Group Normalization \cite{DBLP:journals/corr/abs-1803-08494}}


\section{Output regularization}

\r{confidence penalty on predictions that are extrememly confident\cite{pereyra2017regularizing}. Originally an RL idea to promote expoloration. In SL, we would prefer fast convergence i) anneal confidence penalty ii) only penalize at a certain confidence threshold (lower entropy threshold). Intuitive (or not), can improve generalization.}

%TODO:
\r{label smoothing\cite{szegedy2016rethinking}}

\r{Adding label noise\cite{xie2016disturblabel}}

\r{smooth labels -- either via a ``teacher model''\cite{hinton2015distilling} or using it's own distribution\cite{reed2014training}}

\r{virtual adversarial training\cite{miyato2018virtual}}

\chapter{Statsy Stuff}

% TOOD: decide if/where this belongs, organization, titles, etc

\emph{Collinearity} --- When two or more predictor variables are closely related to one another (highly correlated to one another) they are said to be collinear.

\emph{Population vs Sample} -- the population (usually denoted $N$) is the collection of all the items of interest in a study where as the sample is a subset of a population (usually denoted $n$). The numbers obtained when working with a population are called the `parameters' and the numbers obtained when working with a sample are a called `statistics'. \r{a random sample is obtained when each member of the sample is chosen from the population by chance and accurately reflects the population}


\TD{Probability Density Function (PDF)}


\section{Experimental Design}

\r{Basic Principals}

\r{
	\begin{itemize}[noitemsep,topsep=0pt]
		\item Randomization
		\begin{itemize}[noitemsep,topsep=0pt]
			\item Every treatment has the same probability
		\end{itemize}
		\item Replication
		\begin{itemize}[noitemsep,topsep=0pt]
			\item Repeat the experiment ``enough'' times
		\end{itemize}
		\item Control
		\begin{itemize}[noitemsep,topsep=0pt]
			\item Reduce/eliminate variance from other factors
		\end{itemize}
	\end{itemize}
}

\section{``tests''}

\r{accept or reject null hypothesis}

\TD{degrees of freedom, \r{number of outcomes, minus 1?}}

\r{critical values \r{typically use the 0.05 column, i.e. $95$ sure you're accepting or rejecting}}

\subsection{T test}

% TODO: read https://medium.com/@wyess/demystifying-statistical-analysis-2-the-independent-t-test-expressed-in-linear-regression-434a3b302289

\TD{Gosset -- published under "student". student t test\cite{student1908probable}}

\r{a ratio of signal to noise --- is the differnce in means between two groups significant }

\r{use t-table -- one or two-tailed, with degrees of freedom, to obtain critical value}

\r{degrees of freedom}

\TD{Assumptions: randomly selected}
\r{\begin{itemize}[noitemsep,topsep=0pt]
		\item Measurement scale is coninuous/ordinal scale
		\item randomly selected (representative of population)
		\item normal distribution,
		\item similar variance
	\end{itemize}
}
% 		\item similar size of datapoints

\r{independent}

\r{dependent (paired) -- example same population multiple times}


\r{Correlated (or paired) T-Test}
\begin{equation}
	{\textrm{t-value} = \frac{ \textrm{diff in means}}{\textrm{variability between groups}} = \frac{ |\bar{X}_1 - \bar{X}_2|}{ 
			\sqrt{ \frac{ {std_1}^2 }{ n_1 } + \frac{ {std_2}^2 }{ n_2 }}}}
	\label{eq:paired_t_test}
\end{equation}
\r{source: https://magoosh.com/statistics/how-to-perform-an-independent-sample-t-test/}


\subsection{Chi-Squared Test}

\r{Carl Pearsons Chi-Squared Test. is the variation in your data due to chance?}


\section{Power Analysis}

\r{often used to determine the size of the experiment necessary to detect a given effect/treatment. May also be used determine the power of an experiment after it has been performed (given the effect/treatment size and the size of the experiement)}

\r{power --- the probability of detecting an effect/treatment, provided an effect/treatment is present}

\r{\textit{e.g.} a power value of $0.7$ would signify that $70\%$ of the time, the there is a significant difference between the effect/treatment group and the control group}

\r{power, effect size, sample size, and alpha}


\section{Analysis of Variance (ANOVA)}

\r{Splits and observed aggregate into systematic factors and random factors. Determine the influence that an independent variable has. Fisher analysis of variance\cite{fisher1992statistical}}

\r{if no true variance exists, the ANOVA's F-ratio should be 1 or near 1.}

\begin{equation}
	{\textrm{ANOVA coefficient} = \frac{ \textrm{MST}}{\textrm{MSE}} = \frac{  \textrm{Mean Sum of Squares due to Treatment}}{ 
			 \textrm{Mean Sum of Squres due to Error}}}
	\label{eq:anova}
\end{equation}


\section{Transforms}

\TD{Box-Cox transformation\cite{box1964analysis}: George Box and David Cox}

\r{lambda $\lambda$, which varies from $\neg5$ to $5$}

\r{lambda is selected on which value gives the best approximation of a normal distribution}


\r{Formula for positive data}
\begin{equation}
	\begin{cases} 
		\frac
		{y^\lambda - 1}
		{\lambda},  & \lambda \neq 0; \\
	    \log y, & \lambda = 0. \\
	\end{cases}
\end{equation}



\r{Formula that could be used for negative y-values}
\begin{equation}
	\begin{cases} 
		\frac
		{({{y + \lambda_2}})^{\lambda_1} - 1}
		{\lambda_1},  & {\lambda_1} \neq 0; \\
		\log ({y+{\lambda_2}}), & {\lambda_1} = 0. \\
	\end{cases}
\end{equation}



\section{Variance Reduction Methods}

\TD{An experiement who's treatment effect is small relative to the metrics variance is considered weaker/underpowered, when compared to an experiment who's variance is smaller -- (this explanation needs work)}

% \includegraphics[width=0.5\textwidth]{example-image-a}\hfil
\TD{add X and Y names (Effect and Variance)}
\begin{figure}
	\begin{subfigure}{6cm}
		\centering\includegraphics[width=5cm]{example-image-a}
		\caption{Low Effect, High Variance: Two wide mostly overlapping distributions}
	\end{subfigure}
	\begin{subfigure}{6cm}
		\centering\includegraphics[width=5cm]{example-image-b}
		\caption{High Effect, High Variance: Two wide mostly non-overlapping overlapping distributions}
	\end{subfigure}
	
	\begin{subfigure}{6cm}
		\centering\includegraphics[width=5cm]{example-image-c}
		\caption{Low Effect, Low Variance: Two narrow mostly overlapping overlapping distributions}
	\end{subfigure}
	\begin{subfigure}{6cm}
		\centering\includegraphics[width=5cm]{example-image-c}
		\caption{High Effect, Low Variance: Two narrow mostly non-overlapping overlapping distributions}
	\end{subfigure}
\end{figure}

\r{CUPED methods (\textbf{C}ontrolled \textbf{U}sing \textbf{P}re-\textbf{E}xperiment \textbf{D}ata)\cite{deng2013improving}}

\r{Out of DoorDash --- CUPAC methods (\textbf{C}ontrol \textbf{U}sing \textbf{P}redictions as \textbf{C}ovariates)\cite{tangcontrol}}

\r{reduction of pre-experiment variance to help strengthen experiments power}


\section{Distributions}

\r{http://www.math.wm.edu/~leemis/chart/UDR/UDR.html}

\TD{https://medium.com/@srowen/common-probability-distributions-347e6b945ce4}

\subsection{Common Distributions}

\r{There are many, many different types of distributions. However, there are a number of distributions that are commonly used}

\subsubsection{Normal Distribution / Gaussian Distribution}

\emph{Normal Distribution} --- \TD{Normal, or Guassian, distribution. Data is symmetrical where half the values are greater than the mean and half the values are less than the mean. The median, mode, and mean are all equal}

\subsubsection{Uniform Distribution}

\TD{Uniform}

\r{For example, rolling a single dice}

\subsubsection{Bernoulli Distribution}

\TD{Bernoulli}

\subsubsection{Poisson Distribution}

\TD{Bernoulli}

%\include{appendix/experiments}

%%%%%%%%%%%%%

% TODO: I'm still not sure how/where to structure this

\section{Dense}

\TD{TODO}

\section{Convolutions}

% this reads strangely --> DNN on an image may not take advantage of the ``stationarity'' (statistics) of an image.

\r{When using a standard dense layer, all inputs are treated independently. However, adjacent pixels, on average, are highly highly correlated. For example, if there is a texture in the image, a similar pattern of pixels may occur repeatedly. Convolutions architecturally build in an implicit spatial structure to consider these spatial.}

% TODO: I'm not sure how I'm going to structure these yet or where I'll be placing them

% TODO: https://arxiv.org/abs/1904.11486
% https://www.youtube.com/watch?v=HjewNBZz00w


\TD{LeNet-5 \cite{lecun1998gradient}}

\r{Convolutions are built upon a lie -- that is we refer to the opperation as a convolution, yet it is in fact a cross-correlation operation since we don't rotate the kernel 180$\deg$. However, it is convention to refer to the operation as a convolution. For more, please see section \ref{conv_vs_cross}}

\r{translational invariance --- a property that relates to how a systems decisions are insensitive to the location of a features within an input. That is, if we're looking for an object or feature, our system shouldn't change if the object is in different locations within the input}

\TD{``Filter factorization'' (not the exact same definition of mathematical factorization)-- one $5\times5$ filter vs $2$ $3\times3$ filters stacked.  in the $5\times5$ there are $5\times5 = 25$ parameters, in the $3\times3$, there are $3\times3 \times 2 = 18$ learnable parameters, resulting in a ``cheaper'' operation.}

\TD{Neocognitron -- CNN paper prior to ``CNN''\cite{fukushima1982neocognitron}}

% Survey on CNNs
% TODO: a lot here -- good read
\TD{A Survey of the Recent Architectures of Deep Convolutional Neural Networks \cite{DBLP:journals/corr/abs-1901-06032}}


\TD{Squeeze-and-Excitation Networks \cite{DBLP:journals/corr/abs-1709-01507}}


% Graham Taylor
\r{weighted averaging operation in time or space}


\r{translation equivariant --- }

\TD{BlurPool --- ``fix is anti-aliasing by low-pass filtering before downsampling'' ---Making Convolutional Networks Shift-Invariant Again \cite{DBLP:journals/corr/abs-1904-11486}}


\r{spatial hierarchies --- \TD{TODO: figure raw data, abstract edges+, then more distinct images, then closer output to the output, then the final label}}


\r{typcially a feature extraction phase (consisting of convolutional and pooling layers) followed by a classifier block (dense layers).}

%%%% popular layer types
\textcolor{green}{TODO: feature maps, (height, width, and depth (also called channels axis)). Stride, filter size, depth. talk about parameters}

\r{The output feature map (every dimension in the depth axis is a feature/filter) --- after a convolution operation the depth of a layer is no longer representative of a color channel (like RGB), it is now representative of a feature extracted by the convolutional operation, these are called filters.}

\TD{Strided Convolution\cite{springenberg2014striving}}

\TD{Dilated Convolution --- `atrous' convolution. (famously used by wavenet), which is convenient in time series analysis.}

\r{weight tieing}


\textcolor{green}{TODO: figure}

\begin{figure}[htp]
	\centering
	\includegraphics[width=0.5\textwidth]{example-image-a}\hfil
	\caption{Figure example of convolution operation on 2d image \textcolor{green}{TODO}}
	\label{fig:conv_2d_example_calc}
\end{figure}

\begin{figure}[htp]
	\centering
	\includegraphics[width=0.5\textwidth]{example-image-b}\hfil
	\caption{Figure example of convolution operation on 3d image \textcolor{green}{TODO}}
	\label{fig:conv_2d_depth_example_calc}
\end{figure}

\textcolor{green}{TODO: examples of how different filter values and strides can effect the output dimensions.}




\section{Pooling}

\TD{TODO: examples of max vs average pooling}

%%%%%% research
\textcolor{blue}{Pooling may not fully determine learned deformation stability -- possibly filter smoothness\cite{ruderman2018learned}}

\r{downsampling}

\r{Why? importance of reducing the number of params.}

\TD{L2-pooling}

\TD{L2-pooling over the features or channels.}

\TD{additional --- learned/parameterized pooling}

\begin{figure}[htp]
	\centering
	\includegraphics[width=0.5\textwidth]{example-image-a}\hfil
	\caption{Figure example of max pooling operation on 2d image \textcolor{green}{TODO: I want this figure to be basic 2d}}
	\label{fig:pooling_max_2d_ex_a}
\end{figure}

\begin{figure}[htp]
	\centering
	\includegraphics[width=0.5\textwidth]{example-image-b}\hfil
	\caption{Figure example of average pooling operation on 3d image \textcolor{green}{TODO: I want this figure to be 3d}}
	\label{fig:pooling_avg_3d_ex_a}
\end{figure}


\r{may be better to use convolutional layers in place of the pooling layers\cite{springenberg2014striving}}

\section{Recurrent Cells}

% TODO: read this
% Recurrent / Echo state networks / ESN
\TD{The ``echo state'' approach to analysing and training recurrent neural networks-with an erratum note \cite{jaeger2001echo}}
\TD{Deep Echo State Network (DeepESN): A \cite{DBLP:journals/corr/abs-1712-04323}}

\subsection{Cell Advancements}

\subsubsection{LSTM}

% TODO: Nice overview of LSTMs: https://colah.github.io/posts/2015-08-Understanding-LSTMs/

Introduced in 1997 %\cite{hochreiter1997long}

\r{detect long term dependencies in sequence}

\r{two state vectors, short and long term}

\r{Main motivation: learning what to store in the long-term state and what to ``forget''.}

\r{at each time step, some information is ``stored'' and some information is ``forgotten''.}

\paragraph{variants}

\TD{Depth-Gated LSTM \cite{DBLP:journals/corr/YaoCVDD15}}

\TD{A Clockwork RNN \cite{DBLP:journals/corr/KoutnikGGS14}}

\TD{LSTM: A Search Space Odyssey \cite{DBLP:journals/corr/GreffSKSS15} --- survey of LSTM variants --- all variants are essentially equal.}


\paragraph{other directions}

% interesting paper on ``grid LSTMs'' -- not sure why they never become popular
\TD{Grid Long Short-Term Memory \cite{Kalchbrenner2016GridLS}}

\paragraph{Fully Connected Layers}


\begin{enumerate}[noitemsep,topsep=0pt]
	\item Main
	\item \textit{Gate Controllers}
	\begin{enumerate}[noitemsep,topsep=0pt]
		\item Forget
		\item Input
		\item Output
	\end{enumerate}
\end{enumerate}

\r{The gate controllers use a logistic activation fuction (output a range from 0 to 1). This output is then fed through an element-wise multiplication function and thus if the value is $0$, the gate is ``closed'', and $1$ if the gate is ``open''.}

\r{These gates are able to potentially:}

\begin{enumerate}[noitemsep,topsep=0pt]
	\item Recognize an important input
	\item Store the important input in a long-term state ()
	\item Preserve the information for as long as it's needed
	\item Extract the important information when needed
\end{enumerate}


\subparagraph{Main}

\begin{figure}
	\centering
	\includegraphics[width=0.5\textwidth]{example-image-a}\hfil
	\caption{\TD{Main Layer DIAGRAM}}
	%\label{}
\end{figure}

\r{This allows for the same basic functionality as a ``standard'' RNN cell --- however, the output, rather than being only sent to the next cell, is now partially stored in the long-term state.}


\subparagraph{Forget}

\r{Determines which part of the long-term state is forgotten/erased.}

\begin{figure}
	\centering
	\includegraphics[width=0.5\textwidth]{example-image-a}\hfil
	\caption{\TD{Forget Layer DIAGRAM}}
	%\label{}
\end{figure}



\subparagraph{Input}

\r{Determines which part of the output from the \textbf{main layer} are kept in the long-term state.}

\begin{figure}
	\centering
	\includegraphics[width=0.5\textwidth]{example-image-a}\hfil
	\caption{\TD{Input Layer DIAGRAM}}
	%\label{}
\end{figure}

\subparagraph{Output}

\r{Determines which part of the long term state is ``relevant'' (read and output).}

\begin{figure}
	\centering
	\includegraphics[width=0.5\textwidth]{example-image-a}\hfil
	\caption{\TD{Output Layer DIAGRAM}}
	%\label{}
\end{figure}


\paragraph{Other}

\subparagraph{Peephole Connections}

\r{In basic LSTM cells, the gate controller can only look at the input and previous short-term state. Peephole connections, proposed in 2000 \TD{cite gers2000recurrent} add an extra connection that allows for the gate controller to also see information from the long term state as well. }

\r{The previous long-term state also becomes an input to the forget and input gate. The current long-term state becomes an intput to the output gate.}



\subsubsection{GRU}

\r{The GRU (gated recurrent unit) is a varient of the LSTM cell \TD{cite - cho2014learning}. The main modifications include:}

\begin{itemize}[noitemsep,topsep=0pt]
	\item Both state vectors are merged into one state vector
	\item A single gate controller determines the \textbf{Forget} and \textbf{Input} gate
	\begin{itemize}[noitemsep,topsep=0pt]
		\item If the gate output is a 1, the input is open and the forget gate is closed. If the gate output is 0, the input gate is closed and the forget gate is open
	\end{itemize}
	\item \r{The output gate is removed and a new controller exists that controls which part of ht previous state will be ``shown'' to the main layer}. At each timestep the full state vector is output.
\end{itemize}

\subsection{Notes -- add}

\r{A recent paper \TD{greff2017lstm}, compares three LSTM variants and makes three main observations:}

\begin{itemize}[noitemsep,topsep=0pt]
	\item no significant architecture improvements over LSTMs
	\item forget gate and the output activation function are the most critical components
	\item \TD{hyperparams...}
\end{itemize}




\section{Capsule Networks}

% TODO: capsule networks
\TD{Dynamic Routing Between Capsules \cite{DBLP:journals/corr/abs-1710-09829}}

\section{Attention}

\r{``An attention function can be described as mapping a query and a set of key-value pairs to an output,
	where the query, keys, values, and output are all vectors. The output is computed as a weighted sum
	of the values, where the weight assigned to each value is computed by a compatibility function of the
	query with the corresponding key.'' \cite{DBLP:journals/corr/VaswaniSPUJGKP17}}

\TD{Self-attention Does Not Need $O(n^{2})$ Memory~\cite{Rabe2021SelfattentionDN}}

%TODO: another blog to checkout https://distill.pub/2016/augmented-rnns/

\r{overview can be found here\cite{weng2018attention}}


\TD{The original attention mechanism is introduced\cite{Bahdanau2015NeuralMT}.}

% TF attention implementation (https://www.tensorflow.org/tutorials/text/nmt_with_attention)

\TD{Effective Approaches to Attention-based Neural Machine Translation \cite{DBLP:journals/corr/LuongPM15}}

\TD{Massive Exploration of Neural Machine Translation Architectures \cite{DBLP:journals/corr/BritzGLL17}}

% TODO: index for transformer
% 'self-attention'
\TD{Attention Is All You Need -- Transformer network --- multi-head self-attention mechanism, key-value pairs \cite{DBLP:journals/corr/VaswaniSPUJGKP17}}

% self-attention \TD{Self-attention, less commonly intra-attention}
\TD{Long Short-Term Memory-Networks for Machine Reading \cite{DBLP:journals/corr/ChengDL16}}


%\TD{Nice table comparing mechanisms https://lilianweng.github.io/lil-log/2018/06/24/attention-attention.html}

\TD{in above post\cite{weng2018attention}: soft vs hard attention and global vs local attention}

% ``heads learn redundant key/query projections'' --> share
% https://github.com/epfml/collaborative-attention
\TD{Multi-Head Attention: Collaborate Instead of Concatenate \cite{Cordonnier2020MultiHeadAC}}

% soft vs hard and global vs local

\TD{Describes two variants: a ``hard'' stochastic attention mechanism (trainable via ``maximizing an approximate variational lower bound'' or REINFORCE) and a ``soft'' deterministic attention mechanism(trainable by standard back-propagation) \cite{DBLP:journals/corr/XuBKCCSZB15}. Soft attention --- scores to all entities (is differenetiable but expensive) and hard attention --- only selects one entity (non-differentiable (and complicated, reinforcement learning), but requires less computation at inference)}


% TODO: does this make sense?
\TD{Non-linear projection for K,Q, and V~\cite{DBLP:journals/corr/abs-2111-10017}}


\subsubsection{Scoring Functions}

% TODO: https://lilianweng.github.io/lil-log/2018/06/24/attention-attention.html#summary
\TD{table from \cite{weng2018attention}}


\subsection{Self-Attention}

\r{sometimes refered to as ``intra-attention''\cite{DBLP:journals/corr/VaswaniSPUJGKP17}. Keys, queries and values are all derived from the same sequence. \TD{Self-attention transforms a sequence to create a representation of itself.}}



\subsection{transformers}

% possibly useful: http://nlp.seas.harvard.edu/2018/04/03/attention.html

\TD{survey of recent transformer architectures \TD{Efficient Transformers: A Survey \cite{Tay2020EfficientTA}}}


% Factorized Attention to self-attention
\TD{Generating Long Sequences with Sparse Transformers \cite{DBLP:journals/corr/abs-1904-10509}}

% include reccurence:  "enables learning dependency beyond a fixed length" + "relative position encodings"
\TD{Transformer-XL: Attentive Language Models Beyond a Fixed-Length Context \cite{DBLP:journals/corr/abs-1901-02860}}

% extends DBLP:journals/corr/abs-1901-02860 -- 
% https://github.com/guolinke/TUPE
\TD{Compressive Transformers for Long-Range Sequence Modeling \cite{Rae2020CompressiveTF}}

% linear attention
\TD{Transformers are RNNs: Fast Autoregressive Transformers with Linear Attention \cite{Katharopoulos2020TransformersAR}}

% 
\TD{Transformer with Untied Positional Encoding (TUPE) --- Rethinking Positional Encoding in Language Pre-training \cite{Ke2020RethinkingPE}}



\TD{Reformer: The Efficient Transformer \cite{Kitaev2020ReformerTE}}


% TODO: top-down attention
% related to self-attention
% https://twitter.com/thomaskipf/status/1277570203665170432
\TD{Object-Centric Learning with Slot Attention \cite{Locatello2020ObjectCentricLW}}

\TD{Recurrent Independent Mechanisms \cite{Goyal2019RecurrentIM}}

% DETR -- also object detection
\TD{End-to-End Object Detection with Transformers \cite{Carion2020EndtoEndOD}}


% TODO: read https://lilianweng.github.io/lil-log/2020/04/07/the-transformer-family.html


\subsection{Positional Encodings}

\TD{Positional embedding and positional encoding tend to be used interchangably. However, typically an encoding means ``fixed'' while an embedding means ``learned'' or ``trainable''.}

% TODO: example of how word order matters (not is a good example)

\r{Attention/transformers view the inputs as sets, that is there is no order associated with each input. All information enters the attention block at once. This is in contrast to something like a recurrent model, in which the order of the inputs is implicit.}

\r{trade off: potentially faster (remove the dependancy of doing operations sequentially) and can also possibly help capture longer range dependancies (without additional complexity e.g. skip connections)}

\r{(re)introducing order to the input by including additional information -- the ``positional embedding''.}

\r{NOTE: Great blog posts on this subject~\cite{kazemnejad_2021, kernes_2021, kernes_2021B}}

\subsubsection{Positional Encoding Value}

\r{why not add linear/progressive value signifying order?}

\r{This would be called an aboslute positional embedding}

\r{Include index information [0, n], where n is the length of the sentance (minus 1). This could lead to magnitude issues. Where the singal from the word embeddings is ``washed out'' by the positional embedding.  Another consideration is that (may or may not be an issue depending on the application) is that you'd like to ensure you have the largest sequence in the training set that you expect to see in evaluation set. For example, if you only see sequences of length $25$ in the training data and then see a sequence of length $32$ during inference. The model will be unsure what to do with values $25 - 31$ (zero indexing). Depending on how you include the positional embedding (e.g. additive or concat), the model may misinterpret the values or be largely/entirely unsure what to make of these previously unseen values.}
	
	
\r{To address this you could either increase the magnitude of the word embeddings or normalize/scale the positional embedding.}

\r{However, niether are ideal.}

\r{Increasing the magnitude of the word embeddings would possibly work, though you may consider issues with exploding values in the network, but you'd still have a similar issue to what would happen if you normalized the positional embedding. }

\r{That is, the normalized positional embeddings may encode different information when the sentances are longer or shorter -- the delta between words in a 5 word sentance vs a 20 word sentance doesn't have a consistent meaning}

% NOTE: haven't read this yet (I don't think, though the link is purple...)
\TD{Self-Attention with Relative Position Representations~\cite{DBLP:journals/corr/abs-1803-02155}}

\r{Ideally the embedding would be able to account for all the issues we discussed.}

\begin{itemize}[noitemsep,topsep=0pt]
	\item consistent delta between each position
		\begin{itemize}[noitemsep,topsep=0pt]
			\item regardless of sequence length, if an instance is one instance away from another, the positional encoding should be the same e.g. in a length four sequence the positional encoding should be the same from instances $1$ and $2$ as it is for instances $19$ and $20$ in a length $22$ sequence.
		\end{itemize}
	\item generalize to sequence lengths unseen in training
\end{itemize}

\r{additionally, we'd prefer to have each instance in the sequence be unique. That is the positional encoding for one instance shouldn't be the same as another in the same sequence (e.g. two words in a sentance).}

\paragraph{Positional Encoding Value(s)}

\r{Rather than use a single value, a possible solution is to use an array of values.}

\TD{Relative positional encoding (rather than absolute).}


\TD{What if we were to use a binary array to represent each location?}

\TD{issue with binary}

\TD{}


\TD{CAPE: Encoding Relative Positions with Continuous Augmented Positional Embeddings~\cite{DBLP:journals/corr/abs-2106-03143}}

% NOTE: possibly relevant: https://aclanthology.org/2021.emnlp-main.266.pdf

\paragraph{Sinusoidal}

\TD{include figure with multiple frequencies and points on the x and y axis leading to embeddings}

\subsection{Positional Embeddings (learned ``encodings'')}

% possibly useful: https://theaisummer.com/positional-embeddings/

\TD{Learning to Encode Position for Transformer with Continuous Dynamical Model~\cite{DBLP:journals/corr/abs-2003-09229}}


\TD{What Do Position Embeddings Learn? An Empirical Study of Pre-Trained Language Model Positional Encoding~\cite{DBLP:journals/corr/abs-2010-04903}}

\subsubsection{Including Positional Embeddings}

% someones thoughts on  additive vs concat: https://www.reddit.com/r/MachineLearning/comments/cttefo/d_positional_encoding_in_transformer/exs7d08/

\paragraph{Additive}

\TD{saves memory (over concatenation -- less dimensions)}

\TD{figure}

\paragraph{Concatenation}

\TD{figure}


\section{MLP-Mixer}

\r{MLPs that are used to ``mix'' tokens (spatial) and ``mix'' channels (features)}

% possible blog: https://wandb.ai/wandb_fc/pytorch-image-models/reports/Is-MLP-Mixer-a-CNN-in-Disguise---Vmlldzo4NDE1MTU

% MLP resurgence
\TD{Do You Even Need Attention? A \cite{DBLP:journals/corr/abs-2105-02723}}

\TD{gMLP (Pay Attention to MLPs) \cite{DBLP:journals/corr/abs-2105-08050}}

\TD{MLP-Mixer: An all-MLP Architecture for Vision \cite{DBLP:journals/corr/abs-2105-01601}}

\TD{RepMLP: Re-parameterizing Convolutions into Fully-connected Layers for Image Recognition \cite{DBLP:journals/corr/abs-2105-01883}}

\TD{ResMLP: Feedforward networks for image classification with data-efficient training \cite{DBLP:journals/corr/abs-2105-03404}}
Conncurrent papers released looking to replace attention with MLPs.

\TD{Do You Even Need Attention? A Stack of Feed-Forward Layers Does Surprisingly Well on ImageNet \cite{MelasKyriazi2021DoYE}}




\section{Mixture of Experts (MoE)}

\TD{Breaking down a problem (task) into multiple sub-problems (sub-tasks), training and expert in each sub-problem, then learning a meta/gating model that routes information to a specific expert and combines outputs}

% Divide and conquer vs meta-learning approach


\TD{High level steps}
\begin{itemize}[noitemsep,topsep=0pt]
	\item Decompose task into subtasks
	\item Learn ``expert'' for each subtask 
	\item Decide which expert to use (gating model or gating expert)
	\item Combine outputs as needed (pool/aggregate/select)
\end{itemize}

\TD{``20 years MoE''~\cite{yuksel2012twenty}}

\TD{Outrageously Large Neural Networks: The Sparsely-Gated Mixture-of-Experts Layer~\cite{shazeer2017outrageously}}




\backmatter 

% TODO: glossary

\printindex

%%%%%%%%%%%%%%%%%%%%%%%%%%%%%%%%%%%%%%%%%%% Bibliography

\bibliographystyle{siam}
%\bibliographystyle{unsrt}
%\bibliographystyle{plain}
%\bibliographystyle{abbrv}

% TODO: multi column bib: https://tex.stackexchange.com/questions/77980/using-bibtex-how-can-i-make-the-bibliography-multicolumn

% ``hack'' to shrink size of bibliography : https://tex.stackexchange.com/questions/329/how-to-change-font-size-for-bibliography
{\scriptsize
	\bibliography{TCF,arxiv}}
%\bibliography{TCF}



\end{document} 
