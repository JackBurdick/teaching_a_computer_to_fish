\chapter{Foundational Methods}

%% Maybe (Foundational Methods --- supervised)

%%%%%%%%%%%%%%%%%%%%%%%%%%%%%% Regression
\section{Regression}

\textcolor{blue}{Three cases of the {generalized linear model}\index{generalized linear model}, simple, multiple, and polynomial linear regression.}

\textcolor{blue}{TODO: define generalized linear model}

\subsection{Simple Linear Regression}

% C2 of Mastering ML
\textcolor{blue}{Model a \emph{linear} relationship between a response variable and a feature representing an explanatory variable. The relationship is modeled with a linear surface called a hyperplane \ALR.}

\textcolor{blue}{Simple linear regression consists of two total dimensions (a dimension for the response variable and another for the explanatory) -- the hyperplane, as explained above, has one dimension (line)}

\textcolor{blue}{May also be called univariate regression (one variable).}

\textcolor{red}{convex loss function}

\textcolor{red}{history: ``method of least squares'', Legendre and Gauss, astronomy --- 1936 Fisher proposed ``linear discriminant analysis) --- 1940s (various authors - logistic regression) --- 1970s (Nelder and Wedderburn ``generalized linear models'' == entire class of statistical learning methods) (TODO: verify, find papers, ISLRp6) --- 1980s (Breinman, Friedman, Olshen, and Stone) == ``classification and regression trees''. 1986 (Hastie and Tibshirani == ``generalized additive models'' == non-linear extensions to generalized linear models)}

\begin{equation}
{Y \approx \beta_0 + \beta_1 x}
\label{eq:slr_ex}
\end{equation}

\textcolor{blue}{$\approx$ can be read as ``\emph{is approximately modeled as}''. $Y$ is a quantitative response (output/prediction) and $X$ predictor variable(input/feature). $\beta_0$ and $\beta_1$ are two unknown constants representing the intercept and slope, respectively. These unknown values that determine the behavior of the model are known as the model \emph{parameters} or \emph{coefficients}}

\subsubsection{OLS}

\TD{weighted sum of features plus a bias}


\begin{equation}
	\begin{split}
		\hat{y} & = \theta + \theta_1 x_1 + \theta_2 x_2 + ... + \theta_n + x_n  \\
		  \textrm{pred} & = \textrm{bias} + \textrm{feature weight} * \textrm{feature value}
	\end{split}
\end{equation}

\textcolor{blue}{{Ordinary Lease Squares (OLS)}\index{Ordinary Lease Squares (OLS)}, or {Linear Least Squares}\index{Linear Least Squares} is a method for estimating the parameters for a simple linear regression model.}

\textcolor{blue}{Solving OLS for simple linear regression ($y=\beta_0 + \beta_1 x$).}

\textcolor{blue}{First we'll solve for the slope $\beta_1$, where $\beta_1$ is can be found using Eq.\ref{eq:slr_ols_slope}.}

\begin{equation}
{\beta_1 =  \frac{cov(x,y)}{var(x)}}
\label{eq:slr_ols_slope}
\end{equation}

%% TODO: JACK -- these two def var and covar need to be moved

\textcolor{blue}{Variance (Eq.\ref{eq:variance_def}) is the measure of how far the set of values are spread apart -- if all the numbers in a set were equal, their variance would be zero.}

\begin{equation}
{var(x) = \frac{\sum_{i=1}^{n}(x_i - \hat{x})^2}{n-1}}
\label{eq:variance_def}
\end{equation}


\textcolor{blue}{Covariance (Eq.\ref{eq:covariance_def}) is the measure of how much two variable change together -- if two variables increase together, their covariance is positive}

\begin{equation}
{cov(x) = \frac{\sum_{i=1}^{n}(x_i - \hat{x})(y_i - \hat{y})}{n-1}}
\label{eq:covariance_def}
\end{equation}

\textcolor{blue}{After solving for $\beta_1$, $\beta_0$ can be found by rearranging the original equation and \textcolor{red}{substituting in the means of $x$ and $y$}($y=\beta_0 + \beta_1 X$) to become Eq.\ref{eq:slr_ols_intercept}}

\begin{equation}
{\beta_0 =  \bar{y} - \beta_1 \bar{x}}
\label{eq:slr_ols_intercept}
\end{equation}

\subsubsection{Cost}

\textcolor{blue}{Cost or loss function (See \textcolor{red}{local ref?}) is used to define and quantitatively measure the error of the model -- the differences between the predicted and ground truth values. The differences between the training is called the residuals\index{residuals} or training errors where as the differences observed between the test predictions and ground truths are called the prediction or test errors.}

\textcolor{blue}{A common measure of the models fitness may be the {residual sum of squares (RSS)}\index{residual sum of squares (RSS)} (Eq.\ref{eq:rss_def}, where $y_i$ is the observed value and $f(x_i)$ is the predicted value)}

\begin{equation}
{\sum_{i=1}^{n}{(y_i - f(x_i))^2}}
\label{eq:rss_def}
\end{equation}

% see p62 of ISL for more

\subsubsection{Evaluation}

\textcolor{blue}{Several methods exist for measuring the models predictive capability (see \textcolor{red}{local ref?} for more details.)}


\subsection{Multiple Linear Regression}

\textcolor{blue}{Using $n$ predictors:}

\textcolor{blue}{generalization of simple linear regression. Uses multiple features to predict the response variable}

\textcolor{blue}{linear regression with multiple variables may be called "multivariate linear regression"}



\begin{equation}
{Y \approx \beta_0 + \beta_1 X_1 + \beta_2 X_2 + \cdots + \beta_n X_n}
\label{eq:mlr_ex}
\end{equation}


\subsection{Polynomial Regression}

% TODO: this needs to be written more clearly

\textcolor{blue}{Special case of multiple linear regression, models a linear relationship between a response variable and polynomial feature terms}

\textcolor{blue}{a linear model that, using polynomial feature terms, can model non-linear relationships.}

\textcolor{blue}{It is important to note that when representing features as polynomials, feature scaling becomes increasingly important. e.g. if a feature is on a 0-100 scale and the feature is cubed, the value is now on a 0-1000000 scale.}

\textcolor{blue}{Quadratic regression (second-order polynomial) shown in equation (EQ\ref{eq:quad_regression_def})}

\begin{equation}
{Y \approx \alpha + \beta_1 X + \beta_2 X^2}
\label{eq:quad_regression_def}
\end{equation}










%%%%%%%%%%%%%%%%%%%%%%%%%%%%%% Logistic Regression
\section{Logistic Regression}

\r{Despite the `regression' bit in the name, logistic regression (logit regression) is a classification model}

\r{Similar to linear regression \ALR, logistic regression computes the weightedf sum of the input features plus a bias term. However, rather than output the result directly, a logistic of the result is output. The logistic, is sigmoid \ALR that outputs a value between 0 and 1}

\r{estimates the probability that an instance $x$ belongs in a class}


\r{odds, or odds ratio\index{odds ratio} (Eq.~\ref{eq:odds_ratio}), where $p$ is representative of the probability of a positive (event we aim to predict) event and is defined as the probability of the event occuring divided by the probability of the event not occuring (see Eq~\ref{eq:odds_ratio}).  As an example, if the probability of an event happening it $10\%$, then the odds of the event happening are $\frac{0.10}{1-0.10} = {1:9}$}

\begin{equation}
{\frac{p}{1-p}}
\label{eq:odds_ratio}
\end{equation}

\r{A logit\index{logit} is the log of the odds of the event happening. (Eq.~\ref{eq:logit_def}) (log-odds)}

\begin{equation}
{logit(p)=\log{\frac{p}{1-p}}}
\label{eq:logit_def}
\end{equation}


\begin{figure}[htp]
	\centering
	    \includegraphics[width=0.33\textwidth]{example-image-a}\hfil
		\includegraphics[width=0.33\textwidth]{example-image-b}\hfil
	\caption{\TD{The logit value is on the range $-inf$ to $inf$ and sigmoid is on the range $0$ to $1$}}
	\label{fig:logit_vs_sigmoid}
\end{figure}

\r{logistic function (sigmoid function) (Eq.~\ref{eq:sigmoid_def}) -- the inverse of a logit function and corresponds to the probability that a certain sample belongs to a particular, positive, class. If the response variable value meets or exceeds the {discrimination threshold}\index{discrimination threshold}, the positive class is predicted. As described later, the \ALR{} softmax function is used to extend to multi-class}

\begin{equation}
{S(x)={\frac{1}{1+e^{-x}}}={\frac{e^x}{e^x+1}}}
\label{eq:sigmoid_def}
\end{equation}

% TODO: placement/link around sigmoid func
\textcolor{red}{regularization is important in logistic regression since the activation function will never reach zero and attempting to do so (e.g. longer training) can lead to weights being driven to $-inf$ or $+inf$. also, near the asymptotes, the gradient is quite small}

\begin{equation}
cost =	\left\{
	\begin{array}{ll}
		-\log (\hat{p}) & \textrm{if }  y = 1 \\
		-\log (1 - \hat{p}) & \textrm{if }  y = 0 \\
	\end{array} 
	\right.
\end{equation}


\begin{equation}
	\begin{split}
		\textrm{log loss} & =  \textrm{avg over all instances} ( \textrm{cost} ) \\
		& =  \frac{1}{m} \sum_{i=1}^{m} ( \textrm{cost} ) \\
		& =  \frac{1}{m} \sum_{i=1}^{m}(   (\textrm{target}) \times \textrm{cost}_ \textrm{true} +  (1 - \textrm{target}) \times \textrm{cost}_ \textrm{false} ) \\
		& =  \frac{1}{m} \sum_{i=1}^{m}( y^{(i)} \log ({\hat{p}}^{(i)}) +(1-y^{(i)}) \log (1 - \hat{p}^{(i)}) )
	\end{split}
\end{equation}

\r{unlike linear regression, there is no presently known closed form equation for computing the parameters that minimizes the cost function.}

\r{the equation for the partial derivatives is the same as for linear regression, only with the addition of the sigmoid}

\begin{equation}
	\begin{split}
		 \textrm{derivative}_ \textrm{partial} & =  \textrm{avg over all instances} ( \textrm{error}_\textrm{pred} *  \textrm{feature}) \\
		& =  \textrm{avg}((\sigma ( \textrm{pred}) -  \textrm{target}) *  \textrm{feature}) \\
		& = \frac{1}{m} \sum_{i=1}^{m}(\sigma ( \textrm{pred}) -  \textrm{target}) *  \textrm{feature} \\
		& = \frac{1}{m} \sum_{i=1}^{m}(\sigma ( \theta^T x^{(i)}) -  y^{(i)})) *  x_j 
	\end{split}
\end{equation}

\subsection{Softmax Regression}

%TODO: index

\r{Softmax regression which may also be called Multinomial Logistic Regression, creates multi-class prediction by predicting one class from $n$ classes.}

\TD{the softmax function effectively drives small values to/near zero and pushes large values toward 1 -- where the sum of all values is equal to 1.}

\r{Let's pretend we want to make predictions over multiple classes}

\r{The softmax function (may also be called the normalized exponential)}

\r{The cost function is similar to above, now only averaging over each class}

\begin{equation}
	\begin{split}
		\textrm{cross entropy cost} & =  \textrm{avg}_\textrm{instance}  \textrm{avg}_\textrm{class}( \textrm{cost} ) \\
		& =  \frac{1}{m} \sum_{i=1}^{m}  \frac{1}{k} \sum_{i=1}^{k}  ( \textrm{cost} ) \\
		& =  \frac{1}{m} \sum_{i=1}^{m}  \frac{1}{k} \sum_{i=1}^{k}  ( (\textrm{p. instance belongs to class k}) \log ({\hat{p}}^{(i)}_k) ) \\
		& =  \frac{1}{m} \sum_{i=1}^{m}  \frac{1}{k} \sum_{i=1}^{k}  ( y^{(i)}_k \log ({\hat{p}}^{(i)}_k) ) 
	\end{split}
\end{equation}

\r{in the equation above, it's worth noting that when $k$ is equal to $2$, the cost function is equivalent to The Logistic Regression cost function \ALR}


\r{\ALR cross entropy}


%%%%%%%%%%%%%%%%%%%%%%%%%%%%%% KNN
\input{./foundations/nearest_neighbor}

%%%%%%%%%%%%%%%%%%%%%%%%%%%%%% Support Vector Machines
\input{./foundations/svm}

%%%%%%%%%%%%%%%%%%%%%%%%%%%%%% Naive Bayes
\input{./foundations/naive_bayes}

%%%%%%%%%%%%%%%%%%%%%%%%%%%%%% Decision Trees
\input{./foundations/decision_trees}


\chapter{Artificial Neural Networks}

\textcolor{blue}{If a perceptron is analogous to a single neuron, an artificial neural network (either feedforward or feedback) would be analogous to a brain.}

\r{powerful and general framework for representing non-linear mappings (function approximation) from input features to outputs, where the form of the mapping is controlled by adjustable parameters (weights and biases). Determining the values for these parameters is the ``learning'' or training.}

%%%%%%%%%%%%%%%%%%%%%%%%%%%%%% perceptron
\input{./foundations/perceptron}

%%%%%%%%%%%%%%%%%%%%%%%%%%%%%% overview
\section{Artificial Neural Networks (ANN)}

\textcolor{blue}{Principals and basic feed forward networks}

\r{The most computationally expensive component is calculating the gradient of the loss function with respect to the parameters of the network}

% see page 233 of Understanding Machine Learning
\r{Artificial neural networks are {universal approximators}\index{universal approximators} -- \textcolor{red}{expand}}

\r{universal approximation theorem \textcolor{green}{(Hornik 1989, Cybenko, 1989)}. Regardless of the function that is attempted to being learned, a large MLP will be able to \textbf{represent} this function. However, it is not guaranteed that the large MLP, despite being a universal \textcolor{red}{approximator} capable of representing the function, is able to \textit{learn} the function}


\subsection{Multi-layer Perceptron}

\r{surprisingly/dangerously robust to bugs}

\r{Not a single multi-layer perceptron with multiple layers, rather it is a network composed of multiple layers of perceptrons. Multi-layer perceptrons, through use of sucessive transformations (multipel layers of adaptive weights) address some of the limitations presented with a single layer perceptron \TD{local ref}.  MLPs, even composed of just two layers, are capable of approximating any continuous functional mapping --- the restriction being that the network must be feed-forward (described in \TD{local ref}) ensuring the outouts are possible to calcualted from as explicit functions of the inputs.}

\r{universal approximator\cite{hornik1991approximation}}

\r{justification for deeper networks --- can be exponentially more compact.}

\subsection{Architecture}

\r{{input layer}\index{input layer}, {hidden layer}\index{hidden layer}, {output layer}\index{output layer}}

\r{the input layer is not counted in the number of layers in a network}

\begin{figure}[htp]
	\centering
	\includegraphics[width=0.5\textwidth]{example-image-a}\hfil
	\caption{\TD{TODO: diagram of neural network showing layers}}
	\label{fig:foundations_ann_overview}
\end{figure}


\subsection{Components}

\begin{figure}[htp]
	\centering
	\includegraphics[width=0.5\textwidth]{example-image-a}\hfil
	\caption{\TD{TODO: labeled diagram of nodes (weights and biases), connections, activation functions}}
	\label{fig:foundations_ann_overview}
\end{figure}


\subsubsection{Nodes / units}

\paragraph{Initialization}

\TD{TODO: initialization methods and for different layers}


\subsubsection{Activation Function}

\TD{TODO: I think this is where I'll talk about activation functions}

%% need for non-linearity
\r{If all the activation functions in the hidden layers of the network were to be linear then it is possible to create a equivalent network without the hidden units. This is due to the principle that the composition of successive linear transformations is itself a linear transformation \TD{show + detail more clearly}. \textcolor{red}{the activation functions of the hidden and output layers may be different.}}

\TD{TODO: step function to sigmoid function -- smoothed version of the step function -- can understand how an input changes the output.}

\r{When considering networks only consisting of threshold activations, we run into the {credit assignment}~\index{credit assignment problem} during training. That is we have no way of determining which of the hidden units is more/less responsible for the incorrect output.  A solution to this issue is to use differentiable activation functions, this then allows for the activation of the output to become differentiable functions of both the input variables and the parameters (weights and biases).}

%% TODO: placement
\r{A sigmoidal hidden unit can be used to approximate a hidden linear unit by scaling the input parameters (weights and biases) to be very small such that the values are small and lie on the linear part of the sigmoidal curve near the origin. Similarly a step function may be approximated by scaling the input parameters (weights and biases) to be very large such that the values are either in the activated or not activated state. Nearly any continuous functional mapping can be represetned by a network consisting of two layers of sigmoidal hidden units.  A network consisting of three or more sigmoidal hidden units can approximate any smooth mapping \TD{Lapedes and Farber 1988}}

\TD{local ref to a more in depth discussion of activation functions.}

%% this likely doesn't belong here
\begin{figure}[htp]
	\centering
	\includegraphics[width=0.3\textwidth]{example-image-a}\hfil
	\includegraphics[width=0.3\textwidth]{example-image-b}\hfil
	\includegraphics[width=0.3\textwidth]{example-image-c}\hfil
	\caption{\TD{TODO: three images of possible decision boundaries created by NN with threshold act.fn and 1,2, and 3 layers. one is a single linear hyperplane, 2 is a non convex and 3 is a disjoint}}
	\label{fig:foundations_ann_layers_decision_region}
\end{figure}

\r{networks having three or more layers of weights can create non-convex and disjoint decision regions. \TD{see Huang and Lippmann 1988 for examples of 2 layers.}. Networks with two layers are not capable of creating arbitrary decision boundries \TD{Gibson and Cownan 1990, Blum and Li, 1991} (also see \TD{fig ref}). However, if the activation function is converted to a sigmoidal activation, it is possible to arbitrarily closely approximate an given decision boundry.}

%%%%%%%%%%%%%%%%%%

\textcolor{blue}{Activation functions are XXXXXXXX}

\subsubsection{Why Non-linear}

\textcolor{blue}{Non-linear is necessary XXXXXXXXXX}


\subsubsection{Popular Activation Functions}

\r{Activation functions can be grouped into two main categories -- smooth and not smooth. Smooth activation functions (such as sigmoid) are differentiable at every point along the function where as the other activation functions are not differentiable at every location (relu).}

% history
%differentiable everywhere, monotonic, and smooth.


\r{linear (see above), }

\textcolor{blue}{ReLu, better because \textcolor{red}{help prevent saturation}, but still have problems \textcolor{red}{can "die" at 0.} }

\textcolor{blue}{ELU fuctions. they prevent the "dying" problem by being \textcolor{red}{non-zero} but their main drawback is that they are more computationally expensive due to the calculation of the exponent.}

\paragraph{Smooth Non-linear}

\subparagraph{Sigmoid}

\textcolor{blue}{The sigmoid\index{sigmoid} activation function.}

\textcolor{blue}{calibrated probability estimate}


% {{{act_smooth_sigmoid}}}
\begin{figure}
	\centering
	\includegraphics[width=0.65\textwidth]{./sync_imgs/act/smooth/sigmoid.png}
	\label{fig:act_smooth_sigmoid}
\end{figure}

% {{{act_smooth_tangent}}}
\begin{figure}
	\centering
	\includegraphics[width=0.65\textwidth]{./sync_imgs/act/smooth/tangent.png}
	\label{fig:act_smooth_tangent}
\end{figure}

\subparagraph{ELU}

\textcolor{blue}{\textcolor{red}{CITE}. Smooth, monotonic, and non-zero in the negative portion of the input. The main drawback is that they are more computationally expensive (due to calculating the exponential)}


\begin{equation}
{
	ELU = f(x) = \left\{
	\begin{array}{ll}
	\alpha(e^x - 1) x & \quad $for$ \ x < 0 \\
	x & \quad $for$ \ x \ge 0
	\end{array}
	\right.
}
\label{eq:act_elu_def}
\end{equation}


% {{{act_smooth_elu}}}
\begin{figure}
	\centering
	\includegraphics[width=0.65\textwidth]{./sync_imgs/act/smooth/elu.png}
	\label{fig:act_smooth_elu}
\end{figure}

% {{{act_smooth_selu}}}
\begin{figure}
	\centering
	\includegraphics[width=0.65\textwidth]{./sync_imgs/act/smooth/selu.png}
	\label{fig:act_smooth_selu}
\end{figure}


\subparagraph{Softplus}

\textcolor{blue}{continuous and differentiable at zero. However, due to the natural log and exponential function, there is added computation compared to th ReLU.}

% typcially discouraged in practice since ReLU achieves similar results and is less computationally expensive

\begin{equation}
{
	Softplus = f(x) = \ln{(1+e^x)}
}
\label{eq:act_softplus_def}
\end{equation}


% {{{act_smooth_softplus}}}
\begin{figure}
	\centering
	\includegraphics[width=0.65\textwidth]{./sync_imgs/act/smooth/softplus.png}
	\label{fig:act_smooth_softplus}
\end{figure}

% {{{act_smooth_softsign}}}
\begin{figure}
	\centering
	\includegraphics[width=0.65\textwidth]{./sync_imgs/act/smooth/softsign.png}
	\label{fig:act_smooth_softsign}
\end{figure}


\paragraph{Not Smooth Non-linear}

\subparagraph{ReLU}

\begin{equation}
{
	ReLU = f(x) = \left\{
	\begin{array}{ll}
	0 & \quad $for$ \ x < 0 \\
	x & \quad $for$ \ x \ge 0
	\end{array}
	\right.
}
\label{eq:act_relu_def}
\end{equation}

% {{{act_notsmooth_relu}}}
\begin{figure}
	\centering
	\includegraphics[width=0.65\textwidth]{./sync_imgs/act/notsmooth/relu.png}
	\label{fig:act_notsmooth_relu}
\end{figure}

\subparagraph{Leaky ReLU}

\textcolor{blue}{The Leaky ReLU (Eq~\ref{eq:act_leaky_relu_def}) was designed in attempt to address the dying ReLU issue \textcolor{red}{CITE}. Rather than simply outputting a zero in the negative range, the Leaky ReLU will will have a small non-zero slope (user specified) -- allowing weight updating and training to continue.}

\textcolor{green}{TODO: randomized Leaky ReLU \textcolor{red}{cite} --- $\alpha$ (from PReLU) is sampled from a uniform distribution randomly. The net-effect could be considered similar to drop out since, technically, there is a different network for each value of $\alpha$, resulting in an ensemble of sorts. At test time, the values for $\alpha$ are averaged.}

\begin{equ}[!ht]
	\begin{equation}
	{
		Leaky ReLU = f(x) = \left\{
		\begin{array}{ll}
		N x & \quad $for$ \ x < 0 \\
		x & \quad $for$ \ x \ge 0
		\end{array}
		\right.
	}
	\label{eq:act_leaky_relu_def}
	\end{equation}
	\caption{where $N$ is a constant. $N$ is typically set to 0.01}
\end{equ}

% {{{act_notsmooth_leakyrelu}}}
\begin{figure}
	\centering
	\includegraphics[width=0.65\textwidth]{./sync_imgs/act/notsmooth/leakyrelu.png}
	\label{fig:act_notsmooth_leakyrelu}
\end{figure}

\subparagraph{ReLU6}

\textcolor{blue}{In general, this function is referred to as a {ReLUN}\index{ReLUN} function, where $N$ is some constant. However, in practice, $6$, was determined to be the optimal value.\textcolor{red}{CITE}. \textcolor{red}{This capped value, may help learn the sparse values sooner.} By having the upper limit bounded, the prepare the network for a fixed point precision for inference --- if the upper limit is unbounded, then you may loose too many bits to \textcolor{red}{Q} portion of the fixed point number.}


\textcolor{blue}{Similar to the ReLU fuction, only the output is capped to six in the positive domain.}

\begin{equation}
{
	ReLU6 = f(x) = min{(max{(0,x)},6)}
}
\label{eq:act_ReLU6_def}
\end{equation}

% {{{act_notsmooth_relu6}}}
\begin{figure}
	\centering
	\includegraphics[width=0.65\textwidth]{./sync_imgs/act/notsmooth/relu6.png}
	\label{fig:act_notsmooth_relu6}
\end{figure}

\subparagraph{PReLU}

\begin{equ}[!ht]
	\begin{equation}
	{
		PReLU = f(x) = \left\{
		\begin{array}{ll}
		\alpha x & \quad $for$ \ x < 0 \\
		x & \quad $for$ \ x \ge 0
		\end{array}
		\right.
	}
	\label{eq:act_prelu_def}
	\end{equation}
	\caption{where $\alpha$ is a parameterized --- a learned parameter from training.}
\end{equ}

\r{Parametric Rectified Linear Unit (PReLU) \cite{he2015delving}}

\textcolor{blue}{$\alpha$, rather than being hard coded, is determined during training by the data. The logic being that the value would be more optimal than we could set \textcolor{red}{CITE}}

% {{{act_notsmooth_prelu}}}
\begin{figure}
	\centering
	\includegraphics[width=0.65\textwidth]{./sync_imgs/act/notsmooth/prelu.png}
	\label{fig:act_notsmooth_prelu}
\end{figure}


\TD{Self-Normalizing Neural Networks \cite{DBLP:journals/corr/KlambauerUMH17}}




%%%%%%%%%%%%%%%%


\subsection{Characterization}

\subsubsection{Types: Feed-forward vs Feedback}

\textcolor{blue}{Feed-forward --- Directed acyclic graph of artificial neurons. Feedback contain feedback connections that are fed back into itself. When feedforward are include these feedback connections, they become considered recurrent neural networks.}

\paragraph{Feed-forward}

\r{``general framework for representing non-linear functional mappings between a set of input variables and a set of output variables''}

\subparagraph{Layered networks}

\begin{figure}[htp]
	\centering
	\includegraphics[width=0.5\textwidth]{example-image-a}\hfil
	\caption{\TD{TODO: layered network diagram}}
	\label{fig:foundations_ann_layered_network}
\end{figure}

\r{Whereas a single layer network is composed of linear combination of input variables, that are then, transformed by a non-linear activation function, more general functions are creating layered networks that are composed of successive layers of processing units (adaptive weights) with connections running from every unit in one layer to every unit in the next.}

\subparagraph{General topologies}

\begin{figure}[htp]
	\centering
	\includegraphics[width=0.5\textwidth]{example-image-b}\hfil
	\caption{\TD{TODO: general topology}}
	\label{fig:foundations_ann_general_topology}
\end{figure}

\r{general topologies}

\paragraph{Feedback}

\subsubsection{Terminology}

\r{Considered \textit{networks} since they are typically composed of many different functions --- creating a ``network''.}

\r{Considered \textit{neural} since they are \textbf{loosely} inspired by neuroscience.}

\r{layer --- a layer may be considered a group of units that act in parallel. The layer will extract representations from the input, that are (in theory) more useful to the specific task.  Chaining together these layers results in a form of progressive \IDI{data distillation}.}

\r{Visible and Hidden Layers. Visible layers are called visible since they contain variables that are ``visible'', where as the hidden layers extract increasingly abstract features -- hidden since their values are not given in the raw data, but rather an output from a previous layer.}



\subsection{Learning: Backpropagation}

% see p196[184] of Mastering ML w/SKL
\TD{TODO: whoooo, this is going to be a big one. understand how each component contributes to the error and adjust accordingly.}

\r{popularized by \TD{Rumelhard, Hinton and Williams (1986)}, but similar ideas were discussed earlier by \TD{Werbos 1974}, and \TD{Parker 1985}}

\r{error backpropagation is used for evaluating the dervivatives of an error function with respect to the parameters (weights and biases) of the network}

\r{Iterative algorithm consisting of two main components --- the forward, then reverse, pass.}

% see p.141(156) - 146(161) of NNbishop
\TD{MORE}

\TD{Example}

\subsubsection{Forward pass}

\r{In the forward pass inputs are propagated through the network and derivatives of the error functuion, with respect to the parameters (weights and biases) are evaluated. Propagation o ferrors through the network, calculating the derivatives, can be applied to may different error functions.}

\r{it becomes important to use a computationally efficient method for evaluating these derivatives \TD{local ref}}

\r{During this stage is when the errors are propagated through the network.}

\subsubsection{Backward pass}

\r{In the Backward pass, the previously calcuated derivatives are used to compute the adjustments to the parameters --- propagated in reverse through the network (from cost function to input layer) and each node is updated -- \TD{TODO: expand}.}

\r{backpropagation\cite{alber2018backprop}}

\r{Many optimization schemes \TD{local ref} may be used to adjust the parameters by using the calculated derivatives from the forward pass.}

\r{The calculated derivatives are used by the majority of training algorithms}

%% p116 of neural networks, p131 on tablet
\TD{Hessian matrix is a matrix containing the second derivative of a function. The second derivative provides information about the curvature of the function.}

% see p197-201[180] of Mastering ML w/SKL
\TD{TODO: figure showing sample calculation}


% See p207 of DL



\r{\IDI{symbolic differentiation} --- compute a gradient function for the chain (chain rule) mapping parameter values to gradient values}

\subsubsection{Back-propagation efficiency}

%% see para in p146(161) of bishop NN

\subsubsection{Chain Rule}

\r{See \textcolor{red}{local ref to math prereq section}}

\TD{TODO: chain rule}

\r{Backpropagation is typically used with an optimization algorithm (see \textcolor{red}{local ref?})}

\subsection{Autodiff}

\TD{Autodifferentiation (\IDI{autodiff})}

\begin{itemize}[noitemsep,topsep=0pt]
	\item Manual differentiation
	\item Finite difference approximation
	\item Autodiff
	\begin{itemize}[noitemsep,topsep=0pt]
		\item Forward-mode autodiff
		\item Reverse-mode autodiff
	\end{itemize}
\end{itemize}

\TD{automatic differentiation may be used to estimate derivatives numerically as the derivative is not always known.}

\r{Tensorflow uses \IDI{reverse-mode differentiation}.  Calculate the contribution that each parameter had on the loss value}

\subsubsection{Manual differentiation}

\r{calculus ``by hand''}

\subsubsection{Finite difference approximation}

\r{using infinitely close points on a function to calculate the slope of line passing through these points.}

%TODO: very nice code example in hands on ML that is worth including here

\TD{specific equation, Newton's difference quotient}

\subsubsection{Forward-mode autodiff}

% TODO: same flaw as above, more accurate, but requires just as many passes -- which is ``unfeasible'' for large scale NN

\TD{dual numbers}

\subsubsection{Reverse-mode autodiff}

% TODO: only two passes needed

%%%%%%%%%%%%%%%%%%%%%%%%%%%%%% feedforward
\input{./foundations/feedforward}

%%%%%%%%%%%%%%%%%%%%%%%%%%%%%% feedback
\section{Feedback or Recurrent}

\textcolor{green}{TODO: Overview}

\r{RNNs or ``\textit{\textbf{r}}ecurrent \textit{\textbf{n}}eural \textit{\textbf{n}}etworks'' are used for a variety of purposes but are typically designed with sequences of data as an input in mind. They are similar in concept to a standard/feed-forward netowrk, with the major distinction being that they also have connections that point ``backwards'' i.e. they have connections that feed into themselves.}

\r{Are capable fo working on sequences of arbitrary lengths, rather than fixed-sized inputs}

\r{universal approximator~\cite{doya1993universality}}

\r{``superiority of gated models over vanilla RNN models is almost exclusively driven by trainability''~\cite{Collins2017CapacityAT}, `` Our results suggest that, contrary to common
	belief, the capacity of RNNs to remember their input history is not a practical limiting factor on their
	performance.'' ~\cite{Collins2017CapacityAT}}

% TODO: I'm not sure where the other citation for this is... but somewhere...
\TD{systematically removing pieces of an LSTM~\cite{DBLP:journals/corr/GreffSKSS15}}

\TD{An empirical exploration of recurrent network architectures~\cite{jozefowicz2015empirical}}
\TD{``Thus we recommend adding a bias of 1 to the forget gate of every LSTM in every application''~\cite{jozefowicz2015empirical}}
\TD{``if there are [RNN] architectures that are much better than the LSTM, then they are not trivial to find''~\cite{jozefowicz2015empirical}}


\subsection{Foundation}

\r{An example of an RNN diagram is shown in \TD{fig}. However, this representation is misleading since it does not show ``every'' connection in the model --- most notably, the recurrent connections.  RNNs may also be often represented in diagrams as ``unrolled'' (\TD{fig}). The unrolled RNN is easier to visualize how these recurrent connections are included.  This makes it easier to understand how each timestep is dependent on not only the current input (at the particular time step), but also dependent on ``all'' previous time steps. It is often stated that at a certain timestep (n), the output has ``memory'' since it is a function of all the previous time steps.}


\footnotetext{the term ``all'' is emphasized here since it is the goal to include information from all previous time steps. This is true in theory, however, this is not always the case in practice. This is discussed further in \ALR{}}

\subsection{Simple RNN and Recurrent Neuron}

\TD{Diagram of the inside of a RNN neuron}


\subparagraph{Overview}

\TD{todo}


\section{Common Problems}

Two well known main problems with RNNs.

\begin{enumerate}[noitemsep,topsep=0pt]
	\item Maintaining states are expensive
	\item Vanishing and/or exploding gradients
\end{enumerate}

\TD{hardware acceleration}

\subsection{Maintaining States}


\subsection{Addressing Vanishing and Exploding Gradients}

\r{Propagating signals through a long/deep network without loosing (vanishing gradient) or overamplifying the signal (exploding gradient) is difficult.  There have been a few advances to address this issue.}

\begin{enumerate}[noitemsep,topsep=0pt]
	\item Architecture (different cell types, memory schemes)
	\item Initialization Strategies
	\item Activation Function
\end{enumerate}



\chapter{Common Operations/Components}

% TODO: I'm still not sure how/where to structure this

\section{Dense}

\TD{TODO}

\section{Convolutions}

% this reads strangely --> DNN on an image may not take advantage of the ``stationarity'' (statistics) of an image.

\r{When using a standard dense layer, all inputs are treated independently. However, adjacent pixels, on average, are highly highly correlated. For example, if there is a texture in the image, a similar pattern of pixels may occur repeatedly. Convolutions architecturally build in an implicit spatial structure to consider these spatial.}

% TODO: I'm not sure how I'm going to structure these yet or where I'll be placing them

% TODO: https://arxiv.org/abs/1904.11486
% https://www.youtube.com/watch?v=HjewNBZz00w


\TD{LeNet-5 \cite{lecun1998gradient}}

\r{Convolutions are built upon a lie -- that is we refer to the opperation as a convolution, yet it is in fact a cross-correlation operation since we don't rotate the kernel 180$\deg$. However, it is convention to refer to the operation as a convolution. For more, please see section \ref{conv_vs_cross}}

\r{translational invariance --- a property that relates to how a systems decisions are insensitive to the location of a features within an input. That is, if we're looking for an object or feature, our system shouldn't change if the object is in different locations within the input}

\TD{``Filter factorization'' (not the exact same definition of mathematical factorization)-- one $5\times5$ filter vs $2$ $3\times3$ filters stacked.  in the $5\times5$ there are $5\times5 = 25$ parameters, in the $3\times3$, there are $3\times3 \times 2 = 18$ learnable parameters, resulting in a ``cheaper'' operation.}

\TD{Neocognitron -- CNN paper prior to ``CNN''\cite{fukushima1982neocognitron}}

% Survey on CNNs
% TODO: a lot here -- good read
\TD{A Survey of the Recent Architectures of Deep Convolutional Neural Networks \cite{DBLP:journals/corr/abs-1901-06032}}


\TD{Squeeze-and-Excitation Networks \cite{DBLP:journals/corr/abs-1709-01507}}


% Graham Taylor
\r{weighted averaging operation in time or space}


\r{translation equivariant --- }

\TD{BlurPool --- ``fix is anti-aliasing by low-pass filtering before downsampling'' ---Making Convolutional Networks Shift-Invariant Again \cite{DBLP:journals/corr/abs-1904-11486}}


\r{spatial hierarchies --- \TD{TODO: figure raw data, abstract edges+, then more distinct images, then closer output to the output, then the final label}}


\r{typcially a feature extraction phase (consisting of convolutional and pooling layers) followed by a classifier block (dense layers).}

%%%% popular layer types
\textcolor{green}{TODO: feature maps, (height, width, and depth (also called channels axis)). Stride, filter size, depth. talk about parameters}

\r{The output feature map (every dimension in the depth axis is a feature/filter) --- after a convolution operation the depth of a layer is no longer representative of a color channel (like RGB), it is now representative of a feature extracted by the convolutional operation, these are called filters.}

\TD{Strided Convolution\cite{springenberg2014striving}}

\TD{Dilated Convolution --- `atrous' convolution. (famously used by wavenet), which is convenient in time series analysis.}

\r{weight tieing}


\textcolor{green}{TODO: figure}

\begin{figure}[htp]
	\centering
	\includegraphics[width=0.5\textwidth]{example-image-a}\hfil
	\caption{Figure example of convolution operation on 2d image \textcolor{green}{TODO}}
	\label{fig:conv_2d_example_calc}
\end{figure}

\begin{figure}[htp]
	\centering
	\includegraphics[width=0.5\textwidth]{example-image-b}\hfil
	\caption{Figure example of convolution operation on 3d image \textcolor{green}{TODO}}
	\label{fig:conv_2d_depth_example_calc}
\end{figure}

\textcolor{green}{TODO: examples of how different filter values and strides can effect the output dimensions.}




\section{Pooling}

\TD{TODO: examples of max vs average pooling}

%%%%%% research
\textcolor{blue}{Pooling may not fully determine learned deformation stability -- possibly filter smoothness\cite{ruderman2018learned}}

\r{downsampling}

\r{Why? importance of reducing the number of params.}

\TD{L2-pooling}

\TD{L2-pooling over the features or channels.}

\TD{additional --- learned/parameterized pooling}

\begin{figure}[htp]
	\centering
	\includegraphics[width=0.5\textwidth]{example-image-a}\hfil
	\caption{Figure example of max pooling operation on 2d image \textcolor{green}{TODO: I want this figure to be basic 2d}}
	\label{fig:pooling_max_2d_ex_a}
\end{figure}

\begin{figure}[htp]
	\centering
	\includegraphics[width=0.5\textwidth]{example-image-b}\hfil
	\caption{Figure example of average pooling operation on 3d image \textcolor{green}{TODO: I want this figure to be 3d}}
	\label{fig:pooling_avg_3d_ex_a}
\end{figure}


\r{may be better to use convolutional layers in place of the pooling layers\cite{springenberg2014striving}}

\section{Recurrent Cells}

% TODO: read this
% Recurrent / Echo state networks / ESN
\TD{The ``echo state'' approach to analysing and training recurrent neural networks-with an erratum note \cite{jaeger2001echo}}
\TD{Deep Echo State Network (DeepESN): A \cite{DBLP:journals/corr/abs-1712-04323}}

\subsection{Cell Advancements}

\subsubsection{LSTM}

% TODO: Nice overview of LSTMs: https://colah.github.io/posts/2015-08-Understanding-LSTMs/

Introduced in 1997 %\cite{hochreiter1997long}

\r{detect long term dependencies in sequence}

\r{two state vectors, short and long term}

\r{Main motivation: learning what to store in the long-term state and what to ``forget''.}

\r{at each time step, some information is ``stored'' and some information is ``forgotten''.}

\paragraph{variants}

\TD{Depth-Gated LSTM \cite{DBLP:journals/corr/YaoCVDD15}}

\TD{A Clockwork RNN \cite{DBLP:journals/corr/KoutnikGGS14}}

\TD{LSTM: A Search Space Odyssey \cite{DBLP:journals/corr/GreffSKSS15} --- survey of LSTM variants --- all variants are essentially equal.}


\paragraph{other directions}

% interesting paper on ``grid LSTMs'' -- not sure why they never become popular
\TD{Grid Long Short-Term Memory \cite{Kalchbrenner2016GridLS}}

\paragraph{Fully Connected Layers}


\begin{enumerate}[noitemsep,topsep=0pt]
	\item Main
	\item \textit{Gate Controllers}
	\begin{enumerate}[noitemsep,topsep=0pt]
		\item Forget
		\item Input
		\item Output
	\end{enumerate}
\end{enumerate}

\r{The gate controllers use a logistic activation fuction (output a range from 0 to 1). This output is then fed through an element-wise multiplication function and thus if the value is $0$, the gate is ``closed'', and $1$ if the gate is ``open''.}

\r{These gates are able to potentially:}

\begin{enumerate}[noitemsep,topsep=0pt]
	\item Recognize an important input
	\item Store the important input in a long-term state ()
	\item Preserve the information for as long as it's needed
	\item Extract the important information when needed
\end{enumerate}


\subparagraph{Main}

\begin{figure}
	\centering
	\includegraphics[width=0.5\textwidth]{example-image-a}\hfil
	\caption{\TD{Main Layer DIAGRAM}}
	%\label{}
\end{figure}

\r{This allows for the same basic functionality as a ``standard'' RNN cell --- however, the output, rather than being only sent to the next cell, is now partially stored in the long-term state.}


\subparagraph{Forget}

\r{Determines which part of the long-term state is forgotten/erased.}

\begin{figure}
	\centering
	\includegraphics[width=0.5\textwidth]{example-image-a}\hfil
	\caption{\TD{Forget Layer DIAGRAM}}
	%\label{}
\end{figure}



\subparagraph{Input}

\r{Determines which part of the output from the \textbf{main layer} are kept in the long-term state.}

\begin{figure}
	\centering
	\includegraphics[width=0.5\textwidth]{example-image-a}\hfil
	\caption{\TD{Input Layer DIAGRAM}}
	%\label{}
\end{figure}

\subparagraph{Output}

\r{Determines which part of the long term state is ``relevant'' (read and output).}

\begin{figure}
	\centering
	\includegraphics[width=0.5\textwidth]{example-image-a}\hfil
	\caption{\TD{Output Layer DIAGRAM}}
	%\label{}
\end{figure}


\paragraph{Other}

\subparagraph{Peephole Connections}

\r{In basic LSTM cells, the gate controller can only look at the input and previous short-term state. Peephole connections, proposed in 2000 \TD{cite gers2000recurrent} add an extra connection that allows for the gate controller to also see information from the long term state as well. }

\r{The previous long-term state also becomes an input to the forget and input gate. The current long-term state becomes an intput to the output gate.}



\subsubsection{GRU}

\r{The GRU (gated recurrent unit) is a varient of the LSTM cell \TD{cite - cho2014learning}. The main modifications include:}

\begin{itemize}[noitemsep,topsep=0pt]
	\item Both state vectors are merged into one state vector
	\item A single gate controller determines the \textbf{Forget} and \textbf{Input} gate
	\begin{itemize}[noitemsep,topsep=0pt]
		\item If the gate output is a 1, the input is open and the forget gate is closed. If the gate output is 0, the input gate is closed and the forget gate is open
	\end{itemize}
	\item \r{The output gate is removed and a new controller exists that controls which part of ht previous state will be ``shown'' to the main layer}. At each timestep the full state vector is output.
\end{itemize}

\subsection{Notes -- add}

\r{A recent paper \TD{greff2017lstm}, compares three LSTM variants and makes three main observations:}

\begin{itemize}[noitemsep,topsep=0pt]
	\item no significant architecture improvements over LSTMs
	\item forget gate and the output activation function are the most critical components
	\item \TD{hyperparams...}
\end{itemize}




\section{Capsule Networks}

% TODO: capsule networks
\TD{Dynamic Routing Between Capsules \cite{DBLP:journals/corr/abs-1710-09829}}

\section{Attention}

\r{``An attention function can be described as mapping a query and a set of key-value pairs to an output,
	where the query, keys, values, and output are all vectors. The output is computed as a weighted sum
	of the values, where the weight assigned to each value is computed by a compatibility function of the
	query with the corresponding key.'' \cite{DBLP:journals/corr/VaswaniSPUJGKP17}}

\TD{Self-attention Does Not Need $O(n^{2})$ Memory~\cite{Rabe2021SelfattentionDN}}

%TODO: another blog to checkout https://distill.pub/2016/augmented-rnns/

\r{overview can be found here\cite{weng2018attention}}


\TD{The original attention mechanism is introduced\cite{Bahdanau2015NeuralMT}.}

% TF attention implementation (https://www.tensorflow.org/tutorials/text/nmt_with_attention)

\TD{Effective Approaches to Attention-based Neural Machine Translation \cite{DBLP:journals/corr/LuongPM15}}

\TD{Massive Exploration of Neural Machine Translation Architectures \cite{DBLP:journals/corr/BritzGLL17}}

% TODO: index for transformer
% 'self-attention'
\TD{Attention Is All You Need -- Transformer network --- multi-head self-attention mechanism, key-value pairs \cite{DBLP:journals/corr/VaswaniSPUJGKP17}}

% self-attention \TD{Self-attention, less commonly intra-attention}
\TD{Long Short-Term Memory-Networks for Machine Reading \cite{DBLP:journals/corr/ChengDL16}}


%\TD{Nice table comparing mechanisms https://lilianweng.github.io/lil-log/2018/06/24/attention-attention.html}

\TD{in above post\cite{weng2018attention}: soft vs hard attention and global vs local attention}

% ``heads learn redundant key/query projections'' --> share
% https://github.com/epfml/collaborative-attention
\TD{Multi-Head Attention: Collaborate Instead of Concatenate \cite{Cordonnier2020MultiHeadAC}}

% soft vs hard and global vs local

\TD{Describes two variants: a ``hard'' stochastic attention mechanism (trainable via ``maximizing an approximate variational lower bound'' or REINFORCE) and a ``soft'' deterministic attention mechanism(trainable by standard back-propagation) \cite{DBLP:journals/corr/XuBKCCSZB15}. Soft attention --- scores to all entities (is differenetiable but expensive) and hard attention --- only selects one entity (non-differentiable (and complicated, reinforcement learning), but requires less computation at inference)}


% TODO: does this make sense?
\TD{Non-linear projection for K,Q, and V~\cite{DBLP:journals/corr/abs-2111-10017}}


\subsubsection{Scoring Functions}

% TODO: https://lilianweng.github.io/lil-log/2018/06/24/attention-attention.html#summary
\TD{table from \cite{weng2018attention}}


\subsection{Self-Attention}

\r{sometimes refered to as ``intra-attention''\cite{DBLP:journals/corr/VaswaniSPUJGKP17}. Keys, queries and values are all derived from the same sequence. \TD{Self-attention transforms a sequence to create a representation of itself.}}



\subsection{transformers}

% possibly useful: http://nlp.seas.harvard.edu/2018/04/03/attention.html

\TD{survey of recent transformer architectures \TD{Efficient Transformers: A Survey \cite{Tay2020EfficientTA}}}


% Factorized Attention to self-attention
\TD{Generating Long Sequences with Sparse Transformers \cite{DBLP:journals/corr/abs-1904-10509}}

% include reccurence:  "enables learning dependency beyond a fixed length" + "relative position encodings"
\TD{Transformer-XL: Attentive Language Models Beyond a Fixed-Length Context \cite{DBLP:journals/corr/abs-1901-02860}}

% extends DBLP:journals/corr/abs-1901-02860 -- 
% https://github.com/guolinke/TUPE
\TD{Compressive Transformers for Long-Range Sequence Modeling \cite{Rae2020CompressiveTF}}

% linear attention
\TD{Transformers are RNNs: Fast Autoregressive Transformers with Linear Attention \cite{Katharopoulos2020TransformersAR}}

% 
\TD{Transformer with Untied Positional Encoding (TUPE) --- Rethinking Positional Encoding in Language Pre-training \cite{Ke2020RethinkingPE}}



\TD{Reformer: The Efficient Transformer \cite{Kitaev2020ReformerTE}}


% TODO: top-down attention
% related to self-attention
% https://twitter.com/thomaskipf/status/1277570203665170432
\TD{Object-Centric Learning with Slot Attention \cite{Locatello2020ObjectCentricLW}}

\TD{Recurrent Independent Mechanisms \cite{Goyal2019RecurrentIM}}

% DETR -- also object detection
\TD{End-to-End Object Detection with Transformers \cite{Carion2020EndtoEndOD}}


% TODO: read https://lilianweng.github.io/lil-log/2020/04/07/the-transformer-family.html


\subsection{Positional Encodings}

\TD{Positional embedding and positional encoding tend to be used interchangably. However, typically an encoding means ``fixed'' while an embedding means ``learned'' or ``trainable''.}

% TODO: example of how word order matters (not is a good example)

\r{Attention/transformers view the inputs as sets, that is there is no order associated with each input. All information enters the attention block at once. This is in contrast to something like a recurrent model, in which the order of the inputs is implicit.}

\r{trade off: potentially faster (remove the dependancy of doing operations sequentially) and can also possibly help capture longer range dependancies (without additional complexity e.g. skip connections)}

\r{(re)introducing order to the input by including additional information -- the ``positional embedding''.}

\r{NOTE: Great blog posts on this subject~\cite{kazemnejad_2021, kernes_2021, kernes_2021B}}

\subsubsection{Positional Encoding Value}

\r{why not add linear/progressive value signifying order?}

\r{This would be called an aboslute positional embedding}

\r{Include index information [0, n], where n is the length of the sentance (minus 1). This could lead to magnitude issues. Where the singal from the word embeddings is ``washed out'' by the positional embedding.  Another consideration is that (may or may not be an issue depending on the application) is that you'd like to ensure you have the largest sequence in the training set that you expect to see in evaluation set. For example, if you only see sequences of length $25$ in the training data and then see a sequence of length $32$ during inference. The model will be unsure what to do with values $25 - 31$ (zero indexing). Depending on how you include the positional embedding (e.g. additive or concat), the model may misinterpret the values or be largely/entirely unsure what to make of these previously unseen values.}
	
	
\r{To address this you could either increase the magnitude of the word embeddings or normalize/scale the positional embedding.}

\r{However, niether are ideal.}

\r{Increasing the magnitude of the word embeddings would possibly work, though you may consider issues with exploding values in the network, but you'd still have a similar issue to what would happen if you normalized the positional embedding. }

\r{That is, the normalized positional embeddings may encode different information when the sentances are longer or shorter -- the delta between words in a 5 word sentance vs a 20 word sentance doesn't have a consistent meaning}

% NOTE: haven't read this yet (I don't think, though the link is purple...)
\TD{Self-Attention with Relative Position Representations~\cite{DBLP:journals/corr/abs-1803-02155}}

\r{Ideally the embedding would be able to account for all the issues we discussed.}

\begin{itemize}[noitemsep,topsep=0pt]
	\item consistent delta between each position
		\begin{itemize}[noitemsep,topsep=0pt]
			\item regardless of sequence length, if an instance is one instance away from another, the positional encoding should be the same e.g. in a length four sequence the positional encoding should be the same from instances $1$ and $2$ as it is for instances $19$ and $20$ in a length $22$ sequence.
		\end{itemize}
	\item generalize to sequence lengths unseen in training
\end{itemize}

\r{additionally, we'd prefer to have each instance in the sequence be unique. That is the positional encoding for one instance shouldn't be the same as another in the same sequence (e.g. two words in a sentance).}

\paragraph{Positional Encoding Value(s)}

\r{Rather than use a single value, a possible solution is to use an array of values.}

\TD{Relative positional encoding (rather than absolute).}


\TD{What if we were to use a binary array to represent each location?}

\TD{issue with binary}

\TD{}


\TD{CAPE: Encoding Relative Positions with Continuous Augmented Positional Embeddings~\cite{DBLP:journals/corr/abs-2106-03143}}

% NOTE: possibly relevant: https://aclanthology.org/2021.emnlp-main.266.pdf

\paragraph{Sinusoidal}

\TD{include figure with multiple frequencies and points on the x and y axis leading to embeddings}

\subsection{Positional Embeddings (learned ``encodings'')}

% possibly useful: https://theaisummer.com/positional-embeddings/

\TD{Learning to Encode Position for Transformer with Continuous Dynamical Model~\cite{DBLP:journals/corr/abs-2003-09229}}


\TD{What Do Position Embeddings Learn? An Empirical Study of Pre-Trained Language Model Positional Encoding~\cite{DBLP:journals/corr/abs-2010-04903}}

\subsubsection{Including Positional Embeddings}

% someones thoughts on  additive vs concat: https://www.reddit.com/r/MachineLearning/comments/cttefo/d_positional_encoding_in_transformer/exs7d08/

\paragraph{Additive}

\TD{saves memory (over concatenation -- less dimensions)}

\TD{figure}

\paragraph{Concatenation}

\TD{figure}


\section{MLP-Mixer}

\r{MLPs that are used to ``mix'' tokens (spatial) and ``mix'' channels (features)}

% possible blog: https://wandb.ai/wandb_fc/pytorch-image-models/reports/Is-MLP-Mixer-a-CNN-in-Disguise---Vmlldzo4NDE1MTU

% MLP resurgence
\TD{Do You Even Need Attention? A \cite{DBLP:journals/corr/abs-2105-02723}}

\TD{gMLP (Pay Attention to MLPs) \cite{DBLP:journals/corr/abs-2105-08050}}

\TD{MLP-Mixer: An all-MLP Architecture for Vision \cite{DBLP:journals/corr/abs-2105-01601}}

\TD{RepMLP: Re-parameterizing Convolutions into Fully-connected Layers for Image Recognition \cite{DBLP:journals/corr/abs-2105-01883}}

\TD{ResMLP: Feedforward networks for image classification with data-efficient training \cite{DBLP:journals/corr/abs-2105-03404}}
Conncurrent papers released looking to replace attention with MLPs.

\TD{Do You Even Need Attention? A Stack of Feed-Forward Layers Does Surprisingly Well on ImageNet \cite{MelasKyriazi2021DoYE}}




\section{Mixture of Experts (MoE)}

\TD{Breaking down a problem (task) into multiple sub-problems (sub-tasks), training and expert in each sub-problem, then learning a meta/gating model that routes information to a specific expert and combines outputs}

% Divide and conquer vs meta-learning approach


\TD{High level steps}
\begin{itemize}[noitemsep,topsep=0pt]
	\item Decompose task into subtasks
	\item Learn ``expert'' for each subtask 
	\item Decide which expert to use (gating model or gating expert)
	\item Combine outputs as needed (pool/aggregate/select)
\end{itemize}

\TD{``20 years MoE''~\cite{yuksel2012twenty}}

\TD{Outrageously Large Neural Networks: The Sparsely-Gated Mixture-of-Experts Layer~\cite{shazeer2017outrageously}}



\chapter{Applied Neural Networks}

% TODO: I'm still not sure how/where to structure this

\section{Dense}

\TD{TODO}

\section{Convolutions}

% this reads strangely --> DNN on an image may not take advantage of the ``stationarity'' (statistics) of an image.

\r{When using a standard dense layer, all inputs are treated independently. However, adjacent pixels, on average, are highly highly correlated. For example, if there is a texture in the image, a similar pattern of pixels may occur repeatedly. Convolutions architecturally build in an implicit spatial structure to consider these spatial.}

% TODO: I'm not sure how I'm going to structure these yet or where I'll be placing them

% TODO: https://arxiv.org/abs/1904.11486
% https://www.youtube.com/watch?v=HjewNBZz00w


\TD{LeNet-5 \cite{lecun1998gradient}}

\r{Convolutions are built upon a lie -- that is we refer to the opperation as a convolution, yet it is in fact a cross-correlation operation since we don't rotate the kernel 180$\deg$. However, it is convention to refer to the operation as a convolution. For more, please see section \ref{conv_vs_cross}}

\r{translational invariance --- a property that relates to how a systems decisions are insensitive to the location of a features within an input. That is, if we're looking for an object or feature, our system shouldn't change if the object is in different locations within the input}

\TD{``Filter factorization'' (not the exact same definition of mathematical factorization)-- one $5\times5$ filter vs $2$ $3\times3$ filters stacked.  in the $5\times5$ there are $5\times5 = 25$ parameters, in the $3\times3$, there are $3\times3 \times 2 = 18$ learnable parameters, resulting in a ``cheaper'' operation.}

\TD{Neocognitron -- CNN paper prior to ``CNN''\cite{fukushima1982neocognitron}}

% Survey on CNNs
% TODO: a lot here -- good read
\TD{A Survey of the Recent Architectures of Deep Convolutional Neural Networks \cite{DBLP:journals/corr/abs-1901-06032}}


\TD{Squeeze-and-Excitation Networks \cite{DBLP:journals/corr/abs-1709-01507}}


% Graham Taylor
\r{weighted averaging operation in time or space}


\r{translation equivariant --- }

\TD{BlurPool --- ``fix is anti-aliasing by low-pass filtering before downsampling'' ---Making Convolutional Networks Shift-Invariant Again \cite{DBLP:journals/corr/abs-1904-11486}}


\r{spatial hierarchies --- \TD{TODO: figure raw data, abstract edges+, then more distinct images, then closer output to the output, then the final label}}


\r{typcially a feature extraction phase (consisting of convolutional and pooling layers) followed by a classifier block (dense layers).}

%%%% popular layer types
\textcolor{green}{TODO: feature maps, (height, width, and depth (also called channels axis)). Stride, filter size, depth. talk about parameters}

\r{The output feature map (every dimension in the depth axis is a feature/filter) --- after a convolution operation the depth of a layer is no longer representative of a color channel (like RGB), it is now representative of a feature extracted by the convolutional operation, these are called filters.}

\TD{Strided Convolution\cite{springenberg2014striving}}

\TD{Dilated Convolution --- `atrous' convolution. (famously used by wavenet), which is convenient in time series analysis.}

\r{weight tieing}


\textcolor{green}{TODO: figure}

\begin{figure}[htp]
	\centering
	\includegraphics[width=0.5\textwidth]{example-image-a}\hfil
	\caption{Figure example of convolution operation on 2d image \textcolor{green}{TODO}}
	\label{fig:conv_2d_example_calc}
\end{figure}

\begin{figure}[htp]
	\centering
	\includegraphics[width=0.5\textwidth]{example-image-b}\hfil
	\caption{Figure example of convolution operation on 3d image \textcolor{green}{TODO}}
	\label{fig:conv_2d_depth_example_calc}
\end{figure}

\textcolor{green}{TODO: examples of how different filter values and strides can effect the output dimensions.}




\section{Pooling}

\TD{TODO: examples of max vs average pooling}

%%%%%% research
\textcolor{blue}{Pooling may not fully determine learned deformation stability -- possibly filter smoothness\cite{ruderman2018learned}}

\r{downsampling}

\r{Why? importance of reducing the number of params.}

\TD{L2-pooling}

\TD{L2-pooling over the features or channels.}

\TD{additional --- learned/parameterized pooling}

\begin{figure}[htp]
	\centering
	\includegraphics[width=0.5\textwidth]{example-image-a}\hfil
	\caption{Figure example of max pooling operation on 2d image \textcolor{green}{TODO: I want this figure to be basic 2d}}
	\label{fig:pooling_max_2d_ex_a}
\end{figure}

\begin{figure}[htp]
	\centering
	\includegraphics[width=0.5\textwidth]{example-image-b}\hfil
	\caption{Figure example of average pooling operation on 3d image \textcolor{green}{TODO: I want this figure to be 3d}}
	\label{fig:pooling_avg_3d_ex_a}
\end{figure}


\r{may be better to use convolutional layers in place of the pooling layers\cite{springenberg2014striving}}

\section{Recurrent Cells}

% TODO: read this
% Recurrent / Echo state networks / ESN
\TD{The ``echo state'' approach to analysing and training recurrent neural networks-with an erratum note \cite{jaeger2001echo}}
\TD{Deep Echo State Network (DeepESN): A \cite{DBLP:journals/corr/abs-1712-04323}}

\subsection{Cell Advancements}

\subsubsection{LSTM}

% TODO: Nice overview of LSTMs: https://colah.github.io/posts/2015-08-Understanding-LSTMs/

Introduced in 1997 %\cite{hochreiter1997long}

\r{detect long term dependencies in sequence}

\r{two state vectors, short and long term}

\r{Main motivation: learning what to store in the long-term state and what to ``forget''.}

\r{at each time step, some information is ``stored'' and some information is ``forgotten''.}

\paragraph{variants}

\TD{Depth-Gated LSTM \cite{DBLP:journals/corr/YaoCVDD15}}

\TD{A Clockwork RNN \cite{DBLP:journals/corr/KoutnikGGS14}}

\TD{LSTM: A Search Space Odyssey \cite{DBLP:journals/corr/GreffSKSS15} --- survey of LSTM variants --- all variants are essentially equal.}


\paragraph{other directions}

% interesting paper on ``grid LSTMs'' -- not sure why they never become popular
\TD{Grid Long Short-Term Memory \cite{Kalchbrenner2016GridLS}}

\paragraph{Fully Connected Layers}


\begin{enumerate}[noitemsep,topsep=0pt]
	\item Main
	\item \textit{Gate Controllers}
	\begin{enumerate}[noitemsep,topsep=0pt]
		\item Forget
		\item Input
		\item Output
	\end{enumerate}
\end{enumerate}

\r{The gate controllers use a logistic activation fuction (output a range from 0 to 1). This output is then fed through an element-wise multiplication function and thus if the value is $0$, the gate is ``closed'', and $1$ if the gate is ``open''.}

\r{These gates are able to potentially:}

\begin{enumerate}[noitemsep,topsep=0pt]
	\item Recognize an important input
	\item Store the important input in a long-term state ()
	\item Preserve the information for as long as it's needed
	\item Extract the important information when needed
\end{enumerate}


\subparagraph{Main}

\begin{figure}
	\centering
	\includegraphics[width=0.5\textwidth]{example-image-a}\hfil
	\caption{\TD{Main Layer DIAGRAM}}
	%\label{}
\end{figure}

\r{This allows for the same basic functionality as a ``standard'' RNN cell --- however, the output, rather than being only sent to the next cell, is now partially stored in the long-term state.}


\subparagraph{Forget}

\r{Determines which part of the long-term state is forgotten/erased.}

\begin{figure}
	\centering
	\includegraphics[width=0.5\textwidth]{example-image-a}\hfil
	\caption{\TD{Forget Layer DIAGRAM}}
	%\label{}
\end{figure}



\subparagraph{Input}

\r{Determines which part of the output from the \textbf{main layer} are kept in the long-term state.}

\begin{figure}
	\centering
	\includegraphics[width=0.5\textwidth]{example-image-a}\hfil
	\caption{\TD{Input Layer DIAGRAM}}
	%\label{}
\end{figure}

\subparagraph{Output}

\r{Determines which part of the long term state is ``relevant'' (read and output).}

\begin{figure}
	\centering
	\includegraphics[width=0.5\textwidth]{example-image-a}\hfil
	\caption{\TD{Output Layer DIAGRAM}}
	%\label{}
\end{figure}


\paragraph{Other}

\subparagraph{Peephole Connections}

\r{In basic LSTM cells, the gate controller can only look at the input and previous short-term state. Peephole connections, proposed in 2000 \TD{cite gers2000recurrent} add an extra connection that allows for the gate controller to also see information from the long term state as well. }

\r{The previous long-term state also becomes an input to the forget and input gate. The current long-term state becomes an intput to the output gate.}



\subsubsection{GRU}

\r{The GRU (gated recurrent unit) is a varient of the LSTM cell \TD{cite - cho2014learning}. The main modifications include:}

\begin{itemize}[noitemsep,topsep=0pt]
	\item Both state vectors are merged into one state vector
	\item A single gate controller determines the \textbf{Forget} and \textbf{Input} gate
	\begin{itemize}[noitemsep,topsep=0pt]
		\item If the gate output is a 1, the input is open and the forget gate is closed. If the gate output is 0, the input gate is closed and the forget gate is open
	\end{itemize}
	\item \r{The output gate is removed and a new controller exists that controls which part of ht previous state will be ``shown'' to the main layer}. At each timestep the full state vector is output.
\end{itemize}

\subsection{Notes -- add}

\r{A recent paper \TD{greff2017lstm}, compares three LSTM variants and makes three main observations:}

\begin{itemize}[noitemsep,topsep=0pt]
	\item no significant architecture improvements over LSTMs
	\item forget gate and the output activation function are the most critical components
	\item \TD{hyperparams...}
\end{itemize}




\section{Capsule Networks}

% TODO: capsule networks
\TD{Dynamic Routing Between Capsules \cite{DBLP:journals/corr/abs-1710-09829}}

\section{Attention}

\r{``An attention function can be described as mapping a query and a set of key-value pairs to an output,
	where the query, keys, values, and output are all vectors. The output is computed as a weighted sum
	of the values, where the weight assigned to each value is computed by a compatibility function of the
	query with the corresponding key.'' \cite{DBLP:journals/corr/VaswaniSPUJGKP17}}

\TD{Self-attention Does Not Need $O(n^{2})$ Memory~\cite{Rabe2021SelfattentionDN}}

%TODO: another blog to checkout https://distill.pub/2016/augmented-rnns/

\r{overview can be found here\cite{weng2018attention}}


\TD{The original attention mechanism is introduced\cite{Bahdanau2015NeuralMT}.}

% TF attention implementation (https://www.tensorflow.org/tutorials/text/nmt_with_attention)

\TD{Effective Approaches to Attention-based Neural Machine Translation \cite{DBLP:journals/corr/LuongPM15}}

\TD{Massive Exploration of Neural Machine Translation Architectures \cite{DBLP:journals/corr/BritzGLL17}}

% TODO: index for transformer
% 'self-attention'
\TD{Attention Is All You Need -- Transformer network --- multi-head self-attention mechanism, key-value pairs \cite{DBLP:journals/corr/VaswaniSPUJGKP17}}

% self-attention \TD{Self-attention, less commonly intra-attention}
\TD{Long Short-Term Memory-Networks for Machine Reading \cite{DBLP:journals/corr/ChengDL16}}


%\TD{Nice table comparing mechanisms https://lilianweng.github.io/lil-log/2018/06/24/attention-attention.html}

\TD{in above post\cite{weng2018attention}: soft vs hard attention and global vs local attention}

% ``heads learn redundant key/query projections'' --> share
% https://github.com/epfml/collaborative-attention
\TD{Multi-Head Attention: Collaborate Instead of Concatenate \cite{Cordonnier2020MultiHeadAC}}

% soft vs hard and global vs local

\TD{Describes two variants: a ``hard'' stochastic attention mechanism (trainable via ``maximizing an approximate variational lower bound'' or REINFORCE) and a ``soft'' deterministic attention mechanism(trainable by standard back-propagation) \cite{DBLP:journals/corr/XuBKCCSZB15}. Soft attention --- scores to all entities (is differenetiable but expensive) and hard attention --- only selects one entity (non-differentiable (and complicated, reinforcement learning), but requires less computation at inference)}


% TODO: does this make sense?
\TD{Non-linear projection for K,Q, and V~\cite{DBLP:journals/corr/abs-2111-10017}}


\subsubsection{Scoring Functions}

% TODO: https://lilianweng.github.io/lil-log/2018/06/24/attention-attention.html#summary
\TD{table from \cite{weng2018attention}}


\subsection{Self-Attention}

\r{sometimes refered to as ``intra-attention''\cite{DBLP:journals/corr/VaswaniSPUJGKP17}. Keys, queries and values are all derived from the same sequence. \TD{Self-attention transforms a sequence to create a representation of itself.}}



\subsection{transformers}

% possibly useful: http://nlp.seas.harvard.edu/2018/04/03/attention.html

\TD{survey of recent transformer architectures \TD{Efficient Transformers: A Survey \cite{Tay2020EfficientTA}}}


% Factorized Attention to self-attention
\TD{Generating Long Sequences with Sparse Transformers \cite{DBLP:journals/corr/abs-1904-10509}}

% include reccurence:  "enables learning dependency beyond a fixed length" + "relative position encodings"
\TD{Transformer-XL: Attentive Language Models Beyond a Fixed-Length Context \cite{DBLP:journals/corr/abs-1901-02860}}

% extends DBLP:journals/corr/abs-1901-02860 -- 
% https://github.com/guolinke/TUPE
\TD{Compressive Transformers for Long-Range Sequence Modeling \cite{Rae2020CompressiveTF}}

% linear attention
\TD{Transformers are RNNs: Fast Autoregressive Transformers with Linear Attention \cite{Katharopoulos2020TransformersAR}}

% 
\TD{Transformer with Untied Positional Encoding (TUPE) --- Rethinking Positional Encoding in Language Pre-training \cite{Ke2020RethinkingPE}}



\TD{Reformer: The Efficient Transformer \cite{Kitaev2020ReformerTE}}


% TODO: top-down attention
% related to self-attention
% https://twitter.com/thomaskipf/status/1277570203665170432
\TD{Object-Centric Learning with Slot Attention \cite{Locatello2020ObjectCentricLW}}

\TD{Recurrent Independent Mechanisms \cite{Goyal2019RecurrentIM}}

% DETR -- also object detection
\TD{End-to-End Object Detection with Transformers \cite{Carion2020EndtoEndOD}}


% TODO: read https://lilianweng.github.io/lil-log/2020/04/07/the-transformer-family.html


\subsection{Positional Encodings}

\TD{Positional embedding and positional encoding tend to be used interchangably. However, typically an encoding means ``fixed'' while an embedding means ``learned'' or ``trainable''.}

% TODO: example of how word order matters (not is a good example)

\r{Attention/transformers view the inputs as sets, that is there is no order associated with each input. All information enters the attention block at once. This is in contrast to something like a recurrent model, in which the order of the inputs is implicit.}

\r{trade off: potentially faster (remove the dependancy of doing operations sequentially) and can also possibly help capture longer range dependancies (without additional complexity e.g. skip connections)}

\r{(re)introducing order to the input by including additional information -- the ``positional embedding''.}

\r{NOTE: Great blog posts on this subject~\cite{kazemnejad_2021, kernes_2021, kernes_2021B}}

\subsubsection{Positional Encoding Value}

\r{why not add linear/progressive value signifying order?}

\r{This would be called an aboslute positional embedding}

\r{Include index information [0, n], where n is the length of the sentance (minus 1). This could lead to magnitude issues. Where the singal from the word embeddings is ``washed out'' by the positional embedding.  Another consideration is that (may or may not be an issue depending on the application) is that you'd like to ensure you have the largest sequence in the training set that you expect to see in evaluation set. For example, if you only see sequences of length $25$ in the training data and then see a sequence of length $32$ during inference. The model will be unsure what to do with values $25 - 31$ (zero indexing). Depending on how you include the positional embedding (e.g. additive or concat), the model may misinterpret the values or be largely/entirely unsure what to make of these previously unseen values.}
	
	
\r{To address this you could either increase the magnitude of the word embeddings or normalize/scale the positional embedding.}

\r{However, niether are ideal.}

\r{Increasing the magnitude of the word embeddings would possibly work, though you may consider issues with exploding values in the network, but you'd still have a similar issue to what would happen if you normalized the positional embedding. }

\r{That is, the normalized positional embeddings may encode different information when the sentances are longer or shorter -- the delta between words in a 5 word sentance vs a 20 word sentance doesn't have a consistent meaning}

% NOTE: haven't read this yet (I don't think, though the link is purple...)
\TD{Self-Attention with Relative Position Representations~\cite{DBLP:journals/corr/abs-1803-02155}}

\r{Ideally the embedding would be able to account for all the issues we discussed.}

\begin{itemize}[noitemsep,topsep=0pt]
	\item consistent delta between each position
		\begin{itemize}[noitemsep,topsep=0pt]
			\item regardless of sequence length, if an instance is one instance away from another, the positional encoding should be the same e.g. in a length four sequence the positional encoding should be the same from instances $1$ and $2$ as it is for instances $19$ and $20$ in a length $22$ sequence.
		\end{itemize}
	\item generalize to sequence lengths unseen in training
\end{itemize}

\r{additionally, we'd prefer to have each instance in the sequence be unique. That is the positional encoding for one instance shouldn't be the same as another in the same sequence (e.g. two words in a sentance).}

\paragraph{Positional Encoding Value(s)}

\r{Rather than use a single value, a possible solution is to use an array of values.}

\TD{Relative positional encoding (rather than absolute).}


\TD{What if we were to use a binary array to represent each location?}

\TD{issue with binary}

\TD{}


\TD{CAPE: Encoding Relative Positions with Continuous Augmented Positional Embeddings~\cite{DBLP:journals/corr/abs-2106-03143}}

% NOTE: possibly relevant: https://aclanthology.org/2021.emnlp-main.266.pdf

\paragraph{Sinusoidal}

\TD{include figure with multiple frequencies and points on the x and y axis leading to embeddings}

\subsection{Positional Embeddings (learned ``encodings'')}

% possibly useful: https://theaisummer.com/positional-embeddings/

\TD{Learning to Encode Position for Transformer with Continuous Dynamical Model~\cite{DBLP:journals/corr/abs-2003-09229}}


\TD{What Do Position Embeddings Learn? An Empirical Study of Pre-Trained Language Model Positional Encoding~\cite{DBLP:journals/corr/abs-2010-04903}}

\subsubsection{Including Positional Embeddings}

% someones thoughts on  additive vs concat: https://www.reddit.com/r/MachineLearning/comments/cttefo/d_positional_encoding_in_transformer/exs7d08/

\paragraph{Additive}

\TD{saves memory (over concatenation -- less dimensions)}

\TD{figure}

\paragraph{Concatenation}

\TD{figure}


\section{MLP-Mixer}

\r{MLPs that are used to ``mix'' tokens (spatial) and ``mix'' channels (features)}

% possible blog: https://wandb.ai/wandb_fc/pytorch-image-models/reports/Is-MLP-Mixer-a-CNN-in-Disguise---Vmlldzo4NDE1MTU

% MLP resurgence
\TD{Do You Even Need Attention? A \cite{DBLP:journals/corr/abs-2105-02723}}

\TD{gMLP (Pay Attention to MLPs) \cite{DBLP:journals/corr/abs-2105-08050}}

\TD{MLP-Mixer: An all-MLP Architecture for Vision \cite{DBLP:journals/corr/abs-2105-01601}}

\TD{RepMLP: Re-parameterizing Convolutions into Fully-connected Layers for Image Recognition \cite{DBLP:journals/corr/abs-2105-01883}}

\TD{ResMLP: Feedforward networks for image classification with data-efficient training \cite{DBLP:journals/corr/abs-2105-03404}}
Conncurrent papers released looking to replace attention with MLPs.

\TD{Do You Even Need Attention? A Stack of Feed-Forward Layers Does Surprisingly Well on ImageNet \cite{MelasKyriazi2021DoYE}}




\section{Mixture of Experts (MoE)}

\TD{Breaking down a problem (task) into multiple sub-problems (sub-tasks), training and expert in each sub-problem, then learning a meta/gating model that routes information to a specific expert and combines outputs}

% Divide and conquer vs meta-learning approach


\TD{High level steps}
\begin{itemize}[noitemsep,topsep=0pt]
	\item Decompose task into subtasks
	\item Learn ``expert'' for each subtask 
	\item Decide which expert to use (gating model or gating expert)
	\item Combine outputs as needed (pool/aggregate/select)
\end{itemize}

\TD{``20 years MoE''~\cite{yuksel2012twenty}}

\TD{Outrageously Large Neural Networks: The Sparsely-Gated Mixture-of-Experts Layer~\cite{shazeer2017outrageously}}




\chapter{Unsupervised}

\r{The reality is that data is not always labeled.}

\r{unsupervised learning is capable of finding hidden patterns in the underlying structure of hte data.}

\r{attempts to represent data with increaingly fewer parameters}

\r{Discovering hidden structures or patterns in unlabeled training data.}

\TD{Neighborhood-Based Methods \ref{nearest_neighbors} \r{lazy learners  -- learn how to label new instances based on proximity to existing instances}}

% TODO: placement / may need to rename+restructure sections
\textcolor{blue}{unsupervised methods may be commonly used in two main settings:}
\begin{enumerate}[noitemsep,topsep=0pt]
	\item Data Exploration
	\begin{itemize}[noitemsep,topsep=0pt]
		\item Visualization (clustering \textcolor{red}{local ref})
	\end{itemize}
	\item Preprocessing (e.g. prior to a supervised method): unsupervised pretraining may be considered a form of regularization
	\begin{itemize}[noitemsep,topsep=0pt]
		\item Compressing (dimensionality reduction \textcolor{red}{local ref})
		\item Creating new/different representations
	\end{itemize}
\end{enumerate}

\r{regularization, feature engineering, detecting outliers -- also used for detecting how different new (incoming) training data is from the current distribution.}

\r{popular applications --- anomaly detection, group segmentation, preprocessing (dimensionality reduction)}

\subsection{TODO}

\TD{evaluating unsupervised learning systems are harder to evaluate than supervised learning systems.}

\r{evaluating supervised methods is often subjective. An approach to skirt this is issue may be to label the test set manually and then use any desired supervised learning metric.}

%%%%%%%%%%%%%%%%%%%%%%%%%%%%%% clustering
\input{./foundations/unsupervised/clustering}

%%%%%%%%%%%%%%%%%%%%%%%%%%%%%% Dimensionality Reduction
\input{./foundations/unsupervised/dimensionality_reduction}


% TODO: format / other
%\input{./foundations/unsupervised/applications}


\chapter{Semi-supervised}

\input{./foundations/semi_supervised}


\chapter{Common Architectures}


\TD{Overview}

% TODO: I'm still not sure if I should divide this up by ``types'' of architectures or types of problems and the architectures used to solve them.. maybe both? unsure...


Images and Videos
\begin{itemize}[noitemsep,topsep=0pt]
	\item Classification
	\item Segmentation
	\begin{itemize}[noitemsep,topsep=0pt]
		\item Semantic segmentation (where we don't differentiate between instances)
		\item Instance segmentation
	\end{itemize}
	\item Object detection
\end{itemize}


\subsection{Spatial Data}

\r{types of problems -- spatial to spatial (1:1), spatial to sequence, spatial to value (single or multiple)}

\subsubsection{Convolutional Approaches}



\subsubsection{Image Classification}
% Graham Taylor talk
\r{alexnet, network-in-network, inception/google lenet (end with $1 \times 1 \times N_{classes}$ into a global average pooling layer), vggnet}

% TODO: I'm not sure where this belongs
\TD{$1 \times 1 \times$ convolution explaination --- useful for adjusting the number of features in the feature dimension, either up or down.}

\TD{Inception module --- rather than manually choosing proper\``best'' filter size, ``let the model choose''.}

\r{Resnet (\TD{Deep Residual Learning for Image Recognition \cite{DBLP:journals/corr/HeZRS15}}, similar to highway networks) --- skip connections ``residual modual/block'' (dynamically adjusting depth) (others in this time: fractal nets, stochastic XXXX nets, where skip connections are shortcut paths)}

\r{DenseNet --- where these skip connections are taken to an extreme. also, concatenation, not summation}

% TODO: not sure where this belongs yet
\paragraph{skip connections}


\r{three popular explainations (as described in \cite{lakshmanan2021practical}).}
\begin{itemize}[noitemsep,topsep=0pt]
	\item Addition opperation makes the task ``easier''
	\begin{itemize}[noitemsep,topsep=0pt]
		\item Detailed by the authors of the original paper. That it is easier to predict the ``residue''/delta between the input and output rather than the solely the desired output.
	\end{itemize}
	\item Residual connections make the network effectively shallower 
	\begin{itemize}[noitemsep,topsep=0pt]
		\item Described in \TD{Residual Networks are Exponential Ensembles of Relatively Shallow Networks \cite{DBLP:journals/corr/VeitWB16}. The connections effectively create an ensemble of shallower networks. The network then learns to select the best path/subnetwork for each instance.}
	\end{itemize}
	\item Loss topological ``smoothing'' (TODO: change term)
	\begin{itemize}[noitemsep,topsep=0pt]
		\item \TD{show figure. \TD{Visualizing the Loss Landscape of Neural Nets \cite{DBLP:journals/corr/abs-1712-09913}}}
	\end{itemize}
\end{itemize}



\TD{SqueezeNet \cite{DBLP:journals/corr/IandolaMAHDK16} ``fire modules'', contraction and expansion}

\r{Squeeze and Excitation network --- adaptive recalibration of feature maps. \cite{DBLP:journals/corr/abs-1709-01507} }

% fit into an architecture ``timeline''
\TD{ResNeXt\cite{xie2017aggregated}}


\subsection{Object Detection}

\r{sliding window and use each window as input to a classifier. But the problem is that we need to apply the classifier to a huge number of locations and scales and aspect ratios. A potential solution to this problem is to use region proposals.}

\r{region proposals were introduced by R-CNN \TD{cite}, where conventional methods were used to propose regions for the CNN to classify. It's important to note that most regions are not square may need to be warped to be insert to the CNN}

\r{Bbox regression is also used --- how much to offset the region proposal.}


\r{fast R-CNN proposed regions by using the feature maps from a layer from a CNN that was previously trained (VGG or Resnet, for example). after identifying the regions, then crop-resize, then CNN over each region, then get output (class + bbox regression.)}

\r{faster r-cnn. Insert a region proposal network (RPN) to predict proposals from features --- in this architecture, there are actually four losses to jointly train 1) object/net RPN, 2) box/net RPN, 3) Class prediction, and 4) final bounding box score.}


% TODO: YOLO object detection
\TD{You Only Look Once: Unifie \cite{DBLP:journals/corr/RedmonDGF15}}
% TODO more object detection
\TD{Mask R-CNN \cite{DBLP:journals/corr/HeGDG17}}
\TD{Faster R-CNN: \cite{DBLP:journals/corr/RenHG015}}
\TD{SSD: \cite{DBLP:journals/corr/LiuAESR15}}

% 
\TD{Focal Loss (RetinaNet) \cite{DBLP:journals/corr/abs-1708-02002}}


\subsection{Segmentation}

\r{fully convolution --- dowsample to upsample: 1) efficency when using convs on smaller dimensional inputs 2) latent representation}

\r{CNN and RPN --- nice results \TD{cite}}

\r{mask R-CNN --- also used for pose prediction}

% code: https://github.com/facebookresearch/detectron2/tree/master/projects/PointRend
% blog: https://ai.facebook.com/blog/using-a-classical-rendering-technique-to-push-state-of-the-art-for-image-segmentation/
\TD{PointRend: Image Segmentation as Rendering \cite{Kirillov2019PointRendIS}}


\section{Text: Natural Language Processing}

\r{bag of words -- RNN -- RNN augmented (LSTM, GRU, etc) -- attention -- transformers }


\section{Structured data: Tablular}


\section{Other Common Architectures}




% chapter: model compression
\chapter{Model ``Compression'' or ``Distillation''}
% this section will likely be moved around
motivation

\TD{Distilling the Knowledge in a Neural Network \cite{Hinton2015DistillingTK}}

\begin{itemize}[noitemsep,topsep=0pt]
	\item Model (output) performance
	\begin{itemize}[noitemsep,topsep=0pt]
		\item regularization (through reduction of number of parameters)
	\end{itemize}
	\item Model performance
	\begin{itemize}[noitemsep,topsep=0pt]
		\item memory/storage --- by creating physically smaller networks
		\item energy --- computation, latency, which supports edge depolyments
	\end{itemize}
\end{itemize}



\section{Quantization}


\r{reduces the precision of parameters in a model}

\r{latency, memory}

\r{some fixed-point accellerators become available in edge settings}

\subsection{When}

\subsubsection{Initialization}

\TD{Pruning Neural Networks at Initialization: Why are We Missing the Mark? \cite{Frankle2020PruningNN}}

\TD{Robust Pruning at Initialization \cite{Hayou2021RobustPA}}

\subsubsection{during training}

\subsubsection{post-training}

\r{calibration data}

\section{Weight Pruning}

%TODO: https://github.com/he-y/Awesome-Pruning -- repo of recent pruning work

% intial skeleton influence by:
\TD{The initial structure and citations were heavily influenced by \cite{lange2020_lottery_ticket_hypothesis}}

\TD{SNIP: Single-shot Network Pruning based on Connection Sensitivity \cite{DBLP:journals/corr/abs-1810-02340}}

\r{reduces the overall number of parameters}

\r{in practice, this often refers to setting the parameters of a particular ``node'' to zero and making it untrainable during training.}

\TD{Comparing Rewinding and Fine-tuning in Neural Network Pruning \cite{Renda2020ComparingRA} --- Learning rate rewinding}

\TD{Deconstructing Lottery Tickets: Zero \cite{DBLP:journals/corr/abs-1905-01067} --- SuperMasks}

\TD{The Early Phase of Neural Network Training \cite{Frankle2020TheEP}}

% `` identify highly sparse trainable subnetworks at initialization, without ever training, or indeed without ever looking at the data''
\TD{Synaptic Flow Pruning (SynFlow) --- Pruning neural networks without any data by iteratively conserving synaptic flow \cite{Tanaka2020PruningNN}}

% code: https://github.com/varungohil/Generalizing-Lottery-Tickets
\TD{One ticket to win them all: generalizing lottery ticket initializations across datasets and optimizers \cite{Morcos2019OneTT}}



\subsubsection{What to remove}



\paragraph{relationship of nodes to nodes pruned}

\subparagraph{unstructured}

\r{no consideration for relationship between pruned weights}

\subparagraph{structured}

\r{prunes ``groups'' of weights.}

\paragraph{relationship of nodes to architecture}

\subparagraph{local}

\r{enforcment of a percent of weights pruned at each layer}

\subparagraph{global}

\r{total percent of weights are pruned, no restriction layer wise.}

\subsubsection{Deciding what to remove}

\r{common thought: large magnitude parameters are ``more important'' and thus should be removed less --- which is counterintuitive to concepts such as l2 regularization which penalize large magnitude parameters.}

\TD{other techniques}

\subsubsection{When to prune}

\begin{itemize}[noitemsep,topsep=0pt]
	\item Before training
	\item During training
	\item After training
\end{itemize}

\subsubsection{Pruning mechanism}

\begin{itemize}[noitemsep,topsep=0pt]
	\item One shot (prune all at once)
	\item Iterative
\end{itemize}

\section{Topology}

\r{more efficient model topology}

\subsection{Distillation}

% TODO: these are popular citations in the space, I haven't read many of them yet. putting them here for a start when I come back to the topic
\TD{Like What You Like: Knowledge Distill via Neuron Selectivity Transfer \cite{DBLP:journals/corr/HuangW17a}}
\TD{FitNets: Hints for Thin Deep Nets \cite{Romero2015FitNetsHF}}
\TD{Similarity-Preserving Knowledge Distillation \cite{DBLP:journals/corr/abs-1907-09682}}
\TD{Correlation Congruence for Knowledge Distillation \cite{DBLP:journals/corr/abs-1904-01802}}
\TD{A gift from knowledge distillation: Fast optimization, network minimization and transfer learning\cite{yim2017gift}}
\TD{Relational Knowledge Distillation \cite{DBLP:journals/corr/abs-1904-05068}}
\TD{Paraphrasing Complex Network: Network Compression via Factor Transfer \cite{DBLP:journals/corr/abs-1802-04977}}
\TD{Contrastive Representation Distillation \cite{DBLP:journals/corr/abs-1910-10699}}


\subsubsection{Student Teacher}

\TD{not sure if this belongs as its own section or not}

\TD{Knowledge Distillation and Student-Teacher Learning for Visual Intelligence: A Review and New Outlooks \cite{Wang2020KnowledgeDA}}

\TD{Self-Training with Weak Supervision~\cite{DBLP:journals/corr/abs-2104-05514}}



\paragraph{Knowledge Transfer}

% TODO: I'm unclear on the distinction here, is knowlege transfer the transfer of knowledge (like transfer learning but student/teacher?) and knowledge distillation the process of distilling that knowledge?

%% knowledge transfer -- but really distillation is kt? unclear here. need to read more
\TD{Probabilistic Knowledge Transfer for Deep Representation Learning \cite{DBLP:journals/corr/abs-1803-10837}}
\TD{Knowledge Transfer via Distillation of Activation Boundaries Formed by Hidden Neurons \cite{DBLP:journals/corr/abs-1811-03233}}
\TD{Variational Information Distillation for Knowledge Transfer \cite{DBLP:journals/corr/abs-1904-05835}}

\subsection{Tensor Decomposition}
% Not sure what this is -- from coursera


% TODO: this fits somewhere in this chapter, but not necessarily `here'
\subsection{The Lottery Ticket Hypothesis}

% RigL -- training sparse networks
\TD{Rigging the Lottery: Making All Tickets Winners \cite{Evci2019RiggingTL}}


\TD{Original paper --- The Lottery Ticket Hypothesis: Training Pruned Neural Networks \cite{DBLP:journals/corr/abs-1803-03635} --- iterative pruning. Dense network at initialization contains a number of ``winning tickets''}


\TD{Stabilizing the Lottery Ticket Hypothesis \cite{DBLP:journals/corr/abs-1903-01611}}

\TD{IMP with rewind}

% CODE: https://github.com/RICE-EIC/Early-Bird-Tickets
\TD{Drawing early-bird tickets: Towards more efficient training of deep networks \cite{You2020DrawingET}}

\TD{Playing the lottery with rewards and multiple languages: lottery tickets in RL and NLP \cite{Yu2020PlayingTL}}


\section{Sparsity}

% TODO: not sure where this section belongs quite yet

\TD{Training Neural Networks with Fixed Sparse Masks~\cite{DBLP:journals/corr/abs-2111-09839}}



% chapter: training dynamics
\input{./foundations/training_dynamics}

\input{./foundations/adversarial_examples}


\chapter{Generative}

\r{Nice resource: "Generative Deep Learning"\cite{foster2019generative} and \TD{An Introduction to Deep Generative Modeling \cite{DBLP:journals/corr/abs-2103-05180}}}

\section{Generative Adversarial Networks (GANs)}

\TD{Generative Adversarial Networks \cite{Goodfellow2014GenerativeAN}}

\TD{NIPS 2016 Tutorial: Generative Adversarial Networks \cite{DBLP:journals/corr/Goodfellow17}}

\TD{``Inverse PM -- semi-famous interaction, stemming from this review describing/inquiring about the relation to https://web.archive.org/web/20160411075236/http://media.nips.cc/nipsbooks/nipspapers/paper\_files/nips27/reviews/1384.html '' "predictability minimisation" or PM\cite{schmidhuber1992learning}}

% TODO: how as this not been done yet? I think I've written some pretty nice text on this before
\r{Discriminator and generator}

\r{Tries to learn the underlying structure of the data}

%TODO:

\TD{Self-Attention Generative Adversarial Networks \cite{Zhang2019SelfAttentionGA} uses attention\cite{DBLP:journals/corr/abs-1711-07971}}

\TD{Unsupervised Representation Learning with Deep Convolutional Generative Adversarial Networks \cite{Radford2015UnsupervisedRL}}

\TD{Mode collapse}

% TODO: wgan
\TD{Wasserstein GAN \cite{Arjovsky2017WassersteinG}}
\TD{Improved Training of Wasserstein GANs \cite{DBLP:journals/corr/GulrajaniAADC17}}
% TODO: DCGAN
\TD{Unsupervised Representation Learning with Deep Convolutional Generative Adversarial Networks \cite{Radford2016UnsupervisedRL}}

\section{Variational AutoEncoder}

% TODO: I've written on this somewhere.... 9Oct21

\TD{Auto-Encoding Variational Bayes \cite{Kingma2014AutoEncodingVB}}

\TD{Tutorial on Variational Autoencoders \cite{Doersch2016TutorialOV}}

% consistency in latent space w/without augementations in VAE
\TD{Consistency Regularization for Variational Auto-Encoders \cite{DBLP:journals/corr/abs-2105-14859}}
