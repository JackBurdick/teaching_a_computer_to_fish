\chapter{Foundational Methods}

%% Maybe (Foundational Methods --- supervised)

%%%%%%%%%%%%%%%%%%%%%%%%%%%%%% Regression
\section{Regression}



\subsection{Simple Linear Regression}

% C2 of Mastering ML
\textcolor{blue}{Model a \emph{linear} relationship between a response variable and a feature representing an explanatory variable. The relationship is modeled with a linear surface called a hyperplane (A subspace that consists of one dimension less than the dimensionality of the space it occupies e.g. a line in a 2D plot, or a 2D plane in a 3D environment).}

\textcolor{blue}{Simple linear regression consists of two total dimensions (a dimension for the response variable and another for the explanatory) -- the hyperplane, as explained above, has one dimension (line)}

\begin{equation}
{Y \approx \beta_0 + \beta_1 x}
\label{eq:slr_ex}
\end{equation}

\textcolor{blue}{$\approx$ can be read as ``\emph{is approximately modeled as}''. $Y$ is a quantitative response (output/prediction) and $X$ predictor variable(input/feature). $\beta_0$ and $\beta_1$ are two unknown constants representing the intercept and slope, respectively. These unknown values that determine the behavior of the model are known as the model \emph{parameters} or \emph{coefficients}}

\subsubsection{OLS}

\textcolor{blue}{{Ordinary Lease Squares (OLS)}\index{Ordinary Lease Squares (OLS)}, or {Linear Least Squares}\index{Linear Least Squares} is a method for estimating the parameters for a simple linear regression model.}

\textcolor{blue}{Solving OLS for simple linear regression ($y=\beta_0 + \beta_1 x$).}

\textcolor{blue}{First we'll solve for the slope $\beta_1$, where $\beta_1$ is can be found using Eq.\ref{eq:slr_ols_slope}.}

\begin{equation}
{\beta_1 =  \frac{cov(x,y)}{var(x)}}
\label{eq:slr_ols_slope}
\end{equation}

\textcolor{blue}{Variance (Eq.\ref{eq:variance_def}) is the measure of how far the set of values are spread apart -- if all the numbers in a set were equal, their variance would be zero.}

\begin{equation}
{var(x) = \frac{\sum_{i=1}^{n}(x_i - \hat{x})^2}{n-1}}
\label{eq:variance_def}
\end{equation}


\textcolor{blue}{Covariance (Eq.\ref{eq:covariance_def}) is the measure of how much two variable change together -- if two variables increase together, their covariance is positive}

\begin{equation}
{cov(x) = \frac{\sum_{i=1}^{n}(x_i - \hat{x})(y_i - \hat{y})}{n-1}}
\label{eq:covariance_def}
\end{equation}

\textcolor{blue}{After solving for $\beta_1$, $\beta_0$ can be found by rearranging the original equation and \textcolor{red}{substituting in the means of $x$ and $y$}($y=\beta_0 + \beta_1 X$) to become Eq.\ref{eq:slr_ols_intercept}}

\begin{equation}
{\beta_0 =  \bar{y} - \beta_1 \bar{x}}
\label{eq:slr_ols_intercept}
\end{equation}

\subsubsection{Cost}

\textcolor{blue}{Cost or loss function (See \textcolor{red}{local ref?}) is used to define and quantitatively measure the error of the model -- the differences between the predicted and ground truth values. The differences between the training is called the residuals\index{residuals} or training errors where as the differences observed between the test predictions and ground truths are called the prediction or test errors.}

\textcolor{blue}{A common measure of the models fitness may be the {residual sum of squares (RSS)}\index{residual sum of squares (RSS)} (Eq.\ref{eq:rss_def}, where $y_i$ is the observed value and $f(x_i)$ is the predicted value)}

\begin{equation}
{\sum_{i=1}^{n}{(y_i - f(x_i))^2}}
\label{eq:rss_def}
\end{equation}

% see p62 of ISL for more

\subsubsection{Evaluation}

\textcolor{blue}{Several methods exist for measuring the models predictive capability (see \textcolor{red}{local ref?} for more details.)}

\subsection{Multiple Linear Regression}

\textcolor{blue}{Using $n$ predictors:}

\begin{equation}
{Y \approx \beta_0 + \beta_1 X_1 + \beta_2 X_2 + \cdots + \beta_n X_n}
\label{eq:mlr_ex}
\end{equation}


\subsection{Polynomial Regression}

%%%%%%%%%%%%%%%%%%%%%%%%%%%%%% Logistic Regression
\section{Logistic Regression}

\textcolor{blue}{Despite the `regression' bit in the name, logistic regression is a classification model}

\textcolor{blue}{odds ratio\index{odds ratio} (Eq.~\ref{eq:odds_ratio}), where $p$ is representative of the probability of a positive (event we aim to predict) event.}

\begin{equation}
{\frac{p}{1-p}}
\label{eq:odds_ratio}
\end{equation}

\textcolor{blue}{A logit\index{logit} function (Eq.~\ref{eq:logit_def}) is the logarithm of the odds ratio (log-odds)}

\begin{equation}
{logit(p)=\log{\frac{p}{1-p}}}
\label{eq:logit_def}
\end{equation}

\textcolor{blue}{logistic function (sigmoid function) (Eq.~\ref{eq:sigmoid_def}) -- the inverse of a logit function and corresponds to the probability that a certain sample belongs to a particular class}

\begin{equation}
{S(x)={\frac{1}{1+e^{-x}}}={\frac{e^x}{e^x+1}}}
\label{eq:sigmoid_def}
\end{equation}


%%%%%%%%%%%%%%%%%%%%%%%%%%%%%% KNN
\section{Nearest Neighbors}
\label{nearest_neighbors}

\textcolor{blue}{KNN is a lazy learner\index{lazy learner} (a special case of instance-based nonparametric model (see \textcolor{red}{local ref?})): the model memorizes the training dataset rather than learn a discriminative function. Uses instances that are nearest the test instance to predict the value of the response variable.}

\textcolor{blue}{The intuition (and subsequently the assumption made by this method) behind nearest neighbors is that instances look alike must be alike.}

\textcolor{blue}{Can be used for classification (binary, multi-class, multi-label) or regression. To adapt a KNN for regression, the approach is modified to return a prediction of the average target value from the nearest neighbors, rather than the majority label.}

\textcolor{blue}{Due to the curse-of-dimensionality, In practice, nearest neighbors will usually be performed after a dimensionality reduction preprocessing step.}

\TD{collaborative filtering}

\TD{content-based filtering}

%\subsection{Use Examples}

%\textcolor{blue}{search}
%\textcolor{blue}{Recommender systems}

\textcolor{blue}{Simple algorithm that neighbors are representations of training instances in a metric space}

\subsection{Overview}


% page 188 of FofMLforpred data analytics

\textcolor{blue}{set local models (neighborhoods), where each is defined by a subset of the training data (in nearest neighbors, this is a single instance).}

\textcolor{blue}{A global prediction model based on the full dataset creates a decision boundary between regions of the features space which designates the \textcolor{red}{label?(term)}}

\subsubsection{Distance metric}

\textcolor{blue}{A distance metric (See \textcolor{red}{local ref?} for more information) is used to determine the closeness or the distance between instances.}

\subsection{K-Nearest Neighbors}

\textcolor{blue}{Since the nearest neighbor algorithm relies on local models, each defined by a single training instance, it is quite sensitive to noise. To address this issue, rather than rely on a single instance, a \textcolor{red}{label} is calculated from a set of $k$ nearest neighbors}

\textcolor{blue}{KNN classification involves i) choosing the number of $k$ (nearest neighbors) and a distance metric, ii) finding the $k$ nearest neighbors, and iii) assigning a class label by majority vote.}

\textcolor{blue}{The optimal value for $k$ will greatly influence the bias-variance trade-off. If the value is too low, there exists a risk of the algorithm being too sensitive to noise and overfitting, where as if the value of $k$ is too high, there is the possibility that a true pattern will be lost.}

\textcolor{blue}{$k$ is usually defined as an odd number in order to prevent ties.}

\subsubsection{Regression Considerations}

\textcolor{blue}{Feature values are associated with a real value instead of a label. The prediction is the mean or weighted mean of response variable of the $k$ nearest neighbors.}

\subsection{Considering Imbalanced Data}

\textcolor{blue}{If dealing with imbalanced data, as $k$ increases, the majority label will dominate the feature space.}

\subsubsection{Weighted K-Nearest Neighbors}


\subsubsection{Distance Weighted K-Nearest Neighbors}

\textcolor{blue}{The weight of each instance is a function of the inverse distance to the instance from the specified location. An easy way to implement this is to calculate the reciprocal of the squared distance (Eq.~\ref{eq:weighted_dist_knn}), where $n$ is the neighbor and $m$ is the specified location}

\begin{equation}
{\frac{1}{{dist(m,n)}^2}}
\label{eq:weighted_dist_knn}
\end{equation}

\textcolor{blue}{When calculating weight of each instance, $k$ is set to be a large value and may even be equal to the number of instances in the training set so that all training instances are included in the prediction process.}

\textcolor{blue}{Votes from neighbors that are close to the specified location are assigned a high weight, while distant neighbors are assigned a lower weight value.}

\subsection{Considerations}

% see page 225 of Understanding Machine Learning

\r{}

\subsubsection{Memory}

\subsection{Other Variations}

% see p196 of FofMLforpred data analytics for more information
\textcolor{blue}{k-d tree, k-dimmensional tree -- balanced binary tree}

\textcolor{blue}{R-Trees}

\textcolor{blue}{B-Trees}

\textcolor{blue}{M-Trees}

\textcolor{blue}{VoRTrees}

%%%%%%%%%%%%%%%%%%%%%%%%%%%%%% Support Vector Machines
\section{Support Vector Machines (SVM)}

\textcolor{blue}{Support Vector Machine (SVM)\index{Support Vector Machine (SVM)}. In order to minimize misclassification errors, the optimization objective is to maximize the margin (distance between the decision boundary (separating hyperplane) and the nearest training samples. These margins are called support vectors). Maximizing the margins, in theory, tend to have lower generalization error, where smaller margins may be more prone to overfitting.}

\textcolor{blue}{Can be used for classification and regression}

\textcolor{blue}{(Slack parameter?)}

\textcolor{blue}{Variable can be used to control the width of the margin and help tune the bias-variance trade-off.}

\textcolor{green}{TODO: figure showing difference in width of margins}

\textcolor{blue}{potential negatives -- don't scale particularly well: large datasets may present runtime and memory complexity issues, careful preprocessing and parameter tuning is important.}

\subsection{Maximizing Geometric Margin}

% p183[171] of mastering ml with skl
\textcolor{red}{``quadratic programming problem''}

\subsubsection{Sequential Minimal Optimization}

\textcolor{blue}{Sequential Minimal Optimization (SMO)\index{Sequential Minimal Optimization (SMO)} is an algorithm used find the parameters that maximize the geometric margin}

\subsection{Kernel SVM}

\textcolor{blue}{kernelized to solve nonlinear classification problems}

\subsubsection{The `Kernel Trick'}

\textcolor{green}{TODO: paras about the kernel trick}

\textcolor{blue}{see sec \textcolor{red}{local ref} for more information on the `kernel trick'}



%%%%%%%%%%%%%%%%%%%%%%%%%%%%%% Naive Bayes
%%%%%%%%%%%%%%%%%%%%%%%%%%%%%% Naive Bayes
\section{Naive Bayes}

\textcolor{blue}{typically performs well on small datasets. Compared to logistic regression it is more biased.}

\subsection{Bayes' Theorem}

\textcolor{blue}{Bayes' theorem is a formula (Eq.\ref{eq:bayes_def}) that is used for calculating the probability of an event using prior knowledge and related conditions.}

\begin{equation}
{P(A|B)=\frac{P(B|A)P(A)}{P(B)}}
\label{eq:bayes_def}
\end{equation}

\begin{equation}
{P(y|x_1,\dots,x_n)=\frac{P(x_1,\dots,x_n|y)P(y)}{P(x_1,\dots,x_n)}}
\label{eq:bayes_exp_def}
\end{equation}

% see p131(119) of Mastering ML with SKL
\textcolor{green}{TODO: more...}

% see p131(119) of Mastering ML with SKL
\textcolor{blue}{Variants: multinomial, Gaussian, and Bernoulli.}

\textcolor{blue}{The model is considered Naive because it assumes that all the features are conditionally independent given the response variable.}

\textcolor{blue}{NB assumes all training instances are {independent and identically distributed (i.i.d)}\index{independent and identically distributed (i.i.d)} -- training instances must be independent from each other and drawn from the same probability distribution.}

%%%%%%%%%%%%%%%%%%%%%%%%%%%%%% Decision Trees
%%%%%%%%%%%%%%%%%%%%%%%%%%%%%% Decision Trees
\section{Decision Trees}

\textcolor{blue}{\textcolor{green}{(TODO: revise this para!)} Decision trees make classification decisions based on a series of questions that separate the data into subsets. These questions are chained and result in a tree of questions/decisions where the leaves are considered pure i.e. they contain samples that belong to the same class.}

\textcolor{blue}{Minimum Description Length (MDL) - describes how ``deep'' the tree is allowed to grow.}

\textcolor{blue}{Importance of pruning -- Decision trees can be very deep and can easily lead to overfitting. To help prevent this situation, a limit is set for the maximal depth of a tree. }

\textcolor{blue}{The main advantage to using decision trees is that they are easily interpretable and may often resemble a program developed by hand.}

\subsection{Criterion -- Maximizing Information Gain}

\textcolor{blue}{Term - Information gain -- difference between the impurity of the parent node and the sum of the child node impurities -- the lower the impurity of the child nodes compared to the parent node, the higher the information gain}

\textcolor{blue}{Three commonly used splitting criteria used in binary decision trees: (i) Gini Impurity, (ii) entropy, and (iii) classification error}

\subsubsection{Gini Impurity}

\subsubsection{Entropy}

\subsubsection{Classification Error}

\subsubsection{ID3 Algorithm}

\textcolor{blue}{ID3 or ``Iterative Dichotomizer 3''}


\textcolor{blue}{top-down, recursive, depth-first partitioning of the dataset.}

\textcolor{blue}{Assumes categorical features without any missing values.}

\textcolor{red}{Can be extended to handle continuous descriptive features and continuous target features}

\subsubsection{C4.5 Algorithm}

\textcolor{blue}{Variant of the {ID3 algorithm}\index{ID3 algorithm} that can handle continuous categorical descriptive features and missing features}

\textcolor{red}{uses post-pruning}

\textcolor{blue}{{J48}\index{J48} is an open source implementation of the C4.5 algorithm}

\subsubsection{CART Algorithm}

\textcolor{blue}{The CART algorithm is another variant of the ID3 algorithm.}

\textcolor{blue}{Uses the Gini index}

\textcolor{blue}{Can handle continuous target features}

\subsection{Pruning}

\textcolor{blue}{Can help address overfitting. Pruning is an attempt to reduce the size of the tree while still maintaining a similar empirical error.}

\subsubsection{Pre-pruning}

\subsubsection{Post-pruning}

\section{Random Forests}

% see Breiman, 2001
\textcolor{blue}{Ensemble method. Combine various decision trees, where some may be weak learners\index{weak learner} (\textcolor{green}{def}) and some may be strong learners\index{strong learner} (\textcolor{green}{def}). The final classification will be determined by majority vote from the number of trees.}






\chapter{Artificial Neural Networks}

\textcolor{blue}{If a perceptron is analogous to a single neuron, an artificial neural network (either feedforward or feedback) would be analogous to a brain.}

\r{powerful and general framework for representing non-linear mappings (function approximation) from input features to outputs, where the form of the mapping is controlled by adjustable parameters (weights and biases). Determining the values for these parameters is the ``learning'' or training.}

%%%%%%%%%%%%%%%%%%%%%%%%%%%%%% perceptron
\section{Perceptron}

\textcolor{green}{TODO: overview}

\textcolor{green}{Obligatory Biology figure}

\subsection{History}

\textcolor{blue}{McCullock and Pitts -- nerve cell as a simple logic gate with binary outputs.  Where multiple input signals are accumulated and if they exceed a certain threshold an output signal is generated}

\r{$x_i$ represent the input (maybe raw features or output from previous neurons), weights $w_i$ represetn the strengths of the interconnections (analygous to synapses) between the neurons, and the bias $w_0$ represent the threshold for a neuron to be activated.}

\textcolor{blue}{Frank Rosenblatt at Cornell Aeonautical Laboratory in late 1950s -- motivated by efforts to simulate the human brain -- the perceptron: an algorithm that would learn the optimal weight coefficients to the input features in order to determine whether an output signal should be produced. The early activation function \textcolor{red}{See more in local ref?} was a simple unit step function.}

\textcolor{blue}{Adaline (\textbf{ADA}ptive \textbf{LI}near \textbf{NE}uron) \textcolor{green}{(TODO: WIdrow and Hoff (1960))} -- rather than the weights being updated based on a unit step function like the perceptron, the weights are updated based on a linear activation function. A continuous output value (rather than discrete) is used to compute the model error and update the weights.}

\textcolor{blue}{Linear activation function is used for weight updates but a unit step function can still be used to predict the class labels.\textcolor{red}{TODO: figure showing this}}

\subsection{Overview}

\textcolor{blue}{advantage: perceptrons are capable of online learning --- the models parameters can be updated on a single training instance rather than a batch of instances}

\textcolor{blue}{Though not used too frequently in practice, the perceptron is a building block that later sections in this chapter on artificial neural networks are built upon.}

\subsection{Activation Function Basics}

\textcolor{blue}{Basic activation function (Eq.\ref{eq:act_func_basic_def}) where $w$ represents the model's parameters and $b$ represents a bias terms and $act$ is representative of the activation function.}

\begin{equation}
{y = act (\sum_{i=1}^{n}(w_i x_i + b)}
\label{eq:act_func_basic_def}
\end{equation}

\textcolor{blue}{There are many different types of activation functions that may be used. A more in-depth discussion on activation functions is discussed in \textcolor{red}{local ref?}}

\textcolor{blue}{The linear combination of parameters and inputs may sometimes be referred to as the preactivation}

\textcolor{blue}{Rosenblatt's original activation function is the {Heaviside step function}\index{Heaviside step function} ({unit step function}\index{unit step function}), show in Eq.\ref{eq:heaviside_step_func}] \textcolor{green}{TODO: figure}}

\begin{equation}
{out(x) = \left\{
	\begin{array}{ll}
	1 & \quad $when $ x \geq 0 \\
	0 & \quad x < 0
	\end{array}
	\right.}
\label{eq:heaviside_step_func}
\end{equation}

% see p164[152] of Mastering ML w/SKL *para*


\subsection{Limitations}

\textcolor{blue}{}

\textcolor{blue}{nonlinear activation function is necessary to break linearity since computing a series of weighted sums is equivalent to computing a single weighted sum}

\r{around the 1960s --- shown that they could solve a number of problems readily, while unable to solve other problems (that superfically appeared to be a similar level of difficulty). \TD{Perceptrons (book), Minsky and Papert 1969}}

\r{can only classify data sets by a linear hyperplane. However, it is possible to solve linearly inseparable data, provided the data can first be preprocessed. The difficulty lies in the fact that the (pre)processing elemetns are fixed in advance and cannot adapt.}

\subsection{Extending to model linearly inseparable data}

\subsubsection{Kernelization}

\textcolor{blue}{projecting linearly inseparable data into a higher dimensional space (with the intent to make the data linearly separable)}

\subsubsection{Directed Graph}

\textcolor{blue}{Artificial neural network (discussed next, in \textcolor{red}{local ref?}) is a universal function approximator make of a directed graph of perceptrons}

\subsection{Notes}

\textcolor{blue}{If the activation function is a logistic sigmoid activation function, the model is the same as logistic regression.  The difference, however, is that the perceptron can be trained with an online, error driven, approach}

%%%%%%%%%%%%%%%%%%%%%%%%%%%%%% overview
\section{Artificial Neural Networks (ANN)}

\textcolor{blue}{Principals and basic feed forward networks}

\textcolor{blue}{The most computationally expensive component is calculating the gradient of the loss function with respect to the parameters of the network}

% see page 233 of Understanding Machine Learning
\textcolor{blue}{Artificial neural networks are {universal approximators}\index{universal approximators} -- \textcolor{red}{expand}}

\textcolor{blue}{universal approximation theorem \textcolor{green}{(Hornik 1989, Cybenko, 1989)}. Regardless of the function that is attempted to being learned, a large MLP will be able to \textbf{represent} this function. However, it is not guaranteed that the large MLP, despite being a universal \textcolor{red}{approximator} capable of representing the function, is able to \textit{learn} the function}

\subsection{Multi-layer Perceptron}

\textcolor{blue}{Not a single multi-layer perceptron with multiple layers, rather it is a network composed of multiple layers of perceptrons.}

\subsection{Architecture}

\textcolor{blue}{{input layer}\index{input layer}, {hidden layer}\index{hidden layer}, {output layer}\index{output layer}}

\textcolor{blue}{the input layer is not counted in the number of layers in a network}

\textcolor{green}{TODO: diagram of neural network showing layers}

\subsection{Components}

\textcolor{green}{TODO: labeled diagram of nodes (weights and biases), connections, activation functions}

\subsubsection{Nodes / units}

\paragraph{Initialization}

\textcolor{green}{TODO: initialization methods and for different layers}



\subsubsection{Activation Function}

\textcolor{green}{TODO: I think this is where I'll talk about activation functions}

\textcolor{green}{TODO: explain the need for non-linearity.}

\textcolor{green}{TODO: step function to sigmoid function -- smoothed version of the step function -- can understand how an input changes the output.}


\subsection{Characterization}

\subsubsection{Types: Feed-forward vs Feedback}

\textcolor{blue}{Feed-forward --- Directed acyclic graph of artificial neurons. Feedback contain feedback connections that are fed back into itself. When feedforward are include these feedback connections, they become considered recurrent neural networks.}

\subsubsection{Terminology}

\textcolor{blue}{Considered \textit{networks} since they are typically composed of many different functions --- creating a ``network''.}

\textcolor{blue}{Considered \textit{neural} since they are \textbf{loosely} inspired by neuroscience.}

\textcolor{blue}{layer --- a layer may be considered a group of units that act in parallel. The layer will extract representations from the input, that are (in theory) more useful to the specific task.  Chaining together these layers results in a form of progressive {data distillation}\index{data distillation}.}

\textcolor{blue}{Visible and Hidden Layers. Visible layers are called visible since they contain variables that are ``visible'', where as the hidden layers extract increasingly abstract features -- hidden since their values are not given in the raw data, but rather an output from a previous layer.}



\subsection{Learning: Backpropagation}

% see p196[184] of Mastering ML w/SKL
\textcolor{green}{TODO: whoooo, this is going to be a big one. understand how each component contributes to the error and adjust accordingly.}

\textcolor{blue}{Iterative algorithm consisting of two main components --- in order, the forward, then the reverse pass}

\textcolor{blue}{In the forward pass inputs are propagated through the network}

\textcolor{blue}{In the Backward pass, errors from the forward function and  are propagated in reverse through the network (from cost function to input layer) and each node is updated -- \textcolor{green}{TODO: expand}.}

\subsubsection{Backward Propagation}

% see p197-201[180] of Mastering ML w/SKL

\textcolor{green}{TODO: figure showing sample calculation}


% See p207 of DL

\textcolor{blue}{also sometimes called {reverse-mode differentiation}\index{reverse-mode differentiation}.  Calculate the contribution that each parameter had on the loss value}

\textcolor{blue}{{symbolic differentiation}\index{symbolic differentiation} --- compute a gradient function for the chain (chain rule) mapping parameter values to gradient values}

\subsubsection{Chain Rule}

\textcolor{red}{See \textcolor{red}{local ref to math prereq section}}

\textcolor{green}{TODO: chain rule}

\textcolor{blue}{Backpropagation is typically used with an optimization algorithm (see \textcolor{red}{local ref?})}

\subsection{Multi-layer perceptrons}

%%%%%%%%%%%%%%%%%%%%%%%%%%%%%% feedforward
\section{Artificial Neural Networks (ANN)}

\textcolor{blue}{Will be focusing on principals and basic feed forward networks}


\textcolor{blue}{A feedforward network can be represented as a directed acyclic graph -- a directed graph in which there exist no cycles in the underlying graph, with nodes representing neurons and connected by edges}

\textcolor{blue}{The most computationally expensive component is calculating the gradient of the loss function with respect to the parameters of the network}

% see page 233 of Understanding Machine Learning
\textcolor{blue}{Artificial neural networks are {universal approximators}\index{universal approximators} -- \textcolor{red}{expand}}

\subsection{Types: Feedback vs Feed-forward}

\subsubsection{Feed-forward}

\textcolor{blue}{Directed acyclic graph of artificial neurons.}

\subsubsection{Feedback}

\subsection{Learning: Backpropagation}

\subsection{Multi-layer perceptrons}

%%%%%%%%%%%%%%%%%%%%%%%%%%%%%% feedback
\section{Feedback or Recurrent}

\textcolor{green}{TODO: Overview}


\chapter{Unsupervised}

\r{The reality is that data is not always labeled.}

\r{unsupervised learning is capable of finding hidden patterns in the underlying structure of hte data.}

\r{attempts to represent data with increaingly fewer parameters}

\textcolor{blue}{Discovering hidden structures or patterns in unlabeled training data.}

\TD{Neighborhood-Based Methods \ref{nearest_neighbors} \r{lazy learners  -- learn how to label new instances based on proximity to existing instances}}

% TODO: placement / may need to rename+restructure sections
\textcolor{blue}{unsupervised methods may be commonly used in two main settings:}
\begin{enumerate}[noitemsep,topsep=0pt]
	\item Data Exploration
	\begin{itemize}[noitemsep,topsep=0pt]
		\item Visualization (clustering \textcolor{red}{local ref})
	\end{itemize}
	\item Preprocessing (e.g. prior to a supervised method): unsupervised pretraining may be considered a form of regularization
	\begin{itemize}[noitemsep,topsep=0pt]
		\item Compressing (dimensionality reduction \textcolor{red}{local ref})
		\item Creating new/different representations
	\end{itemize}
\end{enumerate}

r{regularization, feature engineering, detecting outliers -- also used for detecting how different new (incoming) training data is from the current distribution.}

%%%%%%%%%%%%%%%%%%%%%%%%%%%%%% clustering
\section{Clustering}

\textcolor{green}{todo: Overview}

\subsection{K-means}

\textcolor{green}{todo: Overview}

\subsubsection{Local Optima}

\textcolor{blue}{unlucky initialization}

\subsection{Selecting K}

\textcolor{green}{todo: Overview}

\subsubsection{Elbow Method}

\subsection{Evaluating}

\textcolor{blue}{There are no labels - can still evaluate using intrinsic measures}

\subsubsection{Silhouette Coefficient}

%%%%%%%%%%%%%%%%%%%%%%%%%%%%%% Dimensionality Reduction
\section{Dimensionality Reduction}

\r{Dimensionality reduction is typically used to reduce the dimensions in a feature representation while retaining as much information as possible.}

\r{Motivation: i) mitigate issues caused by the curse of dimensionality ii) compress data iii) visualize and explore datasets and improve interpretability -- interpreting data in high dimensions is harder than in lower dimension spaces (particularly three or less). May also be used to compress data before being used by another learning algorithm.}

\r{Example --- projecting 3D data into a 2D space.}

\r{project the raw high-dimensuional input data into  a lower-dimensional space by only \TD{iteratively removing the least explainatory dimensions.}}

\TD{Two major branches of dimensionality --- linear and nonlinear}

\subsection{Principal Component Analysis}

\textcolor{green}{{Principal Component Analysis (PCA)}\index{Principal Component Analysis (PCA)} may also be known as the {Karhunen-Love Transform (KLM)}\index{Karhunen-Love Transform (KLM)} }

%p232[220] of Masstering ML w/SKL
\textcolor{red}{``PCA is most useful when'the variance of the dataset is distributed unevenly across the dimensions'' }

% TODO: get var/covar defs (currently in regression tex)
\textcolor{blue}{carvariances between each par of dimensions in a dataset are described in a {covarience matrix}\index{covarience matrix} }

\r{retains as much of the variation as possible (some is lost)}

\TD{several variants --- incremental PCA, nonlinear (kernel PCA), sparse (sparse PCA)}

\subsection{TODO: others}
\TD{Singular value decomposition SVD}

\TD{Random Projection}

%%%% % plus index - unsupervised method
% see p156 of DL for more
\TD{manifold learning (or nonlinear dimensionality reduction)--- {manifold}\index{manifold}, though having a more formal mathematical meaning, will be considered a connected region for our machine learning purposes. --- Nonlinear transormation. PCA and random projection project the data linearly from high to low dimension.}

\TD{isomap -- type of manifold learning. estimates the geodesic or curved distance (rather than euclidean) between a point and its neighbors}

\TD{t-distributed stochastic neighbor embedding (t-SNE) --- non-linear --- }

\TD{dictionary learning --- learning sparse representation (binary vectors) --- easily identify vectors with the most nonzero values}

% TODO: find a good text on this -- this should be an in depth section
% TODO: dataset for individual voices ina coffeehouse
\TD{independent component analysis (ICA) --- separate blended signals into individual components (signal processing)}

\TD{Latent Dirichlet allocation}


\chapter{Semi-supervised}

\section{Semi-Supervised}

\r{Semi-Supervised Learning (SSL)}

\TD{TODO: overview}

\subsection{Examples}

\TD{TODO: Examples}


\subsection{TODO}

% TODO:
% looks promising: https://towardsdatascience.com/pseudo-labeling-to-deal-with-small-datasets-what-why-how-fd6f903213af
% https://www.kaggle.com/cdeotte/pseudo-labeling-qda-0-969
\TD{Psuedo-Label \cite{lee2013pseudo} -- summary: use labeled data to create labels for unlabeled data and then retrain}
\r{$loss_{batch} = loss_{label} + weight * loss_{unlabel}$ }
