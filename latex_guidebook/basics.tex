\chapter{Basics}

\section{Overview}

%%%%%%%%%%%%%%%%%%%%%%%% obligatory "No Free Lunch"
I'm not sure it's possible to discuss machine learning without at least mentioning the ``No Free Lunch'' theorem, which states ``No single classifier works best across all possible scenarios''

%%%%%%%%%%%%%%%%%%%%%%%% Types of data
\textcolor{blue}{categorical or numerical. Numerical can be discrete or continuous}

%%%%%%%%%%%%%%%%%%%%%%%% Measurement Levels
\textcolor{blue}{qualitative or quantitative.} 

\textcolor{blue}{Qualitative can be nominal (aren't numbers and can't be put in any order -- e.g. the seasons: spring, summer, fall, winter) or ordinal (groups and categories that follow a strict order -- e.g. difficult levels: hard, medium, or easy)}

\textcolor{blue}{Quantitative are represented by numbers but can be interval (0 is meaningless -- e.g. temperature in C or F, where true zero is not 0) or ratio (has a true 0 -- e.g. temperature in K, weight or length)}


%%%%%%%%%%%%%%%%%%%%%%%% Acquiring Data
\section{Data Acquisition}


\subsection{Resources}

\subsection{Generating Fake Data}

\textcolor{green}{TODO: generating fake data with SKL}

\textcolor{blue}{make\_blobs}

% {{{datagen_blobs_2dcode}}}
\begin{lstlisting}[style=pyInStyle]
X, y = datagen.make_blobs(centers=4, n_samples=100, n_features=2,cluster_std=1.0,
                          center_box=(-10, 10),
                          random_state=42, shuffle=True)
\end{lstlisting}

% {{{datagen_blobs_2dimg}}}
\begin{figure}
\centering
\includegraphics[width=0.65\textwidth]{./sync_imgs/datagen/blobs/2dimg.png}
\label{fig:datagen_blobs_2dimg}
\end{figure}

\textcolor{blue}{Data can also be generated in three (multiple) dimensions}

% {{{datagen_blobs_3dcode}}}
\begin{lstlisting}[style=pyInStyle]
X, y = datagen.make_blobs(centers=4, n_samples=100, n_features=3, random_state=42)
\end{lstlisting}

% {{{datagen_blobs_3dimg}}}
\begin{figure}
\centering
\includegraphics[width=0.65\textwidth]{./sync_imgs/datagen/blobs/3dimg.png}
\label{fig:datagen_blobs_3dimg}
\end{figure}

%\textcolor{blue}{More dataset types can be generated, the documentation can be found at (http://scikit-learn.org/stable/modules/classes.html#module-sklearn.datasets)}


\textcolor{blue}{see \textcolor{red}{local ref?} for more examples on how to generate data}



%%%%%%%%%%%%%%%%%%%%%%%% Data Pre-processing
\section{Data Pre-processing}

\textcolor{blue}{Data is rarely obtained in a form that is necessary for optimal performance of a learning algorithm. Data can be missing, can contain a mix of categorical and quantitative, can contain values on vastly different scales, etc.}

\textcolor{blue}{It is important to note that any parameters related to data pre-processing, such as feature scaling and dimensionality reduction, are obtained solely from observing the training set. The parameters for these methods obtained on the training set are then later applied to the test set. This is important since if these preprocessing parameters were obtained on the entire dataset and included the test set, the the model performance may be overoptimistic since then when applying the methods to the unseen data.}

\subsection{Handling Missing Data}

\subsubsection{Filtering Out}

\textcolor{blue}{Simply removing any entries that are missing data. This is convenient and easy but may not be practical -- any time data is being removed, potentially useful information is lost and too much data may be removed.}

\textcolor{green}{TODO: Code in jupyter on how to do this with pandas and dropna -- key params - how, thresh, subset}

\subsubsection{Filling In}

\textcolor{blue}{Estimating the missing data}

\subsection{Handling Categorical Data}

\subsubsection{Encoding}

\subsection{Feature Scaling, Normalization}

\subsubsection{Min-Max scaling (Normalization)}

\textcolor{blue}{values are shifted and rescaled so they end up on a [0,1] range}

\subsubsection{Standardization}

\textcolor{blue}{(Eq.~\ref{eq:preprocess_standardization}) first, subtract the sample mean, then divide by standard deviation variance}

\textcolor{blue}{pros: unlike min-max, not bound to specific range}

\textcolor{blue}{standardized values always have a zero mean and a standard deviation of 1.}

\textcolor{blue}{gives our data the property of a standard normal distribution}

\begin{equation}
{X' = \frac{X - \mu}{\sigma}}
\label{eq:preprocess_standardization}
\end{equation}

\textcolor{green}{TODO: create code sample - numpy, and sklearn methods}


\subsection{Others}

\subsubsection{Removing Duplicates}

\subsubsection{Outliers}

\subsubsection{Discretization and Binning}


%%%%%%%%%%%%%%%%%%%%%%%% Data Type Considerations + Feature Extraction
\section{Feature Extraction from Various Datatypes}

\textcolor{green}{TODO: Feature Extraction}

\subsection{Feature Engineering}

\textcolor{blue}{acquisition and/or systematic improvement of features}

\textcolor{blue}{TODO: features are learned,not engineered in deep learning models}

% TODO: placement and naming
\subsubsection{Kernel}

\textcolor{blue}{the following can't be separated linearly as is.}

% {{{kernelized_2class4clust_2dimg}}}
\begin{figure}[h]
\centering
\includegraphics[width=0.65\textwidth]{./sync_imgs/kernelized/2class4clust/2dimg.png}
\label{fig:kernelized_2class4clust_2dimg}
\end{figure}

\textcolor{blue}{but what if we produce a new feature (feature 1 ** 2)}

% {{{kernelized_2class4clust_3dimg}}}
\begin{figure}
\centering
\includegraphics[width=0.65\textwidth]{./sync_imgs/kernelized/2class4clust/3dimg.png}
\label{fig:kernelized_2class4clust_3dimg}
\end{figure}


\subsubsection{Feature Crosses}

% TODO: figure of linearly separable dataset and or a xor dataset
% makes some non-linear problems (xor) linear

\textcolor{blue}{combine two or more categorical features. Feature crossing is only possible when working with categorical features. When working with continuous features, the values can be discretized prior to the feature cross}


% how much to translate feature 1 and 2 are parameters that need to be learned
% "discreteize" the input space
% feature crosses "memorize" -- "Memorization works when LOTS of data for a 
% single cell in the input space & the distribution of data is statistically significant."
% not used as often in traditional ML, but powerful on large datasets
% Rome/New york yellow/white Taxi example

\textcolor{blue}{number of inputs.}
% example of 24hrs a day, 7 days a week = 168 inputs in a feature cross of the two.
% TF uses a sparse representation for inputs to address this (one hot encoding and feature crosses)
% input will only activate one input at a time, thus the input is very sparse

\textcolor{blue}{It is possible that the feature cross may cause the model to overfit the data.}

% TODO: show example of this happening -- X1,X2 (2 blobs ) = good, X1X1,X2X2,X1X2 = overfit

\textcolor{blue}{it is possible to look at the relative weights for the inputs and determine how much each feature is contributing to the decision. This can help determine if maybe the features cross isn't necessary -- L1 regularization (see \textcolor{red}{local ref}) may work to zero out this feature as well.}

% TODO: implementation details and choosing the number of hashbuckets.
% if too small, there could be collisions, "rule of thumb" 1/2sqrt(N) and 2N
% trade off is memorization vs sparsity

% adding an embedded layer (real values, learned)
% learns how to ``embedd'' the feature cross in a lower dimesnsional space
% the features learned in embedded features may be useful to other problems from a seperate/maybe related domain
% using learned embeddings in one city for another city on the same types of inputs

\subsection{Images}

\textcolor{green}{TODO: Images}


\subsubsection{Video}

\textcolor{green}{TODO: Video}


\subsection{Natural Language}

\textcolor{green}{TODO: Natural Language}

\subsubsection{Terminology}

\textcolor{blue}{A {corpus}\index{corpus} is a collection of documents. {vocabulary}\index{vocabulary} is a corpus's unique words}

\subsubsection{Pre-processing}

\textcolor{green}{TODO: Pre-processing}

\textcolor{blue}{converting all letters to lowercase}

\textcolor{blue}{stemming and lemmatization --- Condensing word forms (derived and inflected) into a single feature. These methods are used to reduce the dimensionality of the features space.}

\paragraph{Stop Word Filtering}

\textcolor{blue}{todo: removing words that are common throughout the language as well as potentially to most of the documents in a corpus. Typically stop words do not convey meaning through their meaning, but rather through their grammatical meaning.}

\paragraph{Tokenization}

\textcolor{blue}{Tokenization is the process of splitting and grouping characters together into meaningful sequences. \textcolor{red}{If a document is tokenized, the result is a set of tokens (words).} Tokens are not limited to words however, and may also be shorter sequences like punctuation characters and affixes.}

\textcolor{green}{TODO: Tokenization example}

\paragraph{Lemmatization}

\textcolor{green}{TODO: Lemmatization. converting words into their base form --- determining the lemma (morphological root) of an inflected word.}

\paragraph{Stemming}

\textcolor{green}{TODO: Stemming. There exist many stemming algorithms. Stemming removes all character patterns that appear to be affixes to a word. Note: the resulting word may or may not be a valid word e.g. \textcolor{red}{XXXXXXXX}.}

\subparagraph{Porter Stemming}

\subsubsection{Encoding}

\paragraph{Encoding Methods}

\subparagraph{Bag-of-Words}

\textcolor{blue}{{bag-of-words}\index{bag-of-words} similar to one-hot-encoding, it encodes words that appear in text as one feature for each word of interest. Does not encode any other information like syntax, grammar, or order of the words.}

\textcolor{blue}{Bag-of-Words encodes the corpus's vocabulary as a feature vector to represent each document. The intuition for using bag-of-words is that documents that contain similar words are likely to be similar to one another.}


\paragraph{tf-idf}

\textcolor{green}{TODO: tf-idf\index{tf-idf} (Eq.\ref{eq:tf_idf_def}) Inverse Document Frequency is a measure of how common/rare a term is in a corpus --- explain importance}

\begin{equation}
{log\frac{N}{1|XXXXXXXXTODOXXXXXXXXXX|}}
\label{eq:tf_idf_def}
\end{equation}

\subsubsection{Embedding}

% TODO: this section may need to be promoted

% an embedding can be created for any categorical column

% ``embeddings cab be thought of as latent features''

% good starting point for number of dimmensions may be cube root of the possible values

\textcolor{blue}{Embeddings are }

\subparagraph{glove}

\textcolor{green}{TODO: glove}

\subparagraph{word2vec}

\textcolor{green}{TODO: word2vec}

\subsubsection{Other Notes}

% 'hashing trick' --- see p59 of Mastering ML with SKL

\subsection{Audio}


\textcolor{green}{TODO: Audio}



%%%%%%%%%%%%%%%%%%%%%%%% Data sampling and partitioning
\section{Partitioning Data}

\subsection{Sampling}

\textcolor{blue}{Training, validation, test}

\section{Some Terms}

\emph{input variable(s)} -- predictors, independent variables, features, regressors, controlled variables, exposure variables or simply variables.
 
\emph{output variable(s)} -- response or dependent variable. May also be known as regressands, criterion variables, measured variables, responding variables, explained variables, outcome variables, experimental variables, labels.

\textcolor{blue}{Both input and output variables may take on continuous or discrete values.}

\emph{relationship} $Y = f(x) + \epsilon$ \textcolor{blue}{estimate $f$. prediction and inference}.

\textcolor{blue}{\emph{reducible error\index{reducible error}} -- the estimated function $\hat{f}$ will likely not be perfect, and the reducible error is the error that could be corrected.  The \emph{irreducible error} is an error that can not be corrected. The irreducible error may be larger than zero due to \emph{unmeasured variables} \emph{e.g.} varibles that were not measured and \emph{unmeasurable variation} \emph{e.g.} an individual's feelings/emotions or variation in the production of a product. The irreducible error provides an upper bound on the performance of the predicted $\hat{f}$}

\section{Type of Learning}

\textcolor{blue}{Three main types of machine learning: supervised, unsupervised, and reinforcement.}

% From ML for Predictive Data Analytics
\textcolor{blue}{Another way to group types of learning -- Information-based, similarity-based, probability-based, and error-based}

\subsection{Supervised}

\textcolor{blue}{observe input variables with corresponding output values. A program that predicts an output for a in input by learning from pairs of labeled inputs and outputs. Classification \textcolor{red}{ref} and regression \textcolor{red}{ref} are subcategories of supervised learning}

\subsection{Unsupervised}

\textcolor{blue}{observe input variables without corresponding output values and attempts to discover patterns in the data.}

% p14 of mastering ml agorithms
\textcolor{blue}{There is no error signal to measure, rather, performance metrics report some attribute of structure discovered in the data, such as the distances within and between clusters.}
 
% clustering
\subsubsection{Clustering}

\textcolor{blue}{Finding sub groups where observations are more similar to eachother based on some similarity measure. Clustering is sometimes referred to as ``unsupervised classification'' and is often used to explore a dataset.}

\textcolor{blue}{An example of clustering may be to group a collection of documents into categories, or songs into genres.}

% principal components
\subsubsection{Dimensionality Reduction}

\textcolor{blue}{where the goal is to reduce the dimensionality of the data while retaining as much as the relevant information as possible}

\textcolor{blue}{A high number of features may be computationally costly. Ability to generalize may be reduced if some of the features capture noise or are irrelevant to the underlying relationship. The goal could be to find the features that account for the greatest changes in the response variable}


\subsection{Semi-supervised Learning}

\textcolor{blue}{`semi-supervised learning', another type of learning, makes use of both supervised and unsupervised data.}

\subsection{Reinforcement}

\textcolor{blue}{Reinforcement learning does not learn from labeled pairs of inputs and outputs, rather it learns from `feedback' from decisions that are not explicitly corrected.}

\textcolor{blue}{Goal -- develop an \emph{agent} that improves it's performance based on interactions with an \emph{environment} based on a \emph{reward}}

\subsection{Supervised vs Unsupervised}

\subsection{Classification vs Regression}

\subsubsection{Regression} 

Regression, also called regression analysis \textcolor{red}{local ref?} involves predicting a continuous or quantitative output value. For example attempting to find a relationship between a given predictor/explainatory variables (age, job title, zip code) and a continuous response (an individuals outcome).

\subsubsection{Classification} 

Classification involves predicting categorical (discrete) or qualitative output value (such as a non-numerical value). 

\textcolor{blue}{Binary classification (benign vs malignant) and multi-class classification (identifying many different skin diseases).}

\subsection{Multi-label classification}
\textcolor{blue}{TODO: {multi-label classification}\index{multi-label classification} --- where a classifier assigns multiple labels to each instance}


\subsubsection{Approaches: Problem transformation}

\textcolor{blue}{There are two main approaches to multi-label classification}

\textcolor{blue}{{Problem transformation}\index{Problem transformation} modify the original multi-label problem to a set of single-label classification problems.}

\paragraph{Unique set/combination of labels}

\textcolor{green}{TODO: table and example.}

\textcolor{blue}{Two main concerns with this methodology: i) increasing the number of classes is impractical and will often have very few instances and ii) the classifier can only predict combinations that were seen in the training data.}

\paragraph{Many Binary Classifiers}

% p124[112] of Mastering ML with SKL
\textcolor{blue}{Train a classifier for each label in the training set. The final prediction is the combination of all the predictions from the binary classifiers.}

\textcolor{blue}{The main concern with this approach is that the relationships between labels is ignored.}

\subsubsection{Evaluating Multi-label Classification}

\textcolor{blue}{see \textcolor{red}{local ref}.}

% page 94 of AGtext
One-versus-all \emph{OvA} (also \emph{one-versus-rest}) -- 

One-versus-one (OvO) -- train a binary classifier for every pair


\textcolor{blue}{Binary classification can be extended to multi-class classification via the OvR method.}

%\subsubsection{Bayes Classifier}

\section{Training}

\textcolor{red}{Cost and Loss functions -- I'm not sure why I didn't have this yet?}

\textcolor{red}{example of plots - step by step}

\textcolor{red}{contour maps}

\section{Quality of Fit}

%% regression example

\subsection{Regression Example}

\textcolor{blue}{Mean Squared Error.$\hat{f}(x_i)$ is the prediction that $\hat{f}$ produces for the $i$th sample. The output will be small for predicted values that are similar to the ground truth}

\begin{equation}
{MSE = \frac{1}{n}\sum_{i=1}^{n}(y_i - \hat{f}(x_i))^2}
\label{eq:MSE_def}
\end{equation}

%% classification example

\subsection{Classification Example}

\textcolor{blue}{The proportion of mistakes that are made.}

\begin{equation}
{error\_rate = \frac{1}{n}\sum_{i=1}^{n}(y_i \ne \hat{y_i})}
\label{eq:class_error_rate_def}
\end{equation}

\textcolor{blue}{$\hat{y_i}$ is the predicted classification label for the $i$th observation using our predictor/model $\hat{f}$ and $y_i$ is the ground truth label}

\section{Describing Learners}

\subsection{Parametric vs non-parametric}

\subsubsection{parametric}

\textcolor{blue}{parametric models are models that learn a fixed number of parameters, independent from the number of training instances, that able to classify new data points without requiring the original dataset anymore. First, a function form is selected (linear, polynomial, etc.), then the coefficients for the function are learned form the training data.}
	
\textcolor{blue}{Examples of parametric models may be simple artificial neural networks, naive bayes, logistic regression, etc.}

\subsubsection{nonparametric}

%% unsure about this! 
\textcolor{red}{Nonparametric models are not models without parameters, rather they are models were the number of parameters are not fixed, they may grow with the number of training instances}

\textcolor{blue}{An Example of a nonparametric model may be k-Nearest neighbors -- where the model does not assume anything about the form of the mapping function and makes predictions based on the k most similar training instances.}

\textcolor{blue}{A disadvantage to this type of approach is that the computational complexity for classifying new samples grow linearly with the number of samples in the training set.}

\textcolor{blue}{May be useful when little is known about the underlying relationship in the data and there is an abundance of data.}

\subsection{Eager vs Lazy Learners}

\textcolor{green}{TODO: Eager vs Lazy overview}
\textcolor{blue}{Training an eager learner is often more computationally expensive, but typically prediction with the resulting model is inexpensive.}

\subsubsection{Eager Learners}

\textcolor{blue}{Eager learners estimate the parameters of a model that generalize to a training set --- build an input-independent model}

\subsubsection{Lazy Learners}

\textcolor{blue}{Also known as Instance-based Learners}

\textcolor{blue}{do not spend time traioning, but may predict responses slowly (relatively) compared to eager learners}

\textcolor{blue}{Lazy learners store the training dataset with little to no processing.}


\subsection{Generative vs Discriminative Models}

\textcolor{green}{TODO: Generative vs Discriminative models overview}
%\textcolor{blue}{}

\subsubsection{Discriminative Models}

\textcolor{green}{TODO: Discriminative Models --- learn a decision boundary that is used to \textit{discriminate} between classes. There exist both probabilistic and non-probabilistic discriminative models}

\paragraph{Probabilistic Discriminative}

\textcolor{blue}{Probabilistic discriminative models learn to estimate the conditional probability i.e. which class is most probable given the input features.}

\paragraph{Non-probabilistic Discriminative}

\textcolor{blue}{Non-probabilistic discriminative models directly map features to classes.}

\subsubsection{Generative Models}

% see p129[117] of Mastering ML w/SKL
\textcolor{green}{TODO: Generative Models --- do not learn a decision boundary, rather, they model the joint probability distribution of the features and classes i.e. they model how the classes generate features. Then, using Bayes' theorem, they are able to estimate the conditional probability of a class given the features.}


% see p130[118] of Mastering ML w/SKL
\textcolor{blue}{One advantage of generative models is that they can be used to generate new examples of data}

\subsection{Strong vs Weak Learners}

\textcolor{green}{TODO: Strong vs Weak learners (classifier, predictor, etc.) overview}
%\textcolor{blue}{}

\subsubsection{Strong Learners}

\textcolor{green}{TODO: Strong Learners are models that are arbitrarily better than weak learners.}

\subsubsection{Weak Learners}

\textcolor{green}{TODO: Weak Learners are models (typically simple models) that perform only slightly better than random chance.}


\section{Online Learning}

% See p.246 of Understanding Machine learning
\textcolor{blue}{difference to \textcolor{red}{PAC learning?}}

\section{Ensemble Methods}

\textcolor{green}{TODO: overview - discussed in more detail in \textcolor{red}{local ref?}}

\section{Kernel Trick}

\textcolor{blue}{Transform the training data onto a higher dimensional feature space}

% see p177[165] of mastering ML with SKL

\textcolor{blue}{The kernel is a function that XXXXXXXXXX}

\textcolor{blue}{Choosing an appropriate kernel can be challenging}

% see p180[168] of mastering ML w/SKL for more on kernels
\textcolor{blue}{Some commonly used kernels include polynomial, sigmoid, Guassian, and linear kernels}

\textcolor{blue}{commonly used in SVMs (see \textcolor{red}{local ref}), the kernel trick can be used with any model that can be expressed in terms of the dot product of two feature vectors.}

%%%%%%%%%%%%%%%%%%% Hyper-parameters
\section{Hyper-Parameters}

\TD{A disciplined approach to neural network hyper-parameters: Part 1 \cite{DBLP:journals/corr/abs-1803-09820}}


\subsection{Parameters: "tuning knobs"}

\subsubsection{Learning Rate}
\label{hp_learning_rate}

\TD{TODO: Learning rate overview}

% TODO: Learning rate practical advice

% TODO: figure showing cost vs iteration for a LR that is too small, just right, and too large

% TODO: Learning rate figure showing how if the learning rate is too high, you'll likely see the cost diverage when plotted vs iterations

% TODO: schedules

\r{In general, if the LR is too small, convergence (with something like gradient descent) may be slow.  If LR is too large, then convergence may not occur and the reduction in error may oscillate wildly or may even diverge.}

\TD{The large learning rate phase of deep learning: the catapult mechanism \cite{Lewkowycz2020TheLL}}

\paragraph{Schedule}

\r{Rather than keep the same learning rate during all of training, the learning rate is adjusted during training according to a ``schedule''.}

\paragraph{Descriminative/Differential}

\r{FastAI -- ``discriminative'' however, typically shows up as ``differential'' learning rate. Rather than use the same learning rate for all layers/components, different layers/components use different learning rates. (e.g. use a lower learning rate for pretrained layers and a normal/higher learning rate for the customlayers that follow.)}

\paragraph{research}

% TODO: this section may not belong here - may belong in an "advanced section"

%%%% learning rates
\textcolor{blue}{cyclic learning rate~\cite{smith2017cyclical}}

\textcolor{blue}{sgdr: stochastic gradient descent with restarts~\cite{loshchilov2016sgdr} (SGDR). The learning rate is decreased from the max value along a curve (cosine, shown in Eq.\ref{eq:sgdr_def}, where $n_{max}^i$ and $n_{min}^i$ are ranges for the learning rate, $T_i$ represents epochs, $T_{cur}$ is how many epochs have been performed since the last restart). The authors also suggest making each next cycle longer than the previous cycle by a constant $T_mul$ may be beneficial.}

\r{LR annealing, Cosine anealing ($1/2$ cosine curve)}

% \TD{`differential learning rate'/different learning rates at different levels of the network} blog: https://blog.slavv.com/differential-learning-rates-59eff5209a4f

\begin{equation}
{n_t = n_{min}^i + 1/2(n_{max}^i - n_{min}^i)(1 + cos(\frac{T_{cur}}{T_i}\pi))}
\label{eq:sgdr_def}
\end{equation}

\subsubsection{Batch size}

\textcolor{green}{TODO: batch size overview}

\r{anywhere from a single instance to the entire training set size.}

\textcolor{blue}{optimal batch size is problem dependent}

\textcolor{blue}{TODO: notebook and plots showing how the smoothness is affected when comparing batch sizes of 1 vs 10 vs 20 etc.}

% related to shuffling - the gradients are computed on a batch and so a batch should be representative of the data

%%%%% small batch size
\r{small minibatch sizes (between 2 and 32) may be better than large batch sizes~\cite{masters2018revisiting}.}

\r{``generalization gap'' may not be due to large mini-batches, rather, due to the number of updates made to the system~\cite{hoffer2017train}}

%%%% minibatch
\r{Batch training is almost always slower to converge than on-line/mini-batch training, which is likely due to the fact that on-line/mini-batches learning will follow the error surface, allowing for larger learning rates, and thus faster convergence~\cite{wilson2003general}.}

% incrementing batchsize over time
\r{Increasing the batch size may achieve similar benefits to decaying the learning rate ~\cite{smith2017don} -- which could lead to use of larger batch sizes, reducing the number of parameter updates and therefore reducing training time.}

\r{minibatches use the hardware more efficiently}


\subsection{Hyper-Parameter Optimization}

% \r{opinion: perfomed last to eek out extra performance}



\subsubsection{Coordinate Descent}

All hyper-parameters remain fixed, except for the hyper-parameter of interest. The hyper-parameter of interest is then adjusted such that the validation error is minimized.

\r{bayesian $>$ random $>$ grid}

\subsubsection{Grid Search}

\textcolor{blue}{{Grid search}\index{Grid search} Exhaustive search that trains+evaluates a model for each combination of specified hyperparameter configurations and combinations defined by a Cartesian product of the sets of possible values for each hyperparameter.}

\subsubsection{Randomized Search}

\r{{Randomized search}\index{Randomized search} }

\r{TODO: figure demonstrating difference between grid and randomized search}

\TD{TODO: grid vs random search figure}

\subsubsection{Other Methods: Automated / Model-based Methods}

\textcolor{blue}{See \textcolor{red}{local ref? --- advanced methods and research}}

\paragraph{Bayesian Methods}

\TD{todo:}




%%%%%%%%%%%%%%%%%%%%%%%% Optimizers

\section{Estimating Model Parameters}
\section{Initialization}

\TD{initialization --- how we define/set the initial values of parameters}

\TD{largely focused on neuralnetworks initialization}

\TD{TODO: initialization methods and importance}

\TD{figure showing the importance of initialization strategies for different architectures after \textit{n} layers}

\r{motivated partially by reducing the possibility of exploding or vanishing gradients.}

\TD{AutoInit: Analytic Signal-Preserving Weight Initialization for Neural Networks \cite{Bingham2021AutoInitAS}}

\TD{basic idea: initialize with small random values, typically from uniform or gaussian --- more advanced: hueristics based on characteristics --- motivation: }


\subsection{Parameter types (the initialization of)}

\TD{TODO: different types of parameters may benefit from different strategies}

\paragraph{Weights}

\TD{TODO: fully connected, convolution}

Break symmetry -- two things:
\begin{itemize}[noitemsep,topsep=0pt]
	\item \r{Non-zero}
	\item \r{some diversity}
\end{itemize}

\paragraph{Biases}

% HUGO talk
\TD{initializing with negative values may encourage sparsity?}

\TD{fan in and fan out}

\subsection{Normal Vs Uniform}


\subsection{Strategies}

% TODO: Nice write up: https://machinelearningmastery.com/weight-initialization-for-deep-learning-neural-networks/
% also possibly useful: https://machinelearningmastery.com/why-initialize-a-neural-network-with-random-weights/
\TD{write up\cite{brownlee2021WeightInit}}

\TD{TODO: strategies overview}


\TD{fixup initialization \cite{zhang2019fixup}}

\TD{LeCun \cite{lecun2012efficient}}

\subsubsection{Glorot or Xavier}

\TD{\cite{glorot2010understanding}}

\r{xavier: derived based on linear activations (which isn't true for modern architectures)}



\subsubsection{he}

\TD{Kaiming initialization \cite{he2015delving}}

\subsubsection{Implementation}



\section{Estimating Model Parameters}

\subsection{Optimizers}

\textcolor{blue}{Estimate the values of the model's parameters that minimize the value of the cost function}

\subsubsection{Gradient Descent}

\textcolor{blue}{Gradient Descent --- overview --- optimization algorithm that can be used to estimate the local minimum of a function}

\textcolor{blue}{Iteratively updates the model parameters by calculating the partial derivatives of the cost function at each step during training}

\textcolor{blue}{Gradient descent is only guaranteed to find the local minimum of the cost function.}


\paragraph{Batch Gradient Descent}

\textcolor{blue}{batch gradient descent --- taking a step (update the weights) opposite (down) the gradient calculated from the entire training set}

\textcolor{blue}{Batch gradient descent is deterministic --- will produce the same paramter values if the same dataset is used multiple times.}


\paragraph{Stochastic Gradient Descent}

\textcolor{blue}{Stochastic Gradient Descent (sometimes called iterative or on-line gradient descent) --- rather than update the weights based on the sum of the accumulated errors, the weights are updated for each training sample}

\textcolor{blue}{Stochastic gradient descent is deterministic --- may produce the different parameter values if the same dataset is used multiple times. May not minimize the cost function as well as gradient descent but the approximation is often ``close enough''.}


\paragraph{Mini-batch Gradient Descent}

\textcolor{blue}{mini-batch gradient descent --- compromise between batch and stochastic gradient descent where the gradient is calculated over a batch of training data}

\textcolor{blue}{Since the gradient is calculated on a single example, the error surface will appear noisier than if it was calculated over a batch or the entire training set.}

\textcolor{blue}{When using stochastic gradient descent, it is important to shuffle the data after each epoch.}

%%%%%%%%%%%%%%%%%%%%%%%% Evaluation

\r{Importance of dataset partitioning \textcolor{red}{local ref?}}

% \textcolor{blue}{The best performance measure will vary depending on the task. For instance, in a medical setting, it may be life threating to classify an event as ``healthy'' when the patient is not healthy.}

\r{A performance measure is used to capture, empirically, how well a prediction made by the model aligns with the expected, ground truth, value.}

\r{Evaluation metrics allow for intuitive explaination of the results to those who may be non/less-technical}

\subsection{Creating a Test Set}

% rough para
\r{The most important rule regarding evaluating models, is to ensure that the data used to evaluate the model has never been used before to influence the during training or selection -- this means it was not used during training to update the parameters and it was not used to influence which models are `best' (like a validation set may be used for)}

\r{The performance of a model on a test set may be indicative of how well the model can generalize to unseen data. (This assumes your data sample is representative of the data population)}

\r{Hold-out test set -- created by randomly sampling the dataset. Again, it is important to emphasize that the instances in the test set are never used in the training process and are instead reserved for use only during the evaluation phase.}

\r{peeking\index{peeking}, is an issue that arises when part or all of the test set is included in the training set. This means the model has already seen the data on which the model will be evaluated and so it is possible, probable, that the model will produce high evaluation scores, which will likely translate to an overoptimistic estimation of the models performance when used in production.}

\r{Evaluating the performance of a model can be challenging and will vary depending on the task. For instance, accuracy may not always be the best measure of performance -- consider a medical setting in which sensitivity may be more important since a false negative may be life threatening where as a false positive may only require additional observation.}

\r{When comparing various models, it may be challenging to rank them on a single performance measure. \textcolor{green}{TODO: more.}}

\subsection{Qualitative Evaluation}

\r{generalization is a measure of how well the system preforms on previously unseen data. generalization error.}



\subsubsection{(Over$|$Under)fitting and Capacity}

\r{{Model capacity}\index{model capacity} helps control how likely a model is to overfit or underfit. Where a model with low capacity may have difficulty fitting a a training set and a model with high capacity may ``overfit'' the data by essentially memorizing the training data.}

\r{Model capcity is closely related to model complexity and the models {hypothesis space}index{hypothesis space} (The set of functions available to the learning algorithm --- \textcolor{green}{TODO: expand - for example a linear vs polynomial model})}

\TD{TODO: figure showing training and validation error and 1) optimal capacity, 2) under and overfitting region 3)generalization gap, 4) capacity}

\paragraph{Overfitting}

\r{Overfitting\index{Overfitting} refers to a case in which a model fits the training data very well but does not fit validation/test set. If a model is overfitting, it is said to have a high variance and is analogous to memorizing the training set.}

\r{Overfitting can arise from modeling data with too many parameters/too complex of a model.}

\r{learning ``particularities in the training set''}

\TD{TODO: figure showing an example of overfitting}

% addressing overfitting: 1) reduce number of features (manual selection or w/model selection algor) 2) regularization

\begin{figure}[htp]
	\centering
	\includegraphics[width=0.3\textwidth]{example-image-a}\hfil
	\includegraphics[width=0.3\textwidth]{example-image-b}\hfil
	\includegraphics[width=0.3\textwidth]{example-image-c}\hfil
	\caption{Figure example showing the same 2d dataset and an underfitting, overfitting, and ``good'' fitting. \textcolor{green}{TODO} circles=training, x=test -- include scores for each.. slight curve' under=linear, over=extreme poly, good=``smooth''}
	\label{fig:basics_eval_fitting_examples}
\end{figure}

\paragraph{Underfitting}

\r{Underfitting\index{Underfitting} refers to a case in which a model does not fit the training data well. If a model is underfitting, it is said to have a high bias}

\r{Underfitting can arise from modeling data with too few parameters/too simple of a model.}

\TD{TODO: figure showing an example of underfitting}

\subparagraph{Solution}

\TD{method for better optimization and increasing model capacity: greedy layer-wise --- unsupervised pre-training}

\r{better optimization --- use better optimization methods \ALR}


\subsubsection{Bias Variance Trade-off}

\r{Two fundamental causes of prediction error in a model -- the bias and the variance.}

\paragraph{Variance}
\r{variance\index{Variance} refers to the amount the model would change (consistency or variability) if it was re-trained/estimated multiple using a different subsets of the training data set. A model that has high variance is sensitive to randomness in the training data}

\r{A model with high variance may be described as highly flexible and will likely overfit the data.}


\paragraph{Bias}
\r{Bias\index{Bias} refers to the amount of error that is introduced by approximating a problem with a model that is simpler than the (complex) problem}

\r{A model with high bias will produce similar errors for instances regardless of the training data that is used to train the model -- the model is more strongly ``biased'' to its own assumptions of the relationship (as defined by the model), than the relationship the data may be indicating. A model with high bias may also be described as inflexible and will likely underfit the data.}


% not word-for-word, but example adapted from p35 of ISL
\textcolor{red}{For example, linear regression assumes a linear relationship between the features and labels. However, it is unlikely that a true linear relationship exists and so using linear regression to model this type of particular problem will likely introduce some bias.}

\paragraph{Trade-Off}

% TODO: see page 34 of ISL for eq and explaination here

\r{In general, as a more ``flexible'' model is used, the variance will increase and the bias will decrease.}

\r{One reason to choose a more restrictive model is that they are often more interpretable.}

\begin{figure}[htp]
	\centering
	\includegraphics[width=0.4\textwidth]{example-image-a}\hfil
	\includegraphics[width=0.4\textwidth]{example-image-b}\hfil
	\caption{\TD{side by side figure: a: complex vs simple training on trainnig data (nth poly vs linear), b: same models on test data}}
	\label{fig:basics_eval_tradeoff_examples}
\end{figure}


% see page 36 of ISL
\r{It is easy to obtain a model with low bias but high variance (\emph{e.g.} drawing a squiggly line through every training observation) and it is easy to obtain a model with low variance but high bias (\emph{e.g.} drawing a straight line approximating every training observation) but it is difficult to obtain a model that has both low variance and low bias.}

\textcolor{blue}{It should be noted that in a real world example, it may not be possible to explicitly calculate the test error, bias, or variance.}



\begin{figure}[htp]
	\centering
	\includegraphics[width=0.3\textwidth]{example-image-a}\hfil
	\includegraphics[width=0.3\textwidth]{example-image-b}\hfil
	\includegraphics[width=0.3\textwidth]{example-image-c}\hfil
	\caption{\TD{same 2D dataset with 3 layers and the hidden layer in a has few nodes, b: normal amount of nodes, and c: many nodes} \r{illistrative that the number of connections and complexity increases the chances for overfitting also increases}}
	\label{fig:basics_eval_nodesinhidden}
\end{figure}


\begin{figure}[htp]
	\centering
	\includegraphics[width=0.2\textwidth]{example-image-a}\hfil
	\includegraphics[width=0.2\textwidth]{example-image-b}\hfil
	\includegraphics[width=0.2\textwidth]{example-image-c}\hfil
	\includegraphics[width=0.2\textwidth]{example-image-a}\hfil
	\caption{\TD{same 2D dataset with one, two, three and four hidden layers}}
	\label{fig:basics_eval_numlayers}
\end{figure}

\r{observing a direct trade-off between overfitting and model complexity.}

\r{When we talk about deep learning, we're talking about deep and powerful models that are attempting to solve complex problems that are prone to overfitting and thus usually employ additional countermeasures, such as regularization, to help prevent overfitting.}


\TD{TODO: para about using regularization here/finding the right balance \textcolor{red}{local ref to regularization?}}



%%%%%%%%%%% Metrics (subsec nested under sec.Eval)
\emph{Cost} is frequently used interchangeably with loss. Technically, loss refers to the error on a single example and cost is the average of the loss across the entire training set.

% page 94 of AGtext
One-versus-all \emph{OvA} (also \emph{one-versus-rest})

One-versus-one (OvO) -- train a binary classifier for every pair


\section{Metrics}

%% Confusion matrix
\begin{table}
	\centering
	\begin{tabular}{l|l|c|c|}
		\multicolumn{2}{c}{}&\multicolumn{2}{c}{Ground Truth}\\ 
		\cline{3-4}
		\multicolumn{2}{c|}{}&Positive&Negative\\ 
		\cline{2-4}
		\multirow{2}{*}{\rotatebox{90}{Pred}}& Positive & $TP$ & $FP$ \\ 
		\cline{2-4}
		& Negative & $FN$ & $TN$ \\ 
		\cline{2-4}
	\end{tabular}
	\caption{Example confusion matrix}
	\label{tab:sample_conf_matrix}
\end{table}


\begin{itemize}
	
\item \textit{Accuracy}, (Eq.~\ref{eq:accuracy}): the ratio of correct predictions to the total number of predictions.

\begin{equation}
{\frac{TP+TN}{TP+TN+FP+FN}}
\label{eq:accuracy}
\end{equation}

\item \textit{Sensitivity}, (Eq.~\ref{eq:sensitivity}): the ratio of true positives that are correctly identified.

\begin{equation}
{\frac{TP}{TP+FN}}
\label{eq:sensitivity}
\end{equation}

\item \textit{Precision}, (Eq.~\ref{eq:precision}): the ratio of positives that are, in fact, positive. If the classifier predicts positive, how often is is correct?

\begin{equation}
{\frac{TP}{TP+FP}}
\label{eq:precision}
\end{equation}

\item \textit{AUC (Area Under the Curve)}, is a single value representing the area under an ROC curve. Though generally referred to as the AUC, the term is correctly abbreviated AUROC, specifying that the curve is an ROC curve.
\end{itemize}

