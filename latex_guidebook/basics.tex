\chapter{Basics}

%%%%%%%%%%%%%%%%%%%%%%%% Types of data
\textcolor{blue}{categorical or numerical. Numerical can be discrete or continuous}

%%%%%%%%%%%%%%%%%%%%%%%% Measurement Levels
\textcolor{blue}{qualitative or quantitative.} 

\textcolor{blue}{Qualitative can be nominal (aren't numbers and can't be put in any order -- e.g. the seasons: spring, summer, fall, winter) or ordinal (groups and categories that follow a strict order -- e.g. difficult levels: hard, medium, or easy)}

\textcolor{blue}{Quantitative are represented by numbers but can be interval (0 is meaningless -- e.g. temperature in C or F, where true zero is not 0) or ratio (has a true 0 -- e.g. temperature in K, weight or length)}


%%%%%%%%%%%%%%%%%%%%%%%% Acquiring Data
\section{Data Acquisition}


\subsection{Resources}

\subsection{Generating Fake Data}

\textcolor{green}{TODO: generating fake data with SKL}

\textcolor{blue}{make\_blobs}

% {{{datagen_blobs_2dcode}}}
\begin{lstlisting}[style=pyInStyle]
X, y = datagen.make_blobs(centers=4, n_samples=100, n_features=2,cluster_std=1.0,
                          center_box=(-10, 10),
                          random_state=42, shuffle=True)
\end{lstlisting}

% {{{datagen_blobs_2dimg}}}
\begin{figure}
\centering
\includegraphics[width=0.65\textwidth]{./sync_imgs/datagen/blobs/2dimg.png}
\label{fig:datagen_blobs_2dimg}
\end{figure}

\textcolor{blue}{Data can also be generated in three (multiple) dimensions}

% {{{datagen_blobs_3dcode}}}
\begin{lstlisting}[style=pyInStyle]
X, y = datagen.make_blobs(centers=4, n_samples=100, n_features=3, random_state=42)
\end{lstlisting}

% {{{datagen_blobs_3dimg}}}
\begin{figure}
\centering
\includegraphics[width=0.65\textwidth]{./sync_imgs/datagen/blobs/3dimg.png}
\label{fig:datagen_blobs_3dimg}
\end{figure}

%\textcolor{blue}{More dataset types can be generated, the documentation can be found at (http://scikit-learn.org/stable/modules/classes.html#module-sklearn.datasets)}


\textcolor{blue}{see \textcolor{red}{local ref?} for more examples on how to generate data}



%%%%%%%%%%%%%%%%%%%%%%%% Data Pre-processing
\section{Data Pre-processing}

\textcolor{blue}{Data is rarely obtained in a form that is necessary for optimal performance of a learning algorithm. Data can be missing, can contain a mix of categorical and quantitative, can contain values on vastly different scales, etc.}

\textcolor{blue}{It is important to note that any parameters related to data pre-processing, such as feature scaling and dimensionality reduction, are obtained solely from observing the training set. The parameters for these methods obtained on the training set are then later applied to the test set. This is important since if these preprocessing parameters were obtained on the entire dataset and included the test set, the the model performance may be overoptimistic since then when applying the methods to the unseen data.}

\subsection{Handling Missing Data}

\subsubsection{Filtering Out}

\textcolor{blue}{Simply removing any entries that are missing data. This is convenient and easy but may not be practical -- any time data is being removed, potentially useful information is lost and too much data may be removed.}

\textcolor{green}{TODO: Code in jupyter on how to do this with pandas and dropna -- key params - how, thresh, subset}

\subsubsection{Filling In}

\textcolor{blue}{Estimating the missing data}

\subsection{Handling Categorical Data}

\subsubsection{Encoding}

\subsection{Feature Scaling, Normalization}

\subsubsection{Min-Max scaling (Normalization)}

\textcolor{blue}{values are shifted and rescaled so they end up on a [0,1] range}

\subsubsection{Standardization}

\textcolor{blue}{(Eq.~\ref{eq:preprocess_standardization}) first, subtract the sample mean, then divide by standard deviation variance}

\textcolor{blue}{pros: unlike min-max, not bound to specific range}

\textcolor{blue}{standardized values always have a zero mean and a standard deviation of 1.}

\textcolor{blue}{gives our data the property of a standard normal distribution}

\begin{equation}
{X' = \frac{X - \mu}{\sigma}}
\label{eq:preprocess_standardization}
\end{equation}

\textcolor{green}{TODO: create code sample - numpy, and sklearn methods}


\subsection{Others}

\subsubsection{Removing Duplicates}

\subsubsection{Outliers}

\subsubsection{Discretization and Binning}



%%%%%%%%%%%%%%%%%%%%%%%% Data sampling and partitioning
\section{Partitioning Data}

\subsection{Sampling}

\textcolor{blue}{Training, validation, test}

\section{Some Terms}

\emph{input variable(s)} -- predictors, independent variables, features, or simply variables.

\emph{output variable(s)} -- response or dependent variable

\emph{relationship} $Y = f(x) + \epsilon$ \textcolor{blue}{estimate $f$. prediction and inference}.

\textcolor{blue}{\emph{reducible error\index{reducible error}} -- the estimated function $\hat{f}$ will likely not be perfect, and the reducible error is the error that could be corrected.  The \emph{irreducible error} is an error that can not be corrected. The irreducible error may be larger than zero due to \emph{unmeasured variables} \emph{e.g.} varibles that were not measured and \emph{unmeasurable variation} \emph{e.g.} an individual's feelings/emotions or variation in the production of a product. The irreducible error provides an upper bound on the performance of the predicted $\hat{f}$}


\section{Supervised vs Unsupervised}

\subsection{supervised} -- 

\subsection{unsupervised} -- observe input variables without corresponding output values.

% clustering

% principal components

\section{Classification vs Regression}

\subsection{Regression} -- predicting a continuous or quantitative output value

\subsection{Classification} -- predicting categorical or qualitative output value (such as a non-numerical value)

\subsection{Bayes Classifier}



\section{Training}

\section{Quality of Fit}

%% regression example

\subsection{Regression Example}

\textcolor{blue}{Mean Squared Error.$\hat{f}(x_i)$ is the prediction that $\hat{f}$ produces for the $i$th sample. The output will be small for predicted values that are similar to the ground truth}

\begin{equation}
{MSE = \frac{1}{n}\sum_{i=1}^{n}(y_i - \hat{f}(x_i))^2}
\label{eq:MSE_def}
\end{equation}

%% classification example

\subsection{Classification Example}

\textcolor{blue}{The proportion of mistakes that are made.}

\begin{equation}
{error\_rate = \frac{1}{n}\sum_{i=1}^{n}(y_i \ne \hat{y_i})}
\label{eq:class_error_rate_def}
\end{equation}

\textcolor{blue}{$\hat{y_i}$ is the predicted classification label for the $i$th observation using our predictor/model $\hat{f}$ and $y_i$ is the ground truth label}

\section{(Over|Under)fitting}

\subsection{Overfitting}

\textcolor{blue}{Overfitting refers to a case in which a model fits the training data very well but does not fit validation/test set}

\subsection{Underfitting}

\section{Bias Variance Trade-off}

\subsection{Variance}
\textcolor{blue}{variance refers to the amount the model would change if it was trained/estimated using a different training data set}

\subsection{Bias}
\textcolor{blue}{Bias refers to the amount of error that is introduced by approximating a problem with a model that is simpler than the complex problem} 
% not word-for-word, but example adapted from p35 of ISL
\textcolor{red}{For example, linear regression assumes a linear relationship between the features and labels. However, it is unlikely that a true linear relationship exists and so using linear regression to model this type of particular problem will likely introduce some bias.}

\subsection{Trade-Off}

% TODO: see page 34 of ISL for eq and explaination here

\textcolor{blue}{In general, as a more ``flexible'' model is used, the variance will increase and the bias will decrease.}

% see page 36 of ISL
\textcolor{blue}{It is easy to obtain a model with low bias but high variance (\emph{e.g.} drawing a squiggly line through every training observation) and it is easy to obtain a model with low variance but high bias (\emph{e.g.} drawing a straight line approximating every training observation) but it is difficult to obtain a model that has both low variance and low bias.}

\textcolor{blue}{It should be noted that in a real world example, it maynot be possible to explicitly calculate the test error, bias, or variance.}

\subsection{Parametric vs non-parametric}



%%%%%%%%%%%%%%%%%%%%%%%% Metrics
\emph{Cost} is frequently used interchangeably with loss. Technically, loss refers to the error on a single example and cost is the average of the loss across the entire training set.

% page 94 of AGtext
One-versus-all \emph{OvA} (also \emph{one-versus-rest})

One-versus-one (OvO) -- train a binary classifier for every pair


\section{Metrics}

%% Confusion matrix
\begin{table}
	\centering
	\begin{tabular}{l|l|c|c|}
		\multicolumn{2}{c}{}&\multicolumn{2}{c}{Ground Truth}\\ 
		\cline{3-4}
		\multicolumn{2}{c|}{}&Positive&Negative\\ 
		\cline{2-4}
		\multirow{2}{*}{\rotatebox{90}{Pred}}& Positive & $TP$ & $FP$ \\ 
		\cline{2-4}
		& Negative & $FN$ & $TN$ \\ 
		\cline{2-4}
	\end{tabular}
	\caption{Example confusion matrix}
	\label{tab:sample_conf_matrix}
\end{table}


\begin{itemize}
	
\item \textit{Accuracy}, (Eq.~\ref{eq:accuracy}): the ratio of correct predictions to the total number of predictions.

\begin{equation}
{\frac{TP+TN}{TP+TN+FP+FN}}
\label{eq:accuracy}
\end{equation}

\item \textit{Sensitivity}, (Eq.~\ref{eq:sensitivity}): the ratio of true positives that are correctly identified.

\begin{equation}
{\frac{TP}{TP+FN}}
\label{eq:sensitivity}
\end{equation}

\item \textit{Precision}, (Eq.~\ref{eq:precision}): the ratio of positives that are, in fact, positive. If the classifier predicts positive, how often is is correct?

\begin{equation}
{\frac{TP}{TP+FP}}
\label{eq:precision}
\end{equation}

\item \textit{AUC (Area Under the Curve)}, is a single value representing the area under an ROC curve. Though generally referred to as the AUC, the term is correctly abbreviated AUROC, specifying that the curve is an ROC curve.
\end{itemize}

